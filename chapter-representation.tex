%%%%%%%%%%%%%%%%%%%%%%%%%%%%%%
%%%%%                    %%%%%
%%%%%   Representation   %%%%%
%%%%%                    %%%%%
%%%%%%%%%%%%%%%%%%%%%%%%%%%%%%

\section{Representation}
\label{chap_representation}



\subsection{Divisor Types}

\begin{itemize}
  \item Arita's classification
\end{itemize}



\subsection{Reduced Divisors}

\begin{itemize}
  \item How to determine type of $\bar D$.
  \item Definition of reduced divisor
  \item Types 0, 11, 21, 22, 31 reduced
\end{itemize}



\subsection{Typical Divisors}

\begin{itemize}
  \item Oyono's definition
  \item Kamal's definition
  \item Comparison to Arita
  \item Type *1 can typically be represented by first two polys.
\end{itemize}



\subsection{Geometric Interpretation}

\begin{itemize}
  \item Type 0 : nothing 
  \item Type 11 : a single rational affine point
  \item Type 21 : two points with distinct x-coordinates or a double point with non-vertical tangent line
  \item Type 22 : two points with same x-coordinates or a double point with vertical tangent line
  \item Type 31 : three distinct non-colinear points. Non-distinct : no straight line passes through the points with correct multiplicity
  \item Type 32 : three colinear points, distinct x-coordinates. Or non-distinct points interpolated by a non-vertical line
  \item Type 33 : three colinear points, same x-coordinates, or non-distinct points interpolated by vertical line.
\end{itemize}



\subsection{Type 31 Divisors}

\begin{itemize}
  \item
\end{itemize}



\subsection{Type 61 Divisors}

\begin{itemize}
  \item
\end{itemize}


%%%%%%%%%%%%%%%%%%%%%%%%
%%%%%              %%%%%
%%%%%   Divisors   %%%%%
%%%%%              %%%%%
%%%%%%%%%%%%%%%%%%%%%%%%

\section{The Divisor Class Group}
\label{chap_divisors}



\subsection{Divisors}

Let $C$ be a $C_{3,4}$ curve.
A \defn{divisor} $D$ of $C$ is a formal sum of points in $C(\bar K)$, including possibly the point at infinity, $P_\infty$.
If $P$, $Q$, and $R$ are points on the curve $C$, then examples of divisors include
  \[ \begin{array}{c} P + Q + R - 3P_\infty \\ P + 3Q - 2R \\ Q \\ 0 \end{array}. \]
More generally, a divisor has the form
  \[ D = \sum_{P \in C(\bar K)} n_P P,\]
where only finitely many $n_P$'s are non-zero.

A divisor is an element free Abelian group generated by the set of points $C(\bar K)$.
This Abelian group is denoted by $\Div(C)$.
Addition and negation are defined in the obvious way:
\[ \left( \sum_{P \in C(\bar K)} n_P P \right) + \left( \sum_{P \in C(\bar K)} m_P P \right) = \sum_{P \in C(\bar K)} (n_P + m_P) P \]
\[ -\left( \sum_{P \in C(\bar K)} n_P P \right) = \sum_{P \in C(\bar K)} (-n_P) P. \]

The coefficient $n_P$ of $P$ is the \defn{order} of the divisor $D$ at $P$, denoted by $\ord_P(D)$.
For example, if $D = P + 3Q - 2R$, then $\ord_Q(D) = 3$ and $\ord_R(D) = -2$.
The \defn{degree} of a divisor $D$, denoted by $\deg(D)$, is the sum of its orders at each point on the curve:
  \[ \deg(D) = \sum_{P \in C(\bar K)} \ord_P(D). \]
For example, $\deg(P + 3Q - 2R) = 1 + 3 - 2 = 2$.

It is easily verified that the map $\deg$ and the family of maps $\ord_P$ have the additive properties
  \[ \ord_P(A + B) = \ord_P(A) + \ord_P(B) \]
  \[ \deg(A + B) = \deg(A) + \deg(B). \]
In particular, $\ord_P$ and $\deg$ are group homomorphisms $\Div(C) \to \bb Z$.

We are able to put a partial order on divisors.
For two divisors $D = \sum n_P P$ and $D' = \sum m_P P$,
we order them
  \[ D \leq D' \iff \forall P \in C(\bar K) : n_P \leq m_P. \]
The divisor $D$ precedes $D'$ if it has lesser order than $D'$ at every point.
This partial order is compatible with addition, in the sense that for any divisors $A, B, D$,
  \[ A \leq B \implies A + D \leq B + D \]
  \[ A \leq B \implies -B \leq -A. \]
If $D \geq 0$, then $D$ is called an \defn{effective} divisor.

Divisors of a curve, together with this partial order, form a lattice --
every pair of divisors have a unique join and meet,
which we call their \defn{least common multiple} and \defn{greatest common divisor}.
Given two divisors $D$ and $D'$, their least common multiple is the unique, smallest divisor $L$ such that $D \leq L$ and $D' \leq L$.
Their greatest common divisor is the unique, largest divisors $G$ such that $G \leq D$ and $G \leq D'$.
Explicitly,
\begin{align*}
  \lcm(D, D') &= \sum_{P \in C(\bar K)}\max\{\ord_P(D), \ord_P(D')\}P \\
  \gcd(D, D') &= \sum_{P \in C(\bar K)}\min\{\ord_P(D), \ord_P(D')\}P.
\end{align*}
Just as integers $a$ and $b$ satisfy the law
  \[ \left| ab \right| = \lcm(a,b)\gcd(a,b), \]
divisors satisfy the law
  \[ D + D' = \lcm(D, D') + \gcd(D, D'). \]
This property will play a prominent role in adding divisors \note{in chapter ???}.

There is a short chain of subgroups of $\Div(C)$,
\[ \Princ(D) \subset \Div_K^0(C) \subset \Div^0(C) \subset \Div(C), \]
which we will now describe.

Since $\deg : \Div(C) \to \bb Z$ is a group homomorphism, its kernel is a subgroup of $\Div(C)$.
The kernel, of course, is the subgroup of divisors of degree zero, which we denote by $\Div^0(C) := \ker \deg$.

Let $\sigma \in \Gal(\bar K / K)$.
In \note{an earlier chapter}, we define the action of $\sigma$ of $C(\bar K)$.
This may be extended to an action on $\Div(C)$ in a natural way.
If $D = \sum n_P P$, then define
\[ \sigma(D) = \sum_{P \in C(\bar K)} n_P \sigma(P). \]
Just as $\sigma$ permutes points in $C(\bar K)$, so too does it permute divisors in $\Div(C)$.
In this way, an automorphism in $\Gal(\bar K/K)$ is also an automorphism of $\Div(C)$.

Given an automorphism $f$ on a group $G$, the \defn{fixed-point subgroup} of $f$ is
  \[ G^f := \{ g \in G ~|~ f(g) = g \}. \]
Given a set $S$ of automorphisms on $G$, this may be generalized even further:
  \[ G^S := \{ g \in G ~|~ \forall f \in S : f(g) = g \}. \]
So $G^S$ is the set of group elements in $G$ fixed by every automorphism in $S$.

We say that a divisor $D$ is \defn{defined over $K$} if $D$ is fixed by every automorphism in $\Gal(\bar K/K)$.
Divisors defined over $K$ therefore form a subgroup $\Div_K(C) \subset \Div(C)$.
  \[ \Div_K(C) := \Div(C)^{\Gal(\bar K/K)} \]

\begin{example}
  Let $K$ be any field, $L$ any algebraic extension and $\sigma \in \Gal(\bar K/L)$.
  By definition, $\sigma$ fixes $L$.
  If $P$ is any point with coordinates in $L$, then $\sigma(P) = P$.
  If $D$ is any divisor consisting only of points with coordinates in $L$, then $\sigma(D) = D$ and $D$ is defined over $L$.
\end{example}
\begin{example}
  Let $K = \bb F_2$ and let $L = K(\alpha)$ be an algebraic extension with $\alpha^2 + \alpha = 1$.
  Let $C$ be the curve $x^4 + y^3 + x + 1$ over $K$.
  Let $P$ be the point $(\alpha : 1 : 1)$ on $C$ and let $D$ be the divisor $D = P$.
  There is an automorphism $\sigma \in \Gal(\bar K/K)$ that maps $\alpha \mapsto \alpha + 1$, so 
    \[ \sigma(D) = \sigma(P) = (\sigma(\alpha) : \sigma(1) : \sigma(1)) = (\alpha + 1 : 1 : 1) \neq D. \]
  Hence $D$ is not defined over $K$.
\end{example}
\begin{example}
  Let $K$, $L$, $C$, and $P$ be as in the previous example.
  Let $Q = (\alpha + 1 : 1 : 1)$, which is also a point on $C$.
  Let $D$ be the divisor $D = P + Q$.
  Every automorphism $\sigma$ in $\Gal(\bar K/K)$ maps either $\alpha$ to itself or to $\alpha + 1$.
  In the former case,
  \begin{align*}
    \sigma(D) &= \sigma(P) + \sigma(Q) \\
              &= (\sigma(\alpha) : \sigma(1) : \sigma(1)) + (\sigma(\alpha + 1) : \sigma(1) : \sigma(1)) \\
              &= (\alpha : 1 : 1) + (\alpha + 1 : 1 : 1) \\
              &= P + Q = D.
  \end{align*}
  In the latter case,
  \begin{align*}
    \sigma(D) &= \sigma(P) + \sigma(Q) \\
              &= (\sigma(\alpha) : \sigma(1) : \sigma(1)) + (\sigma(\alpha + 1) : \sigma(1) : \sigma(1)) \\
              &= (\alpha + 1 : 1 : 1) + (\alpha : 1 : 1) \\
              &= Q + P = D.
  \end{align*}
  So $D$ is defined over $K$.
\end{example}

The intersection of subgroups is again a subgroup, so define
  \[ \Div_K^0(C) := \Div_K(C) \cap \Div^0(C). \]
These are the divisors defined over $K$ of degree zero.

If $f \in K(C)$ is a rational function on $C$,
then $f$ has finitely may zeros and poles along $C$, and,
counting multiplicities, there are as many zeros as there are poles.
\note{What theorem is this?}
Define the \defn{order} of $f$ at $P$ by
\[ \ord_P(f) = \begin{cases}
  n & \text{$f$ has a zero of order $n$ at $P$} \\
  -n & \text{$f$ has a pole of order $n$ at $P$}
\end{cases}. \]
Define the \defn{divisor of $f$} as
\[ \div(f) = \sum_{P \in C(\bar K)} \ord_P(f) P. \]
Since $f$ has finitely many zeros and poles, this formal sum belongs properly to $\Div(C)$,
thus it is correctly called a divisor.
Since these poles and zeros add to zero, this is a degree zero divisor.
\note{Need a proof that this is defined over $K$.}
Therefore $\div (f)$ is an element of $\Div_K^0(C)$.

If $D$ is a divisor and there is an $f$ such that $D = \div(f)$,
then $D$ is called a \defn{principal divisor}.
Principal divisors form a subgroup \note{(prove it is closed under addition)} $\Princ(C) \subset \Div_K^0(C)$.

Finally, we have describe the subsets
\[ \Princ(D) \subset \Div_K^0(C) \subset \Div^0(C) \subset \Div(C). \]
The \defn{divisor class group} of $C$ is the quotient group
\[ \Cl(C) = \frac {\Div_K^0(C)} {\Princ(C)}. \]
In the literature, the divisor class group is often called the \defn{Jacobian} $\Jac(C)$ of $C$
\note{e.g. Abu Salem, Khuri-Makdisi, Arita, Basiri, Flon, Oyono, Harasawa}
or the \defn{Picard group} $\Pic^0(C)$ of $C$.
\note{Does Galbraith use this?}
\note{Hess and Weir use DCG.}
In this thesis, we will use the term divisor class group,
as the Picard group is usually defined in terms of line bundles
and the Jacobian in terms of a Jacobian variety,
neither perspective being adopted here.



\subsection{Remarks}

Let $D$ be a divisor and suppose $\ord_P(D) = -1$ for some point $P = (x_0 : y_0 : 1)$.
Consider the vertical line through $P$.
This line is given by the polynomial $f = x - x_0$ and intersects $C$ at three points (counting multiplicity), $P$, $Q$, and $R$.
Then $\div f = P + Q + R - 3P_\infty$.
In the divisor class group,
  \[ \div f \equiv 0 \equiv P + Q + R - 3P_\infty. \]
Thus $-P \equiv Q + R - 3P_\infty$.

Because of this, anytime an affine point appears in a divisor with some negative multiplicity,
it may be replaced by positive points minus some multiple of the point at infinity.
Every divisor may therefore be written in the form
  \[ D = P_1 + \ldots + P_n - nP_\infty \]
where the points $P_i$ are not necessarily distinct.
In other words, $D$ can be written in the form
  \[ D = D^+ - D^\infty \]
where $D^+$ is an effective divisor and $D^\infty = \deg(D^+)P_\infty$ (also effective).

Unless otherwise specified, we will assume that a divisor $D$ is of this form.
Since $D^\infty$ is determined uniquely by $D^+$,
we will furthermore drop the $D^\infty$ part and denote $D$ by its positive part only.
That is, if $D = P + Q + R - 3P_infty$, we will instead write simply $D = P + Q + R$ and call $D$ a degree 3 divisor.






\subsection{Prime Divisors}

We have a partial order on divisors, with the property that $A < B \implies \deg A < \deg B$.
On the subgroup of degree zero divisors, this partial order is uninteresting, since there are no two distinct degree zero divisors with $A < B$.

If $D$ is a degree zero divisor, then we can write $D = D_{\text{aff}} - D_\infty$, where $D_\infty = \deg D_{\text{aff}}P_\infty$.
The $D_\infty$ part is uniquely determined by the $D_{\text{aff}}$ part.
This leads to a more useful partial order on degree zero divisors.
For $A, B \in \Div^0(C)$
  \[ A \leq_{\text{aff}} B \iff A_{\text{aff}} \leq B_{\text{aff}}. \]
However, we will write $\leq$ rather than $\leq_{\text{aff}}$.

Now for degree zero divisors defined over $K$, we say that a divisor $0 < D \in \Div_K^0(C)$ is \defn{prime}
if for divisors $0 < A, B \in \Div_K^0(C)$,
  \[ D = A + B \implies A = D \text{ or } B = D. \]
The prime divisors are the least non-zero divisors in the lattice $(\Div_K^0(C), \leq)$.

\begin{proposition}
  Let $P$ be an affine point on $C$ and let $[P] = \sum_{P_0 \in \orb(P)}(P_0 - P_\infty)$.
  Then $[P]$ is a prime divisor.
\end{proposition}
\note{Show the converse as well! Make this if and only if.}
\begin{proof}
  It is clear that $0 < [P]$, since
  \[ 0 < \sum_{P_0 \in \orb(P)}P_0. \]
  Now suppose $A, B \in \Div_K^0(C)$ are such that $0 < A, B$ and $A + B = D$.
  Then $P$ is in the support of either $A$ or $B$.
  Without loss of generality, suppose $P \in \supp(A)$.
  Since $A$ is defined over $K$, the support of $A$ must also contain all other points in $\orb(P)$,
  whence $A = [P]$.
\end{proof}

Just as non-zero ideals in a Dedekind domain can be uniquely factored into products of prime ideals,
non-zero divisors in $\Div_K^0(C)$ can be factored into sums of prime divisors.
This relationship is explored in the next chapter.

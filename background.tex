
%%%%%%%%%%%%%%%%%%%%%%%%%%
%%%%%                %%%%%
%%%%%   Background   %%%%%
%%%%%                %%%%%
%%%%%%%%%%%%%%%%%%%%%%%%%%

\section{Background and Motivation}

\note{Generalizing Sato-Tate to higher genus curves requires Sato-Tate groups.
The theory is outside the scope of this thesis.}

A projective line over $\bb F_q$ has $q + 1$ rational points.
An elliptic curve over $\bb F_q$ has approximately as many rational points as does a line.
The actual number of rational points on an elliptic curve differs from that of the line by an error term
that is small compared to the size of the field $\bb F_q$.
The following theorem due to Hasse places bounds on this error term.
\begin{theorem}[Hasse's Theorem]
  Let $E$ be an elliptic curve over a finite field $\bb F_q$.
  The number of points on $E$ differs from $q + 1$ by at most $2 \sqrt q$.
  \[ | \#E(\bb F_q) - (q + 1) | \leq 2 \sqrt q \]
\end{theorem}

This theorem generalizes in the following way.
\begin{theorem}[Hasse-Weil Bound]
  Let $C$ be a curve of genus $g$ over a finite field $\bb F_q$.
  The number of points on $C$ differs from $q + 1$ by at most $2 g \sqrt q$.
  \[ | \#C(\bb F_q) - (q + 1) | \leq 2 g \sqrt q \]
\end{theorem}

Expressed differently, the Hasse-Weil Bound says that the quantity
\begin{equation}
  \frac {\#C(\bb F_q) - (q + 1)}{2 g \sqrt q}
\end{equation}
is between -1 and 1, so that for some angle $\theta_p$,
\begin{equation}
  \label{eq_sato_tate_ratio}
 \cos \theta_p = \frac {\#C(\bb F_q) - (q + 1)}{2 g \sqrt q}.
\end{equation}

Now given a fixed curve $C$, one can ask, what the probability is as $p$ varies that the error term should lie in some interval.
Given real numbers $-1 \leq a < b \leq 1$ (or angles $0 \leq \alpha < \beta \leq \pi$),
what is the probability that the ratio $\cos \theta_p \in [a, b]$ (or $\theta_p \in [\alpha, \beta]$)?
The Sato-Tate Conjecture (proven for genus 1), says that this probability follows a $\sin^2$ distribution.

\begin{conjecture}[Sato-Tate]
  Let $E$ be an elliptic curve without complex multiplication.
  \note{Complete this thought...}
  \[ \lim_{N \to \infty} \frac {\#\{ p \leq N : \alpha \leq \theta_p \leq \beta \}} {\pi(N)} = \frac 2 \pi \int_{\alpha}^{\beta} \sin^2 \theta\,d\theta. \]
\end{conjecture}

Divisor class group computations are used to compute the L-series of a genus 3 curve.
L-series is raised in
  * Birch and Swinnerton-Dyer conjecture
  * Koblitz-Zywina conjecture
  * Lang-Trotter conjecture
  * Sato-Tate conjecture

%%%%%%%%%%%%%%%%%%%%%%%%%%
%%%%%                %%%%%
%%%%%   Conclusion   %%%%%
%%%%%                %%%%%
%%%%%%%%%%%%%%%%%%%%%%%%%%

\section{Conclusion}
\label{chap_conclusion}

We have been able to accomplish the two main goals of this thesis:
to generalize the techniques of Abu Salem and Khuri-Makdisi to apply atypical divisors as classified by Arita,
and to find an improvement in the typical case as well.

With regards to the generalization to atypical divisors,
nowhere in Chapters \ref{chap_addition}, \ref{chap_doubling}, and \ref{chap_reduction}
did we impose any requirements on the characteristic of the field over which the curve is defined
or the curve equation being of long or short form.
Moreover, the algorithms described in those chapters terminate with correct outputs
even when the divisors involved are non-disjoint or contain points with order greater than 1.
Explicit formulae for each of the possible atypical cases\footnote{
Either adding one or more atypical divisors, or adding divisors whose sum turns out to be atypical.}
have been derived, but due to the sheer number of these cases,\footnote{
There are 13 cases arising from adding two type 31 divisors alone,
corresponding to $\type(\lcm(D, D')) = 41, 42, \ldots, 65$.}
these formulae may be found at \cite{github};
to include them all in this thesis would likely add another 100 or more pages,
and anyone who needs those formulae would probably prefer raw code over a pdf file or ink on paper.

With regards to the improvement in the typical case,
preliminary testing shows that we are able compute divisors faster than the previous state-of-the-art.
This was accomplished primarily by avoiding computing some unnecessary polynomial coefficients,
and by collection addition/doubling and reduction into a single step to eliminate a finite field inversion operation.
The representation of a typical divisor by a three-polynomial Gr\"obner basis $\pid{f,g,h}$,
rather than by a two-polynomial generating set $\pid{f,g}$ as in \cite{salem07},
allows for the matrix $M_{\text{add}}$ to be computed especially fast.
We require many more multiplications than the previous state-of-the-art,
but many fewer additions and one fewer inversion.
Likely, the trade-off of an inversion for several multiplications contributed the largest portion of the speed-up,
while lowering the number of additions likely contributed a non-trivial amount as well.

Sage code for the typical addition and doubling case is also available at \cite{github}.
Anyone wishing to perform computations in the divisor class group of a $C_{3,4}$ curve,
or wishing to translate that code into another language such as Magma,
may begin by downloading the implementation there.




%%%%%%%%%%%%%%%%%%%%%%%%%%%
%%%%%                 %%%%%
%%%%%   Future Work   %%%%%
%%%%%                 %%%%%
%%%%%%%%%%%%%%%%%%%%%%%%%%%

\subsection{Future Work}
\label{sec_future_work}

There are four avenues for possible improvements to $C_{3,4}$ curve arithmetic.
Let us end the thesis on a high note by listing them from least to most promising.

There is room for improvement in the arithmetic in the atypical cases.
In Chapter \ref{chap_typical_case}, we eliminated an expensive inversion operation
by combining addition/doubling and reduction into a single step.
The formulae in \cite{github} for handling the atypical cases perform addition/doubling and reduction discretely.
An inversion could also be eliminated from these cases
in much the same way as was done in Chapter \ref{chap_typical_case} for the typical case.
However, this is a lot of work for very little gain,
considering that any application for this arithmetic will likely spend very little time computing these atypical cases.

Representating divisors by Gr\"obner bases means carrying around some redundant information.
Typically, we represent a divisor by an ideal $\pid{f,g,h}$ where $h$ may be computed from $f$ and $g$.
This is a time-space trade-off.\footnote{
A careful reading of the formulae in \cite{salem07}, where the authors represent a divisor $D$ by only $\pid{f,g}$,
shows that the authors are computing $h$ every time they add another divisor to $D$.}
In computational applications where memory is a factor, a smaller representation may be desirable.
%Two motivations to use Gr\"obner bases are
%that they allow typical and atypical reduced degree 3 divisors to be added in a uniform way,
%and they contained the minimum polynomial of $I_D$, which is of importance in the reduction step.
It was known to me at the beginning of this project that any ideal in a Dedekind domain is generated by two or fewer elements.
It was not known to me that an even stronger statement is true:
given any $f \in I_D$, there is a $g \in I_D$ such that $I_D = \pid{f,g}$ ---
one may choose one of the generators arbitrarily.\footnote{
See Theorem 8.5.1 in \cite{alaca04}. It is proven that every ideal $I$ in a Dedekind domain is generated by two elements,
but the proof is constructive. An element $\alpha \in I$ is chosen arbitrarily and another element $\beta$ is found
such that $I = \pid{\alpha, \beta}$.}
One of the motivations for using Gr\"obner bases was that one of its generators is always the minimum polynomial in the ideal.
I suspect that every type 31 divisor can be represented instead by one of three forms:
$\pid{x^2 + \ldots, xy + \ldots}$, $\pid{x^2 + \ldots, y^2 + \ldots}$, or $\pid{x^2 + \ldots, xy^2 + \ldots}$,
where the polynomial $x^2 + \ldots$ is minimal.
Thus one may reclassify type 31 divisors into 3 subtypes and similarly reclassify type 41, 51, and 61 divisors.
I do not know that one will gain any runtime improvements this way,
but it would allow for semi-typical divisors to be represented more compactly.

In the typical case, we are able to quickly compute $M_{\text{add}}$, but computing $M_{\text{doub}}$ is slow in comparison.
Our calculation of $M_{\text{doub}}$ does not make much use of the third polynomial $h$ in the representation of a type 31 divisor,
though it may be possible to leverage it some more.
Presently, to compute $M_{\text{doub}}$, we find $g'$ and $h'$ such that $fh' \equiv gg'$,
and make use of a map $d$ that maps $f \mapsto g'$ and $g \mapsto h'$.
One promising improvement is to instead let $d$ be the map
\[ d : \begin{array}{l}
  f \mapsto s't - st' \\
  g \mapsto t'r - tr' \\
  h \mapsto r's - rs'
\end{array} \]
where $r, s, t, r', s', t'$ are as in Lemmas \ref{lem_fgh_is_fg_31} and \ref{lem_fgh_is_fh_31}.
Each of these six polynomials are very fast to compute.
Whereas we currently need 36M+44A to compute $g'$, $h'$, and $M_{\text{doub}}$,
it would only require about 30M+41A to compute $r,s,t,r',s',t',M_{\text{doub}}$.
Preliminary testing has shown that using this map for $d$ produces correct results.
However, this discovery was made late in the writing of this thesis
and lacks a proof of correctness and thorough testing.

The current state of $C_{3,4}$ curve arithmetic is still significantly
more expensive than the state for genus 3 hyperelliptic curves.
Colleagues at the University of Calgary have applied Shanks' NUCOMP algorithm \cite{shanks89}
to genus 3 hyperelliptic curve divisor arithmetic \cite{jacobson07}.
Upcoming results \cite{lindner_communication} have the cost of addition down to
approximately 1I+52M (ramified curves) and 1I+67M (split curves),
half as many multiplications as are currently required for $C_{3,4}$ curves.
One may wonder whether NUCOMP may be applied to $C_{3,4}$ curve arithmetic as well.

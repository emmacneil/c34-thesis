%%%%%%%%%%%%%%%%%%%%%%
%%%%%            %%%%%
%%%%%   Ideals   %%%%%
%%%%%            %%%%%
%%%%%%%%%%%%%%%%%%%%%%

\section{The Ideal Class Group}
\label{chap_ideals}

Divisors may be cumbersome to compute with.
The coordinates of the points in the support of the divisor may live in some algebraic extension of the base field.
In the arithmetic described in this thesis, we would only need to work in extensions of degree at most 3.
Still, working in $\bb F_{q^3}$ is more expensive than working in $\bb F_q$.
It would be preferable to work in the base field only, if it can be helped.

It can be helped by representing divisors by ideals of the curve's coordinate ring.
There is a correspondence between divisors defined over $K$ and rational functions in $K(x,y)/C$.
These rational functions, of course, have coefficients in $K$.
The equivalence relation on the divisor class group allows us to simplify things even more.
We will see that, just as we can clear out negative points assume a divisor $D$ is written as $D^+ - D^\infty$,
we can clear out denominators and assume our rational functions are merely polynomials.

In this chapter, we will define the ideal class group of the curve,
show that the ideal class group and divisor class group are isomorphic,
and show how to move back and forth between the groups as desired.



\subsection{Prime Ideals, Prime Divisors}

In order to show that the divisor class group and ideal class group (once defined) are isomorphic,
we will exhibit an explicit isomorphism between the two groups.
Since the coordinate ring $K[C]$ is a Dedekind domain, its ideals have a unique factorization.
We first define a map from prime ideals to divisors before extending this map to ideals and fractional ideals.

Let $\frak p$ be a prime ideal of $K[C]$.
Define the \defn{divisor of $\frak p$} to be
  \[ \div \frak p = \sum_{P \in C - P_\infty} \min_{f \in \frak p - \{0\}}\{\ord_P(f)\}(P - P_\infty). \]
By construction, this divisor is of degree zero.
We will see that it is accurate to call such a divisor a \defn{prime divisor} --
this section will show that the divisors of prime ideals are exactly the prime divisors as defined in the previous chapter.

\begin{comment}
The theorems that follow highlight what characterizes prime divisors.
Let $P$ be a point in $\bar K \times \bar K$.
The \defn{orbit} of $P$ is the set
  \[ \orb(P) = \{ \sigma(P) ~|~ \sigma \in \Gal(\bar K/K)\}. \]
Since $P$ resides in a finite algebraic extension of $K$, this orbit is finite.
Given an affine point $P$, define the divisor
  \[ [P] = \sum_{P_0 \in \orb(P)} (P_0 - P_\infty). \]
The support of $[P]$ is exactly the orbit of $P$, each point appearing with multiplicity 1,
minus some appropriate multiple of the point at infinity, $P_\infty$.
The theorems that follow will demonstrate that the prime divisors of a curve are exactly the divisors of the form $[P]$ for some point $P$.
This, we demonstrate by defining a map from divisors of this form back to prime ideals,
and showing that this acts as a two-sided inverse to our map on prime ideals.
\end{comment}
\begin{comment}
First, a simple lemma. \note{This and its corollary are also in Chapter - C34 Curves.}
\begin{lemma}
  Let $f \in K[x,y]$ and $\sigma \in \Gal(\bar K/K)$.
  Let $P = (x_0, y_0)$ be a point in $\bar K \times \bar K$. Then
  \[ f(\sigma(x_0), \sigma(y_0)) = \sigma(f(x_0, y_0)). \]
\end{lemma}
\begin{proof}
  \begin{align*}
    f(\sigma(x_0), \sigma(y_0))
      &= \sum a_{i,j}\sigma(x)^i\sigma(y)^j \\
      &= \sum \sigma(a_{i,j})\sigma(x)^i\sigma(y)^j
        & \text{$\sigma$ fixes $K$} \\
      &= \sum \sigma(a_{i,j}x^iy^j)
        & \text{$\sigma$ is multiplicative} \\
      &= \sigma \left( \sum a_{i,j}x^iy^j \right)
        & \text{$\sigma$ is additive} \\
      &= \sigma(f(x_0, y_0)).
  \end{align*}
\end{proof}
\begin{corollary}
%  \label{cor_orb}
  Let $f \in K[x,y]$ and $\sigma \in \Gal(\bar K/K)$.
  Let $P$ be an affine point.
  Then $f$ has a zero at $P$ if and only if $f$ has a zero at $\sigma(P)$.
\end{corollary}
\begin{proof}
  ($\implies$) Suppose $f$ has a zero at a point $P = (x_0, y_0)$,
  i.e. $f(x_0, y_0) = 0$. Then
  \[ f(\sigma(x_0), \sigma(y_0)) = \sigma(f(x_0, y_0)) = \sigma(0) = 0. \]  

  ($\impliedby$) Suppose $f$ has a zero at $\sigma(P)$, i.e. $f(\sigma(x_0), \sigma(y_0)) = 0$.
  Then $\sigma$ has an inverse $\sigma\inv \in \Gal(\bar K/K)$ and
  \[ f(x_0, y_0) = \sigma\inv(\sigma(f(x_0, y_0))) = \sigma\inv(f(\sigma(x_0), \sigma(y_0))) = \sigma\inv(0) = 0. \] 
\end{proof}
\end{comment}

If $[P]$ is a prime divisor on $C$, then define
\[ I_{[P]} = \pid{ f \in K[C] ~|~ f(P) = 0 }. \]

\note{Put Some or all of this in earlier section?}
\begin{proposition}
  Let $P$ be an affine point on $C$ and let $\frak p = I_{[P]}$. Then
  \begin{enumerate}[label=(\roman*)]
    \item $\frak p$ is a $K[C]$-ideal;
    \item $\frak p$ is non-zero;
    \item $\frak p$ is prime;
    \item $\frak p$ is maximal;
    \item $\frak m_P = \frak p \cal O_P$;
  \end{enumerate}
\end{proposition}
\begin{proof}
  \begin{enumerate}[label=(\roman*)]
    \item
    Suppose $f \in \frak p$ and $g \in \frak p$.
    Then $(f + g)(P) = f(P) + g(P) = 0$, so $f + g \in \frak p$.
    Suppose $f \in \frak p$ and $h \in K[C]$.
    Then $(hf)(P) = h(P)f(P) = h(P)\cdot 0 = 0$, so $hf \in \frak p$.
    
    \item
    Let $P = (x_0, y_0)$.
    Let $m(x) \in K[x]$ be the minimal polynomial of $x_0$.
    Let $\tilde m(x,y)$ be the image of $m$ in $K[C]$.
    Then $\tilde m$ is non-zero and $\tilde m(P) = \tilde m(x_0, y_0) = m(x_0) = 0$,
    so $\tilde m \in \frak p$.
    
    \item
    Suppose $fg \in \frak p$ for some $f, g \in K[C]$.
    Then $(fg)(P) = 0 = f(P)g(P)$.
    Since $f(P)$ and $g(P)$ are field elements, one of them must be non-zero.
    Therefore one of $f$ and $g$ is in $\frak p$.
    
    \item
    In a Dedekind domain, all non-zero prime ideals are maximal.
    
    \item
    \note{TODO}
  \end{enumerate}
\end{proof}
\note{Maybe combine this prop with the previous, and also cut unneeded facts/shorten the previous.}
\begin{proposition}
  Let $P = (x_0, y_0)$ be an affine point on $C$.
  Let $\frak q = \pid{x - x_0, y - y_0}$ as a $\bar K[C]$-ideal. Then
  \[ I_{[P]} = \frak q \cap K[C]. \]
\end{proposition}
\begin{proof}
  \begin{align*}
    I_{[P]}
      &= \pid{ f \in K[C] | f(P) = 0 } \\
      &= \pid{ f | f \in K[C], f(P) = 0 } \\
      &= \pid{ f | f \in K[C], f \in \frak q } \\
      &= \frak q \cap K[C].
  \end{align*}
\end{proof}

We have now that $I_{[-]}$ defines map from some subset of divisors to prime ideals.
We show now that this map is one-to-one.
\begin{proposition}
  Let $P$ and $Q$ be affine points on $C$ and suppose $\orb(P) \neq \orb(Q)$.
  Then $I_{[P]} \neq I_{[Q]}$.
\end{proposition}
\begin{proof}
  Let $X$ and $Y$ be the sets
    \[ X = \{ x_i ~|~ (x_i, y_i) \in \orb(P) \} \triangle \{ x_i ~|~ (x_i, y_i) \in \orb(Q) \} \]
    \[ Y = \{ y_i ~|~ (x_i, y_i) \in \orb(P) \} \triangle \{ y_i ~|~ (x_i, y_i) \in \orb(Q) \}, \]
    where $\triangle$ denotes the symmetric difference of sets.
  So, e.g., $X$ is the set of $x$-coordinates found in the orbit of $P$ or $Q$, but not both.
  Suppose that $X$ is non-empty and contains an element $x_0$.
  Let $m(x_0)$ be the minimal polynomial of $x_0$, viewed as a polynomial in $K[C]$.
  Then $m$ is zero everywhere on the orbit of $P$, but is non-zero on the orbit of $Q$.
  Simlarly, if $Y$ contains an element $y_0$, then the minimal polynomial of $y_0$ gives the same result.

  Now suppose $X$ and $Y$ are both empty.
  Then there is a point $Q_0 \in \orb(Q)$ with the same $x$-coordinate as $P$, but whose $y$-coordinate is a conjugate of $P$'s.
  Without loss of generality, let $P = (x_1, y_1)$ and $Q_0 = (x_1, \sigma(y_1))$.
  Let $\frak q_1 = \pid{x - x_1, y - y_1}$ and $\frak q_2 = \pid{x - x_1, y - \sigma(y_1)}$. Then
  \begin{align*}
    I_{[P]} &= \frak q_1 \cap K[C] \\
    I_{[Q]} &= \frak q_2 \cap K[C],
  \end{align*}
  \[ \frak q_1 + \frak q_1 = \bar K[C], \]
  and
  \begin{align*}
    I_{[P]} + I_{[Q]}
      &= (\frak q_1 \cap K[C]) + (\frak q_2 \cap K[C]) \\
      &= (\frak q_1  + \frak q_1) \cap K[C] \\
      &= \bar K[C] \cap K[C] \\
      &= K[C] \neq I_{[P]}.
  \end{align*}
  Hence $I_{[P]} \neq I_{[Q]}$.
\end{proof}

\begin{corollary}
  Let $P$ and $Q$ be affine points on $C$. The following are equivalent:
  \begin{enumerate}[label=(\roman*)]
    \item $Q \in \orb(P)$;
    \item $\orb(P) = \orb(Q)$;
    \item $[P] = [Q]$;
    \item $I_{[P]} = I_{[Q]}$.
  \end{enumerate}
\end{corollary}
\begin{proof}
  \begin{description}
    \item [(i) $\implies$ (ii):]
      For some $\sigma \in \Gal(\bar K/K)$, we have $Q = \sigma(P)$ and $\sigma\inv(Q) = P$.
      
      Suppose $R \in \orb(P)$.
      Then $R = \phi(P)$ for some $\phi \in \Gal(\bar K/K)$ and
      \[ \sigma\phi\inv R = \sigma\phi\inv\phi(P) = \sigma(P) = Q \in \orb(Q). \]
      
      Suppose $R \in \orb(Q)$.
      Then $R = \phi(Q)$ for some $\phi \in \Gal(\bar K/K)$ and
      \[ \sigma\inv\phi\inv R = \sigma\inv\phi\inv\phi(Q) = \sigma\inv(Q) = P \in \orb(P). \]
    
    %\item [(ii) $\implies$ (i):]
    %  $Q \in \orb(Q)$ and $\orb(Q) = \orb(P)$, so $Q \in \orb(P)$.

    \item [(ii) $\implies$ (iv):]
      \begin{align*}
        I_{[P]}
          &= \pid{ f \in K[C] ~|~ f(P) = 0 } \\
          &= \pid{ f \in K[C] ~|~ \forall P_0 \in \orb(P) : f(P_0) = 0 } \\
          &= \pid{ f \in K[C] ~|~ \forall P_0 \in \orb(Q) : f(P_0) = 0 } \\
          &= \pid{ f \in K[C] ~|~ f(Q) = 0 } \\
          &= I_{[Q]}
      \end{align*}
      
    \item [(iv) $\implies$ (i):]
      By the previous proposition.
    
    \item [(ii) $\implies$ (iii):]
      By definition.

    \item [(iii) $\implies$ (ii):]
      \begin{align*}
        \sum_{P_0 \in \orb(P)}(P_0 - P_\infty) &= \sum_{Q_0 \in \orb(Q)}(Q_0 - P_\infty) \\
        \sum_{P_0 \in \orb(P)}P_0 &= \sum_{Q_0 \in \orb(Q)}Q_0 \\
        \{P_0 \in \orb(P)\} &= \{Q_0 \in \orb(Q)\} \\
        \orb(P) &= \orb(Q)
      \end{align*}
  \end{description}
\end{proof}

The following proposition essentially shows that divisors of prime ideals are prime divisors as defined in Chapter \ref{chap_divisors}.
\begin{proposition}
  Let $\frak p$ be a non-zero prime ideal and let $P$ be an affine point on $C$. Then
  \[ P \in \supp(\div \frak p) \iff \frak p = I_{[P]}. \]
\end{proposition}
\begin{proof}
  ($\implies$)
  Suppose $P$ is in the support of $\div \frak p$.
  Let $\frak q = I_{[Q]}$.
  By an above proposition, $\frak q$ is prime.
  Since every polynomial in $\frak p$ vanishes at $P$, we have $\frak p \subseteq \frak q$.
  However, every non-zero prime ideal of $K[C]$ is maximal, so $\frak p = \frak q$.
  
  ($\impliedby$)
  Suppose $\frak p = I_{[P]}$.
  For every non-zero polynomial $f \in \frak p$, $f(P) = 0$, hence $\nu_P(f) > 0$.
  Now the order of $\div \frak p$ at $P$ is
  \[ \ord_P(\div \frak p) = \min_{0 \neq f \in \frak p}\{ \nu_P(f) \} > ,0 \]
  therefore $P \in \supp(\div \frak p)$.
\end{proof}

Now if $\frak p = I_{[P]}$, then $\frak p = I_{[\sigma(P)]}$ for every conjugate point of $P$,
so that the entire orbit of $P$ is in the support of $\div \frak p$.
Conversely, if $Q$ is not in the orbit of $P$, then $\frak p \neq I_{[Q]}$, so $Q \not\in \supp(\div \frak p)$.

\note{Expansion of previous prop.}
\begin{lemma}
  \label{lem_order_is_1}
  Let $P$ be an affine point on $C$ and let $\frak p = I_{[P]}$.
  Then $\ord_P(\div \frak p) = 1$.
\end{lemma}
\begin{proof}
  Clearly, $\ord_P(\div \frak p) \geq 1$.
  We must show that there is a polynomial in $\frak p$ whose valuation at $P$ is exactly 1.
  
  Let $P = (x_0, y_0)$ and consider the lines determined by $x - x_0$ and $y - y_0$.
  Since $C$ is non-singular, at most one of these lines is tangent to $C$ at $P$.
  Without loss of generality, suppose $x - x_0$ is not tangent to $C$ at $P$.
  Let $m(x, y)$ be the minimum polynomial of $x_0$, seen as an element of $K[C]$.
  Then $\nu_P(m) = \nu_P(x - x_0) = 1$.
  Moreover, $m(x, y)$ is zero on the orbit of $P$,
  so $m(x, y) \in I_{[P]} = \frak p$.
\end{proof}

\begin{proposition}
  Let $\frak p$ be a non-zero prime ideal and let $P$ be an affine point on $C$.
  The following are equivalent
  \begin{enumerate}[label=(\roman*)]
    \item $P \in \supp(\div \frak p)$;
    \item $\frak p = I_{[P]}$.
    \item $\div \frak p = [P]$.
  \end{enumerate}
\end{proposition}
\begin{proof}
  \begin{description}
    \item [(i) $\implies$ (ii):]
      Suppose $P$ is in the support of $\div \frak p$.
      Let $\frak q = I_{[Q]}$.
      By an above proposition, $\frak q$ is prime.
      Since every polynomial in $\frak p$ vanishes at $P$, we have $\frak p \subseteq \frak q$.
      However, every non-zero prime ideal of $K[C]$ is maximal, so $\frak p = \frak q$.

    \item [(ii) $\implies$ (i):]
      Suppose $\frak p = I_{[P]}$.
      For every non-zero polynomial $f \in \frak p$, $f(P) = 0$, hence $\nu_P(f) > 0$.
      Now the order of $\div \frak p$ at $P$ is
      \[ \ord_P(\div \frak p) = \min_{0 \neq f \in \frak p}\{ \nu_P(f) \} > ,0 \]
      therefore $P \in \supp(\div \frak p)$.

    \item [(ii) $\implies$ (iii):]
      Suppose $\frak p = I_{[P]}$.
      Since $I_{[P]} = I_{[\sigma(P)]}$ for every $\sigma \in \Gal(\bar K/K)$,
      we have $\sigma(P) \in \supp(\div \frak p)$.
      That is, the entire orbit of $P$ is in the support of $\div \frak p$.
      Conversely, if $Q$ is not in the orbit of $P$, then $\frak p \neq I_{[Q]}$,
      so that $Q \not \in \supp(\div \frak p)$.
      By Lemma \ref{lem_order_is_1}, $P$ appears with multiplicity 1, so that $\div \frak p = [P]$.
      
    \item [(iii) $\implies$ (i):]
      \note{Obvious.}
  \end{description}
\end{proof}

\begin{proposition}
  Let $P$ be an affine point in $C(\bar K)$ and let $\frak p$ be a non-zero prime ideal of $K[C]$. Then
  \begin{enumerate}[label=(\roman*)]
    \item $I_{\div \frak p} = \frak p$;
    \item $\div I_{[P]} = [P]$.
  \end{enumerate}
\end{proposition}
\begin{proof}
  \begin{enumerate}[label=(\roman*)]
    \item
      Let $\frak p$ be a non-zero prime idea of $K[C]$.
      Then there is an affine point $P \in \supp(\div \frak p)$,
      and $\frak p = I_{[P]}$. Then
      \[ \div I_{[P]} = \div \frak p = [P]. \]
    
    \item
      Let $P$ be an affine point in $C(\bar K)$.
      Let $\frak p = I_{[P]}$. Then $\frak p$ is a non-zero prime ideal and
      \[ I_{\div \frak p} = I_{[P]} = \frak p. \]
  \end{enumerate}
\end{proof}

\begin{comment}
This proposition shows that every prime divisor has the form $[P]$ for some affine point $P$.
\begin{proposition}
  Let $\frak p$ be a non-zero prime ideal and let $P$ be an affine point in the support of $\div \frak p$.
  Then $\div \frak p = [P]$.
\end{proposition}
\begin{proof}
  By an above proposition, $P$ is in the support of $\div \frak p$ with order 1.
  So too are the other points in its orbit. \note{Prove this.}
  We must show that these are all the affine points.
  
  Suppose $\div \frak p = [P] + [Q]$ for some $Q \not\in [P]$.
  Then $Q \in \supp(\div \frak p)$.
  Then by an above proposition $\frak p = I_{[Q]}$.
  However $\frak p = I_{[P]}$.
  So $I_{[P]} = I_{[Q]}$, and by an above proposition, $Q \in [P]$, a contradiction.
\end{proof}

Now we show that these maps are inverses of one another.

\begin{proposition}
  Let $\frak p$ be a non-zero prime ideal. Then
  \[ I_{\div \frak p} = \frak p. \]
\end{proposition}
\begin{proof}
  The divisor $\div \frak p$ is non-trivial with an affine point $P$ in its support.
  Then by above proposition, $\div \frak p = [P]$,
  and by another $\frak p = I_{[P]}$.
  Putting these together,
  \[ \frak p = I_{[P]} = I_{\div \frak p}. \]
\end{proof}

\begin{proposition}
  Let $P$ be an affine point on $C$. Then
  \[ \div(I_{[P]}) = [P]. \]
\end{proposition}
\begin{proof}
  Let $\frak p := I_{[P]}$.
  Then $\frak p$ is prime and non-zero.
  By an above proposition, $P \in \supp(\div \frak p)$.
  By another above proposition, $\div \frak p = [P]$.
  Putting these together,
  \[ \div(I_{[P]}) = \div \frak p = [P]. \]
\end{proof}
\end{comment}


\begin{comment}
\begin{lemma}
  Let $D = \div \frak p$ be a prime divisor.
  If $P$ is an affine point in the support of $D$, then $\ord_P(D) = 1$.
\end{lemma}
Certainly if $P \in \supp(D)$, then $\ord_P(D) \geq 1$.
To prove this lemma, we look for a polynomial with order exactly 1 at $P$.
One such polynomial is the uniformizer at $P$.
\begin{proof}
  Let $f$ be any non-zero polynomial in $\frak p$.
  \note{Need caveat that $\frak p$ is non-zero. This mistake is probably elsewhere in this section.}
  Let $n = \ord_P(f)$. Then $n \geq 1$.
  Let $u$ be a uniformizer at $P$.
  Then $f = \frac r s u^n$ for some $r$ and $s$ that are non-zero at $P$.
  Then $sf = ru^n$, which implies $ru^n \in \frak p$.
  However, $r$ is non-zero at $P$, so $r \not \in \frak p$.
  Since $\frak p$ is prime, we get $u \in \frak p$.
  However, $\ord_P(u) = 1$, so
  \[ \ord_P(D) = \min_{f \in \frak p - \{ 0 \}} \ord_P(f) = \ord_P(u) = 1. \]
\end{proof}

\begin{lemma}
  Let $f \in K[x,y]$ and $\sigma \in \Gal(\bar K/K)$.
  Let $P = (x_0, y_0)$ be a point in $\bar K \times \bar K$. Then
  \[ f(\sigma(x_0), \sigma(y_0)) = \sigma(f(x_0, y_0)). \]
\end{lemma}
\begin{proof}
  \begin{align*}
    f(\sigma(x_0), \sigma(y_0))
      &= \sum a_{i,j}\sigma(x)^i\sigma(y)^j \\
      &= \sum \sigma(a_{i,j})\sigma(x)^i\sigma(y)^j
        & \text{$\sigma$ fixes $K$} \\
      &= \sum \sigma(a_{i,j}x^iy^j)
        & \text{$\sigma$ is multiplicative} \\
      &= \sigma \left( \sum a_{i,j}x^iy^j \right)
        & \text{$\sigma$ is additive} \\
      &= \sigma(f(x_0, y_0)).
  \end{align*}
\end{proof}
\begin{corollary}
  \label{cor_orb}
  Let $f \in K[x,y]$ and $\sigma \in \Gal(\bar K/K)$.
  Let $P$ be an affine point.
  Then $f$ has a zero at $P$ if and only if $f$ has a zero at $\sigma(P)$.
\end{corollary}
\begin{proof}
  ($\implies$) Suppose $f$ has a zero at a point $P = (x_0, y_0)$,
  i.e. $f(x_0, y_0) = 0$. Then
  \[ f(\sigma(x_0), \sigma(y_0)) = \sigma(f(x_0, y_0)) = \sigma(0) = 0. \]  

  ($\impliedby$) Suppose $f$ has a zero at $\sigma(P)$, i.e. $f(\sigma(x_0), \sigma(y_0)) = 0$.
  Then $\sigma$ has an inverse $\sigma\inv \in \Gal(\bar K/K)$ and
  \[ f(x_0, y_0) = \sigma\inv(\sigma(f(x_0, y_0))) = \sigma\inv(f(\sigma(x_0), \sigma(y_0))) = \sigma\inv(0) = 0. \] 
\end{proof}

\note{Not sure where to place following lemma.}
\begin{lemma}
  Let $P$ be an affine point.
  Let $I_P$ be the ideal of polynomials vanshing on $\orb(P)$.
  Then $I_P$ is a prime ideal.
\end{lemma}
\begin{proof}
  Suppose $fg \in I_P$.
  Then $0 = (fg)(P) = f(P)g(P)$.
  Since $f(P)$ and $g(P)$ are field elements in $\bar K$, either $f(P) = 0$ or $g(P) = 0$.
  By Corollary \ref{cor_orb}, either $f$ vanishes on $\orb(P)$ or $g$ vanishes on $\orb(P)$.
  Therefore at least one of $f$ or $g$ is in $I_P$.
\end{proof}
\end{comment}



%%%%%%%%%%%%%%%%%%%%%%%%%%%%%%%%%%%%%%%%%%%%%%%%%%%%%%%%%%%%%%%%%%%%%%%%%%%%%%%

\subsection{Ideals and Divisors}

\note{Error in this section. Monoid of divisors consists only of divisors \emph{greater than or equal to 0}.}
The coordinate ring $K[C]$ is a Dedekind domain.
The non-zero ideals of $K[C]$ may be factored into a product of prime ideals, and this factorization is unique.
Our maps between prime ideals and prime divisors can now be extended to act on any non-zero ideal of $K[C]$ or any prime divisor of $\Div_K^0(C)$.

Let $I$ be a non-zero ideal of $K[C]$.
Let its factorization into prime ideals be $\frak p_1^{k_1} \dots \frak p_n^{k_n}$.
Then define the divisor of $I$ to be
\[ \div I = \sum_{i=1}^n k_i \div \frak p_i. \]
The divisor of $I$ is the sum of the divisors of its prime factors.
As for the whole ring $K[C]$ itself,
its prime factorization is the empty product which maps to the empty sum:
  \[ \div (K[C]) = 0. \]
Note that the divisor of $I$ is of degree zero and defined over $K$.

In the other direction, let $D$ be a divisor in $\Div_K^0(C)$.
Then it factors into a sum of prime divisors, say $D = k_1[P_1] + \dots + k_n[P_n]$.
Define the ideal of $D$ to be
\[ I_D = \prod_{i=1}^n I_{[P_i]}^{k_i}. \]
The divisor 0 is the empty sum.
Let it map to the empty product, which is the whole ring $K[C]$:
\[ I_{0} = K[C]. \]

Let $\cal I_C$ be the monoid of non-zero ideals of $K[C]$.
We now have maps $\div(-) : \cal I_C \to \Div_K^0(C)$ and $I_{(-)} : \Div_K^0(C) \to \cal I_C$.
\begin{theorem}
  The maps $\div(-)$ and $I_{(-)}$ are isomorphisms of monoids and mutual inverses.
\end{theorem}
\begin{proof}
  Let $I \in \cal I_C$. Let its prime factorization be $\prod \frak p_i^{k^i}$. Then
  \begin{align*}
    I &= \prod_{i=1}^n \frak p_i^{k_i} \\
    \div I &= \sum_{i=1}^n k_i[P_i]
      & P_i \in \supp(\div \frak p_i) \\
    I_{\div I} &= \prod_{i=1}^n I_{[P_i]}^{k_i} \\
               &= \prod_{i=1}^n \frak p_i^{k_i} \\
               &= I.
  \end{align*}
  Let $D \in \Div_K^0(C)$. Let its prime factorization be $\sum k_i[P_i]$. Then
  \begin{align*}
    D &= \sum_{i=1}^n k_i[P_i] \\
    I_D &= \prod_{i=1}^n I_{[P_i]}^{k_i} \\
    \div(I_D) &= \sum_{i=1}^n k_i [P_i] \\ &= D.
  \end{align*}
\end{proof}
\begin{comment}
\begin{proof}
  The proof is quite immediate after factoring each ideal.
  It has already been established that $\div$ maps the identity $K[C]$ of $\cal I_C$ to the identity $0$ of $\Div_K^0(C)$.
  Let $\frak a$ and $\frak b$ be non-zero ideals with prime factorizations
  \[ \frak a = \prod \frak p_i^{k_i}, ~~~ \frak b = \prod \frak q_i^{\ell_i}. \]
  Then
  \begin{align*}
    \div (\frak a \frak b)
      &= \div \left( \left( \prod \frak p_i^{k_i} \right) \left( \prod \frak q_j^{\ell_j} \right) \right) \\
      &= \sum k_i \div \frak p_i + \sum \ell_j \div \frak q_i \\
      &= \div \frak a + \div \frak b.
  \end{align*}
\end{proof}
\end{comment}



%%%%%%%%%%%%%%%%%%%%%%%%%%%%%%%%%%%%%%%%%%%%%%%%%%%%%%%%%%%%%%%%%%%%%%%%%%%%%%%

\subsection{Fractional Ideals and $\Div_K^0(C)$}

We may extend the maps even further to fractional ideals and the entirety of $\Div_K^0(C)$.

Let $\cal J_C$ denote the Abelian group of fractional ideals of $K[C]$.
Let $\frak a \in \cal J_C$.
Then $\frak a$ is of the form $\pid{\frac 1 f} \frak b$ for some polynomial $f \in K[C]$ and some integral ideal $\frak b$ of $K[C]$.
Define
\[ \div \frak a = \div \frak b - \div f. \]

\begin{proposition}
  This map is well-defined.
\end{proposition}
\begin{proof}
  Suppose that $\frak a$ is a fractional ideal,
  $\frak b$ and $\frak c$ are integral ideals,
  $f$ and $g$ are non-zero polynomials, and
    \[ \frac 1 f \frak b = \frak a = \frac 1 g \frak c. \]
  Then $g \frak b = f \frak c$ are integral ideals and
  \begin{align*}
    \div \left( \frac 1 f \frak b \right)
      &= \div \frak b - \div f \\
      &= \div \frak b - \div f + \div g - \div g \\
      &= \div (g \frak b) - \div f - \div g \\
      &= \div (f \frak c) - \div f - \div g \\
      &= \div f + \div \frak c - \div f - \div g \\
      &= \div \frak c - \div g \\
      &= \div \left( \frac 1 g \frak c \right).
  \end{align*}
\end{proof}

Once again, note that for a fractional ideal $\frak a \in \cal J_C$,
the divisor $\div \frak a$ is in $\Div_K^0(C)$.
This establishes a function (of sets) from $\cal J_C$ to $\Div_K^0(C)$.
The domain and codomain are both Abelian groups, so we show that this function is in fact a group homomorphism.

\begin{theorem}
  The map
    \[ \div(-) : \cal J_C \to \Div_K^0(C) \]
  is a group homomorphism.
\end{theorem}
\begin{proof}
  It has been established already $\div(-)$ maps the identities $K[C] \mapsto 0$.
  Let $\frak a$ and $\frak b$ be fractional ideals, with
  \[ \frak a = \frac 1 f \frak A, ~~~ \frak b = \frac 1 g \frak B \]
  and where $\frak A$ and $\frak B$ are integral ideals.
  Using the fact that $\div(-)$ is a monoid homomorphism,
  \begin{align*}
    \div(\frak a \frak b)
      &= \div \left( \frac 1 {fg} \frak A \frak B \right) \\
      &= \div (\frak A \frak B) - \div (fg) \\
      &= \div \frak A  + \div \frak B - (\div f + \div g) \\
      &= (\div \frak A - \div f) + (\div \frak B - \div g) \\
      &= \div \frak a + \div \frak b.
  \end{align*}
\end{proof}

\begin{theorem}
  The maps $\div(-) : \cal J_C \to \Div_K^0(C)$ and $I_{(-)} : \Div_K^0(C) \to \cal J_C$
  are group isomorphisms and mutual inverses.
\end{theorem}
\begin{proof}
  Let $\frak a$ be a fractional ideal.
  Then $\frak a = \frac 1 f \frak b$ for some integral ideal $\frak b$ and some $f \in K[C]$ and
  \begin{align*}
    \div \frak a &= \div \frak b - \div f \\
    I_{\div \frak a} &= I_{\div \frak b - \div f} \\
      &= I_{\div \frak b} I_{\div f}\inv \\
      &= \frak b f\inv = I.
  \end{align*}
  Let $D \in \Div_K^0(C)$.
  Then $D = A - B$ for some divisors $0 \leq A$, $0 \leq B$.
  There is a polynomial $f$ with $B \leq \div f$
  \note{(Does this need proof? Just take any polynomial through the support of $B$.)}
  and a divisor $E$ such that $B + E = \div f$.
  Then
  \begin{align*}
    D &= A - B = (A + E) - \div f \\
    I_D &= I_{A + E}I_{\div f}\inv = I_{A + E}f\inv \\
    \div I_D &= \div I_{A + E} - \div f \\
    &= A + E - \div f = D.
  \end{align*}
\end{proof}

In fact, this map is a group \emph{isomorphism}.
To show this, we wish to exhibit a homomorphism in the other direction.

For an effective divisor $D$ defined over $K$, define the \defn{ideal of $D$} to be
\[ I_D = \pid{ f \in K(C)^* ~|~ \forall P \in C(\bar K) - \{P_\infty\} : \nu_P(f) \geq \ord_P(D) }. \]
It is the ideal generated by all rational functions that are regular on the (affine) support of $D$,
vanishing with the appropriate multiplicity at each point.
As $D$ is effective, no function in $I_D$ has a pole at any affine point.
Consequently, every rational function in $I_D$ is equivalent to a polynomial function
and $I_D$ may be seen as an ideal of $K[C]$ as well.

\begin{proposition}
  The map
  \[ I_{(-)} : \Div_{\geq 0}^K \to \cal I_C \]
  is a homomorphism of monoids.
\end{proposition}
\begin{proof}
  For the divisor $0$,
  \begin{align*}
    I_0 &= \pid{ f \in K(C)^* ~|~ \forall P \in C(\bar K) - \{P_\infty\} : \nu_P(f) \geq \ord_P(0) } \\
        &= \pid{ f \in K(C)^* ~|~ \forall P \in C(\bar K) - \{P_\infty\} : \nu_P(f) \geq 0 } \\
        &= \pid{ K[C] } = K[C].
  \end{align*}
  So $I_{(-)}$ maps identity to identity.
  
  Let $A$ and $B$ be effective divisors. Then
  \begin{align*}
    I_AI_B &= \pid{ f \in K(C)^* ~|~ \nu_P(f) \geq \ord_P(A) } \pid{ g \in K(C)^* ~|~ \nu_P(g) \geq \ord_P(B) } \\
           &= \pid{ fg \in K(C)^* ~|~ \nu_P(f) + \nu_P(g) \geq \ord_P(A) + \ord_P(B) } \\
           &= \pid{ fg \in K(C)^* ~|~ \nu_P(fg) \geq \ord_P(A + B)} \\
           &= \pid{ h \in K(C)^* ~|~ \nu_P(h) \geq \ord_P(A + B)} \\
           &= I_{A+B}.
  \end{align*}
  The quantifier ``$\forall P \in C(\bar K) - \{P_\infty\}$'' is omitted here for space constraints.
  \note{Do I need to justify moving from line 3 to 4?}
\end{proof}

Now suppose $D$ is a degree zero divisor defined over $K$.
Then $D$ can be written uniquely in the form
  \[ D = D^+ - D^- + D^\infty, \]
where $D^+$ and $D^-$ are effective divisors and $D^\infty = \deg(D^- - D^+)P_\infty$.
We are merely breaking $D$ up into affine points with positive multiplicity,
affine points with negative multiplicity, and a multiple of the point at infinity.
Define the ideal of $D$ to be
%\[ I_D = I_{D^+}(K[C] : I_{D^-}). \]
\[ I_D = I_{D^+}I_{D^-}\inv. \]
\begin{theorem}
  The map 
    \[ I_{(-)} : \Div_K^0 \to \cal J_C \]
  is a group homomorphism.
\end{theorem}
\begin{proof}
  Let $A$ and $B$ be degree zero divisors defined over $K$.
  Write $A = A^+ - A^- + A^\infty$ and $B = B^+ - B^- + B^\infty$.
  Then
  \begin{align*}
    I_{A + B}
      &= I_{A^+ + B^+}(I_{A^- + B^-})\inv \\
      &= I_{A^+}I_{B^+}(I_{A^-}I_{B^-})\inv \\
      &= I_{A^+} I_{A^-}\inv I_{B^+} I_{B^-}\inv \\
      &= I_A I_B \\
  \end{align*}
\end{proof}



%%%%%%%%%%%%%%%%%%%%%%%%%%%%%%%%%%%%%%%%%%%%%%%%%%%%%%%%%%%%%%%%%%%%%%%%%%%%%%%

\subsection{The Ideal Class Group}

Let $\cal J_C$ be the group of fractional ideals of $K[C]$ and let $\cal P_C$ denote its subgroup of principal ideals.
The \defn{ideal class group} of $K[C]$ is
\[ \cal H_C = \frac {\cal J_C} {\cal P_C}. \]
Since $\cal J_C$ is isomorphic to $\Div_K^0(C)$ and $\cal P_C$ to $\Princ_K(C)$, we have
\[ \cal H_C \simeq \Cl_K^0(C). \]

In the ideal class group, two fractional ideals $\frak a$ and $\frak b$ are equivalent
if there is a rational function $\frac f g \in K(C)$ such that $\frak a = \frac f g \frak b$.
Under this relation, every fractional ideal is equivalent to an integral ideal.
Thus every ideal class has an integral representative.

Since the divisor and ideal class groups are isomorphic, and every ideal class has an integral representative,
we may now represent divisor classes by integral ideals, i.e. ideals generated by polynomials.
There is no need to work with fractional ideals and rational functions.
Representation of divisors by ideals is discussed more in Chapter \ref{chap_representation}.
For now, we show how to compute an integral ideal equivalent to the inverse of an integral ideal. \note{I hate this wording.}

Let $\frak a$ and $\frak b$ be \emph{integral} ideals of a ring $R$.
The \defn{ideal quotient} of $\frak a$ by $\frak b$, also called the \defn{colon ideal}, is
\[ \frak a : \frak b = \{ r \in R ~|~ r \frak b \subseteq \frak a \}. \]
The following proposition sums up several useful, well-known properties of the colon ideal.
\begin{proposition}
  Let $\frak a$, $\frak b$, and $\frak c$ be $R$-ideals. Then
  \begin{enumerate}[label=(\roman*)]
    \item $\frak a : \frak b$ is an $R$-ideal;
    \item $\frak a \subseteq \frak a : \frak b$;
    \item $\frak a : R = \frak a$;
    \item $R : \frak a = R$;
    \item $\frak a \frak b \subseteq \frak c \iff \frak a \subseteq \frak c : \frak b$;
    \item $\frak a : \frak b = R \iff \frak b \subseteq \frak a$;
    \item $\frak a : (\frak b + \frak c) = (\frak a : \frak b) \cap (\frak a : \frak c)$;
    \item $(\frak a \cap \frak b) : \frak c = (\frak a : \frak c) \cap (\frak b : \frak c)$;
    \item $(\frak a : \frak b) : \frak c = \frak a : \frak b \frak c$.
  \end{enumerate}
\end{proposition}
These properties are given as propositions in \cite{cox07},
though the statements are given for a multivariate polynomial ring over a field, $K[x_1, \ldots, x_n]$
rather than for an arbitrary \note{commutative} ring \note{with identity} $R$,
and most of the proofs are left as exercises.
\begin{proof}
  \begin{enumerate}[label=(\roman*)]
    \item Let $a, b \in \frak a : \frak b$.
          Then $aJ, bJ \subseteq I$.
          Then $(a + b)J = aJ + bJ \subseteq I$
          (since $aJ + bJ$ is the join of $aJ$ and $bJ$ in the lattice of $R$-ideals).
          So $a + b \in \frak a : \frak b$.
          
          Let $a \in \frak a : \frak b$, $r \in R$.
          Then $a \frak b \subseteq \frak a$ and $ra \frak b \subseteq a \frak b$,
          so $ra \frak b \subseteq \frak a$ and $ra \in \frak a : \frak b$.

    \item
      We have that
      \begin{align*}
             & \frak a \subseteq \frak a : b \\
        \iff & \forall a \in \frak a : a \frak b \subseteq \frak a \\
        \iff & \forall a \in \frak a : \forall b \in \frak b : ab \in \frak a.
      \end{align*}
      The last statement is true since ideals are closed under multiplication by $R$.
      
    \item
      By part (ii), we have $\frak a \subseteq \frak a : R$.
      Suppose $a \in \frak a : R$.
      Then $aR \subseteq \frak a$.
      In particular, $a = a \cdot 1_R \in \frak a$,
      so $\frak a : R \subseteq \frak a$.
      
    \item
      By definition, $R : \frak a \subseteq R$.
      By part (ii), $R \subseteq R : \frak a$.
      
    \item
      ($\implies$) Let $a \in \frak a$. Then $a \frak b \subseteq \frak a \frak b$,
      and by hypothesis, $\frak a \frak b \subseteq \frak c$,
      so $a \frak b \subseteq \frak c$,
      and $a \in \frak c : \frak b$.
      
      ($\impliedby$) Let $a \in \frak a$.
      By hypothersis, $a \in \frak c : \frak b$, so $a \frak b \subseteq \frak c$.
      Since the choice of $a$ was arbitrary, this means $\frak a \frak b \subseteq \frak c$.
      
    \item
      ($\implies$) Suppose $\frak a : \frak b = R$.
      Then $\frak b = 1_R \frak b \subseteq \frak a$.
      
      ($\impliedby$) Suppose $b \subseteq \frak a$.
      For all $r \in R$, $r \frak b \subseteq \frak b$.
      So $r \frak b \subseteq \frak a$ and $\frak a : \frak b = R$.
      
    \item
      Let $r \in R$. We have
      \begin{align*}
           & r(\frak b + \frak c) \subseteq \frak a \\
      \iff & r \frak b + r \frak c \subseteq \frak a \\
      \iff & r \frak b \subseteq \frak a \text{ and } r \frak c \subseteq \frak a.
      \end{align*}
      So
      \begin{align*}
        \frak a : (\frak b + \frak c)
          &= \{ r \in R ~|~ r(\frak b + \frak c) \subseteq \frak a \} \\
          &= \{ r \in R ~|~ r \frak b \subseteq \frak a \text{ and } r \frak c \subseteq \frak a \} \\
          &= \{ r \in R ~|~ r \frak b \subseteq \frak a \} \cap \{ r \in R ~|~ r \frak c \subseteq \frak a \} \\
          &= (\frak a : \frak b) \cap (\frak a : \frak c).
      \end{align*}
    
    \item
      Similarly to part (vii),
      \begin{align*}
        (\frak a \cap \frak b) : \frak c
          &= \{ r \in R ~|~ r \frak c \subseteq \frak a \cap \frak b \} \\
          &= \{ r \in R ~|~ r \frak c \subseteq \frak a \text{ and } r \frak c \subseteq \frak b \} \\
          &= \{ r \in R ~|~ r \frak c \subseteq \frak a \} \cap \{ r \in R ~|~ r \frak c \subseteq \frak b \} \\
          &= (\frak a : \frak c) \cap (\frak b : \frak c).
      \end{align*}
    
    \item
      Let $r \in R$. We have
      \begin{align*}
           & r \frak c \subseteq \frak a : \frak b \\
      \iff & r \frak b \frak c \subseteq \frak a & \text{by (v)} \\
      \iff & r \in \frak a : \frak b \frak c,
      \end{align*}
      so
      \begin{align*}
        (\frak a : \frak b) : \frak c
          &= \{ r \in R ~|~ r \frak c \subseteq \frak a : \frak b \} \\
          &= \{ r \in R ~|~ r \in \frak a : \frak b \frak c \} \\
          &= \frak a : \frak b \frak c.
      \end{align*}
  \end{enumerate}
\end{proof}

\begin{proposition}
  Let $R$ be a Dedekind domain.
  Let $\frak p$, $\frak q$ be non-zero prime ideals of $R$.
  Then \[ \frak p \frak q : \frak p = \frak q. \]
\end{proposition}
\begin{proof}
  \note{Use ref labels...}
  Clearly, $\frak q \frak p \subseteq \frak p \frak q$.
  Using part (v) of above, this gives $\frak q \subseteq \frak p \frak q : \frak p$.
  However, $\frak p \not\subseteq \frak p \frak q$, so by part (vi),
  $\frak p \frak q : \frak p \neq R$.
  We have
  \[ \frak q \subseteq \frak p \frak q : \frak p \neq R. \]
  Since $\frak q$ is maximal, we get our result.
\end{proof}

\begin{proposition}
  Let $\frak a$ be an ideal of a Dedekind Domain.
  Let $a \in \frak a$.
  Then \[ \frak a (a : \frak a) = \pid a. \]
\end{proposition}
\begin{proof}
  \note{TODO}
\end{proof}

%%%%%%%%%%%%%%%%%%%%%%%%%%%%%%%%%%%%%%%%%%%%%%%%%%%%%%%%%%%%%%%%%%%%%%%%%%%%%%%

\subsection{???}

\begin{theorem}
  Let $F(K[C])$ denote the group of fractional ideals of $K[C]$ and define the maps
  \[ \div(-) : F(K[C]) \to \Div_K^0(C) \]
  \[ I \mapsto \sum_{P \in C - P_\infty} \min_{f \in I - \{0\}}\{\ord_P(f)\}(P - P_\infty) \]
  and
  \[ I_{(-)} : \Div_K^0(C) \to F(K[C]) \]
  \[ D \mapsto \{ f \in K(C)^* ~|~ \div(f) \geq D \} \cup \{ 0 \}. \]
  Then $\div(-)$ is an isomorphism of groups and $I_{(-)}$ is its inverse.
\end{theorem}
\begin{lemma}
  If $I$ is a fractional ideal of $K[C]$, then $\div I$ is a degree zero divisor.
\end{lemma}
\begin{proof}
  This is clear by construction,
  as $\div I$ is a sum of divisors of the form $(P - P_\infty)$,
  each of degree zero.
\end{proof}
\begin{lemma}
  If $I$ is a fractional ideal of $K[C]$, then $\div I$ is defined over $K$.
\end{lemma}
\begin{proof}
  \note{TODO}
\end{proof}
\begin{lemma}
  If $D$ is a degree zero divisor on $C$ defined over $K$,
  then $I_D$ is a fractional ideal of $K[C]$.
\end{lemma}
\begin{proof}
\end{proof}
\begin{lemma}
  Let $\frak a \in \cal J_C$ be a fractional ideal. Then
  \[ I_{\div \frak a} = \frak a. \]
\end{lemma}
\begin{proof}
  Let $a \in \frak a$ be non-zero. Then at every affine point $P$ on $C$,
  \[ \ord_P(\div \frak a) = \min_{0 \neq f \in \frak a}\{ \nu_P(f) \} \leq \nu_P(a). \]
  So $a \in I_{\div \frak a}$.
  
  Let $b \in I_{\div \frak a}$.
  Let $P$ be any point in the support of $\div \frak a$.
  Let $a$ factor into $\frak p_1^{k^1} \dots \frak p_n^{k_n}$. Then
  \begin{align*}
    \nu_P b
      &\geq \ord_P(\div \frak a) \\
      &= \ord_P \left( \div \left( \prod_{i=1}^n \frak p_i^{k_i} \right) \right) \\
      &= \ord_P \left( \sum_{i=1}^n k_i \div \frak p_i \right) \\
      &= \sum_{i=1}^n k_i \ord_P \left( \div \frak p_i \right) \\
      &= k_r \ord_P \left( \div \frak p_r \right)
  \end{align*}
  for some $1 \leq r \leq n$ (since prime ideals correspond to distinct orbits of points).
  
  \note{Try again}
  
  Let $b \in I_{\div \frak a}$.
  Let $a$ factor into $\frak p_1^{k^1} \dots \frak p_n^{k_n}$.
  We will show that $b$ is in each prime power.
  Let $r$ be an integer $1 \leq r \leq n$.
  Let $P$ be any point in the support of $\div \frak p_r$. Then
  \begin{align*}
    \nu_P(b)
      &\geq \ord_P(\div \frak a) \\
      &= \ord_P \left( \div \left( \prod_{i=1}^n \frak p_i^{k_i} \right) \right) \\
      &= \ord_P \left( \sum_{i=1}^n k_i \div \frak p_i \right) \\
      &= \sum_{i=1}^n k_i \ord_P \left( \div \frak p_i \right) \\
      &= k_r \ord_P \left( \div \frak p_r \right),
  \end{align*}
  where the last line follows from the fact that distinct prime ideals correspond to distinct orbits of points.
  Then there is an integer $\ell \geq k_r$ such that $b \in \frak p_r^\ell$, hence $b \in \frak p_r^{k_r}$.
  As $r$ was chosen arbitrarily, $b$ is in each prime power in the factorization of $\frak a$ and therefore in their intersection.
  Finally, since all prime powers are pairwise coprime, $\bigcap \frak p_i^{k_i} = \prod \frak p_i^{k_i} = \frak a$.
\end{proof}
\begin{lemma}
  Let $D \in \Div_K^0(C)$ be a degree zero divisor defined over $K$. Then
  \[ \div(I_D) = D. \]
\end{lemma}
\begin{proof}
  \note{roughly}
  \begin{align*}
    \div(I_D)
      &= \sum_{P \in C - P_\infty} \min_{f \in I_D - \{0\}}\{\ord_P(f)\}(P - P_\infty) \\
      &= D
  \end{align*}
  if and only if at every affine point $P$,
  \[ \min_{f \in I_D - \{0\}}\{\ord_P(f)\} = \ord_P(D), \]
  and at infinity,
  \[ \sum_{P \in C - P_\infty} \min_{f \in I_D - \{0\}}\{\ord_P(f)\} = - \ord_{P_\infty}(D). \]
  For the former, let $P$ be any affine point.
  By definition of $I_D$, we have $\div f \geq D$ for all non-zero $f \in I_D$,
  so we get $\ord_P(f) \geq \ord_P(D)$ for free.
  Suppose this inequality is strict.
  That is, for every non-zero polynomial $f \in I_D$, $\ord_P(f) > \ord_P(D)$.
\end{proof}
\begin{lemma}
  $\div(IJ) = \div(I) + \div(J)$.
\end{lemma}
\begin{proof}
\end{proof}



%%%%%%%%%%%%%%%%%%%%%%%%%%%%%%%%%%%%%%%%%%%%%%%%%%

\subsection{Reorganization of Prime Ideals section}

\begin{proposition}
  Let $P$ be an affine point on $C$ and let $f \in K[C]$. Then
  \[ f(P) = 0 \iff f(\sigma(P)) = 0. \]
\end{proposition}
\begin{proof}
\end{proof}

\begin{proposition}
  Let $P$ be an affine point on $C$ and let $\frak p$ be the set of polynomials that vanish on $\orb(P)$. Then
  \begin{enumerate}[label=(\roman*)]
    \item $\frak p$ is a $K[C]$-ideal;
    \item $\frak p$ is non-zero;
    \item $\frak p$ is prime;
    \item $\frak p$ is maximal;
    \item $\frak m_P = \frak p \cal O_P$.
  \end{enumerate}
\end{proposition}
\begin{proof}
\end{proof}

\begin{proposition}
  Let $P$ and $Q$ be affine points on $C$ and suppose $\orb(P) \neq \orb(Q)$.
  Then $I_{[P]} \neq I_{[Q]}$.
\end{proposition}
\begin{proof}
  Let $X$ and $Y$ be the sets
    \[ X = \{ x_i ~|~ (x_i, y_i) \in \orb(P) \} \triangle \{ x_i ~|~ (x_i, y_i) \in \orb(Q) \} \]
    \[ Y = \{ y_i ~|~ (x_i, y_i) \in \orb(P) \} \triangle \{ y_i ~|~ (x_i, y_i) \in \orb(Q) \}, \]
    where $\triangle$ denotes the symmetric difference of sets.
  So, e.g., $X$ is the set of $x$-coordinates found in the orbit of $P$ or $Q$, but not both.
  Suppose that $X$ is non-empty and contains an element $x_0$.
  Let $m(x_0)$ be the minimal polynomial of $x_0$, viewed as a polynomial in $K[C]$.
  Then $m$ is zero everywhere on the orbit of $P$, but is non-zero on the orbit of $Q$.
  Simlarly, if $Y$ contains an element $y_0$, then the minimal polynomial of $y_0$ gives the same result.

  Now suppose $X$ and $Y$ are both empty.
  Then there is a point $Q_0 \in \orb(Q)$ with the same $x$-coordinate as $P$, but whose $y$-coordinate is a conjugate of $P$'s.
  Without loss of generality, let $P = (x_1, y_1)$ and $Q_0 = (x_1, \sigma(y_1))$.
  
  Suppose, for contradiction, that $I_{[P]} = I_{[Q]}$.
  Let $\frak p = I_{[P]}$.
  Then $\frak p$ is precisely the polynomials in $K[C]$ passing through the point $(x_1, y_1)$.
  Likewise, $\frak p$ is simultaneously the ideal of polynomials in $K[C]$ passing through $(x_1, \sigma(y_1))$.
  Let $\frak q_1 = \pid{x - x_1, y - y_1}$ and $\frak{x - x_1, y - \sigma(y_1)}$ be prime ideals in $\bar K[C]$.
  Then
  \begin{align*}
    \frak p &= \frak q_1 \cap K[C] \\
    \frak p &= \frak q_2 \cap K[C]
  \end{align*}
  and
  \begin{align*}
    \frak p
      &= \frak p + \frak p \\
      &= (\frak q_1 \cap K[C]) + (\frak q_2 \cap K[C]) \\
      &= (\frak q_1 + \frak q_2) \cap K[C]) \\
      &= \bar K[C] \cap K[C] & \text{$\frak q_1, \frak q_2$ coprime} \\
      &= K[C],
  \end{align*}
  a contradiction, since $\frak p$ is prime.
\end{proof}

\begin{proposition}
  Let $P$ and $Q$ be affine points on $C$. The following are equivalent:
  \begin{enumerate}[label=\roman*]
    \item $Q \in \orb(P)$;
    \item $\orb(P) = \orb(Q)$;
    \item $[P] = [Q]$;
    \item $I_{[P]} = I_{[Q]}$.
  \end{enumerate}
\end{proposition}
\begin{proof}
\end{proof}

\begin{proposition}
  Let $\frak p$ be a non-zero prime ideal and let $P$ be an affine point on $C$. Then
  \[ P \in \supp(\div \frak p) \iff \frak p = I_{[P]}. \]
\end{proposition}
\begin{proof}
  ($\implies$)
  Suppose $P$ is in the support of $\div \frak p$.
  Let $\frak q$ be the ideal of polynomials vanishing at $P$.
  By an above proposition, $\frak q$ is prime.
  Moreover,
  \begin{align*}
    q &= \pid{ f \in K[C] ~|~ \text{$f$ vanishes at $P$}} \} \\
      &= \pid{ f \in K[C] ~|~ \text{$f$ vanishes on $[P]$}} \} \\
      &= I_{[P]}
  \end{align*}
  Since every polynomial in $\frak p$ vanishes at $P$, we have $\frak p \subseteq \frak q$.
  However, every non-zero prime ideal of $K[C]$ is maximal, so $\frak p = \frak q$.
  \note{This feels like it was more work than necessary.}
  
  ($\impliedby$)
  Suppose $\frak p = I_{[P]}$.
  For every non-zero polynomial $f \in \frak p$, $f(P) = 0$, hence $\nu_P(f) > 0$.
  Now the order of $\div \frak p$ at $P$ is
  \[ \ord_P(\div \frak p) = \min_{0 \neq f \in \frak p}\{ \nu_P(f) \} > ,0 \]
  therefore $P \in \supp(\div \frak p)$.
\end{proof}

\begin{proposition}
  Let $\frak p$ be a non-zero prime ideal and let $P$ be an affine point in the support of $\div \frak p$.
  Then $\ord_P(\div \frak p) = 1$.
\end{proposition}
\begin{proof}
  Clearly, $\ord_P(\div \frak p) \geq 1$.
  We must show that there is a polynomial in $\frak p$ whose valuation at $P$ is exactly 1.
  \note{Verify that the uniformizer of $\frak m_P$ works.}
\end{proof}

\begin{proposition}
  Let $\frak p$ be a non-zero prime ideal and let $P$ be an affine point in the support of $\div \frak p$.
  Then $\div \frak p = [P]$.
\end{proposition}
\begin{proof}
  By an above proposition, $P$ is in the support of $\div \frak p$ with order 1.
  So too are the other points in its orbit. \note{Prove this.}
  We must show that these are all the affine points.
  
  Suppose $\div \frak p = [P] + [Q]$ for some $Q \not\in [P]$.
  Then $Q \in \supp(\div \frak p)$.
  Then by an above proposition $\frak p = I_{[Q]}$.
  However $\frak p = I_{[P]}$.
  So $I_{[P]} = I_{[Q]}$, and by an above proposition, $Q \in [P]$, a contradiction.
\end{proof}

\begin{proposition}
  Let $\frak p$ be a non-zero prime ideal. Then
  \[ I_{\div \frak p} = \frak p. \]
\end{proposition}
\begin{proof}
  The divisor $\div \frak p$ is non-trivial with an affine point $P$ in its support.
  Then by above proposition, $\div \frak p = [P]$,
  and by another $\frak p = I_{[P]}$.
  Putting these together,
  \[ \frak p = I_{[P]} = I_{\div \frak p}. \]
\end{proof}
\begin{comment}
\begin{proof}
  Let $a \in \frak p$.
  Then at every affine point $P$ on $C$,
  \[ \ord_P(\div \frak p) = \min_{0 \neq f \in \frak p}\{ \nu_P(f) \} \leq \nu_P(a), \]
  so $a \in I_{\div \frak p}$.
  
  Let $b \in I_{\div \frak p}$.
  At every affine point $P$ on $C$, $\nu_P(b) \geq 0$.
  At every affine point $P \in \supp(\div \frak p)$,
  \[ \nu_P(b) \geq \ord_P(D). \]
  
  Let $b \in I_{\div \frak p}$.
  Let $P \in \supp(\div \frak p)$.
  By above proposition, $I_{\div \frak p} = I_{[P]}$, so $b \in I_{[P]}$.
  By another proposition above, $\frak p = I_{[P]}$, so $b \in \frak p$.
  Then $b \in I_{[P]}$.
\end{proof}
\end{comment}

\begin{proposition}
  Let $P$ be an affine point on $C$. Then
  \[ \div(I_{[P]}) = [P]. \]
\end{proposition}
\begin{proof}
  Let $\frak p := I_{[P]}$.
  Then $\frak p$ is prime and non-zero.
  By an above proposition, $P \in \supp(\div \frak p)$.
  By another above proposition, $\div \frak p = [P]$.
  Putting these together,
  \[ \div(I_{[P]}) = \div \frak p = [P]. \]
\end{proof}



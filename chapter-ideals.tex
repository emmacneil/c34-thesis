%%%%%%%%%%%%%%%%%%%%%%
%%%%%            %%%%%
%%%%%   Ideals   %%%%%
%%%%%            %%%%%
%%%%%%%%%%%%%%%%%%%%%%

\section{The Ideal Class Group}
\label{chap_ideals}

Performing arithmetic on divisors themselves is cumbersome.
As shown in Example \ref{ex_defined_over_k},
even if a divisor is defined over a field $K$,
the coordinates of its points may live in an algebraic extension $L/K$.
Performing arithmetic on the points therefore requires operations in $L$,
which is computationally slower than working in $K$.

In this chapter, we will see that the divisor class group of a curve
is isomorphic to the ideal class group of the curve's coordinate ring, $K[C]$.
Every divisor may therefore be represented by polynomials with coefficients in $K$,
rather than by points in $\bar K \times \bar K$.
By interpreting divisors as ideals,
we may perform our divisor arithmetic over the base field $K$.

We will demonstrate the existence of this isomorphism between the groups
by explicitly constructing an isomorphism and its inverse.
As the coordinate ring of a curve is a Dedekind domain,
we first describe how this isomorphism acts on non-zero prime ideals of $K[C]$.
We then extend this isomorphism to act on non-prime ideals, fractional ideals, and then ideal classes.



\subsection{Prime Ideals, Prime Divisors}

There is a relationship between prime ideals in $K[C]$ and prime divisors in $\Div_K^{\geq 0}$.
In fact, the relationship is a bijection, and the goal of this section is to explicitly describe that bijection.
We will define a map $I_{(-)}$ sending prime divisors to non-zero prime ideals
and a map $\div(-)$ sending non-zero prime ideals to prime divisors, then show that these are mutual inverses.
We begin by defining the former map and showing that it does in fact map primes to primes.

\begin{definition}
  Let $[P]$ be a prime divisor on $C$, as per Proposition \ref{prop_prime_divisors}.
  The \defn{ideal of $[P]$} is
  \[ I_{[P]} = \{ f \in K[C] ~|~ f(P) = 0 \}. \]
\end{definition}

This is the set of polynomials in $K[C]$ that vanish on the support of $[P]$.
By Lemma \ref{lem_galois_action_on_polynomial}, $I_{[P]} = I_{[Q]}$ for all $Q \in \orb(P)$.
In Section \ref{sec_local_rings}, we claimed that a point $P$ induces a prime ideal.
That prime ideal is $I_{[P]}$, and any other point in $\orb(P)$ induces the same prime ideal.
The proof that $I_{[P]}$ is prime is straightforward:

\begin{proposition}
  \label{prop_I_P_is_prime}
  Let $P$ be an affine point on $C$ and let $\frak p = I_{[P]}$. Then
  \begin{enumerate}[label=(\roman*)]
    \item $\frak p$ is a $K[C]$-ideal;
    \item $\frak p$ is non-zero;
    \item $\frak p$ is prime;
    \item $\frak p$ is maximal;
  \end{enumerate}
\end{proposition}
\begin{proof}
  \begin{enumerate}[label=(\roman*)]
    \item
    Suppose $f, g \in \frak p$. Then
      \[ (fg)(P) = f(P) + g(P) = 0 + 0 = 0, \]
    so $f + g \in \frak p$.
    Suppose $f \in \frak p$ and $h \in K[C]$. Then
      \[ (hf)(P) = h(P)f(P) = h(P)\cdot 0 = 0, \]
    so $hf \in \frak p$.
    
    \item
    Let $P = (x_0, y_0)$.
    Let $m(x) \in K[x]$ be the minimum polynomial\footnote{
    In the usual algebraic number theory sense of a unique monic polynomial in $K[x]$
    of minimal degree such that $m(x_0) = 0$.}
    of $x_0$.
    We may view $m$ as a polynomial in $K[C]$.
    Then $m$ is non-zero but is zero at $P$, so $m \in \frak p$.
    
    \item
    Suppose $fg \in \frak p$ for some $f, g \in K[C]$.
    Then $(fg)(P) = 0 = f(P)g(P)$.
    Since $f(P)$ and $g(P)$ are field elements, one of them must be zero.
    Therefore one of $f$ and $g$ is in $\frak p$.
    
    \item
    In a Dedekind domain, all non-zero prime ideals are maximal.\qedhere
  \end{enumerate}
\end{proof}

The next proposition gives another characterization of the ideal $I_{[P]}$
in terms of a prime ideal $\frak q$ of $\bar K[C]$ that lies over $I_{[P]}$.
\begin{proposition}
  Let $P = (x_0, y_0)$ be an affine point on $C$.
  Let $\frak q = \pid{x - x_0, y - y_0}$ as a $\bar K[C]$-ideal. Then
  \[ I_{[P]} = \frak q \cap K[C]. \]
\end{proposition}
\begin{proof}
  \begin{align*}
    I_{[P]}
      &= \pid{ f \in K[C] ~|~ f(P) = 0 } \\
      &= \pid{ f ~|~ f \in K[C], f(P) = 0 } \\
      &= \pid{ f ~|~ f \in K[C], f \in \frak q } \\
      &= \pid{ f ~|~ f \in K[C]} \cap \pid{ f ~|~ f \in \frak q } \\
      &= K[C] \cap \frak q.\qedhere
  \end{align*}
\end{proof}
Intuitively, while $I_{[P]}$ is prime as a $K[C]$-ideal, it might be non-prime as a $\bar K[C]$-ideal.
In $\bar K[C]$, $I_{[P]}$ factors into a product of primes, $\prod_{i=1}^n \frak q_i$,
possibly with $n > 1$.
For any of those factors $\frak q_i$, we may restrict back down to $K[C]$ to get $I_{[P]} = \frak q_i \cap K[C]$. 

We have shown that $I_{[-]}$ defines a map from prime divisors to prime ideals.
We now show that it maps distinct prime divisors to distinct prime ideals.
\begin{lemma}
  \label{lemma_orbs_and_ideals}
  Let $P$ and $Q$ be affine points on $C$.
  Then $I_{[P]} = I_{[Q]}$ if and only if $[P] = [Q]$.
\end{lemma}
\begin{proof}
  Certainly if $[P] = [Q]$, then $I_{[P]} = I_{[Q]}$.
  Suppose that $[P] \neq [Q]$.
  Write $P = (x_P, y_P)$ and $Q = (x_Q, y_Q)$.
  Let $m_x$ be the minimum polynomial\footnote{
  Again, in the usual algebraic number theory sense.}
  of $x_P$ and $m_y$ the minimum polynomial of $y_P$.
  
  If $m_x(x_Q) \neq 0$ or $m_y(y_Q) \neq 0$, then we have found a polynomial in $I_{[P]} - I_{[Q]}$ and we are done.
  Suppose $m_x(x_Q) = m_y(y_Q) = 0$.
  Then there are automorphisms $\sigma_x, \sigma_y \in \Gal(\bar K/K)$ such that
  $\sigma_x(x_P) = x_Q$ and $\sigma_y(y_P) = y_Q$.
  These automorphisms must be distinct, since $P \neq Q$.
  Let $\sigma = \sigma_x\inv \circ \sigma_y$ and $R = (x_P, \sigma(y_P))$.
  Then $R \in \orb(Q)$ and $I_{[R]} = I_{[Q]}$.

  Let $\frak p = \pid{x - x_P, y - y_P}$ and $\frak r = \pid{x - x_P, y - \sigma(y_P)}$. Then
  \begin{align*}
    I_{[P]} &= \frak p \cap K[C], \\
    I_{[R]} &= \frak r \cap K[C],
  \end{align*}
  \[ \frak p + \frak r = \bar K[C], \]
  and
  \begin{align*}
    I_{[P]} + I_{[Q]}
      &= I_{[P]} + I_{[R]} \\
      &= (\frak p \cap K[C]) + (\frak r \cap K[C]) \\
      &= (\frak p  + \frak r) \cap K[C] \\
      &= \bar K[C] \cap K[C] \\
      &= K[C] \neq I_{[P]}.
  \end{align*}
  Hence $I_{[P]} \neq I_{[Q]}$.
\end{proof}

In the other direction, we define a map sending non-zero prime ideals to prime divisors.
\begin{definition}
  \label{def_div_p}
  Let $\frak p$ be a non-zero prime ideal of $K[C]$.
  Define the \defn{divisor of $\frak p$} to be
  \[ \div \frak p = \sum_{P \in C - P_\infty} \min_{f \in \frak p - \{0\}}\{\nu_P(f)\}(P - P_\infty). \]
\end{definition}
The affine support of this divisor consists of those points at which \emph{every} polynomial in $\frak p$ is zero.
It is balanced by a negative multiple of the point at infinity,
so that this divisor is of degree zero by construction.
To show that $\div \frak p$ is a prime divisor,
we show $\div \frak p = [P]$ for some finite point $P$ on $C$.
First, we establish a lemma.

\begin{lemma}
  \label{lem_order_is_1}
  Let $P$ be a finite point on $C$ and let $\frak p = I_{[P]}$.
  Then $\ord_P(\div \frak p) = 1$.
\end{lemma}
\begin{proof}
  Since every polynomial in $I_{[P]}$ passes through $P$,
  \[ 1 \geq \min_{f \in I_{[P]} - \{0\}}\{ \nu_P(f) \}
          = \min_{f \in \frak p - \{0\}}\{ \nu_P(f) \}
          = \ord_P(\div \frak p). \]
  To show that $\ord_P(\div \frak p) = 1$,
  we must show that there is a polynomial in $\frak p$ whose valuation at $P$ is exactly 1.
  
  Let $P = (x_0, y_0)$ and consider the lines determined by $x - x_0$ and $y - y_0$.
  Since $C$ is non-singular, at most one of these lines is tangent to $C$ at $P$.
  Without loss of generality, suppose $x - x_0$ is not tangent to $C$ at $P$.
  Let $m(x)$ be the minimum polynomial of $x_0$, seen as an element of $K[C]$. Then
  \begin{align*}
    \nu_P(m)
      &= \nu_P\left( \prod_{i=0}^{\deg m - 1}(x - x_i)\right) \\
      &= \sum_{i=0}^{\deg m - 1} \nu_P(x - x_i) \\
      &= \nu_P(x - x_0)
        & \text{$\nu_P(x - x_i) = 0$ for $i \neq 0$} \\
      &= 1.
  \end{align*}
  Moreover, $m$ is zero on the orbit of $P$, so $m \in I_{[P]} = \frak p$.
\end{proof}

\begin{proposition}
  \label{prop_prime_ideals_prime_divisors}
  Let $\frak p$ be a non-zero prime ideal and let $P$ be an affine point on $C$.
  The following are equivalent
  \begin{enumerate}[label=(\roman*)]
    \item $P \in \supp(\div \frak p)$;
    \item $\frak p = I_{[P]}$.
    \item $\div \frak p = [P]$.
  \end{enumerate}
\end{proposition}
\begin{proof}
  \begin{description}
    \item [(i) $\implies$ (ii):]
      Suppose $P$ is in the support of $\div \frak p$.
      By Definition \ref{def_div_p}, every polynomial in $\frak p$ has a positive valuation at $P$.
      Therefore every polynomial in $\frak p$ is zero at $P$, so $\frak p \subseteq I_{[P]}$.
      Since $\frak p$ and $I_{[P]}$ are both prime ideals and therefore maximal, this implies that they are equal.

    \item [(ii) $\implies$ (i):]
      Suppose $\frak p = I_{[P]}$.
      For every non-zero polynomial $f \in \frak p$, $f(P) = 0$, hence $\nu_P(f) > 0$.
      Now the order of $\div \frak p$ at $P$ is
      \[ \ord_P(\div \frak p) = \min_{0 \neq f \in \frak p}\{ \nu_P(f) \} > 0 \]
      therefore $P \in \supp(\div \frak p)$.

    \item [(i) $\implies$ (iii):]
      Suppose $P \in \supp(\div \frak p)$.
      We must show that for all finite points $Q \in C$, $\ord_Q(\div \frak p) = \ord_Q([P])$.
      
      Let $Q \in C$ be a finite point and suppose $Q \in \supp(\div \frak p)$.
      Since (i) implies (ii), $\frak p = I_{[Q]}$.
      Then by Lemma \ref{lem_order_is_1}, $\ord_Q(\div \frak p) = 1$.
      By Lemma \ref{lemma_orbs_and_ideals}, $[P] = [Q]$, so $\ord_Q([P]) = 1$.
      
      Suppose instead $Q \not \in \supp(\div \frak p)$.
      Then $\ord_Q(\div \frak p) = 0$.
      By Lemma \ref{lemma_orbs_and_ideals}, $[P] \neq [Q]$, so $\ord_Q([P]) = 0$.
\begin{comment}
    \item [(ii) $\implies$ (iii):]
      Suppose $\frak p = I_{[P]}$.
      Since (ii) implies (i), $P \in \supp(\div\frak p)$.
      For every $\sigma \in \Gal(\bar K/K)$, $I_{[P]} = I_{[\sigma(P)]}$,
      so for every $\sigma \in \Gal(\bar K/K)$,  $\sigma(P) \in \supp(\div \frak p)$.
      Therefore the entire orbit of $P$ is in $\supp(\div \frak p)$ and $[P] \leq \div \frak p$.
      
      Conversely, if $Q$ is not in the orbit of $P$, then $\frak p \neq I_{[Q]}$,
      so that $Q \not \in \supp(\div \frak p)$.
      By Lemma \ref{lem_order_is_1}, $P$ appears with multiplicity 1, so that $\div \frak p = [P]$.
\end{comment}
    \item [(iii) $\implies$ (i):]
      Immediate from the definitions.\qedhere
  \end{description}
\end{proof}

Proposition \ref{prop_prime_ideals_prime_divisors} has two important consequences.
It shows that every non-zero prime ideal $\frak p$ of $K[C]$ arises from a prime divisor $[P]$,
and every prime divisor $[P]$ arises from a non-zero prime ideal $\frak p$.
This is summarized in the following theorem, the main result of this section.
It shows that prime ideals and prime divisors are in bijection
via the maps $I_{(-)}$ and $\div(-)$, which are mutual inverses.

\begin{theorem}
  \label{thm_prime_ideals_and_divisors}
  Let $P$ be an affine point in $C(\bar K)$ and let $\frak p$ be a non-zero prime ideal of $K[C]$. Then
  \begin{enumerate}[label=(\roman*)]
    \item $I_{\div \frak p} = \frak p$;
    \item $\div I_{[P]} = [P]$.
  \end{enumerate}
\end{theorem}
\begin{proof}
  \begin{enumerate}[label=(\roman*)]
    \item
      Let $\frak p$ be a non-zero prime ideal of $K[C]$.
      Then there is an affine point $P \in \supp(\div \frak p)$.
      Applying Proposition \ref{prop_prime_ideals_prime_divisors} twice,
      \[ \div I_{[P]} = \div \frak p = [P]. \]
    
    \item
      Let $P$ be an affine point in $C(\bar K)$ and let $\frak p = I_{[P]}$.
      Applying Proposition \ref{prop_prime_ideals_prime_divisors} twice,
      \[ I_{\div \frak p} = I_{[P]} = \frak p. \]
  \end{enumerate}
\end{proof}



%%%%%%%%%%%%%%%%%%%%%%%%%%%%%%%%%%%%%%%%%%%%%%%%%%%%%%%%%%%%%%%%%%%%%%%%%%%%%%%

\subsection{Ideals and Divisors}

The coordinate ring $K[C]$ is a Dedekind domain.
The non-zero ideals of $K[C]$ may be factored into a product of prime ideals, and this factorization is unique.
Our isomorphism between prime ideals and prime divisors can now be extended
to an isomorphism between the monoid of non-zero $K[C]$-ideals
and the monoid $\Div_K^{\geq 0}$ of divisors $D \geq 0$.
In the sections to follow, we extend it even further
to an isomorphism between the ideal and divisor class groups.

Let $\frak a$ be a non-zero ideal of $K[C]$.
Let its factorization into prime ideals be $\frak p_1^{k_1} \dots \frak p_n^{k_n}$.
Then define the divisor of $\frak a$ to be
\[ \div \frak a = \sum_{i=1}^n k_i \div \frak p_i. \]
The divisor of $\frak a$ is the sum of the divisors of its prime factors.
As for the whole ring $K[C]$ itself,
its prime factorization is the empty product which maps to the empty sum:
  \[ \div (K[C]) = 0. \]
Note that the divisor of $\frak a$ is of degree zero and defined over $K$,
by virtue of it being a sum of prime divisors in $\Div_K^0(C)$.

In the other direction, let $D$ be a non-zero divisor in $\Div_K^{\geq 0}(C)$.
Then it factors into a sum of prime divisors, say $D = k_1[P_1] + \dots + k_n[P_n]$.
Define the ideal of $D$ to be
\[ I_D = \prod_{i=1}^n I_{[P_i]}^{k_i}. \]
The divisor 0 is the empty sum.
Let it map to the empty product, which is the whole ring $K[C]$:
\[ I_{0} = K[C]. \]

Let $\cal I_C$ be the monoid of non-zero ideals of $K[C]$.
We now have maps $\div(-) : \cal I_C \to \Div_K^0(C)$ and $I_{(-)} : \Div_K^0(C) \to \cal I_C$.
\begin{theorem}
  The maps $\div(-)$ and $I_{(-)}$ are isomorphisms of monoids and mutual inverses.
\end{theorem}
\begin{proof}
  That these maps are monoid homomorphisms is clear from their definitions.
  We show that they are mutual inverses.

  Let $I \in \cal I_C$. Let its prime factorization be $\prod \frak p_i^{k^i}$. Then
  \begin{align*}
    I &= \prod_{i=1}^n \frak p_i^{k_i} \\
    \div I &= \sum_{i=1}^n k_i[P_i]
      & \text{where }P_i \in \supp(\div \frak p_i) \\
    I_{\div I} &= \prod_{i=1}^n I_{[P_i]}^{k_i} \\
               &= \prod_{i=1}^n \frak p_i^{k_i} \\
               &= I.
  \end{align*}
  Let $D \in \Div_K^0(C)$. Let its prime factorization be $\sum k_i[P_i]$. Then
  \begin{align*}
    D &= \sum_{i=1}^n k_i[P_i] \\
    I_D &= \prod_{i=1}^n I_{[P_i]}^{k_i} \\
    \div(I_D) &= \sum_{i=1}^n k_i [P_i] \\ &= D.
  \end{align*}
\end{proof}
\begin{comment}
\begin{proof}
  The proof is quite immediate after factoring each ideal.
  It has already been established that $\div$ maps the identity $K[C]$ of $\cal I_C$ to the identity $0$ of $\Div_K^0(C)$.
  Let $\frak a$ and $\frak b$ be non-zero ideals with prime factorizations
  \[ \frak a = \prod \frak p_i^{k_i}, ~~~ \frak b = \prod \frak q_i^{\ell_i}. \]
  Then
  \begin{align*}
    \div (\frak a \frak b)
      &= \div \left( \left( \prod \frak p_i^{k_i} \right) \left( \prod \frak q_j^{\ell_j} \right) \right) \\
      &= \sum k_i \div \frak p_i + \sum \ell_j \div \frak q_i \\
      &= \div \frak a + \div \frak b.
  \end{align*}
\end{proof}
\end{comment}



%%%%%%%%%%%%%%%%%%%%%%%%%%%%%%%%%%%%%%%%%%%%%%%%%%%%%%%%%%%%%%%%%%%%%%%%%%%%%%%

\subsection{Fractional Ideals and $\Div_K^0(C)$}

We may extend the maps even further to fractional ideals and the entirety of $\Div_K^0(C)$.

Let $\cal J_C$ denote the Abelian group of fractional ideals of $K[C]$.
Let $\frak a \in \cal J_C$.
Then $\frak a$ is of the form $\pid{\frac 1 f} \frak b$ for some non-zero polynomial $f \in K[C]$ and some integral ideal $\frak b$ of $K[C]$.
Define
\[ \div \frak a = \div \frak b - \div f. \]

Let $D \in \Div_K^0(C)$.
Then $D$ can be written in the form $D = A - \div(f)$ where $A \in \Div_K^{\geq 0}(C)$.
Define
\[ I_D = \pid{\frac 1 f} I_A. \]
We show that these two maps are well-defined.
Afterwards, we show that they are group homomorphisms.
Then, in similar fashion to the previous section, we end by showing they are isomorphisms and mutual inverses.

\begin{proposition}
  The map $\div(-) : \cal J_C \to \Div_K^0(C)$ is well defined.
\end{proposition}
\begin{proof}
  Suppose that $\frak a$ is a fractional ideal,
  $\frak b$ and $\frak c$ are integral ideals,
  $f$ and $g$ are non-zero polynomials, and
    \[ \frak a = \frac 1 f \frak b = \frac 1 g \frak c. \]
  Then $g \frak b = f \frak c$ are integral ideals and
  \begin{align*}
    g \frak b &= f \frak c \\
    \div(g \frak b) &= \div(f \frak c) \\
    \div(g) + \div \frak b &= \div(f) + \div \frak c \\
    \div \frak b - \div(f) &= \div \frak c - \div(g)
%    \div \left( \frac 1 f \frak b \right) &= \div \left( \frac 1 g \frak c \right)
  \end{align*}
\end{proof}

\begin{proposition}
  The map $I_{(-)} : \Div_K^0(C) \to \cal J_C$ is well defined.
\end{proposition}
\begin{proof}
  Suppose that $D, A, B \in \Div_K^0(C)$, $A, B \geq 0$,
  $f, g \in K[C]$ are non-zero, and
    \[ D = A - \div(f) = B - \div(g). \]
  Then
  \begin{align*}
    A + \div(g) &= B + \div(f) \\
    I_{A + \div(g)} &= I_{B + \div(f)} \\
    I_A I_{\div(g)} &= I_B I_{\div(f)} \\
    g I_A &= f I_B \\
    \frac 1 f I_A &= \frac 1 g I_B.
  \end{align*}
\end{proof}

\begin{proposition}
  The map $\div(-) : \cal J_C \to \Div_K^0(C)$ is a group homomorphism.
\end{proposition}
\begin{proof}
  Let $\frak a, \frak b$  be fractional ideals, and let
  \[ \frak a = \frac 1 f \frak a', ~~~ \frak b = \frac 1 g \frak b', \]
  where $\frak a'$ and $\frak b'$ are integral ideals. Then
  \begin{align*}
    \div(\frak a \frak b)
      &= \div \left( \frac 1 {fg} \frak a' \frak b' \right) \\
      &= \div(\frak a' \frak b') - \div (fg) \\
      &= \div \frak a' + \div \frak b' - \div f - \div g \\
      &= (\div \frak a' - \div f) + (\div \frak b' - \div g) \\
      &= \div \frak a + \div \frak b.
  \end{align*}
\end{proof}
\begin{proposition}
  The map $I_{(-)} : \Div_K^0(C) \to \cal J_C$ is a group homomorphism.
\end{proposition}
\begin{proof}
  Let $A, B \in \Div_K^0(C)$ and let
    \[ A = A' - \div f, ~~~ B = B' - \div g, \] 
  where $A', B' \geq 0$. Then
  \begin{align*}
    I_{A + B}
      &= I_{A' + B' - \div(fg)} \\
      &= \frac 1 {fg} I_{A' + B'} \\
      &= \frac 1 {fg} I_{A'}I_{B'} \\
      &= \left( \frac 1 f I_{A'} \right) \left( \frac 1 g I_{B'} \right) \\
      &= I_A I_B.
  \end{align*}
\end{proof}

\begin{theorem}
  \label{thm_ideals_divisors_isomorphic}
  The maps $\div(-) : \cal J_C \to \Div_K^0(C)$ and $I_{(-)} : \Div_K^0(C) \to \cal J_C$
  are group isomorphisms and mutual inverses.
\end{theorem}
\begin{proof}
  Let $\frak a$ be a fractional ideal with
  $\frak a = \frac 1 f \frak a'$,
  where $\frak a'$ is integral. Then
  \begin{align*}
    I_{\div \frak a}
      &= I_{\div \frak a' - \div f} \\
      &= \frac 1 f I_{\div \frak a'} \\
      &= \frac 1 f \frak a' \\
      &= \frak a.
  \end{align*}
  
  Let $D$ be a degree zero divisor defined over $K$ with
  $D = D' - \div f$, where $D' \geq 0$. Then
  \begin{align*}
    \div I_D
      &= \div \left( \frac 1 f I_{D'} \right) \\
      &= \div I_{D'} - \div f \\
      &= D' - \div f \\
      &= D.
  \end{align*}
\end{proof}



%%%%%%%%%%%%%%%%%%%%%%%%%%%%%%%%%%%%%%%%%%%%%%%%%%%%%%%%%%%%%%%%%%%%%%%%%%%%%%%

\subsection{The Ideal Class Group}

Let $\cal J_C$ be the group of fractional ideals of $K[C]$ and let $\cal P_C$ denote its subgroup of principal ideals.
The \defn{ideal class group} of $K[C]$ is
\[ \cal H_C = \frac {\cal J_C} {\cal P_C}. \]
Since $\cal J_C$ is isomorphic to $\Div_K^0(C)$ and $\cal P_C$ to $\Princ_K(C)$, we have
\[ \cal H_C \simeq \Cl_K^0(C). \]

In the ideal class group, two fractional ideals $\frak a$ and $\frak b$ are equivalent
if there is a rational function $\frac f g \in K(C)$ such that $\frak a = \frac f g \frak b$.
Under this relation, every fractional ideal is equivalent to an integral ideal.
Thus every ideal class has an integral representative.
In particular, given an ideal class $[\frak a]$, the inverse class $[\frak a\inv]$ has an integral representative,
given by a colon ideal (see Appendix \ref{appendix_colon_ideal}).

\begin{proposition}
  \label{prop_integral_ideal_inverse}
  Let $\frak a$ be a non-zero ideal of $K[C]$,
  and $[\frak a]$ its image in the ideal class group.
  Let $\alpha \in \frak a$. Then
  \[ [\frak a]\inv = [\alpha : \frak a]. \]
\end{proposition}
\begin{proof}
  By Corollary \ref{cor_ideal_times_quotient_is_principal},
  \[ \frak a(\alpha : \frak a) = (\alpha) \]
  as ideals.
  In the ideal class group,
  \[ [\frak a][\alpha : \frak a] = [\alpha] \equiv [1], \]
  so that $\frak a$ and $\alpha : \frak a$ are inverses of one another.
\end{proof}

Since the divisor and ideal class groups are isomorphic, and every ideal class has an integral representative,
we may now represent divisor classes by integral ideals, i.e. ideals generated by polynomials.
There is no need to work with fractional ideals and rational functions.
In the next chapter, we discuss this representation by polynomials in greater detail.

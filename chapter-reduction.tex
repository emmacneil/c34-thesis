%%%%%%%%%%%%%%%%%%%%%%%%%
%%%%%               %%%%%
%%%%%   Reduction   %%%%%
%%%%%               %%%%%
%%%%%%%%%%%%%%%%%%%%%%%%%

\section{Reduction}
\label{chap_reduction}

In the previous two chapters, we were concerned with adding and doubling reduced divisors,
thereby producing (likely) unreduced divisors.
Given an unreduced divisor $D$, we now wish to find the unique reduced divisor $\bar{\bar D}$ to which 
it is equivalent in the divisor class group.
We will use the letters $u,v,w$ to refer to elements of the reduced Gr\"obner basis
of the ideal of the possibly unreduced divisor $D$.
We will use the letters $f,g,h$ for elements of a Gr\"obner basis of the reduced equivalent $\bar{\bar D}$.

Arita and Abu Salem/Khuri-Makdisi recognized that
flipping a divisor twice would reduce it (though Arita did not use the term ``flipping''),
and that flipping in the divisor class group is analogous to computing quotient ideals in the ideal class group.

In \cite{arita05-2}, Arita reduces divisors by flipping twice.
Given $D$, one computes $\bar D$ followed by $\bar{\bar D}$ in two separate steps.
Each flip operation requires one inversion in the finite field, and several multiplications.
For instance, if $D$ is a type 61 divisor,
it costs 1I+54M (1 finite field inversion and 54 finite field multiplications)
to compute $\bar D$ and 1I+16M to compute $\bar{\bar D}$.

In \cite{salem07}, Abu Salem and Khuri-Makdisi recognize that the field inversions performed
when computing $\bar D$ and $\bar{\bar D}$ are in fact the same values.
When flipping $I_D = \pid{u,v}$ to get $I_{\bar D} = \pid{f,g'}$, they must compute $f_2\inv$,
where $f_2$ is the coefficient of $y$ in $f_2$.
When computing $I_{\bar{\bar D}} = \pid{f,g}$, they must again compute $f_2\inv$.
They therefore save an expensive inversion operation by passing $f_2\inv$ as an input to the second flip operation.
They are able to compute $\bar D$ in 1I+31M and $\bar{\bar D}$ in 0I+7M, a total of 1I+38M.

In \cite{kmakdisi18}, Khuri-Makdisi gives an improvement over the cost of reducing in \cite{salem07}
by viewing reduction in a single step rather than as two flip operations.
The cost of reduction is decreased to 1I+19M, which is, remarkably, cheaper than even a single flip operation.
In this chapter, we will show that the reduction method presented in \cite{kmakdisi18}
may be applied to divisors of all types, with one small exception ---
the same exception encounter in Chapter \ref{chap_doubling}.

In Chapter \ref{chap_doubling}, we had to fall back on slower methods
to double type 31 divisors with $\pid{f,g} \neq \pid{f,g,h} \neq \pid{f,h}$.
Type 31, 41, 51, and 61 divisors with the same property $\pid{u,v} \neq \pid{u,v,w} \neq \pid{u,w}$
present an obstacle for us again here (though see Section \ref{sec_future_work}, Future Work)
and we will reduce those divisors via two discrete flip steps.

We will first present in Section \ref{sec_flipping} a method to compute the flip of a divisor,
which is a generalization of the method in \cite{salem07} to include atypical divisors.
Afterwards, in Section \ref{sec_reduction}, we will see how to reduce a divisor in a single reduction operation,
a generalization of \cite{kmakdisi18} to include almost all atypical divisors.

This chapter makes frequent use of colon ideals.
The reader is reminded that notation and basic properties
of the colon ideal are given in Appendix \ref{appendix_colon_ideal}.


%%%%%%%%%%%%%%%%%%%%%%%%
%%%%%              %%%%%
%%%%%   Flipping   %%%%%
%%%%%              %%%%%
%%%%%%%%%%%%%%%%%%%%%%%%

\subsection{Flipping}
\label{sec_flipping}

When flipping a divisor $D$, we have three cases to consider:
the reduced Gr\"obner basis of $I_D$ consists of 1, 2, or 3 elements.

In the first case, suppose the reduced Gr\"obner basis of $I_D$ consists of 1 element.
That is $I_D = \pid u$ is principal.
In this case, flipping $D$ could not be easier.
By Proposition \ref{prop_colon_ideal}.(vi),
\[ I_{\bar D} = u : I_D = \pid 1. \]
When asked to flip a principal divisor, we return $\bar D = 0$ or equivalently $I_{\bar D} = \pid 1$.

Suppose instead the reduced Gr\"obner basis for $I_D$ has 2 elements, $I_D = \pid{u,v}$.
Then 
\[ I_{\bar D} = u : I_D = u : \pid{u,v} = u : v. \]
Therefore
\[ I_{\bar D} = \{ p \in K[C] ~|~ pv \in \pid u \}. \]
To compute $I_{\bar D}$ is to find polynomials $p$ for which $pv \in \pid u$.
As was the case for addition and doubling, we will restrict our search space
by choosing a bounding monomial $m$.
We know \emph{a priori} what the type of $\bar{\bar D}$ is (Table \ref{tab_large_flip_type}),
so we let $m$ be the largest monomial appearing the reduced Gr\"obner basis of an ideal of that type.
We then look for polynomials $p \in W^m$ for which $pv$ vanishes modulo $u$.
Those polynomials arise as the kernel of the map $\bar M$ in 
\[ \begin{tikzcd}
  W_{\bar D}^m \arrow[hook]{r}{\ker\bar M} &
  W^m \arrow{r}{v \cdot} \arrow[bend left]{rrr}{\bar M} &
  vW^m \arrow{r}{\iota} &
  W^{m_2} \arrow[two heads]{r}{\pi} &
  \frac {W^{m_2}} {uW^{m_1}}
\end{tikzcd}, \]
where $v\cdot$ is the multiplication by $v$ map and $m_1$ and $m_2$ are chosen such that
\begin{equation}
  \label{eq_flip_monomial_choice}
  -\nu_{P_\infty}(m_2) = -\nu_{P_\infty}(m\LM(v)) = -\nu_{P_\infty}(m_1\LM(u)).
\end{equation}

Now suppose that $\pid{u,v,w}$ is a Gr\"obner basis for $I_D$.
Typically, the first two polynomials alone are enough to generate $I_D$.
That is, $\pid{u,v,w} = \pid{u,v}$.
In that case, we may flip $D$ by computing the kernel of the map $\bar M$ above.

Sometimes, $\pid{u,v,w} \neq \pid{u,v}$, but it may still be that $\pid{u,v,w} = \pid{u,w}$.
In this case, too, we may flip $D$ by computing the kernel of the map $\bar M$ above,
though replace $v$ with $w$ in the diagram.

The more difficult case is when $\pid{u,v} \neq \pid{u,v,w} \neq \pid{u,w}$.
In this case, we observe that
\[ I_{\bar D} = u : \pid{u,v,w} = u : \pid{v,w} = (u : v) \cap (u : w). \]
Let $A = \div(u : v) = \div v - D$ and $B = \div(u : w) = \div w - D$.
Then $\bar D = \lcm(A,B)$.
We may compute each of $A$ and $B$ individually using method outlined above,
then compute their least common multiple using methods from Chapter \ref{chap_addition}.
\[ \begin{tikzcd}
  & W_A^m \arrow[hook]{r}{\ker\bar M} &
  W^m \arrow{r}{v \cdot} \arrow[bend left]{rrr}{\bar{M_A}} &
  vW^m \arrow{r}{\iota} &
  W^{m_2} \arrow[two heads]{r}{\pi} &
  \frac {W^{m_2}} {uW^{m_1}} \\
  W_{\bar D}^m = W_A^m \cap W_B^m \arrow[hook]{ru} \arrow[hook]{rd} & \\
  & W_B^m \arrow[hook]{r}{\ker\bar M} &
  W^m \arrow{r}{w \cdot} \arrow[bend right]{rrr}{\bar{M_B}} &
  wW^m \arrow{r}{\iota} &
  W^{m_4} \arrow[two heads]{r}{\pi} &
  \frac {W^{m_4}} {uW^{m_3}}
\end{tikzcd}. \]



%%%%%%%%%%%%%%%%%%%%%%%%%%%%%%%%%%%%%%%%%%
%%%%%                                %%%%%
%%%%%   Reduction without Flipping   %%%%%
%%%%%                                %%%%%
%%%%%%%%%%%%%%%%%%%%%%%%%%%%%%%%%%%%%%%%%%

\subsection{Reduction without Flipping}
\label{sec_reduction}

In the appendix of \cite{kmakdisi18}, Khuri-Makdisi presents an improvement
over his previous joint work with Abu Salem in \cite{salem07}.
The improvement lies entirely in the reduction step.

Let $D$ be a typical degree 6 divisor, with ideal $I_D = \pid{u,v,w}$.
Since $D$ is typical, the first two polynomials alone are enough to generate $I_D$,
so $I_D = \pid{u,v}$.
In prior work \cite{salem07}, $D$ was reduced by flipping twice, computing $\bar D$ and then $\bar{\bar D}$.
In \cite{kmakdisi18}, it is observed that this is more work than is necessary.
It is known \emph{a priori} that $\bar{\bar D}$ is a typical degree 3 divisor.
The minimum polynomial $f \in I_{\bar{\bar D}}$ is therefore known to be of the form $f = x^2 + f_2y + f_1x + f_0$.
It is shown in the appendix of \cite{kmakdisi18} that to compute $\bar{\bar D}$,
it is enough to find a polynomial $g \in K[C]$ such that $fv \equiv gu \pmod F$.
Then
  \[ I_{\bar{\bar D}} = \pid{f, \bar g}, \]
where $\bar g$ is the reduction of $g$ modulo $f$.

This statement holds in greater generality than was considered in \cite{kmakdisi18}.
It applies to any divisor $D$ whose ideal $I_D$ is generated by two polynomials $u,v$,
where $u$ is the minimum polynomial in $I_D$.
We will prove that the methods outlined in the appendix of \cite{kmakdisi18}
may be applied to this wider class of divisors.
First, we will establish a lemma that relates generators of $u : v$ and $v : u$.

\begin{lemma}
  \label{lem_reverse_colon}
  Let $I_D = \pid{u,v}$ and let $\pid{f, r} = u : v$.
  Then there exist polynomials $g, s \in K[C]$ such that
  \begin{align*}
    fv &\equiv gu \pmod F \\
    rv &\equiv su \pmod F
  \end{align*}
  and $\pid{g, s} = v : u$.
\end{lemma}
\begin{proof}
  The existence of $g$ and $s$ satisfying the two congruence relations
  comes from the fact that $\pid{f,r} = u : v$ and the definition of the colon ideal.
  Now $(u : v)\pid{u,v} = \pid u$, so
  \begin{align*}
    \pid u
      &= \pid{f,r}\pid{u,v} \\
      &= \pid{fu, fv, ru, rv} \\
      &= \pid{fu, gu, ru, su} \\
      &= \pid{f, g, r, s}\pid u,
  \end{align*}
  hence
  \begin{equation}
    \label{eq_fgrs_is_1}
    \pid{f,g,r,s} = \pid 1,
  \end{equation}
  and
  \begin{align*}
    \pid v
      &= \pid{fv, gv, rv, sv} \\
      &= \pid{gu, gv, su, sv} \\
      &= \pid{g,s}\pid{u,v}.
  \end{align*}
  Therefore $\pid{g,s} = v : u$.
\end{proof}

\begin{theorem}
  \label{thm_fast_reduction}
  Let $I_D = \pid{u,v}$, where $u$ is the minimum polynomial in $I_D$.
  Let $I_{\bar D} = \pid{f, r}$, where $f$ is the minimum polynomial in $I_{\bar D}$.
  Then there is a polynomial $g$ such that $fv \equiv gu \pmod F$ and $I_{\bar{\bar D}} = \pid{f,g}$.
\end{theorem}
\begin{proof}
  By Lemma \ref{lem_reverse_colon},
  there are polynomials $g, s \in K[C]$ such that $fv \equiv gu$, $rv \equiv su$, and $\pid{g,s} = v : u$.
  The polynomial $g$ is the minimum polynomial in $\pid{g,s}$,
  otherwise $f$ would not be the minimum in $u : v$.
  
  Let $A = \div(g,s)$.
  Then $I_{\bar A} = g : s$.
  By Proposition \ref{prop_flip_properties}, $\bar A = \bar{\bar D}$,
  so $g : s = f : r = I_{\bar{\bar D}}$.
  
  Now certainly $\pid{f,g} \subseteq I_{\bar{\bar D}}$,
  since $f \in f : r = I_{\bar{\bar D}}$ and $g \in g : s = I_{\bar{\bar D}}$.
  
  Let $t \in I_{\bar{\bar D}}$.
  Then $t \in f : r$ and $t \in g : s$, therefore $rt \in \pid f$ and $st \in \pid g$,
  so $\pid{rt, st} \subseteq \pid{f,g}$.
  By Equation \ref{eq_fgrs_is_1},
  \[ \pid t = \pid{ft,gt,rt,st} = \pid{ft,gt} + \pid{rt,st}. \]
  Now $\pid{ft,gt}$ and $\pid{rt,st}$ are both contained in $\pid{f, g}$,
  while their sum $\pid{ft,gt} + \pid{rt,st} = \pid t$ is the smallest ideal containing them.
  Therefore $\pid t \subseteq \pid{f,g}$ and we conclude $t \in \pid{f,g}$.
\end{proof}
\begin{remark}
  It is required that $u$ and $f$ be the minimum polynomials in their respective ideals.
  Recall that the definition of $D$ is $\bar D = \div u - D$
  (or in the ideal class group, $I_{\bar D} = u : I_D$)
  where $u$ is the minimum polynomial.
  If $u$ is not the minimum, then $I_{\bar D} \neq u : v$.
\end{remark}

In Section \ref{sec_reduction_example}, we will see how Theorem \ref{thm_fast_reduction} is applied.
It can be used to quickly reduce divisors of all types, except for atypical divisors of types 31, 41, 51, and 61
where $\pid{u,v} \neq \pid{u,v,w} \neq \pid{u,w}$.
In that case, we flip twice using the method in Section \ref{sec_flipping}.
On a curve over a finite field of order $q$,
we expect this troublesome case to arise with probability approximately $\frac 1 {q^2}$ for large $q$.



%%%%%%%%%%%%%%%%%%%%%%%%%%%%%%%%
%%%%%                      %%%%%
%%%%%   Flipping Example   %%%%%
%%%%%                      %%%%%
%%%%%%%%%%%%%%%%%%%%%%%%%%%%%%%%

\subsection{Flipping Example}
\label{sec_flipping_example}

Let $C$ be the $C_{3,4}$ over $\bb F_{31}$ given by the polynomial equation $F(x,y) = y^3 + x^4 + 1 = 0$.
Let $D$ be the type 61 divisor given by the ideal $I_D = \pid{u,v,w}$, where
\begin{align*}
  u &=  x^3 - 12y^2 + 2xy - 13x^2 - 10y + 7x + 10, \\
  v &= x^2y -  9y^2 - 5xy +  2x^2 -  2x      +  9, \\
  w &= xy^2 -  4y^2 - 9xy -  5x^2 -  3y - 5x + 11.
\end{align*}
In this particular example, we have $\pid{u,v} \neq \pid{u,v,w} \neq \pid{u,w}$.
(See Lemmas \ref{lem_fgh_is_fg_61} and \ref{lem_fgh_is_fh_61}.)
Therefore we compute $I_{\bar D} = (u : v) \cap (u : w)$.
Let $A = \div(u : v)$ and $B = \div(u : w)$.
We will compute $A$ and $B$.

We know from Table \ref{tab_large_flip_type} that $\bar D$ is a type 31 divisor.
From Table \ref{tab_divisor_types},
we also know that no monomial in the reduced Gr\"obner basis of $I_{\bar D}$ is larger than $m = y^2$.
The monomial $x^6$ is chosen as $x^6$ has the same pole order as $m\LM(v) = x^2y^3$.
The other monomial $x^3$ is chosen as $x^3\cdot\LM(u) = x^6$.
(See Equation \ref{eq_flip_monomial_choice}.)

To find $A$, we must compute the kernel of the map $M_{\text{flip}}$ in the diagram
\[ \begin{tikzcd}
  W_A^{y^2} \arrow[hook]{r}{\ker M_{\text{flip}}} &
  W^{y^2} \arrow{r}{v \cdot} \arrow[bend left]{rrr}{M_{\text{flip}}} &
  vW^{y^2} \arrow{r}{\iota} &
  W^{x^6} \arrow[two heads]{r}{\pi} &
  \frac {W^{x^6}} {uW^{x^3}}
\end{tikzcd}. \]
We will need bases for $W^{y^2}$ and $uW^{x^3}$
\begin{align*}
  W^{y^2} &= \Span_K\{ 1, x, y, x^2, xy, y^2 \} \\
  uW^{x^3} &= \Span_K\{ u, xu, yu, x^2u, xyu, y^2u, x^3u \}.
\end{align*}
We must compute the image of the monomial basis of $W^{y^2}$ under the map $M_{\text{flip}}$,
so we must multiply each monomial in the basis of $W^{y^2}$ by $v$ and reduce modulo $u$.
In the matrix below, we encode all the coefficients of $u, xu, \ldots, x^3u$ and $v, xv, \ldots, y^2v$.
All $y^3$ terms are reduced modulo $F$.
\[ \left( \begin{array}{r|rrrrrrr|rrrrrr}
         & u & xu & yu & x^2u & xyu & y^2u & x^3u & v & xv & yv & x^2v & xyv & y^2v \\
  \hline
     1 & 10 &  0 & 12 &  0 &  0 & 10 &  0 &  9 &  0 &  9 &  0 &  0 &  0 \\
     x &  7 & 10 &  0 &  0 & 12 & 29 &  0 & 29 &  9 &  0 &  0 &  9 &  5 \\
     y & 21 &  0 & 10 &  0 &  0 & 12 &  0 &  0 &  0 &  9 &  0 &  0 &  9 \\
   x^2 & 18 &  7 &  0 & 10 &  0 &  0 &  0 &  2 & 29 &  0 &  9 &  0 & 30 \\
    xy &  2 & 21 &  7 &  0 & 10 &  0 &  0 & 26 &  0 & 29 &  0 &  9 &  0 \\
   y^2 & 19 &  0 & 21 &  0 &  0 & 10 &  0 & 22 &  0 &  0 &  0 &  0 &  9 \\
   x^3 &  1 & 18 &  0 &  7 &  0 &  0 & 10 &  0 &  2 &  0 & 29 &  0 &  0 \\
  x^2y &  0 &  2 & 18 & 21 &  7 &  0 &  0 &  1 & 26 &  2 &  0 & 29 &  0 \\
  xy^2 &  0 & 19 &  2 &  0 & 21 &  7 &  0 &  0 & 22 & 26 &  0 &  0 & 29 \\
   x^4 &  0 &  1 & 12 & 18 &  0 & 10 &  7 &  0 &  0 &  9 &  2 &  0 &  0 \\
  x^3y &  0 &  0 &  1 &  2 & 18 &  0 & 21 &  0 &  1 &  0 & 26 &  2 &  0 \\
x^2y^2 &  0 &  0 &  0 & 19 &  2 & 18 &  0 &  0 &  0 &  1 & 22 & 26 &  2 \\
   x^5 &  0 &  0 &  0 &  1 & 12 & 29 & 18 &  0 &  0 &  0 &  0 &  9 &  5 \\
  x^4y &  0 &  0 &  0 &  0 &  1 & 12 &  2 &  0 &  0 &  0 &  1 &  0 &  9 \\
x^3y^2 &  0 &  0 &  0 &  0 &  0 &  1 & 19 &  0 &  0 &  0 &  0 &  1 &  0 \\
   x^6 &  0 &  0 &  0 &  0 &  0 &  0 &  1 &  0 &  0 &  0 &  0 &  0 & 30
\end{array} \right) \]

Now we reduce the columns of the right-hand side of the matrix modulo the columns of the left-hand side.
\[ \left( \begin{array}{r|rrrrrrr|rrrrrr}
         & u & xu & yu & x^2u & xyu & y^2u & x^3u & v & xv & yv & x^2v & xyv & y^2v \\
  \hline
     1 & 10 &  0 & 12 &  0 &  0 & 10 &  0 &  9 &  9 & 17 &  9 & 17 & 26 \\
     x &  7 & 10 &  0 &  0 & 12 & 29 &  0 & 29 & 29 & 21 & 29 & 21 & 11 \\
     y & 21 &  0 & 10 &  0 &  0 & 12 &  0 &  0 &  0 &  1 &  0 &  1 & 20 \\
   x^2 & 18 &  7 &  0 & 10 &  0 &  0 &  0 &  2 &  2 &  1 &  2 &  1 & 26 \\
    xy &  2 & 21 &  7 &  0 & 10 &  0 &  0 & 26 & 26 &  9 & 26 &  9 & 10 \\
   y^2 & 19 &  0 & 21 &  0 &  0 & 10 &  0 & 22 & 22 &  9 & 22 &  9 & 29 \\
   x^3 &  1 & 18 &  0 &  7 &  0 &  0 & 10 &  0 &  0 &  0 &  0 &  0 &  0 \\
  x^2y &  0 &  2 & 18 & 21 &  7 &  0 &  0 &  1 &  1 & 15 &  1 & 15 & 24 \\
  xy^2 &  0 & 19 &  2 &  0 & 21 &  7 &  0 &  0 &  0 & 10 &  0 & 10 & 14 \\
   x^4 &  0 &  1 & 12 & 18 &  0 & 10 &  7 &  0 &  0 &  0 &  0 &  0 &  0 \\
  x^3y &  0 &  0 &  1 &  2 & 18 &  0 & 21 &  0 &  0 &  0 &  0 &  0 &  0 \\
x^2y^2 &  0 &  0 &  0 & 19 &  2 & 18 &  0 &  0 &  0 &  1 &  0 &  1 & 20 \\
   x^5 &  0 &  0 &  0 &  1 & 12 & 29 & 18 &  0 &  0 &  0 &  0 &  0 &  0 \\
  x^4y &  0 &  0 &  0 &  0 &  1 & 12 &  2 &  0 &  0 &  0 &  0 &  0 &  0 \\
x^3y^2 &  0 &  0 &  0 &  0 &  0 &  1 & 19 &  0 &  0 &  0 &  0 &  0 &  0 \\
   x^6 &  0 &  0 &  0 &  0 &  0 &  0 &  1 &  0 &  0 &  0 &  0 &  0 &  0
\end{array} \right) \]
The non-zero rows on the right describe the matrix $M_{\text{flip}}$.
\[ M_{\text{flip}} = \begin{pmatrix}
  9 &  9 & 17 &  9 & 17 & 26 \\
 29 & 29 & 21 & 29 & 21 & 11 \\
  0 &  0 &  1 &  0 &  1 & 20 \\
  2 &  2 &  1 &  2 &  1 & 26 \\
 26 & 26 &  9 & 26 &  9 & 10 \\
 22 & 22 &  9 & 22 &  9 & 29 \\
  1 &  1 & 15 &  1 & 15 & 24 \\
  0 &  0 & 10 &  0 & 10 & 14 \\
  0 &  0 &  1 &  0 &  1 & 20
\end{pmatrix}. \]
It is clear at a glance that most of the rows in this matrix are linearly dependent.
Its reduced row echelon form and kernel are
\[ \text{RREF}(M_{\text{flip}}) =
\begin{pmatrix}
  1 &  1 &  0 &  1 &  0 &  3 \\
  0 &  0 &  1 &  0 &  1 & 20 \\
  0 &  0 &  0 &  0 &  0 &  0 \\
  0 &  0 &  0 &  0 &  0 &  0 \\
  0 &  0 &  0 &  0 &  0 &  0 \\
  0 &  0 &  0 &  0 &  0 &  0 \\
  0 &  0 &  0 &  0 &  0 &  0 \\
  0 &  0 &  0 &  0 &  0 &  0 \\
  0 &  0 &  0 &  0 &  0 &  0
\end{pmatrix}, ~~
\ker(M_{\text{flip}}) = 
\begin{pmatrix}
  -1 & -1 &  0 &  -3 \\
   1 &  0 &  0 &   0 \\
   0 &  0 & -1 & -20 \\
   0 &  1 &  0 &   0 \\
   0 &  0 &  1 &   0 \\
   0 &  0 &  0 &   1
\end{pmatrix} \]
Therefore $W_A^{y^2}$ is spanned by
  \[ W_A^{y^2} = \Span_K \{ x - 1, ~x^2 - 1, ~xy - y, ~y^2 - 20y - 3 \} \]
and $A$ is the type 22 divisor given by
  \[I_A = \pid{x - 1, ~y^2 - 20y - 3}.\]

To find $B$, we follow the same process, computing the kernel of $M_{\text{flip}}$ in
\[ \begin{tikzcd}
  W_B^{y^2} \arrow[hook]{r}{\ker M_{\text{flip}}} &
  W^{y^2} \arrow{r}{w \cdot} \arrow[bend left]{rrr}{M_{\text{flip}}} &
  vW^{y^2} \arrow{r}{\iota} &
  W^{x^5y} \arrow[two heads]{r}{\pi} &
  \frac {W^{x^5y}} {uW^{x^2y}}
\end{tikzcd}. \]
We find that $W_B^{y^2}$ is spanned by
  \[ W_B^{y^2} = \Span_K \{ y + 12x - 8, x^2 + 13x - 14, xy - 9x + 13, y^2 - 13x - 3 \} \]
and $B$ is the type 21 divisor given by
  \[I_A = \pid{y + 12x - 8, x^2 + 13x - 14}.\]
  
Now given $W_A^{y^2}$ and $W_B^{y^2}$ one may compute $W_{\bar D}^{y^2} = W_A^{y^2} \cap W_B^{y^2}$
using standard linear algebra techniques.
Alternatively, given $I_A$ and $I_B$,
one may compute $I_{\bar D} = I_{\lcm(A,B)}$ using methods from Chapter \ref{chap_addition}.

\begin{remark}
  The matrices above were rather large,
  and a lot of unnecessary computational effort goes into filling them completely.
  The first eight rows of the large matrix were linearly dependent, and this will always be the case.
  One may safely ignore all rows above the line representing coefficients of the monomial $x^2y$.
  
  The vector spaces $W_A^{y^2}$ and $W_B^{y^2}$ were 4-dimensional,
  while the reduced Gr\"obner bases they produced consisted only of two elements each.
  Therefore two columns in the kernel we computed were unnecessary.
  One can save time by ignoring the two columns corresponding to the unnecessary kernel basis vectors.
  It is cheaper to derive those two vectors from the other two.
  
  Considerably less computation time needs to be spent on a flip operation than may appear.
\end{remark}



%%%%%%%%%%%%%%%%%%%%%%%%%%%%%%%%%
%%%%%                       %%%%%
%%%%%   Reduction Example   %%%%%
%%%%%                       %%%%%
%%%%%%%%%%%%%%%%%%%%%%%%%%%%%%%%%

\subsection{Reduction Example}
\label{sec_reduction_example}

Let $C$ be the $C_{3,4}$ over $\bb F_{31}$ given by the polynomial equation $F(x,y) = y^3 + x^4 + 1 = 0$.
In Section \ref{sec_fast_doubling_example}, we doubled a type 31 divisor to produce a type 61 divisor.
Let $D$ be that type 61 divisor, given by the ideal $I_D = \pid{u,v,w}$, where
\begin{align*}
  u &=  x^3 -  7y^2 - 14xy +  3x^2 - 8y +  2x +  7, \\
  v &= x^2y - 13y^2 + 12xy - 13x^2 - 4y -  6x - 10, \\
  w &= xy^2 -  5y^2 +  9xy +  7x^2 + 3y + 10x +  5.
\end{align*}
In that example, we could have saved some effort by not computing $w$ at all, for we will not need it in this example.
By Lemma \ref{lem_fgh_is_fg_61}, this divisor is typical, so that $I_D = \pid{u,v}$.
We may ignore $w$ and compute $\bar{\bar D}$ by applying Theorem \ref{thm_fast_reduction}.

By Theorem \ref{thm_fast_reduction},
there exist polynomials $f, g \in K[C]$ such that $fv \equiv gu$ and $I_{\bar{\bar D}} = \pid{f,g}$.
We know \emph{a priori} that $\bar{\bar D}$ is a typical type 31 divisor,
and we may deduce that $f$ and $g$ are of the forms
\begin{align*}
  f &= x^2 + f_2y + f_1x + f_0 \\
  g &= xy + g_3x^2 + g_2y + g_1x + g_0.
\end{align*}
Notice the $x^2$ term in $g$; $g$ is not assumed to be reduced modulo $f$.
We will have to reduce it later.

Now we use the relation $fv \equiv gu \pmod F$ to solve for the coefficients of $f$ and $g$.
We observe that the polynomial $fv - gu$ (as a member of $K[x,y]$) contains the terms
$(-13f_2 + 7g_2)y^3$ and $7xy^3$.
Thus we equate coefficients in 
\[ fv - gu - (7x - 13f_2 + 7g_2)F = 0 \]
to get a system of linear equations
\begin{align*}
  -g_3 - 7 &= 0 \\
  f_2 + 7g_3 + 1 &= 0 \\
  f_1 - g_2 + 14g_3 + 9 &= 0 \\
  13f_2 - g_1 - 7g_2 - 3g_3 - 13 &= 0 \\
  -13f_1 + 12f_2 + 7g_1 + 14g_2 + 8 &= 0 \\
  f_0 + 12f_1 - 13f_2 + 14g_1 - 3g_2 + 8g_3 - 6 &= 0 \\
  -13f_1 - g_0 - 3g_1 - 2g_3 - 6 &= 0.
\end{align*}
Solving this system gives $f = x^2 - 14y + 11x + 10$ and $g = xy - 7x^2 + 15y - 11$.
We then reduce $g$ modulo $f$ to get
\[ I_{\bar{\bar D}} = \pid{x^2 - 14y + 11x + 10, ~xy + 10y + 15x - 3}. \]
These two polynomials alone do not form a Gr\"obner basis for $I_{\bar{\bar D}}$.
For that we need the third polynomial
\begin{align*}
  h &= \frac {(y + 15)f - (x + 1)g} {-14} \\
    &= y^2 + 15y + 9x + 9.
\end{align*}
Note that $y + 15$, $x + 1$ and $14$ are the solutions $r$, $s$, and $t$, respectively,
from Lemma \ref{lem_fgh_is_fg_31}.

%%%%%%%%%%%%%%%%%%%%%%%%%%
%%%%%                %%%%%
%%%%%   C34 Curves   %%%%%
%%%%%                %%%%%
%%%%%%%%%%%%%%%%%%%%%%%%%%

\section{$C_{3,4}$ Curves}
\label{chap_curves}



%%%%%%%%%%%%%%%%%%%%%%%%%%%%%%%%%%%%%%
%%%%%                            %%%%%
%%%%%   Algebraic Plane Curves   %%%%%
%%%%%                            %%%%%
%%%%%%%%%%%%%%%%%%%%%%%%%%%%%%%%%%%%%%

\subsection{Algebraic Plane Curves}

Let $K$ be a perfect field.
\note{Does this need to be perfect? Also, perfect has not been defined in this thesis.}
An \defn{affine algebraic plane curve} over $K$ is the zero set of a polynomial $f \in K[x,y]$.
A \defn{projective algebraic plane curve} over $K$ is the zero set of a homogeneous polynomial $f \in K[x,y,z]$.



%%%%%%%%%%%%%%%%%%%%%%%%%%
%%%%%                %%%%%
%%%%%   Cab Curves   %%%%%
%%%%%                %%%%%
%%%%%%%%%%%%%%%%%%%%%%%%%%

\subsection{$C_{a,b}$ Curves}

A $C_{3,4}$ curve is a special case of a broader class of $C_{a,b}$ curves.

\begin{definition}
  A \defn{$C_{a,b}$ curve} $C$ over a (perfect) field $K$
  is an algebraic projective plane curve
  given by an affine equation
  \[ C(x,y) = \sum_{\substack{0 \leq i \leq b \\ 0 \leq j \leq a \\ ai + bj \leq ab }}c_{i,j}x^iy^j = 0 \]
  where
  \begin{enumerate}[label=(\roman*)]
    \item $c_{b,0}$ and $c_{0,a}$ are non-zero;
    \item $C(x,y)$ is irreducible over $\bar K$;
    \item $C$ has no singular points, except possibly at infinity.
  \end{enumerate}
\end{definition}



%%%%%%%%%%%%%%%%%%%%%%%%%%
%%%%%                %%%%%
%%%%%   C34 Curves   %%%%%
%%%%%                %%%%%
%%%%%%%%%%%%%%%%%%%%%%%%%%

\subsection{$C_{3,4}$ Curves}

\begin{definition}
  A \defn{$C_{3,4}$ curve} $C$ over a (perfect) field $K$
  is an non-singular, irreducible, algebraic projective plane curve
  given by an affine equation
  \[ C(x,y) = c_{0,3}y^3 + c_{4,0}x^4 + c_{1,2}xy^2 + c_{2,1}x^2y + c_{3,0}x^3 + c_{0,2}y^2 + c_{1,1}xy + c_{2,0}x^2 + c_{0,1}y + c_{1,0}x + c_{0,0}, \]
  where $c_{0,3}$ and $c_{4,0}$ are non-zero.
\end{definition}
This is mostly consistent with the definition of the more general $C_{a,b}$ curve above,
save for the fact that we are assuming the point at infinity is also non-singular.
We show below that the point at infinity is always non-singular.

Before proceeding, we will make two simplifying assumptions on the curve equation.
We may assume that $c_{0,3} = 1$, otherwise we may divide the whole curve equation by $c_{0,3}$.
We may also assume that $c_{4,0} = 1$, otherwise we may perform the invertible change of variables $x = \frac X {c_{4,0}^{1/4}}$.
This, however, may require we work over an algebraic extension of $K$ where $c_{4,0}$ has a quartic root.
In light of these assumptions, we now assume $C$ is of the form
\begin{equation}
  \label{eq_c34}
  C(x,y) = y^3 + x^4 + c_8xy^2 + c_7x^2y + c_6x^3 + c_5y^2 + c_4xy + c_3x^2 + c_2y + c_1x + c_0.
\end{equation}

Here, the coefficients are indexed by a single number, rather than a pair, to reflect how they will be represented in code.
In a Sage implementation, part of the data of a $C_{3,4}$ curve will be a {\tt list} of the curve coefficients,
in order of the size of their respective monomials.
\note{Remark on $C_{3,4}$ order.}

In fields of sufficiently large characteristic, one may also assume certain coefficients are zero.
If $\Char K \neq 2$, then the invertible change of variables $x = X - \frac {c_6} 4, y = Y$ gives
\begin{equation}
  \label{eq_c34_char_not_2}
  C(X,Y) = Y^3 + X^4 + d_8XY^2 + d_7X^2Y + d_5Y^2 + d_4XY + d_3X^2 + d_2Y + d_1X + d_0.
\end{equation}
If $\Char K \neq 3$, then the invertible change of variables $x = X, y = Y - \frac{c_8x + c_5}{3}$ gives
\begin{equation}
  \label{eq_c34_char_not_3}
  C(X,Y) = Y^3 + X^4 + d_7X^2Y + d_6X^3 + d_4XY + d_3X^2 + d_2Y + d_1X + d_0.
\end{equation}

If the characteristic of $K$ is neither 2 nor 3, we may perform both substitutions simultaneously. Let
\[ a = \frac {27c_6 - 9c_7c_8 + 2c_8^3} {27} \]
and perform the change of variables
\begin{align*}
  x &= X - \frac a 4 \\
  y &= Y - \frac {c_8} {3} x + \frac {ax_8 - 4c_5} {12}.
\end{align*}
Then this gives $C$ in \defn{short form}
\begin{equation}
  \label{eq_c34_short}
  C(X,Y) = Y^3 + X^4 + d_7X^2Y + d_4XY + d_3X^2 + d_2Y + d_1X + d_0.
\end{equation}

\subsection{Points}

An \defn{affine point} on a $C_{3,4}$ curve is any point in $\bar K \times \bar K$
at which the polynomial defining $C$ vanishes.
A \defn{projective point} is any point in projective space at which the homogeneous polynomial of $C$ vanishes.

The projective point $(1 : 0 : 0)$ is never a point on a $C_{3,4}$ curve.
Thus any projective point $(x:y:z)$ is equivalent to one of the form $(x:y:1)$ or $(x:1:0)$.
Projective points with $z$-coordinate 1 are in bijection with affine points,
so we identify $(x,y)$ with $(x:y:1)$.
Projective points with $z$-coordinate 0 are called \defn{points at infinity}.

We now show that a $C_{3,4}$ curve $C$ has only a single point at infinity.
We homogenize the affine equation of $C$ with respect to a third variable $z$,
to get the projective curve equation of $C$,
  \[ C(x:y:z) = y^3z + x^4 + c_8xy^2z + c_7x^2yz + c_6x^3z + c_5y^2z^2 + c_4xyz^2 + c_3x^2z^2 + c_2yz^3 + c_1xz^3 + c_0z^4 = 0. \]
The points at infinity are those points with $z = 0$, which must satisfy
  \[ x^4 = 0. \]
The points at infinity are therefore those of the form $(0 : y : 0)$, which are all equivalent to $(0 : 1 : 0)$.
That is to say that a $C_{3,4}$ curve has a single point at infinity, $P_\infty = (0 : 1 : 0)$.

We may next show that this point at infinity is always non-singular.
Consider the formal partial derivatives of the projective curve equation.
\begin{align*}
  C_x(x:y:z) &= 4x^3 + c_8y^2z + 2c_7xyz + 3c_6x^2z + c_4yz^2 + 2c_3xz^2 + c_1z^3 \\
  C_y(x:y:z) &= 3y^2z + 2c_8xyz + c_7x^2z + 2c_5yz^2 + c_4xz^2 + c_2z^3 \\
  C_z(x:y:z) &= y^3 + c_8xy^2 + c_7x^2y + c_6x^3 + 2c_5y^2z + 2c_4xyz + 2c_3x^2z + 3c_2yz^2 + 3c_1xz^2 + 4c_0z^3.
\end{align*}
Evaluated at the point $P_\infty$, these are
\begin{align*}
  C_x(0:1:0) &= 0 \\
  C_y(0:1:0) &= 0 \\
  C_z(0:1:0) &= 1 .
\end{align*}
The formal partial derivative $C_z$ is always non-zero at $P_\infty$.
Therefore $C$ is always non-singular at $P_\infty$.

Let $L/K$ be an algebraic extension.
An automorphism $\sigma \in \Gal(L/K)$ may be extended to a function on points of $C$.
For a point $P = (x : y : z)$, define
\[ \sigma(P) = (\sigma(x) : \sigma(y) : \sigma(y)). \]
A point $P$ is called \defn{$L$-rational} or \defn{defined over $L$} if $P \in L \times L$.
Equivalently, a point $P \in \bar K \times \bar K$ is $L$-rational if $P = \sigma(P)$ for all $\sigma \in \Gal(\bar K/L)$.
If $C$ is defined over the base field $K$, then the $K$-rational points are simply called rational.
The point at infinity on a $C_{3,4}$ curve is $K$-rational.
Every point on $C$ is $\bar K$-rational.
The set of $L$-rational points on a curve $C$ is denoted by $C(L)$.

Let $C$ be a curve defined over $K$ and let $P$ be a point in $C(\bar K)$.
Then the \defn{orbit} of $P$ is
\[ \orb(P) = \{ \sigma(P) ~|~ \sigma \in \Gal(\bar K/K). \]



\subsection{The Coordinate Ring}

Let $C$ be a curve defined over a field $K$.
The \defn{coordinate ring} of $C$, denoted by $K[C]$, is the quotient ring
\[ K[C] = \frac {K[x,y]} {\pid{C}}. \]
It is the ring of bivariate polynomials over $K$, modulo the principal ideal generated by the curve equation.

It is well-known fact that, if $C$ is given by an irreducible polynomial, the coordinate ring is a Dedekind domain.
All non-zero ideals of $K[C]$ may be uniquely (up to order) factored into a product of prime ideals.
The coordinate ring has Krull dimension 1, meaning that every non-zero prime ideal is a maximal ideal.
At every prime ideal $\frak p$ of $K[C]$, the localization $K[C]_{\frak p}$ is a discrete valuation ring.

\note{Put this one subsection earlier?}
\begin{lemma}
  Let $f \in K[x,y]$ and $\sigma \in \Gal(\bar K/K)$.
  Let $P = (x_0, y_0)$ be a point in $\bar K \times \bar K$. Then
  \[ f(\sigma(x_0), \sigma(y_0)) = \sigma(f(x_0, y_0)). \]
\end{lemma}
\begin{proof}
  \begin{align*}
    f(\sigma(x_0), \sigma(y_0))
      &= \sum a_{i,j}\sigma(x)^i\sigma(y)^j \\
      &= \sum \sigma(a_{i,j})\sigma(x)^i\sigma(y)^j
        & \text{$\sigma$ fixes $K$} \\
      &= \sum \sigma(a_{i,j}x^iy^j)
        & \text{$\sigma$ is multiplicative} \\
      &= \sigma \left( \sum a_{i,j}x^iy^j \right)
        & \text{$\sigma$ is additive} \\
      &= \sigma(f(x_0, y_0)).
  \end{align*}
\end{proof}
\begin{corollary}
  \label{cor_orb}
  Let $f \in K[x,y]$ and $\sigma \in \Gal(\bar K/K)$.
  Let $P$ be an affine point.
  Then $f$ has a zero at $P$ if and only if $f$ has a zero at $\sigma(P)$.
\end{corollary}
\begin{proof}
  ($\implies$) Suppose $f$ has a zero at a point $P = (x_0, y_0)$,
  i.e. $f(x_0, y_0) = 0$.
  Then at $\sigma(P)$,
  \[ f(\sigma(x_0), \sigma(y_0)) = \sigma(f(x_0, y_0)) = \sigma(0) = 0. \]  

  ($\impliedby$) Suppose $f$ has a zero at $\sigma(P)$, i.e. $f(\sigma(x_0), \sigma(y_0)) = 0$.
  Then $\sigma$ has an inverse $\sigma\inv \in \Gal(\bar K/K)$ and
  \[ f(x_0, y_0) = \sigma\inv(\sigma(f(x_0, y_0))) = \sigma\inv(f(\sigma(x_0), \sigma(y_0))) = \sigma\inv(0) = 0. \] 
\end{proof}

Let $P \in C(\bar K)$.
Let $\frak p$ be the ideal
  \[ \frak p = \pid{ f \in K[C] ~|~ f(P) = 0 }. \]
\note{It can be shown that this is a prime ideal.}
Then $K[C]_{\frak p}$ is a DVR with the unique maximal ideal $\frak m_P = \frak p K[C]_{\frak p}$.
This maximal ideal is principally generated by an element $u$, called the \defn{uniformizer}
(or \defn{uniformizing parameter} or \defn{local paramater}).
Let $f \in K[C]$. The \defn{valuation} of $f$ at $P$ is
  \[ \nu_P(f) = \max_{n \in \bb Z} \{ f \in \frak m_P^n \}. \]

\begin{proposition}
  \[ \nu_P(f) = \nu_{\sigma(P)}(f). \]
\end{proposition}
\begin{proof}
  \note{Follows from Corollary \ref{cor_orb}.}
\end{proof}



\subsection{The Function Field}

The \defn{function field} $K(C)$ of $C$ is the field of fractions of the coordinate ring,
\[ K(C) = \Frac(K[C]). \]
Equivalently, it is the field of fractions of $K[x,y]$, modulo the curve equation,
\[ K(C) = \frac {\Frac(K[x,y])} {\pid{C}} = \frac {K(x,y)}{\pid C}. \]
We will not work much with the function field itself.
We will use it in Chapter \ref{chap_ideals} in defining the ideal class group,
though one of the goals of Chapter \ref{chap_ideals} is to show that we can work entirely in $K[C]$.

Given a point $P$, the valuation function $\nu_P(f)$ on polynomials extends to a valuation on rational functions in $K(C)$.
If $r = f/g$ is a rational function, then define
\[ \nu_P(r) = \nu_P(f) - \nu_P(g). \]

\begin{proposition}
  This map is well-defined.
\end{proposition}
\begin{proof}
  Suppose $\frac f g = \frac h k$. Then
  \begin{align*}
    fk &= gh \\
    \nu_P(fk) &= \nu_P(gh) \\
    \nu_P(f) + \nu_P(k) &= \nu_P(g) + \nu_P(h) \\
    \nu_P(f) - \nu_P(g) &= \nu_P(h) - \nu_P(k) \\
    \nu_P\left(\frac f g\right) &= \nu_P\left(\frac h k\right).
  \end{align*}
\end{proof}



\subsection{Unsorted Stuff}

\begin{proposition}
  Let $C$ be a $C_{3,4}$ curve.
  The line $x = 0$ intersects $C$ at 3 affine places, counting multiplicity.
  The line $y = 0$ intersects $C$ at 4 affine places, counting multiplicity.
  The polynomial $x$ has a pole of order 3 at $P_\infty$.
  The polynomial $y$ has a pole of order 4 at $P_\infty$.
\end{proposition}
\begin{proof}
  When $x = 0$, we have $C(x,y) = C(0,y) = y^3 + c_5y^2 + c_2y + c_0 = 0$,
  which factors over $\bar K$ as $(y - y_1)(y - y_2)(y - y_3) = 0$
  for some field elements $y_1, y_2, y_3$, not necessarily distinct.
  Thus line $x = 0$ intersect $C$ at the points $(0 : y_i : 1)$, $i = 1, 2, 3$.
  
  The result for the line $y = 0$ also holds, \emph{mutatis mutandis}.
  
  \note{Finish this proof}
\end{proof}

Some authors \note{(which?)} characterize $C_{3,4}$ by this proposition.
They define a $C_{3,4}$ curve as a \note{(non-singular?)} algebraic plane curve
with a rational point $P$
for which there exist two functions $x$ and $y$ whose pole orders at $P$ are 3 and 4, respectively.
It can then be shown that the curve is defined by an equation of the form of Equation \ref{eq_c34}.

\begin{proposition}
  Let $P$ be an affine point on a curve $C$ and let
  \[ \frak p = \{ f \in K[C] ~|~ f(P) = 0 \}. \]
  Then $\frak p$ is a non-zero prime ideal of $K[C]$.
\end{proposition}
\begin{proof}
  \begin{description}
    \item[$\frak p$ is non-zero:]
      Let $P = (x_0, y_0)$.
      Then $x_0$ is an element of some finite algebraic extension $L$ of $K$.
      Let $m(x)$ be the minimal polynomial of $x_0$.
      Then $m(x) \in K[x]$ is univariate, non-zero, and may be viewed instead as $m(x,y) \in K[x,y]$.
      Then $m(x_0, y_0) = m(x_0) = 0$, hence $m(x,y) \in \frak p$.
    \item[$\frak p$ is prime:]
      Suppose $fg \in \frak p$. Then
      \begin{align*}
        & (fg)(P) = 0 \\
        \implies & f(P)g(P) = 0 \\
        \implies & f(P) = 0 \text{ or } g(P) = 0 \\
        \implies & f \in \frak p \text{ or } g \in \frak p.
      \end{align*}
  \end{description}
\end{proof}

\begin{proposition}
  Let $P$ be an affine point on a curve $C$ and let
  \[ \cal O_P = \{ \frac f g \in K(C)^* ~|~ g(P) \neq 0 \} \]
  be the \defn{ring of regular functions at $P$}.
  \note{This differs from the usual definition where functions live in $\bar K(C)^*$}.
  Then $\cal O_P$ is a discrete valuation ring.
\end{proposition}
\begin{proof}
  Let $\frak p$ be the prime ideal of polynomials zero at the point $P$.
  Then $\cal O_P$ is the localization of $K[C]$ at $\frak p$.
  The localization of a Dedekind domain at a non-zero prime ideal is a DVR.
\end{proof}

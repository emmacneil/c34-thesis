%%%%%%%%%%%%%%%%%%%%%%%%%%
%%%%%                %%%%%
%%%%%   C34 Curves   %%%%%
%%%%%                %%%%%
%%%%%%%%%%%%%%%%%%%%%%%%%%

\section{$C_{3,4}$ Curves}
\label{chap_curves}

In this chapter, we define the central object of this thesis, $C_{3,4}$ curves.
We begin by describing curves more generally, as well as objects related to curves,
such as their coordinate rings, function fields, and discrete valuations.
In the final section of this chapter, we will define a family of curves called $C_{a,b}$ curves,
of which $C_{3,4}$ curves are a special case.

All fields will be assumed to be perfect.
A field $K$ is called \defn{perfect} if every $K$-irreducible polynomial in $K[x]$
has distinct roots in $\bar K$.
There are many other characterizations of perfect fields (see \cite{hungerford}),
but this is the definition that will best suit our needs in chapters to come.
Every algebraically closed field,
every field of characteristic 0 (e.g. $\bb Q$, $\bb C$)
and every finite field (e.g. $\bb F_q$) is perfect.



%%%%%%%%%%%%%%%%%%%%%%%%%%%%%%%%%%%%%%
%%%%%                            %%%%%
%%%%%   Algebraic Plane Curves   %%%%%
%%%%%                            %%%%%
%%%%%%%%%%%%%%%%%%%%%%%%%%%%%%%%%%%%%%

\subsection{Algebraic Plane Curves}
\label{sec_plane_curves}

Let $K$ be a field.
The \defn{affine plane over $K$}, denoted by $\bb A_K^2$, is the set
\[ \bb A_K^2 = K^2. \]
If $L/K$ is an algebraic extension, then $\bb A_K^2 \subseteq \bb A_L^2$.
The \defn{projective plane over $K$}, denoted by $\bb P_K^2$, is the set of lines in $K^3$ through the origin.
This may be constructed as the set of points in $K^3$ other than the origin,
modulo an equivalence relation whereby two points are equivalent
if and only if they are colinear with the origin. That is,
\[ \bb P_K^2 = (K^3 - \{(0,0,0)\})/\sim \]
where
\begin{equation}
  \label{eq_projective_point_relation}
  (x_1, y_1, z_1) \sim (x_2, y_2, z_2) \iff \exists k \in \bar K : (x_1, y_1, z_1) = (kx_2, ky_2, kz_2).
\end{equation}
The equivalence class of a point $(x, y, z)$ is denoted by $(x : y : z)$.
If $L/K$ is an algebraic extension, then $\bb P_K^2 \subseteq \bb P_L^2$.

There is a bijection between $\bb A_L^2$ and the points in $\bb P_L^2$ whose third coordinates are non-zero.
\[ \phi : \bb A_L^2 \to \bb P_L^2 - \{ (x:y:0) ~|~ x, y \in L \} \]
\begin{align*}
  \phi(x, y) &= (x : y : 1) \\
  \phi\inv(x : y : z) &= \left( \frac x z, \frac y z \right)
\end{align*}
It is straightforward to show that $\phi\inv$ is well-defined.

Let $f(x,y) \in K[x,y]$ be a polynomial.
The \defn{homogenization} of $f$ is the homogeneous polynomial
\[ F(X,Y,Z) = Z^{\deg f} f\left( \frac X Z, \frac Y Z \right) \in K[X,Y,Z]. \]
The homogenization of $0 \in K[x,y]$ is $0 \in K[X,Y,Z]$.
For any homogeneous polynomial $F(X,Y,Z) \in K[X,Y,Z]$,
the \defn{dehomogenization} of $F$ is the polynomial
\[ f(x,y) = F(x, y, 1). \]
The polynomial $f$ might no longer be homogeneous.
The homogenization and dehomogenization operations are mutual inverses.

A \defn{projective algebraic plane curve} over $K$ is a set of points
\[ C_F : \{ (x_0 : y_0 : z_0) \in \bb P_{\bar K}^2 ~|~ F(x_0, y_0, z_0) = 0 \}, \]
for some homogeneous polynomial $F \in K[X, Y, Z]$.
It is the set of points in $\bb P_{\bar K}^2$ at which $F$ is zero.
Notice that this includes points in the algebraic closure of $K$.
The \defn{affine model} of $C_F$ is
\[ C_f : \{ (x, y) ~|~ f(x,y) = 0 \} \cup C_\infty, \]
the set of points at which the dehomogenization $f$ of $F$ is zero,
together with the set $C_\infty$ of \defn{points at infinity}.
This set is $C_\infty = \{ (x:y:0) | F(X,Y,0) = 0 \}$.
These are precisely the points that do not fall under the domain of $\phi\inv$.
The points in $C_f$ are in bijection with the points in $C_F$.
An \defn{affine algebraic plane curve} $C_f$ over $K$
is the affine model of a projective algebraic plane curve $C_F$.
The \defn{projective closure} of $C_f$ is $C_F$.

Because affine and projective algebraic plane curves are so closely related,
essentially two representations of the same object,
we shall refer to both simply as \defn{curves}.
We will define curves by their affine model, i.e. by a polynomial $f \in K[x,y]$.
When the defining polynomial is clear in context, we shall omit the subscript and write $C$ rather than $C_f$.

Although we will define curves by their affine model,
we will usually denote points on the curve by their projective coordinates, in the form $(x:y:z)$.
By the equivalence relation on projective points,
every point in $C$ can be written uniquely in one of the three reduced forms $(x:y:1)$, $(x:1:0)$ or $(1:0:0)$.
Points of the form $(x:y:1)$ are \defn{finite points},
while all other points with $z$-coordinate 0 are \defn{points at infinity}.

If $L \supseteq K$ is an algebraic extension, 
then the set $C(L)$ of $L$-rational points on $C$ is
\[ C(L) = C \cap \bb P_L^2. \]
These are the points on $C$ that are equivalent (via the relation in \ref{eq_projective_point_relation})
to a point with coordinates all in $L$.
Equivalently, these are the points on $C$ whose representations in reduced form have coordinates in $L$.
If $C$ is defined over $K$, then the $K$-rational points are simply called \defn{rational}.

A curve $C = C_f$ is \defn{irreducible} if $f$ is $\bar K$-irreducible,
i.e. if $f$ cannot be written as a product $f = gh$ of lower-degree polynomials $g, h \in \bar K[x,y]$.
If $P$ is a point on $C$,
then $P$ is called \defn{singular} if all formal partial derivatives of the homogenization $F$ of $f$ vanish at $P$.
In this case, the tangent line to $C$ at $P$ does not exist.\footnote{
In this case, one might be interested in the Zariski tangent space instead.}
The curve $C$ is called \defn{singular} if it has at least one singular point.
Otherwise $C$ is called \defn{non-singular} or \defn{smooth}.
Some authors require that algebraic curves be irreducible and sometimes smooth.
Our definition of $C_{3,4}$ curves below will require these conditions.

Let $\sigma \in \Gal(\bar K/K)$ be an automorphism on $\bar K$ that fixes $K$.
Then $\sigma$ also acts on $\bb A_{\bar K}^2$ and $\bb P_{\bar K}^2$ via
\begin{align}
  \label{eq_galois_action_on_point}
  \sigma((x_0, y_0)) &= (\sigma(x_0), \sigma(y_0)) \\
  \sigma((X_0 : Y_0 : Z_0)) &= (\sigma(X_0) : \sigma(Y_0) : \sigma(Z_0)). \nonumber
\end{align}
It is easily verified that the action on $\bb P_{\bar K}^2$ is well-defined.
If $P \in \bb A_{\bar K}^2$ or $P \in \bb P_{\bar K}^2$,
then define the \defn{orbit} of $P$ to be
\[ \orb(P) := \{ \sigma(P) ~|~ \sigma \in \Gal(\bar K/K) \}. \]

\begin{lemma}
  \label{lem_galois_action_on_polynomial}
  Let $f \in K[x,y]$ and $\sigma \in \Gal(\bar K/K)$.
  Let $P = (x_0, y_0)$ be a point in $\bar K \times \bar K$. Then
  \[ f(\sigma(x_0), \sigma(y_0)) = \sigma(f(x_0, y_0)). \]
\end{lemma}
\begin{proof}
  \begin{align*}
    f(\sigma(x_0), \sigma(y_0))
      &= \sum a_{i,j}\sigma(x)^i\sigma(y)^j \\
      &= \sum \sigma(a_{i,j})\sigma(x)^i\sigma(y)^j
        & \text{$\sigma$ fixes $K$} \\
      &= \sum \sigma(a_{i,j}x^iy^j)
        & \text{$\sigma$ is multiplicative} \\
      &= \sigma \left( \sum a_{i,j}x^iy^j \right)
        & \text{$\sigma$ is additive} \\
      &= \sigma(f(x_0, y_0)).
  \end{align*}
\end{proof}
\begin{corollary}
  \label{cor_orb}
  Let $f \in K[x,y]$.
  Then $f$ is zero at $P$ if and only if $f$ is zero at every point in $\orb P$.
\end{corollary}
\begin{comment}
\begin{corollary}
  \label{cor_orb}
  Let $f \in K[x,y]$ and $\sigma \in \Gal(\bar K/K)$.
  Let $P$ be an affine point.
  Then $f$ has a zero at $P$ if and only if $f$ has a zero at $\sigma(P)$.
\end{corollary}
\begin{proof}
  ($\implies$) Suppose $f$ has a zero at a point $P = (x_0, y_0)$,
  i.e. $f(x_0, y_0) = 0$.
  Then at $\sigma(P)$,
  \[ f(\sigma(x_0), \sigma(y_0)) = \sigma(f(x_0, y_0)) = \sigma(0) = 0. \]  

  ($\impliedby$) Suppose $f$ has a zero at $\sigma(P)$, i.e. $f(\sigma(x_0), \sigma(y_0)) = 0$.
  Then $\sigma$ has an inverse $\sigma\inv \in \Gal(\bar K/K)$ and
  \[ f(x_0, y_0) = \sigma\inv(\sigma(f(x_0, y_0))) = \sigma\inv(f(\sigma(x_0), \sigma(y_0))) = \sigma\inv(0) = 0. \] 
\end{proof}
\end{comment}

%%%%%%%%%%%%%%%%%%%%%%%%%%%%%%%
%%%%%                     %%%%%
%%%%%   Coordinate Ring   %%%%%
%%%%%                     %%%%%
%%%%%%%%%%%%%%%%%%%%%%%%%%%%%%%

\subsection{The Coordinate Ring and Function Field}

Let $C$ be a curve defined over a field $K$.
The \defn{coordinate ring} of $C$, denoted by $K[C]$, is the quotient ring
\[ K[C] := \frac {K[x,y]} {\pid{C}}. \]
It is the ring of bivariate polynomials over $K$,
modulo the principal ideal generated by the curve's defining polynomial.

It is a well-known fact in algebraic geometry that,
if $C$ is given by an irreducible polynomial,
its coordinate ring is a Dedekind domain (see \S 8.2 of \cite{galbraith12} or Proposition 8.1 of \cite{neukirch99}).
Therefore all non-zero ideals of $K[C]$ may be uniquely factored into a product of prime ideals.
The coordinate ring has Krull dimension 1, meaning that every non-zero prime ideal is a maximal ideal
(Theorem VIII.6.5 in \cite{hungerford}).



The \defn{function field} $K(C)$ of $C$ is the field of fractions of the coordinate ring,
\[ K(C) := \Frac(K[C]). \]
We will not work much with the function field itself.
We will use it in Chapter \ref{chap_ideals} in defining the ideal class group,
though one of the goals of Chapter \ref{chap_ideals} is to show that we can work entirely in $K[C]$.



%%%%%%%%%%%%%%%%%%%%%%%%%%%
%%%%%                 %%%%%
%%%%%   Local Rings   %%%%%
%%%%%                 %%%%%
%%%%%%%%%%%%%%%%%%%%%%%%%%%

\subsection{Local Rings and Valuations}
\label{sec_local_rings}

Let $K[C]$ be the coordinate ring of an irreducible smooth curve $C$.
Let $\frak p$ be a non-zero prime ideal of $K[C]$.
We may localize $K[C]$ at the prime ideal $\frak p$ to get $K[C]_{\frak p}$,
the \defn{ring of regular functions at $\frak p$},
which is usually denoted by $\cal O_\frak p$ instead.
This can be defined more explicitly by
\[ \cal O_{\frak p} = \left\{ \frac f g \in K(C) ~|~ g \not \in \frak p \right\}. \]
This ring is a local ring, meaning that it has a unique maximal ideal, denoted by $\frak m_{\frak p}$.
Specifically, $\frak m_{\frak p}$ is 
\[ \frak m_{\frak p} = \frak p \cal O_{\frak p} =
   \left\{ \frac f g \in K(C) ~|~ f \in \frak p, ~g \not \in \frak p \right\}. \]

Now let $P$ be a finite point on $C$. Then $P$ induces a prime ideal
$\frak p = \{ f \in K[C] ~|~ f(P) = 0 \}$.
\begin{comment}
\begin{proposition}
  Let $P$ be an affine point on a curve $C$ and let
  \[ \frak p = \{ f \in K[C] ~|~ f(P) = 0 \}. \]
  Then $\frak p$ is a non-zero prime ideal of $K[C]$.
\end{proposition}
\begin{proof}
  \begin{description}
    \item[$\frak p$ is non-zero:]
      Let $P = (x_0, y_0)$.
      Then $x_0$ is an element of some finite algebraic extension $L$ of $K$.
      Let $m(x)$ be the minimal polynomial of $x_0$.
      Then $m(x) \in K[x]$ is univariate, non-zero, and may be viewed instead as $m(x,y) \in K[x,y]$.
      Then $m(x_0, y_0) = m(x_0) = 0$, hence $m(x,y) \in \frak p$.
    \item[$\frak p$ is prime:]
      Suppose $fg \in \frak p$. Then
      \begin{align*}
        & (fg)(P) = 0 \\
        \implies & f(P)g(P) = 0 \\
        \implies & f(P) = 0 \text{ or } g(P) = 0 \\
        \implies & f \in \frak p \text{ or } g \in \frak p.
      \end{align*}
  \end{description}
\end{proof}
\end{comment}
Define the \defn{ring of regular functions at $P$} to be the ring $\cal O_P := \cal O_{\frak p}$,
where $\frak p$ is the prime ideal induced by $P$.
Its unique maximal ideal is $\frak m_P := \frak m_{\frak p}$. Explicitly,
\begin{align*}
  \cal O_P &= \left\{ \frac f g \in K(C) ~|~ g(P) \neq 0 \right\} \\
  \frak m_P &= \left\{ \frac f g \in K(C) ~|~ f(P) = 0, ~g(P) \neq 0 \right\}.
\end{align*}

Given a non-zero prime ideal $\frak p$, $\cal O_{\frak p}$ is not only a local ring,
but also a discrete valuation ring, or DVR (Theorem VIII.6.10 in \cite{hungerford}).
For more on DVRs, see \cite{eisenbud95}.
Rather than define here what discrete valuations and DVRs are in general,
we briefly describe the valuation of a function at a prime ideal.
Let $f \in K[C]$.
The \defn{order} or \defn{valuation} of $f$ at $\frak p$, denoted $\nu_{\frak p}(f)$, is
\[ \nu_{\frak p}(f) := \max\{ r \in \bb N ~|~ f \in \frak m_P^r \}, \]
if $f$ is not identically 0.
The order of 0 at $\frak p$ is defined as $\nu_{\frak p}(0) = \infty$.
Given two polynomials $f, g \in K[C]$, $\nu_{\frak p}$ satisfies the relation
\[ \nu_{\frak p}(fg) = \nu_{\frak p}(f) + \nu_{\frak p}(g), \]
as long as we define $\infty + \infty = \infty$ and $\infty + n = \infty$ for all $n \in \bb Z$.
We can extend this to polynomials in $K(C)$.
Let $f/g \in K(C)$ and define
\[ \nu_{\frak p}\left(\frac f g\right) = \nu_{\frak p}(f) - \nu_{\frak p}(g). \]
This map is well-defined,
for if $\frac f g = \frac h k$, then
\begin{align*}
  fk &= gh \\
  \nu_{\frak p}(fk) &= \nu_P(gh) \\
  \nu_{\frak p}(f) + \nu_P(k) &= \nu_{\frak p}(g) + \nu_{\frak p}(h) \\
  \nu_{\frak p}(f) - \nu_P(g) &= \nu_{\frak p}(h) - \nu_{\frak p}(k) \\
  \nu_{\frak p}\left(\frac f g\right) &= \nu_{\frak p}\left(\frac h k\right).
\end{align*}

Analogously for a finite point $P$ on a curve,
we may define the order of $f/g \in K(C)$ at $P$
by letting $\frak p$ be the prime induced by $P$ and invoking the definition above,
i.e. $\nu_P := \nu_{\frak p}$, where $\frak p$ is the prime ideal induced by $P$.

There are various ways in which we might go about defining the valuation at a point at infinity of a curve.
We will defer defining the valuation there until the end of the chapter.
When so doing, we will make use of the following:
\begin{proposition}
  \label{prop_valuation_min}
  Let $\cal O_{\frak p}$ be a discrete valuation ring with valuation $\nu_{\frak p}$.
  For all $f, g \in K(C)$,
  \[ \nu_{\frak p}(f + g) \geq \min\{ \nu_{\frak p}(f) + \nu_{\frak p}(g) \}. \]
  If $\nu_{\frak p}(f) \neq \nu_{\frak p}(g)$, then
  \[ \nu_{\frak p}(f + g) = \min\{ \nu_{\frak p}(f) + \nu_{\frak p}(g) \}. \]
\end{proposition}
\begin{proof}
  Lemma 1.1.2 in \cite{goldschmidt03}.
\end{proof}

It is an important fact that the valuation of a function $f \in K(C)$ at a point $P$
agrees with the notion of the order of the zero or pole of $f$ at $P$.
If $\nu_P(f) = n < 0$, we say that $f$ has a pole of order $n$ at $P$.
If $\nu_P(f) = n > 0$, we say that $f$ has a zero of order $n$ at $P$.
The function $f$ passes through $P$ if and only if $\nu_P(f) \geq 1$,
and is tangent to $C$ at $P$ if and only if $\nu_P(f) \geq 2$.
\begin{proposition}
  Let $\cal O_{\frak p}$ be a discrete valuation ring with valuation $\nu_{\frak p}$.
  \begin{enumerate}[label=(\roman*)]
    \item
      The maximal ideal of $\cal O_{\frak p}$ is a principal ideal,
      $\frak m_{\frak p} = \pid u$ for some $u \in \cal O_{\frak p}$.
    \item
      For any non-zero $f \in K(C)$,
      $\nu_{\frak p}(f) = n$ if and only if $f = su^n$ for some $s \in \cal O_{\frak p}^*$.
  \end{enumerate}
\end{proposition}
\begin{proof}
  Lemmas 7.3.1 and 7.4.7 in \cite{galbraith12}.
\end{proof}
The generator $u$ of $\frak m_{\frak p}$ is called a \defn{uniformizer} or \defn{local parameter} at $\frak p$.
Uniformizers are not unique.
Any element $u \in \frak m_{\frak p} - \frak m_{\frak p}^2$,
i.e. any element $u$ for which $\nu_{\frak p}(i) = 1$ is a uniformizer.
For a point $P$, this means any function $u$ that passes through but is not tangent to $C$ at $P$ is a uniformizer.

A consequence of Corollary \ref{cor_orb} from the previous section
is that if $\sigma \in \Gal(\bar K/K)$ and $P$ is an affine point on $C$,
then $P$ and $\sigma(P)$ induce the same prime ideal $\frak p$. Thus we have
\begin{proposition}
  \label{prop_valuation_on_orbit}
  Let $P$ be an affine point on $C$ and let $\frak p = \{ f \in K[C] ~|~ f(P) = 0 \}$.
  Then for all $\sigma \in \Gal(\bar K/K)$ and $f \in K(C)$,
  \[ \nu_P(f) = \nu_{\sigma(P)}(f) = \nu_{\frak p}(f). \]
\end{proposition}



%%%%%%%%%%%%%%%%%%%%%%%%%%
%%%%%                %%%%%
%%%%%   C34 Curves   %%%%%
%%%%%                %%%%%
%%%%%%%%%%%%%%%%%%%%%%%%%%

\subsection{$C_{3,4}$ Curves}
\label{sec_c34_curves}

A $C_{3,4}$ curve is a special case of a broader class of $C_{a,b}$ curves.
The class of $C_{3,4}$ curves was first described by Miura \cite{miura97}.
One definition is the following (an equivalent characterization will come at the end of this chapter).

\begin{definition}
  \label{def_cab_curve}
  A \defn{$C_{a,b}$ curve} over a field $K$
  is an algebraic projective plane curve $C$ over $K$
  that is non-singular everywhere except possibly at its points\footnote{
Although we will see shortly that there is only one point at infinity.}
  at infinity, given by a polynomial $F \in K[x,y]$ of the form
  \begin{equation}
    \label{eq_cab}
    F = \sum_{\substack{0 \leq i \leq b \\ 0 \leq j \leq a \\ ai + bj \leq ab }}c_{i,j}x^iy^j
  \end{equation}
  where $0 < a < b$ are coprime and $c_{b,0}$ and $c_{0,a}$ non-zero.
\end{definition}
Here is a useful visualization of Equation \ref{eq_cab}.
The monomials in $C$'s defining polynomial $F$ correspond to integer points
in or on the boundary of the triangle\footnote{
This is the Newton polygon of $F$.}
with corners $(0, 0)$, $(0, a)$, and $(b, 0)$.

\begin{proposition}
  A $C_{a,b}$ curve only has one point at infinity.
\end{proposition}
\begin{proof}
  It is easy to see $F$ has degree $b$
  and that $c_{b,0}x^b$ is the only term in $F$ of degree $b$.
  Let $F^*(X,Y,Z)$ be the homogenization of $F(x,y)$ and evaluate $F^*$ at $Z = 0$.
  We get $F^*(X,Y,0) = c_{b,0}X^b$.
  
  Now suppose $(u:v:0)$ is a point at infinity on the curve.
  Then $F^*(u,v,0) = c_{b,0}u^b = 0$, so $u = 0$ and $(u : v : 0) = (0 : v : 0) = (0 : 1 : 0)$.
\end{proof}
We will denote the unique point at infinity on a $C_{a,b}$ curve by $P_{\infty}$.

A few special cases of $C_{1,b}$ curves are worth mentioning.
When $a = 2$ and $b = 3$, we get an elliptic curve.
When $a = 2$ and $b = 7$, we get a genus 2 ramified hyperelliptic curve.
See Equations \ref{eq_elliptic} and \ref{eq_genus_3_hyperelliptic}.
More generally, for $g \geq 2$, when $a = 2$ and $b = 2g + 1$, we get a genus $g$ ramified hyperelliptic curve.
More importantly in this thesis,
when $a = 3$ and $b = 4$, we get a $C_{3,4}$ curve.

\begin{definition}
  \label{def_c34_curve}
  A \defn{$C_{3,4}$ curve} over a field $K$
  is a smooth projective algebraic plane curve
  given by an affine equation\footnote{
  The subscripts on the coefficients of $C$ are numbered according to the $C_{3,4}$ monomial order,
  Definition \ref{def_cab_order}.}
  \[ C = c_{10}y^3 + c_9x^4 + c_8xy^2 + c_7x^2y + c_6x^3 + c_5y^2 + c_4xy + c_3x^2 + c_2y + c_1x + c_0, \]
  where $c_{9}$ and $c_{10}$ are non-zero.
\end{definition}

Definition \ref{def_c34_curve} may appear to be slightly more restrictive than Definition \ref{def_cab_curve} ---
Definition \ref{def_c34_curve} does not allow for the point at infinity to be singular.
In fact, the point $P_\infty$ on a $C_{3,4}$ is never singular.

\begin{proposition}
  Let $C$ be a $C_{3,4}$ curve over a field $K$.
  The point $P_\infty$ is non-singular.
\end{proposition}
\begin{proof}
  To show that $C$ is non-singular at $P_\infty$,
  we show that one of the formal partial derivatives of $C$ is non-zero at $P_\infty$.
  Since $P_\infty$ is not a finite point,
  this requires that we work with the homogenization of $C$'s defining polynomial,
  \begin{align*}
    \bar C &= c_{10}Y^3Z + c_9X^4 + c_8XY^2Z + c_7X^2YZ + c_6X^3Z + c_5Y^2Z^2 \\ 
           &+ c_4XYZ^2 + c_3X^2Z^2 + c_2YZ^3 + c_1XZ^3 + c_0Z^4.
  \end{align*}
  The formal partial derivative of $\bar C$ with respect to $Z$ is
  \begin{align*}
    \bar C_Z &= c_{10}Y^3 + c_8XY^2 + c_7X^2Y + c_6X^3 + 2c_5Y^2Z \\
             &+ 2c_4XYZ + 2c_3X^2Z + 3c_2YZ^2 + 3c_1XZ^2 + 4c_0Z^3.
  \end{align*}
  Evaluated at $P_\infty$, $\bar C_Z(0 : 1 : 0) = c_{10} \neq 0$.
\end{proof}

We will make some simplifying assumptions on the curve equation.
Since $c_{10}$ is non-zero, we may assume that $c_{10} = 1$,
as multiplying the defining polynomial by $c_{10}\inv$ does not change the set on which $F$ vanishes.
We may also assume that $c_9 = 1$,
otherwise we may perform the invertible change of variables $x \mapsto x / \sqrt[4]{c_{9}}$.
This, however, may require we work over an algebraic extension of $K$ where $c_{4,0}$ has a quartic root.
In light of these assumptions, we now assume $C$ is of the \defn{long form}
\begin{equation}
  \label{eq_c34}
  C = y^3 + x^4 + c_8xy^2 + c_7x^2y + c_6x^3 + c_5y^2 + c_4xy + c_3x^2 + c_2y + c_1x + c_0.
\end{equation}

In fields of sufficiently large characteristic, one may also assume certain coefficients are zero.
If $\Char K \neq 2$, then the invertible change of variables $x = X - \frac {c_6} 4, y = Y$ gives
\begin{equation}
  \label{eq_c34_char_not_2}
  C(X,Y) = Y^3 + X^4 + d_8XY^2 + d_7X^2Y + d_5Y^2 + d_4XY + d_3X^2 + d_2Y + d_1X + d_0,
\end{equation}
for some new coefficients $d_i$.
Notice that there is no $X^3$ term.
If $\Char K \neq 3$, then the invertible change of variables $x = X, y = Y - \frac{c_8X + c_5}{3}$ gives
\begin{equation}
  \label{eq_c34_char_not_3}
  C(X,Y) = Y^3 + X^4 + d_7X^2Y + d_6X^3 + d_4XY + d_3X^2 + d_2Y + d_1X + d_0,
\end{equation}
where there is no $Y^2$ or $XY^2$ term.
If the characteristic of $K$ is neither 2 nor 3, we may perform both substitutions simultaneously. Let
\[ a = \frac {27c_6 - 9c_7c_8 + 2c_8^3} {27} \]
and perform the change of variables
\begin{align*}
  x &= X - \frac a 4 \\
  y &= Y - \frac {c_8} {3} X + \frac {ac_8 - 4c_5} {12}.
\end{align*}
Then this gives $C$ in \defn{short form}
\begin{equation}
  \label{eq_c34_short}
  C(X,Y) = Y^3 + X^4 + d_7X^2Y + d_4XY + d_3X^2 + d_2Y + d_1X + d_0.
\end{equation}

We now address the valuation at $P_\infty$ on a $C_{3,4}$ curve.
We define $\nu_{P_\infty}$ in terms of a uniformizer.
Following \S 7.3 of \cite{galbraith12}, a uniformizer of $\frak m_{P_\infty}$ is $\frac x y$.
Thus $\nu_{P_\infty}\left(\frac x y\right) = 1$.
From this, we can deduce the pole orders of $x$ and $y$ at $P_\infty$.
\begin{proposition}
  Let $P_\infty$ be the point at infinity of a $C_{3,4}$ curve $C$ over $K$.
  The pole orders of $x,y\in K[C]$ are
  \[ \nu_{P_\infty}(x) = -3 ~\text{ and }~ \nu_{P_\infty}(y) = -4. \]
\end{proposition}
\begin{proof}
  First, we note that $\nu_{P_\infty}\left(\frac x y\right) = 1$ implies $\nu_{P_\infty}(x) = \nu_{P_\infty}(y) + 1$.
  Next,
  \begin{align*}
    3\nu_{P_\infty}(y)
      &= \nu_{P_\infty}(y^3) \\
      &= \nu_{P_\infty}(-x^4 - c_8xy^2 - c_7x^2y - \ldots - c_0) \\
      &= \min\{\nu_{P_\infty}(-x^4), \nu_{P_\infty}(- c_8xy^2), \nu_{P_\infty}(-c_7x^2y), \ldots, \nu_{P_\infty}(-c_0)\}
        & \text{Prop. \ref{prop_valuation_min}} \\
      &= \min\{\nu_{P_\infty}(x^4), \nu_{P_\infty}(xy^2), \nu_{P_\infty}(x^2y), \ldots, \nu_{P_\infty}(1)\} \\
      &= \min\{\nu_{P_\infty}(x^4), \nu_{P_\infty}(xy^2), \nu_{P_\infty}(x^2y)\}
  \end{align*}
  The minimum of these three is $\nu_{P_\infty}(x^4)$, for assuming otherwise leads to a contradiction.
  For example, if $3\nu_{P_\infty}(y) = \nu_{P_\infty}(xy^2)$, then $\nu_{P_\infty}(y) = \nu_{P_\infty}(x)$,
  which contradicts $\nu_{P_\infty}(x) = \nu_{P_\infty}(y) + 1$. So
  \[ 3\nu_{P_\infty}(y) = \nu_{P_\infty}(x^4) = 4\nu_{P_\infty}(x)
     = 4(\nu_{P_\infty}(y) + 1), \]
  which gives $\nu_{P_\infty}(y) = -4$, and
  \[ \nu_{P_\infty}(x) = \nu_{P_\infty}(y) + 1 = -4 + 1 = -3. \]
\end{proof}

In general, for a $C_{a,b}$ curve, $\nu_{P_\infty}(x) = -a$ and $\nu_{P_\infty}(y) = -b$.
In fact, this is another characterization of $C_{a,b}$ curves.
\begin{theorem}
  Let $C$ be an affine plane curve over $K$ (not necessarily smooth or irreducible).
  Let $0 < a < b$ be coprime positive integers.
  The following are equivalent.
  \begin{enumerate}[label=(\roman*)]
    \item $C$ is a $C_{a,b}$ curve.
    \item $C$ is irreducible, has exactly one rational point $P_\infty$ at infinity,
          $\nu_{P_\infty}(x) = -a$, and
          $\nu_{P_\infty}(y) = -b$.
  \end{enumerate}
\end{theorem}
\begin{proof}
  Originally proved by Miura in \cite{miura97}.
  An English translation of the proof is provided by Matsumoto in \cite{matsumoto98}.
\end{proof}




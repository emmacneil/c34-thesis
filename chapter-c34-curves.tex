%%%%%%%%%%%%%%%%%%%%%%%%%%
%%%%%                %%%%%
%%%%%   C34 Curves   %%%%%
%%%%%                %%%%%
%%%%%%%%%%%%%%%%%%%%%%%%%%

\section{$C_{3,4}$ Curves}
\label{chap_curves}

In this chapter, we define the central object of this thesis, $C_{3,4}$ curves.
We begin by describing curves more generally, as well as objects related to curves,
such as their coordinate rings, function fields, and discrete valuations.
In the final section of this chapter, we will define a family of curves called $C_{a,b}$ curves,
of which $C_{3,4}$ curves are a special case.

All fields will be assumed to be perfect.
A field $K$ is called \defn{perfect} if every $K$-irreducible polynomial in $K[x]$
has distinct roots in $\bar K$.
There are many other characterizations of perfect fields (see \cite{hungerford}),
but this is the definition that will best suit our needs in chapters to come.
Every algebraically closed field,
every field of characteristic 0 (e.g. $\bb Q$, $\bb C$)
and every finite field (e.g. $\bb F_q$) is perfect.



%%%%%%%%%%%%%%%%%%%%%%%%%%%%%%%%%%%%%%
%%%%%                            %%%%%
%%%%%   Algebraic Plane Curves   %%%%%
%%%%%                            %%%%%
%%%%%%%%%%%%%%%%%%%%%%%%%%%%%%%%%%%%%%

\subsection{Algebraic Plane Curves}

Let $K$ be a field.
The \defn{affine plane over $K$}, denoted by $\bb A_K^2$, is the set
\[ \bb A_K^2 = K^2. \]
If $L/K$ is an algebraic extension, then $\bb A_K^2 \subseteq \bb A_L^2$.
The \defn{projective plane over $K$}, denoted by $\bb P_K^2$, is the set of lines in $K^3$ through the origin.
This may be constructed as the set of points in $K^3$ other than the origin,
modulo an equivalence relation where two points are equivalent if and only if they are colinear with the origin.
That is,
\[ \bb P_K^2 = (K^3 - \{(0,0,0)\})/\sim \]
where
\[ (x_1, y_1, z_1) \sim (x_2, y_2, z_2) \iff \exists k \in K : (x_1, y_1, z_1) = (kx_2, ky_2, kz_2). \]
The equivalence class of a point $(x, y, z)$ is denoted by $(x : y : z)$.
If $L/K$ is an algebraic extension, then $\bb P_K^2 \subseteq \bb P_L^2$.

An \defn{affine algebraic plane curve} over $K$ is a set of points
\[ C_f : \{ (x_0, y_0) \in \bb A_{\bar K}^2 ~|~ f(x_0,y_0) = 0 \}, \]
for some non-constant polynomial $f \in K[x,y]$.
It is the set of points in the affine plane where some non-constant polynomial vanishes.
Note that some of these points may have coordinates not in $K$, but in its algebraic closure.
If $L/K$ is an algebraic extension, then denote $C_f(L) = C_f \cap \bb A_L^2$.
Points in $C_f(L)$ are called \defn{$L$-rational.}
Points in $C_f(K)$ are called \defn{rational} or $K$-rational.

A \defn{projective algebraic plane curve} over $K$ is a set of points
\[ C_f : \{ (x_0 : y_0 : z_0) ~|~ f(x_0, y_0, z_0) = 0 \}, \]
for some homogeneous polynomial $f \in K[X, Y, Z]$.
It is the set of points in the projective plane where some $f$ vanishes.
\note{There is a problem with this definition, since the zero set of $Z \in K[X,Y,Z]$ is a plane, not a curve!}
Note that if $f$ is zero at $(x_0, y_0, z_0)$,
then $f$ is zero at every point in the equivalence class $(x_0 : y_0 : z_0)$,
a consequence of $f$ being homogeneous.
If $L \supseteq K$ is an algebraic extension, then analogously to the case for affine curves,
denote $C_f(L) = C_f \cap \bb P_L^2$.
Points in $C_f(L)$ are called \defn{$L$-rational.}
Points in $C_f(K)$ are called \defn{rational} or $K$-rational.

Affine and projective curves are related.
Given a polynomial $f \in K[x,y]$ defining an affine plane curve $C_f$,
we may construct the homogenization $F$ of $f$,
\[ F = Z^{\deg f} f\left( \frac X Z, \frac Y Z \right) \in K[X,Y,Z]. \]
The \defn{projective closure} of $C_f$ is $\bar{C_f} := C_F$.
\begin{example}
  Let $f = y^2 + x^3 - x - 1 \in \bb Q[x,y]$.
  The affine plane curve $C_f$ is an elliptic curve (see Equation \ref{eq_elliptic}).
  Then $\deg f = 3$ and the projective closure $C_f$ is given by
  \begin{align*}
    F &= Z^3 f\left( \frac X Z, \frac y Z \right) \\
      &= Z^3 \left( \frac Y Z \right)^2 + \left( \frac X Z \right)^3 - \left( \frac X Z \right) - 1 \\
      &= ZY^2 + X^3 - XZ^2 - Z^3.
  \end{align*}
\end{example}
Given a polynomial $F \in K[X,Y,Z]$ defining a projective plane curve $C_F$,
let $f = F(x, y, 1) \in K[x,y]$.
If $f$ is non-constant, then $C_f$ is an \defn{affine slice}
\footnote{Setting $f = F(x, 1, y)$ or $f = F(1, x, y)$ yields a different affine slice.}
of $C_F$.
Homogenization and slicing at $Z = 1$ are mutual inverses.
\begin{example}
  Let $F = ZY^2 + X^3 - XZ^2 - Z^3 \in \bb Q[X,Y,Z]$.
  Then the affine slice of $C_F$ at $z = 1$ is given by the polynomial
  \begin{align*}
    f &= F(x, y, 1) \\
      &= 1 \cdot y^2 + x^3 - x \cdot 1^2 - 1^3 \\
      &= y^2 + x^3 - x - 1.
  \end{align*}
\end{example}

A curve $C_f$ is \defn{irreducible} if $f$ is $\bar K$-irreducible,
i.e. if $f$ cannot be written as a product $f = gh$ of lower-degree polynomials $g, h \in \bar K[x,y]$
(or $g,h \in K[X,Y,Z]$ is $C_f$ is projective).

If $P \in C_f$ is a point on $C_f$,
then $P$ is called \defn{singular} if all formal partial derivatives of $f$ vanish at $P$.
In this case, the tangent line to $C$ at $P$ does not exist.
The curve $C_f$ is called \defn{singular} if it has a singular point.
Otherwise $C_f$ is called \defn{non-singular} or \defn{smooth}.
Some authors require that algebraic curves be irreducible and sometimes non-singular.
Our definition of $C_{3,4}$ curves below will require these conditions.

Because of the correspondence between affine and projective algebraic plane curves,
we will often refer to them simply as curves.
When we define a projective curve by a polynomial in $K[x,y]$,
it is understood that it is the projective closure of that curve that is meant.
Moreover, if $C_f$ is a curve,
we will also overload the symbol $C$ to refer both to the defining polynomial of the curve
and to the curve itself.

Let $\sigma \in \Gal(\bar K/K)$ be an automorphism on $\bar K$ that fixes $K$.
Then we may also have $\sigma$ act on $\bb A_{\bar K}^2$ and $\bb P_{\bar K}^2$ via
\[ \sigma((x_0, y_0)) = (\sigma(x_0), \sigma(y_0)), ~~~
   \sigma((X_0 : Y_0 : Z_0)) = (\sigma(X_0) : \sigma(Y_0) : \sigma(Z_0)). \]
It is easily verified that the action on $\bb P_{\bar K}^2$ is well-defined.
If $P \in \bb A_{\bar K}^2$ or $P \in \bb P_{\bar K}^2$,
then define the \defn{orbit} of $P$ to be
\[ \orb(P) := \{ \sigma(P) ~|~ \sigma \in \Gal(\bar K/K) \}. \]
\note{(I don't know if I should take $\Gal(\bar K/K)$ or $\Gal(L/K)$ for $L$,
the unique(?) Galois extension of minimal degree containing the coordinates of $P$.
Does bumping up from $L$ to $\bar K$ increase the size of the orbit?)}

\begin{lemma}
  Let $f \in K[x,y]$ and $\sigma \in \Gal(\bar K/K)$.
  Let $P = (x_0, y_0)$ be a point in $\bar K \times \bar K$. Then
  \[ f(\sigma(x_0), \sigma(y_0)) = \sigma(f(x_0, y_0)). \]
\end{lemma}
\begin{proof}
  \begin{align*}
    f(\sigma(x_0), \sigma(y_0))
      &= \sum a_{i,j}\sigma(x)^i\sigma(y)^j \\
      &= \sum \sigma(a_{i,j})\sigma(x)^i\sigma(y)^j
        & \text{$\sigma$ fixes $K$} \\
      &= \sum \sigma(a_{i,j}x^iy^j)
        & \text{$\sigma$ is multiplicative} \\
      &= \sigma \left( \sum a_{i,j}x^iy^j \right)
        & \text{$\sigma$ is additive} \\
      &= \sigma(f(x_0, y_0)).
  \end{align*}
\end{proof}
\begin{corollary}
  \label{cor_orb}
  Let $f \in K[x,y]$.
  Then $f$ is zero at $P$ if and only if $f$ is zero at every point in $\orb P$.
\end{corollary}
\begin{comment}
\begin{corollary}
  \label{cor_orb}
  Let $f \in K[x,y]$ and $\sigma \in \Gal(\bar K/K)$.
  Let $P$ be an affine point.
  Then $f$ has a zero at $P$ if and only if $f$ has a zero at $\sigma(P)$.
\end{corollary}
\begin{proof}
  ($\implies$) Suppose $f$ has a zero at a point $P = (x_0, y_0)$,
  i.e. $f(x_0, y_0) = 0$.
  Then at $\sigma(P)$,
  \[ f(\sigma(x_0), \sigma(y_0)) = \sigma(f(x_0, y_0)) = \sigma(0) = 0. \]  

  ($\impliedby$) Suppose $f$ has a zero at $\sigma(P)$, i.e. $f(\sigma(x_0), \sigma(y_0)) = 0$.
  Then $\sigma$ has an inverse $\sigma\inv \in \Gal(\bar K/K)$ and
  \[ f(x_0, y_0) = \sigma\inv(\sigma(f(x_0, y_0))) = \sigma\inv(f(\sigma(x_0), \sigma(y_0))) = \sigma\inv(0) = 0. \] 
\end{proof}
\end{comment}

%%%%%%%%%%%%%%%%%%%%%%%%%%%%%%%
%%%%%                     %%%%%
%%%%%   Coordinate Ring   %%%%%
%%%%%                     %%%%%
%%%%%%%%%%%%%%%%%%%%%%%%%%%%%%%

\subsection{The Coordinate Ring and Function Field}

Let $C$ be a curve defined over a field $K$.
The \defn{coordinate ring} of $C$, denoted by $K[C]$, is the quotient ring
\[ K[C] := \frac {K[x,y]} {\pid{C}}. \]
It is the ring of bivariate polynomials over $K$,
modulo the principal ideal generated by the curve's defining polynomial.

It is a well-known fact that, if $C$ is given by an irreducible polynomial, the coordinate ring is a Dedekind domain.
\note{Many papers state this. I'd like to find a citation with a proof.}
Therefore all non-zero ideals of $K[C]$ may be uniquely factored into a product of prime ideals.
The coordinate ring has Krull dimension 1, meaning that every non-zero prime ideal is a maximal ideal
(Theorem VIII.6.5 in \cite{hungerford}).



The \defn{function field} $K(C)$ of $C$ is the field of fractions of the coordinate ring,
\[ K(C) := \Frac(K[C]). \]
We will not work much with the function field itself.
We will use it in Chapter \ref{chap_ideals} in defining the ideal class group,
though one of the goals of Chapter \ref{chap_ideals} is to show that we can work entirely in $K[C]$.



%%%%%%%%%%%%%%%%%%%%%%%%%%%
%%%%%                 %%%%%
%%%%%   Local Rings   %%%%%
%%%%%                 %%%%%
%%%%%%%%%%%%%%%%%%%%%%%%%%%

\subsection{Local Rings}
\label{sec_local_rings}

Let $K[C]$ be the coordinate ring of an irreducible smooth curve $C$.
Let $\frak p$ be a non-zero prime ideal of $K[C]$.
We may localize $K[C]$ at the prime ideal $\frak p$ to get $K[C]_{\frak p}$,
the \defn{ring of regular functions at $\frak p$},
which is usually denoted by $\cal O_\frak p$ instead.
This can be defined more explicitly by
\[ \cal O_{\frak p} = \left\{ \frac f g \in K(C) ~|~ g \not \in \frak p \right\}. \]
This ring is a local ring, meaning that it has a unique maximal ideal, denoted by $\frak m_{\frak p}$.
More specifically, $\frak m_{\frak p}$ is 
\[ \frak m_{\frak p} = \frak p \cal O_{\frak p} =
   \left\{ \frac f g \in K(C) ~|~ f \in \frak p, ~g \not \in \frak p \right\}. \]

Now let $P$ be a point on $C$. Then $P$ induces a prime ideal
$\frak p = \{ f \in K[C] ~|~ f(P) = 0 \}$.
\begin{comment}
\begin{proposition}
  Let $P$ be an affine point on a curve $C$ and let
  \[ \frak p = \{ f \in K[C] ~|~ f(P) = 0 \}. \]
  Then $\frak p$ is a non-zero prime ideal of $K[C]$.
\end{proposition}
\begin{proof}
  \begin{description}
    \item[$\frak p$ is non-zero:]
      Let $P = (x_0, y_0)$.
      Then $x_0$ is an element of some finite algebraic extension $L$ of $K$.
      Let $m(x)$ be the minimal polynomial of $x_0$.
      Then $m(x) \in K[x]$ is univariate, non-zero, and may be viewed instead as $m(x,y) \in K[x,y]$.
      Then $m(x_0, y_0) = m(x_0) = 0$, hence $m(x,y) \in \frak p$.
    \item[$\frak p$ is prime:]
      Suppose $fg \in \frak p$. Then
      \begin{align*}
        & (fg)(P) = 0 \\
        \implies & f(P)g(P) = 0 \\
        \implies & f(P) = 0 \text{ or } g(P) = 0 \\
        \implies & f \in \frak p \text{ or } g \in \frak p.
      \end{align*}
  \end{description}
\end{proof}
\end{comment}
Define the \defn{ring of regular functions at $P$} to be the ring $\cal O_P := \cal O_{\frak p}$,
where $\frak p$ is the prime ideal induced by $P$.
Its unique maximal ideal is $\frak m_P := \frak m_{\frak p}$. Explicitly,
\begin{align*}
  \cal O_P &= \left\{ \frac f g \in K(C) ~|~ g(P) \neq 0 \right\} \\
  \frak m_P &= \left\{ \frac f g \in K(C) ~|~ f(P) = 0, ~g(P) \neq 0 \right\}.
\end{align*}

Given a non-zero prime ideal $\frak p$, $\cal O_{\frak p}$ is not only a local ring,
but also a discrete valuation ring, or DVR (Theorem VIII.6.10 in \cite{hungerford}).
For more on DVRs, see \cite{eisenbud95}.
Rather than define here what discrete valuations and DVRs are in general,
we briefly describe the valuation of a function at a prime ideal.

Let $f \in K[C]$.
The \defn{order} or \defn{valuation} of $f$ at $\frak p$, denoted $\nu_{\frak p}(f)$, is
\[ \nu_{\frak p}(f) := \max\{ r \in \bb N ~|~ f \in \frak m_P^r \}. \]
The order of 0 at $\frak p$ is defined as $\nu_{\frak p} = \infty$.
Given two polynomials $f, g \in K[C]$, $\nu_{\frak p}$ satisfies the relation
\[ \nu_{\frak p}(fg) = \nu_{\frak p}(f) + \nu_{\frak p}(g), \]
as long as we define $\infty + \infty = \infty$ and $\infty + n = \infty$ for all $n \in \bb Z$.
We can extend this to polynomials in $K(C)$.
Let $f/g \in K(C)$ and define
\[ \nu_{\frak p}\left(\frac f g\right) = \nu_{\frak p}(f) - \nu_{\frak p}(g). \]

\begin{proposition}
  This map is well-defined.
\end{proposition}
\begin{proof}
  Suppose $\frac f g = \frac h k$. Then
  \begin{align*}
    fk &= gh \\
    \nu_{\frak p}(fk) &= \nu_P(gh) \\
    \nu_{\frak p}(f) + \nu_P(k) &= \nu_{\frak p}(g) + \nu_{\frak p}(h) \\
    \nu_{\frak p}(f) - \nu_P(g) &= \nu_{\frak p}(h) - \nu_{\frak p}(k) \\
    \nu_{\frak p}\left(\frac f g\right) &= \nu_{\frak p}\left(\frac h k\right).
  \end{align*}
\end{proof}

Analogously for a point $P$ on a curve,
we may define the order of $f/g \in K(C)$ at $P$
by letting $\frak p$ be the prime induced by $P$ and invoking the definition above,
i.e. $\nu_P := \nu_{\frak p}$, where $\frak p$ is the prime ideal induced by $P$.

It is an important fact that the valuation of a function $f \in K[C]$ at a point $P$
coincides with the order of the zero or pole of $f$ at $P$.
If $\nu_P(f) = n > 0$, we say that $f$ has a zero of order $n$ at $P$.
If $\nu_P(f) = n < 0$, we say that $f$ has a pole of order $n$ at $P$.
It is worth mentioning as well that the $C_{3,4}$ order on monomials arranges monomials by the order of their poles.

A consequence of Corollary \ref{cor_orb} is that if $\sigma \in \Gal(\bar K/K)$ and $P$ is an affine point on $C$,
then $P$ and $\sigma(P)$ induce the same prime ideal $\frak p$. Thus we have
\begin{proposition}
  Let $P$ be an affine point on $C$ and let $\frak p = \{ f \in K[C] ~|~ f(P) = 0 \}$.
  Then for all $\sigma \in \Gal(\bar K/K)$ and $f \in K(C)$,
  \[ \nu_P(f) = \nu_\sigma(P)(f) = \nu_{\frak p}(f). \]
\end{proposition}

\note{It would be nice to define $\nu_{P_\infty}$ at the point at infinite,
but I don't have a definition of this.}



%%%%%%%%%%%%%%%%%%%%%%%%%%
%%%%%                %%%%%
%%%%%   C34 Curves   %%%%%
%%%%%                %%%%%
%%%%%%%%%%%%%%%%%%%%%%%%%%

\subsection{$C_{3,4}$ Curves}

A $C_{3,4}$ curve is a special case of a broader class of $C_{a,b}$ curves.
The class of $C_{3,4}$ curves was first described by Miura \cite{miura97}.
The following theorem is due to Miura \cite{miura97},
and a translation in English was given by Matsumoto \cite{matsumoto98}.
\begin{theorem}
  Let $C$ be an affine plane curve over $K$ (not necessarily smooth or irreducible).
  Let $0 < a < b$ be coprime positive integers.
  The following are equivalent.
  \begin{enumerate}[label=(\roman*)]
    \item $C$ is irreducible,
          has exactly one rational point $P_\infty$ at infinity,
          $\nu_{P_\infty}(x) = -a$, and
          $\nu_{P_\infty}(x) = -b$.
          \note{(Although I have not defined $\nu$ at $P_\infty$ yet.)}
    \item $C$ is defined by a polynomial
          \begin{equation}
            \label{eq_cab}
            C = \sum_{\substack{0 \leq i \leq b \\ 0 \leq j \leq a \\ ai + bj \leq ab }}c_{i,j}x^iy^j
          \end{equation}
          in $K[x,y]$ with $c_{b,0}$ and $c_{0,a}$ non-zero.
  \end{enumerate}
\end{theorem}
Here is a useful visualization of Equation \ref{eq_cab}.
The monomials in $C$'s defining polynomial correspond to integer points
in or on the boundary of the triangle with corners $(0, 0)$, $(0, a)$, and $(b, 0)$.
\begin{comment}
\begin{definition}
  \label{def_cab_curve}
  A \defn{$C_{a,b}$ curve} over a field $K$
  is a projective algebraic plane curve given by an affine polynomial
  \[ C = \sum_{\substack{0 \leq i \leq b \\ 0 \leq j \leq a \\ ai + bj \leq ab }}c_{i,j}x^iy^j \]
  where
  \begin{enumerate}[label=(\roman*)]
    \item $c_{b,0}$ and $c_{0,a}$ are non-zero;
    \item $C$ is irreducible over $\bar K$;
    \item $C$ has no singular points, except possibly at infinity.
  \end{enumerate}
\end{definition}
\end{comment}
\begin{definition}
  \label{def_cab_curve}
  A \defn{$C_{a,b}$ curve} over a field $K$ is a projective algebraic plane curve
  given by a polynomial of the form \ref{eq_cab} with no singular points,
  except possibly at the point $P_\infty$.
\end{definition}

A few special cases are worth mentioning.
When $a = 2$ and $b = 3$, we get an elliptic curve.
When $a = 2$ and $b = 7$, we get a genus 2 split hyperelliptic curve.
See Equations \ref{eq_elliptic} and \ref{eq_genus_3_hyperelliptic}.
More generally, for $g \geq 2$, when $a = 2$ and $b = 2g + 1$, we get a genus $g$ split hyperelliptic curve.
More importantly in this thesis,
in the special case where $a = 3$ and $b = 4$, we get a $C_{3,4}$ curve.
\begin{definition}
  \label{def_c34_curve}
  A \defn{$C_{3,4}$ curve} $C$ over a field $K$
  is a smooth, irreducible, projective algebraic plane curve
  given by an affine equation
  \footnote{The subscripts on the coefficients of $C$ are numbered according to the $C_{3,4}$ monomial order.}
  \[ C = c_{10}y^3 + c_9x^4 + c_8xy^2 + c_7x^2y + c_6x^3 + c_5y^2 + c_4xy + c_3x^2 + c_2y + c_1x + c_0, \]
  where $c_{9}$ and $c_{10}$ are non-zero.
\end{definition}

Definition \ref{def_c34_curve} may appear to be slightly more restrictive than Definition \ref{def_cab_curve} ---
Definition \ref{def_c34_curve} does not allow for the point at infinity to be singular.
We show below that $P_\infty$ on a $C_{3,4}$ is never singular.

Before proceeding, we will make some simplifying assumptions on the curve equation.
Since $c_{10}$ is non-zero, we may assume that $c_{10} = 1$,
as multiplying the defining polynomial by $c_{10}\inv$ does not change the set on which the polynomial vanishes.
We may also assume that $c_9 = 1$,
otherwise we may perform the invertible change of variables $x \mapsto x / \sqrt[4]{c_{9}}$.
This, however, may require we work over an algebraic extension of $K$ where $c_{4,0}$ has a quartic root.
In light of these assumptions, we now assume $C$ is of the \defn{long form}
\begin{equation}
  \label{eq_c34}
  C = y^3 + x^4 + c_8xy^2 + c_7x^2y + c_6x^3 + c_5y^2 + c_4xy + c_3x^2 + c_2y + c_1x + c_0.
\end{equation}

In fields of sufficiently large characteristic, one may also assume certain coefficients are zero.
If $\Char K \neq 2$, then the invertible change of variables $x = X - \frac {c_6} 4, y = Y$ gives
\begin{equation}
  \label{eq_c34_char_not_2}
  C(X,Y) = Y^3 + X^4 + d_8XY^2 + d_7X^2Y + d_5Y^2 + d_4XY + d_3X^2 + d_2Y + d_1X + d_0,
\end{equation}
for some new coefficients $d_i$.
Notice that there is no $X^3$ term.
If $\Char K \neq 3$, then the invertible change of variables $x = X, y = Y - \frac{c_8X + c_5}{3}$ gives
\begin{equation}
  \label{eq_c34_char_not_3}
  C(X,Y) = Y^3 + X^4 + d_7X^2Y + d_6X^3 + d_4XY + d_3X^2 + d_2Y + d_1X + d_0,
\end{equation}
where there is no $Y^2$ or $XY^2$ term.
If the characteristic of $K$ is neither 2 nor 3, we may perform both substitutions simultaneously. Let
\[ a = \frac {27c_6 - 9c_7c_8 + 2c_8^3} {27} \]
and perform the change of variables
\begin{align*}
  x &= X - \frac a 4 \\
  y &= Y - \frac {c_8} {3} X + \frac {ac_8 - 4c_5} {12}.
\end{align*}
Then this gives $C$ in \defn{short form}
\begin{equation}
  \label{eq_c34_short}
  C(X,Y) = Y^3 + X^4 + d_7X^2Y + d_4XY + d_3X^2 + d_2Y + d_1X + d_0.
\end{equation}

%Now let us turn our attention to the points at infinity on a $C_{3,4}$ curve $C$.
%We first answer the question of how many points there are at infinity on $C$.
Now we address the point at infinity on a $C_{3,4}$ curve.
\begin{comment}
\begin{proposition}
  Let $C$ be a $C_{3,4}$ curve over a field $K$.
  Then $C$ has only a single point at infinity, $(0 : 1 : 0)$.
\end{proposition}
We have made no assumptions on the characteristic of $K$,
so we may not assume any coefficients of $C$ are zero.
\begin{proof}
  Consider the homogeneous polynomial $\bar C$ defining the projective closure of $C$,
  \begin{align*}
    \bar C &= Y^3Z + X^4 + c_8XY^2Z + c_7X^2YZ + c_6X^3Z + c_5Y^2Z^2 \\ 
           &+ c_4XYZ^2 + c_3X^2Z^2 + c_2YZ^3 + c_1XZ^3 + c_0Z^4.
  \end{align*}
  At $Z = 0$, we have $\bar C(X,Y,0) = X^4$.
  If $\bar C(X,Y,0) = 0$, then $X = 0$.
  Therefore any solution to $\bar C$ with $Z = 0$ is of the form $(0, Y, 0)$ for some $Y$,
  while every point $(0, Y, 0) \in K^3$ is a solution to $C$.
  In $\bb P_{\bar K}^2$, $(0 : 0 : 0)$ is not a point, and every other $(0 : Y : 0)$ is equivalent to $(0 : 1 : 0)$.
  Therefore, there is only a single point at infinity, $(0 : 1 : 0)$, on $C$.
\end{proof}
Denote the unique point at infinity on a $C_{3,4}$ curve by $P_\infty$.
\end{comment}
\begin{proposition}
  Let $C$ be a $C_{3,4}$ curve over a field $K$.
  The point $P_\infty$ is non-singular.
\end{proposition}
\begin{proof}
  We have made no assumptions on the characteristic of $K$,
  therefore we assume that $C$ is given by a polynomial of the form \ref{eq_c34}.
  To show that $C$ is non-singular at $P_\infty$,
  we show that one of the formal partial derivatives of $C$ is non-zero.
  Since $P_\infty$ is not an affine point,
  this requires that we work with the homogenization of $C$'s defining polynomial,
  \begin{align*}
    \bar C &= Y^3Z + X^4 + c_8XY^2Z + c_7X^2YZ + c_6X^3Z + c_5Y^2Z^2 \\ 
           &+ c_4XYZ^2 + c_3X^2Z^2 + c_2YZ^3 + c_1XZ^3 + c_0Z^4.
  \end{align*}
  The formal partial derivative of $\bar C$ with respect to $Z$ is
  \begin{align*}
    \bar C_Z &= Y^3 + c_8XY^2 + c_7X^2Y + c_6X^3 + 2c_5Y^2Z \\
             &+ 2c_4XYZ + 2c_3X^2Z + 3c_2YZ^2 + 3c_1XZ^2 + 4c_0Z^3.
  \end{align*}
  Evaluated at $P_\infty$, $\bar C_Z(0 : 1 : 0) = 1$.
\end{proof}




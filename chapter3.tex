%%%%%%%%%%%%%%%%%%%%%%%%
%%%%%              %%%%%
%%%%%   Divisors   %%%%%
%%%%%              %%%%%
%%%%%%%%%%%%%%%%%%%%%%%%

\section{Divisors}
\subsection{The Divisor Class Group}

Let $C$ be a non-singular, projective curve over a field $K$.
The \defn{group of divisors} on $C$, denoted by $\Div_{\bar K}(C)$, is the free Abelian group generated by the points of $C(\bar K)$.
A \defn{divisor} of $C$ is any element of this group; it is a finite formal sum of points.
If $P$, $Q$, and $R$ are points on the curve $C$, then examples of divisors include
  \[ \begin{array}{c} P + Q + R \\ P + 3Q - 2R \\ Q \\ 0 \end{array}. \]

If $D$ is a divisor and $P$ is a point on $C$, the \defn{order} of $D$ at $P$, denoted by $\ord_P(D)$, is the coefficient of $P$ in this sum.
For example, if $D = P + 3Q - 2R$, then $\ord_Q(D) = 3$ and $\ord_R(D) = -2$.
A divisor $D$ is called \defn{effective} if $\ord_P(D) \geq 0$ at every point $P$.
So $P + Q + R$, $Q$, and $0$ are effective divisors, while $P + 3Q - 2R$ is not.

The degree of a divisor, denoted by $\deg(D)$, is the sum of the orders of $D$ at all points,
  \[ \deg(D) = \sum_{P \in C}\ord_P(D). \]
For instance, $\deg(P + 3Q - 2R) = 2$.
If $D$ and $D'$ are both divisors of $C$, then $\deg(D + D') = \deg(D) + \deg(D')$.
Consequently, divisors with degree zero form a subgroup $\Div_{\bar K}^0(C)$ of $\Div_{\bar K}(C)$.

If $L$ is an algebraic extension of $K$ and $\sigma \in \Gal(\bar K/L)$,
then $\sigma$ extends an action on points of $C(L)$, via
  \[ \sigma(x : y : z) = (\sigma(x) : \sigma(y) : \sigma(z)). \]
This, in turn, extends to an action on $\Div_{\bar K}(C)$.
If $D$ is the divisor
  \[ D = \sum_{P \in C(\bar K)} n_P P, \]
then define
  \[ \sigma(D) = \sum_{P \in C(\bar K)} n_P \sigma(P). \]
A divisor $D$ is \defn{defined over $L$} if $\sigma(D) = D$ for all $\sigma \in \Gal(\bar K/L)$.
That is, if $D$ remains fixed by every embedding of $L$ in $\bar K$.
The set of divisors on $C$ defined over $L$ is denoted by $\Div_L(C)$.
The set $\Div_L(C)$ forms a subgroup of $\Div_{\bar K}(C)$, and $\Div_L^0(C)$ a subgroup of both $\Div_L(C)$ and $\Div_{\bar K}^0(C)$.

This last definition deserves a few examples.
\begin{example}
  Let $K$ be any field, $L$ any algebraic extension and $\sigma \in \Gal(\bar K/L)$.
  By definition, $\sigma$ fixes $L$.
  If $P$ is any point with coordinates in $L$, then $\sigma(P) = P$.
  If $D$ is any divisor consisting only of points with coordinates in $L$, then $\sigma(D) = D$ and $D$ is defined over $L$.
\end{example}
\begin{example}
  Let $K = \bb F_2$ and let $L = K(\alpha)$ be an algebraic extension with $\alpha^2 + \alpha = 1$.
  Let $C$ be the curve $x^4 + y^3 + x + 1$ over $K$.
  Let $P$ be the point $(\alpha : 1 : 1)$ on $C$ and let $D$ be the divisor $D = P$.
  There is an automorphism $\sigma \in \Gal(\bar K/K)$ that maps $\alpha \mapsto \alpha + 1$, so 
    \[ \sigma(D) = \sigma(P) = (\sigma(\alpha) : \sigma(1) : \sigma(1)) = (\alpha + 1 : 1 : 1) \neq D. \]
  Hence $D$ is not defined over $K$.
\end{example}
\begin{example}
  Let $K$, $L$, $C$, and $P$ be as in the previous example.
  Let $Q = (\alpha + 1 : 1 : 1)$, which is also a point on $C$.
  Let $D$ be the divisor $D = P + Q$.
  Every automorphism $\sigma$ in $\Gal(\bar K/K)$ maps either $\alpha$ to itself or to $\alpha + 1$.
  In the former case,
  \begin{align*}
    \sigma(D) &= \sigma(P) + \sigma(Q) \\
              &= (\sigma(\alpha) : \sigma(1) : \sigma(1)) + (\sigma(\alpha + 1) : \sigma(1) : \sigma(1)) \\
              &= (\alpha : 1 : 1) + (\alpha + 1 : 1 : 1) \\
              &= P + Q = D.
  \end{align*}
  In the latter case,
  \begin{align*}
    \sigma(D) &= \sigma(P) + \sigma(Q) \\
              &= (\sigma(\alpha) : \sigma(1) : \sigma(1)) + (\sigma(\alpha + 1) : \sigma(1) : \sigma(1)) \\
              &= (\alpha + 1 : 1 : 1) + (\alpha : 1 : 1) \\
              &= Q + P = D.
  \end{align*}
  So $D$ is defined over $K$.
\end{example}

A divisor, being a formal sum of points, can be used to record the zeroes and poles of a function.
Let $f \in K(C)$ be a rational function.
The \defn{divisor of $f$} is
  \[ \div f := \sum_{P \in C(\bar K)} v_P(f)P, \]
where $v_P(f)$ is the valuation of $f$ at $P$ (with respect to the curve $C$).
Recall \note{(from an earlier section)} that $v_P(f)$ is the multiplicity of the intersection of $f$ and $C$ at $P$.
\note{(What if $v_P(f) < 0$?)}
If $D$ is the divisor of some rational function, then $D$ is called a \defn{principal divisor}.
In the following, we will see that principal divisors form a subgroup of degree zero divisors.

\begin{proposition}
  Let $k \in K^*$ and let $f, g \in K(C)^*$. Then
  \begin{enumerate}[label=(\roman*)]
    \item $\div(k) = 0$;
    \item $\div(fg) = \div f + \div g$.
  \end{enumerate}
\end{proposition}
\begin{proof}
  \note{TODO}
\end{proof}
\begin{theorem}
  Let $C$ be a curve over $K$ and let $f \in K(C)^*$.
  Then $f$ has finitely many zeroes and poles and $\deg(\div f) = 0$.
\end{theorem}
\begin{proof}
  Galbraith Thm 8.3.14.
\end{proof}

These theorems combined show that the principal divisors are a subset of the degree zero divisors and remain closed as a group.
They therefore form a (normal) subgroup of $\Div_{\bar K}^0(C)$.
The group of principal divisors on $C$ (defined over $K$) is denoted by $\Princ_K(C)$.

At last we arrive at defining the divisor class group.
The \defn{divisor class group} is the quotient group,
  \[ \Jac_K^0(C) = \frac{\Div_K^0(C)}{\Princ_K(C)}. \]
The divisor class group of $C$ is also called the \defn{Jacobian} of $C$ or the \defn{Picard group} of $C$.
It is sometimes denoted by $\Pic_K^0(C)$.

\begin{itemize}
  \item Every divisor equivalent to one of the form $P_1 + \dots + P_n - nP_\infty$.
        (For any term of the form $-mP$, add $m \div f$ for any polynomial $f$ through $P$.)
  \item Riemann Roch Theorem implies divisor equivalent to one with $n \leq g$. (How?)
        Such a divisor is called reduced. (Is it unique?)
\end{itemize}

Operations with divisors.
\begin{itemize}
  \item Addition
  \item Negation
  \item LCM
  \item GCD
\end{itemize}



\subsection{The Ideal Class Group}

Given the point set of a curve, we were able to construct the divisor class group of the curve.
For a ring (more specifically, a Dedekind domain), there is an analogous construction -- the ideal class group of the ring.

Given a ring $R$, there are several binary operations on the set of ideals of $R$.
Among them is multiplication, and the ring $R$ is an identity element under this operation.
The ideals of $R$ therefore form a monoid under multiplication.
The \emph{fractional} ideals of $R$ form a group.

A \defn{fractional ideal} of $R$ is an $R$-submodule $I$ of the field of fractions $\Frac(R)$
such that there is an element $r \in R$ making $rI \subseteq R$.
The principal fractional ideals of $R$ form a subgroup of the fractional ideals.
If $\pid {a/b}$ is a principal fractional ideal, then $b\pid{a/b} = \pid a \subseteq R$.
The \defn{ideal class group} is the group quotient of the fractional ideals by the principal fractional ideals.
Thus, it induces an equivalence relation on fractional ideals.
Two fractional ideals $I$ and $J$ are equivalent if $IJ\inv$ is principal.
We can rephrase this slightly.
Fractional ideals $I$ and $J$ are equivalent if there are elements $a, b$ in $R$ such that $aI = bJ$.
  \[ I \equiv J \iff \exists \frac a b \in \Frac(R) : IJ\inv = \pid{\frac a b} \iff \exists a, b \in R : aI = bJ. \]

In the ideal class group, every fractional ideal is equivalent to an integral ideal. \note{(Define integral.)}
After all, if $I$ is fractional, then there is an $r \in R$ such that $rI \subseteq R$.
But $J = rI$ is an $R$-submodule of $\Frac (R)$ contained in $R$, so it is an $R$-submodule of $R$, i.e. an ideal of $R$.
Since $rI = 1J$, we have $I \equiv J$.
Since every fractional ideal is equivalent to an integral ideal, when working in the ideal class group,
we can get away with working entirely with integral ideals.
(We will see later how to find an integral ideal equivalent to $I\inv$.)

An important fact for this thesis is the following.
\begin{theorem}
  The divisor class group $\Div_K^0(C)$ of a curve $C$ is isomorphic to the ideal class group of its coordinate ring, $K[C]$:
    \[ \Jac_K(C) \cong \Cl(K[C]). \]
\end{theorem}
Divisors defined over $K$ may consist of points lying in an algebraic extension of $K$.
Because of this, performing operations on the points themselves can be computationally expensive.
By operating in the ideal class group instead, we can do all of our computations over the base field $K$.
To prove this theorem, we will demonstrate an isomorphism between the groups.

\begin{theorem}
  Let $\Id(K[C])$ denote the group of fractional ideals of $K[C]$ and define the maps
  \begin{align*}
    \div(-) : \Id(K[C]) &\to \Div_K^0(C) \\
    I &\mapsto \sum_{P \in C} \min_{f \in I}\{v_P(f)\}P
  \end{align*}
  and
  \begin{align*}
    I_{(-)} : \Div_K^0(C) &\to \Id(K[C]) \\
    D &\mapsto \{ f \in K(C)^* ~|~ \forall P \in C : v_P(f) \geq \ord_P(D) \}.
  \end{align*}
  Then $\div(-)$ is an isomorphism of groups and $I_{(-)}$ is its inverse.
\end{theorem}
\begin{lemma}
  If $I$ is a fractional ideal of $K[C]$, then $\div I$ is a degree zero divisor on $C$ defined over $K$.
\end{lemma}
\begin{proof}
  $\div I = \sum_{P \in C} \min_{f \in I}\{v_P(f)\}P$.
\end{proof}
\begin{lemma}
  If $D$ is a degree zero divisor on $C$ defined over $K$,
  then $I_D$ is a fractional ideal of $K[C]$.
\end{lemma}
\begin{proof}
\end{proof}
\begin{lemma}
  $I_{\div I} = I$.
\end{lemma}
\begin{proof}
\end{proof}
\begin{lemma}
  $\div(I_D) = D$.
\end{lemma}
\begin{proof}
\end{proof}
\begin{lemma}
  $\div(IJ) = \div(I) + \div(J)$.
\end{lemma}
\begin{proof}
\end{proof}

Let $[D]$ be a divisor class and assume that $D$ is a reduced divisor, since $[D]$ has such a representative.
Then
  \[ D = P_1 + \dots + P_r - rP_\infty, \]
where $r \leq g$ \note{(define $g$)} and the $P_i$'s are affine, but not necessarily distinct.
Define $I_D$ to be the ideal
  \[ I_D := \{ f \in K[C] ~|~ \forall P \in C : v_P(f) \geq \ord_P(D) \}. \]
In words, the divisor $D$ encodes a bunch of points on $C$ (possibly with multiplicity greater than one)
and $I_D$ is the ideal consisting of all polynomials that intersect $C$ at those points with at least the prescribed multiplicities.

In the other direction, suppose $I$ is an ideal of $K[C]$.
Define $\div(I)$ to be the divisor
  \[ \div(I) = \sum_{P \in C} \min\{ v_P(f) ~|~ f \in I \}P. \]
\note{(As defined, this is not a degree-zero divisor.)}
The divisor $\div(I)$ consists of all points on $C$ at which every polynomial in $I$ vanishes.
If all polynomials in $I$ vanish at a point with multiplicity greater than 1,
then the order of $\div(I)$ at that point is smallest multiplicity with which any of them vanish.

\note{I will need to show this is well-defined. It may be easier not to assume $D$ is effective/reduced.}
  \[ I_D := \{ f \in K(C) ~|~ \forall P \in C : v_P(f) \geq \ord_P(D) \}. \]
  \[ \div(I) = \sum_{P \in C} \min\{ v_P(f) ~|~ f \in I \}P. \]
Need to show $\div(I_D) = D$ and $I_{\div I} = I$.

\begin{itemize}
  \item Divisor represented by ideal of polynomials through points.
  \item Divisor defined over K given by polynomials over K.
\end{itemize}




\subsection{Types of Divisors}

\begin{comment}
\begin{center}
\begin{tabular}{l|l|l}
% DEGREE | TYPE | POLYS
  Degree & Type & Basis \\
  \hline
  0 & 0 & $1$ \\
  \hline
  \multirow{2}{*}{1}
    &\multirow{2}{*}{11}
      & $x + f_0$, \\
    & & $y + g_0$ \\
  \hline
  \multirow{4}{*}{2}
    &\multirow{2}{*}{21}
      & $y + f_1x + f_0$, \\
    & & $x^2 + g_1x + g_0$ \\
    \cline{2-3}
    &\multirow{2}{*}{22}
      & $x + f_0$, \\
    & & $y^2 + g_2y + g_0$ \\
  \hline
  \multirow{6}{*}{3}
    &\multirow{3}{*}{31}
      & $x^2 + f_2y + f_1x + f_0$, \\
    & & $xy + g_2y + g_1x + g_0$, \\
    & & $y^2 + h_2y + h_1x + h_0$ \\
    \cline{2-3}
    &\multirow{2}{*}{32}
      & $y + f_1x + f_0$, \\
    & & $x^3 + g_3x^2 + g_1x + g_0$ \\
    \cline{2-3}
    &\multirow{1}{*}{33}
      & $x + f_0$ \\
  \hline
  \multirow{8}{*}{4}
    &\multirow{3}{*}{41}
      & $xy + f_3x^2 + f_2y + f_1x + f_0$, \\
    & & $y^2 + g_3x^2 + g_2y + g_1x + g_0$, \\
    & & $x^3 + h_3x^2 + h_2y + h_1x + h_0$ \\
    \cline{2-3}
    &\multirow{2}{*}{42}
      & $x^2 + f_2y + f_1x + f_0$, \\
    & & $xy + g_2y + g_1x + g_0$ \\
    \cline{2-3}
    &\multirow{2}{*}{43}
      & $x^2 + f_2y + f_1x + f_0$, \\
    & & $y^2 + g_4xy + g_2y + g_1x + g_0$ \\
    \cline{2-3}
    &\multirow{1}{*}{44}
      & $y + f_1x + f_0$
  \hline
  \multirow{9}{*}{5}
    &\multirow{3}{*}{51}
      & $y^2 + f_4xy + f_3x^2 + f_2y + f_1x + f_0$, \\
    & & $x^3 + g_4xy + g_3x^2 + g_2y + g_1x + g_0$, \\
    & & $x^2y + h_4xy + h_3x^2 + h_2y + h_1x + h_0$ \\
    \cline{2-3}
    &\multirow{2}{*}{52}
      & $xy + f_3x^2 + f_2y + f_1x + f_0$, \\
    & & $y^2 + g_3x^2 + g_2y + g_1x + g_0$ \\
    \cline{2-3}
    &\multirow{2}{*}{53}
      & $xy + f_3x^2 + f_2y + f_1x + f_0$, \\
    & & $x^3 + g_5y^2 + g_3x^2 + g_2y + g_1x + g_0$ \\
    \cline{2-3}
    &\multirow{2}{*}{54}
      & $x^2 + f_2y + f_1x + f_0$, \\
    & & $xy^2 + g_5y^2 + g_4xy + g_2y + g_1x + g_0$ \\
  \hline
  \multirow{10}{*}{6}
    &\multirow{3}{*}{61}
      & $x^3 + f_5y^2 + f_4xy + f_3x^2 + f_2y + f_1x + f_0$, \\
    & & $x^2y + g_5y^2 + g_4xy + g_3x^2 + g_2y + g_1x + g_0$, \\
    & & $xy^2 + h_5y^2 + h_4xy + h_3x^2 + h_2y + h_1x + h_0$ \\
    \cline{2-3}
    &\multirow{2}{*}{62}
      & $y^2 + f_4xy + f_3x^2 + f_2y + f_1x + f_0$, \\
    & & $x^3 + g_4xy + g_3x^2 + g_2y + g_1x + g_0$ \\
    \cline{2-3}
    &\multirow{2}{*}{63}
      & $y^2 + f_4xy + f_3x^2 + f_2y + f_1x + f_0$, \\
    & & $x^2y + g_6x^3 + g_4xy + g_3x^2 + g_2y + g_1x + g_0$ \\
    \cline{2-3}
    &\multirow{2}{*}{64}
      & $xy + f_3x^2 + f_2y + f_1x + f_0$, \\
    & & $x^4 + g_6x^3 + g_5y^2 + g_3x^2 + g_2y + g_1x + g_0$ \\
    \cline{2-3}
    &\multirow{1}{*}{65}
      & $x^2 + f_2y + f_1x + f_0$
\end{tabular}
\end{center}
\end{comment}

\begin{center}
\begin{tabular}{l|l|l||l|l|l}
  %Degree & Type & Basis & Degree & Type & Basis \\
  D. & T. & Basis & D. & T. & Basis \\
  \hline
  0 & 0 & $1$ & \multirow{9}{*}{5} &\multirow{3}{*}{51} & $y^2 + f_4xy + f_3x^2 + f_2y + f_1x + f_0$, \\
  \cline{1-3}
  \multirow{2}{*}{1} &\multirow{2}{*}{11} & $x + f_0$ & & & $x^3 + g_4xy + g_3x^2 + g_2y + g_1x + g_0$, \\
    & & $y + g_0$ & & & $x^2y + h_4xy + h_3x^2 + h_2y + h_1x + h_0$ \\
  \cline{1-3}\cline{5-6}
  \multirow{4}{*}{2} &\multirow{2}{*}{21} & $y + f_1x + f_0$, & & \multirow{2}{*}{52} & $xy + f_3x^2 + f_2y + f_1x + f_0$, \\
    & & $x^2 + g_1x + g_0$ & & & $y^2 + g_3x^2 + g_2y + g_1x + g_0$ \\
    \cline{2-3}\cline{5-6}
    &\multirow{2}{*}{22}  & $x + f_0$, & & \multirow{2}{*}{53} & $xy + f_3x^2 + f_2y + f_1x + f_0$, \\
    & & $y^2 + g_2y + g_0$ & & & $x^3 + g_5y^2 + g_3x^2 + g_2y + g_1x + g_0$ \\
  \cline{1-3}\cline{5-6}
  \multirow{6}{*}{3} &\multirow{3}{*}{31} & $x^2 + f_2y + f_1x + f_0$, & & \multirow{2}{*}{54} & $x^2 + f_2y + f_1x + f_0$, \\
    & & $xy + g_2y + g_1x + g_0$, & & & $xy^2 + g_5y^2 + g_4xy + g_2y + g_1x + g_0$ \\
  \cline{4-6}
    & & $y^2 + h_2y + h_1x + h_0$ & \multirow{10}{*}{6} &\multirow{3}{*}{61} & $x^3 + f_5y^2 + f_4xy + f_3x^2 + f_2y + f_1x + f_0$, \\
    \cline{2-3}
    &\multirow{2}{*}{32} & $y + f_1x + f_0$, & & & $x^2y + g_5y^2 + g_4xy + g_3x^2 + g_2y + g_1x + g_0$, \\
    & & $x^3 + g_3x^2 + g_1x + g_0$ & & & $xy^2 + h_5y^2 + h_4xy + h_3x^2 + h_2y + h_1x + h_0$ \\
    \cline{2-3}\cline{5-6}
    &\multirow{1}{*}{33} & $x + f_0$ & &\multirow{2}{*}{62} & $y^2 + f_4xy + f_3x^2 + f_2y + f_1x + f_0$, \\
  \cline{1-3}
  \multirow{8}{*}{4} &\multirow{3}{*}{41} & $xy + f_3x^2 + f_2y + f_1x + f_0$, & & & $x^3 + g_4xy + g_3x^2 + g_2y + g_1x + g_0$ \\
  \cline{5-6}
    & & $y^2 + g_3x^2 + g_2y + g_1x + g_0$, & &\multirow{2}{*}{63} & $y^2 + f_4xy + f_3x^2 + f_2y + f_1x + f_0$, \\
    & & $x^3 + h_3x^2 + h_2y + h_1x + h_0$ & & & $x^2y + g_6x^3 + g_4xy + g_3x^2 + g_2y + g_1x + g_0$ \\
    \cline{2-3}\cline{5-6}
    &\multirow{2}{*}{42} & $x^2 + f_2y + f_1x + f_0$, & &\multirow{2}{*}{64} & $xy + f_3x^2 + f_2y + f_1x + f_0$, \\
    & & $xy + g_2y + g_1x + g_0$ & & & $x^4 + g_6x^3 + g_5y^2 + g_3x^2 + g_2y + g_1x + g_0$ \\
    \cline{2-3}\cline{5-6}
    &\multirow{2}{*}{43} & $x^2 + f_2y + f_1x + f_0$, & &\multirow{1}{*}{65} & $x^2 + f_2y + f_1x + f_0$ \\
    \cline{4-6}
    & & $y^2 + g_4xy + g_2y + g_1x + g_0$ \\
    \cline{2-3}
    &\multirow{1}{*}{44}
      & $y + f_1x + f_0$
\end{tabular}
%\end{center}
%\begin{center}
\begin{tabular}{l|l|l}
% DEGREE | TYPE | POLYS
    
\end{tabular}
\end{center}
\subsection{The Colon Ideal}

\begin{definition}
  Let $I$ and $J$ be ideals of a ring $R$.
  Then the \defn{ideal quotient} or \defn{colon ideal} of $I$ by $J$ is the set
  \[ (I:J) := \{ a \in R ~|~ rJ \subseteq I \}. \]
  When $I = \pid f$ and $J = \pid g$ are principal ideals, we may write $f : g$ rather than $\pid f : \pid g$.
\end{definition}

First we show that the colon ideal is rightfully called an ideal.
Then we demonstrate a some basic properties of the colon ideal and discuss its relation to divisors.

\begin{proposition}
  Let $I$ and $J$ be ideals of a ring $R$.
  Then $I : J$ is also an ideal of $R$.
\end{proposition}
\begin{proof}
  We must show that $I : J$ is closed under addition and under multiplication by elements of $R$.
  \begin{description}
    \item[Closure under addition:]
      Let $a, b \in I : J$.
      We must show that $a + b \in I : J$.
      That is, we must show that $(a + b)J \subseteq I$.
      
      Letting $a, b \in I : J$, we have that $aJ \subseteq I$ and $bJ \subseteq I$.
      Now let $c \in (a + b)J$.
      Then there is some element $j \in J$ such that $c = (a + b)j$.
      However, since $aJ$ and $bJ$ are subsets of $I$, we have $aj$ and $bj$ being members of $I$.
      Since $I$ is closed under addition, we have $aj + bj = (a + b)j = c \in I$, hence $(a + b)J \subseteq I$.
    
    \item[Closure under scalar multiplication:]
      Let $a \in I : J$ and $r \in R$.
      We must show that $(ra)J \subseteq I$.
      However, this follows immediately from the fact that $(ra)J \subseteq aJ$ and $aJ \subseteq I$.
  \end{description}
\end{proof}

\begin{proposition}
  \label{prop_colon_ideal}
  Let $I$, $J$, and $K$ be ideals of a ring $R$. Then
  \begin{enumerate}[label=(\roman*)]
    \item $K \subseteq I : J$ if and only if $KJ \subseteq I$;
    \item $I \subseteq I : J$;
    \item $I : (J + K) = (I : J) \cap (I : K)$.
  \end{enumerate}
\end{proposition}
\begin{proof}
  Let $I$, $J$, and $K$ be ideals of a ring $R$. Then
  \begin{enumerate}[label=(\roman*)]
    \item
      ($\implies$)
      Suppose $K \subseteq I : J$ and let $a \in KJ$.
      Then there exist elements $k \in K$ and $j \in J$ such that $a = kj$.
      However, $k$ is also a member of $I : J$, so $k$ is such that $kJ \subseteq I$.
      Hence $a = kj \in I$.
      
      ($\impliedby$)
      Suppose that $KJ \subseteq I$ and let $k \in K$.
      We have $kJ \subseteq KJ \subseteq I$, therefore $k \in I : J$.
    \item
      Follows from (i), since $JI \subseteq I$.
    \item
      \note{TODO!}
  \end{enumerate}
\end{proof}

\begin{proposition}
  Let $I$ be an ideal of a ring $R$ and let $f \in I$.
  In the ideal class group of $R$,
    \[ I\inv \equiv f : I. \]
\end{proposition}
\begin{proof}
  Let $J = f : I$.
  By Proposition \ref{prop_colon_ideal}, $JI \subseteq \pid f$.
  This implies that $JI$ is a principal ideal.
  In the ideal class group, we have
    \[ A \equiv B \text{ if and only if } AB\inv \text{ is principal} \]
  for ideals $A$ and $B$.
  Hence $J \equiv I\inv$ and the result follows.
\end{proof}

The divisor class group of a $C_{3,4}$ curve is isomorphic to the ideal class group its coordinate ring.
If we are given a divisor class $[D]$ and we wish to find the class $[-D]$,
this is analogous to being given an ideal class $[I]$ and finding the class of $[I\inv]$.
By the above propositions, given $I$ and an element $f \in I$, a representative of the class $[I\inv]$ is given by $f : I$.
We discuss later how to compute $f : I$.
First we ascribe some geometric meaning to $f : I$.

\begin{theorem}
  Let $D$ be an effective divisor.
  Let $I_D$ be its ideal representation.
  Let $f \in I_D$.
  Then
    \[ \div(f : I_D) = \div f - D. \]
\end{theorem}
\begin{proof}
  \note{TODO}
\end{proof}



%%%%%%%%%%%%%%%%%%%%%%%%%%
%%%%%                %%%%%
%%%%%   Background   %%%%%
%%%%%                %%%%%
%%%%%%%%%%%%%%%%%%%%%%%%%%

\section{Motivation and Prior Work}

This introduction will be light on definitions.
Full details of the topics mentioned may be found later in this thesis or in other cited works.
For now, it should be enough to know that an \defn{algebraic plane curve} is the set of points $(x,y)$
at which $f(x, y) = 0$, for some field $K$ and polynomial $f \in K[x,y]$.
Examples of algebraic plane curves include elliptic curves, hyperelliptic curves,
and the subject of this thesis, $C_{3,4}$ curves.
An \defn{elliptic curve} is the set of zeroes of a polynomial
\begin{equation}
  \label{eq_elliptic}
  y^2 + x^3 + c_4xy + c_3x^2 + c_2y + c_1x + c_0.
\end{equation}
A (ramified genus 3) \defn{hyperelliptic curve} is the set of zeroes of a polynomial
\begin{equation}
  \label{eq_genus_3_hyperelliptic}
  y^2 + x^7 + c_{10}x^3y + c_9x^6 + c_8x^2y + c_7x^5 + c_6xy + c_5x^4 + c_4y + c_3x^3 + c_2x^2 + c_1x + c_0,
\end{equation}
though hyperelliptic curves can be of any genus 2 or greater and come in a variety of flavours (ramified, split, inert).
A \defn{$C_{3,4}$ curve} is the set of zeroes of a polynomial
\begin{equation}
  \label{eq_c34_intro}
  y^3 + x^4 + c_8xy^2 + c_7x^2y + c_6x^3 + c_5y^2 + c_4xy + c_3x^2 + c_2y + c_1x + c_0.
\end{equation}
For each of these classes of curves, one might also demand that they be non-singular
\note{(put a definition of singular in chapter 4 and refer to it here)},
as we will do in Chapter \ref{chap_curves}.

If $C$ is an algebraic plane curve, a \defn{divisor} of a curve is a formal sum of points on $C$.
If $P$ and $Q$ are points on $C$, then $2P$ and $P - Q$ are examples of divisors on $C$.
In Chapter \ref{chap_divisors}, we will place an equivalence relation on divisors,
partitioning them into \defn{divisor classes}.
Together with an addition operation, this forms the \defn{divisor class group} of $C$.



%%%%%%%%%%%%%%%%%%%%%%%%%%
%%%%%                %%%%%
%%%%%   Motivation   %%%%%
%%%%%                %%%%%
%%%%%%%%%%%%%%%%%%%%%%%%%%

\subsection{Motivation}

Let $\bb F_q$ be the finite field of order $q$ and suppose $E$ is an elliptic curve defined over $\bb F_q$,
meaning that $E$ is the set of zeroes of a polynomial in $\bb F_q[x,y]$ of the form in Equation \ref{eq_elliptic}.
The following theorem, due to Helmut Hasse (1898 -- 1979), says that
the number of rational points (i.e. points in $\bb F_q \times \bb F_q$) on $E$
lies in some interval that is small compared to the order of $\bb F_q$.
\begin{theorem}[Hasse]
  Let $E$ be an elliptic curve over a finite field $\bb F_q$.
  The number of rational points on $E$ differs from $q + 1$ by at most $2 \sqrt q$. That is,
  \[ | \#E(\bb F_q) - (q + 1) | \leq 2 \sqrt q. \]
\end{theorem}
In other words,
$q + 1$ is a good estimate for the number of rational points on $E$,
especially when $q$ is large.

\begin{example}
  Let $p = 2^{31} - 1 = 2,147,483,647$, which is prime.
  The elliptic curve defined by $y^2 + x^3 - 1 \in \bb F_p[x,y]$ has 2,147,391,324 rational points.
\end{example}

Let $a_q = \frac {\#E(\bb F_q) - (q + 1)} {2\sqrt q}$.
Then Hasse's Theorem says that $a_p \in [-1, 1]$.
Letting $\theta_q = \arccos a_q$, then $\theta_q \in [0, \pi]$.
Given a subinterval $[\alpha, \beta] \subseteq [0, \pi]$ and a prime $p$,
one can ask what the probabilty is that $\theta_p \in [\alpha, \beta]$.
The Sato-Tate Conjecture suggests that, for most elliptic curves, this probability follows a $\sin^2$ distribution.

\begin{conjecture}[Sato-Tate, \cite{silverman92}]
  Let $E$ be an elliptic curve over $\bb Q$ without complex multiplication.
  For any interval $[\alpha, \beta] \subseteq [0, \pi]$
  \[ \lim_{N \to \infty}
    \frac {\#\{ p \leq N : \alpha \leq \theta_p \leq \beta \}} {\pi(N)} =
    \frac 2 \pi \int_{\alpha}^{\beta} \sin^2 \theta\,d\theta, \]
  where $p$ is prime and $\pi(N)$ is the prime-counting function.
\end{conjecture}

This conjecture on the number of points on an elliptic curve $E$
can be made into a statement about the $L$-series of $E$.
\begin{comment}
The $L$-series of curves are central to many conjectures in number theory.
A complete discussion of $L$-series is outside the scope of this thesis;
we present only enough to motivate the topic of this thesis.
Let $E$ be an elliptic curve defined over the rational numbers $\bb Q$,
meaning $E$ is defined by a polynomial in $\bb Q[x,y]$ of the form of Equation \ref{eq_elliptic}.
\end{comment}
The \defn{$L$-series} of $E$ is defined as
\[ L(E/\bb Q, s)  = \prod_{p \text{ is prime}} \frac 1 {L_p \left( \frac 1 {p^s} \right)},\]
where, for each prime $p$, $L_p$ is a non-zero polynomial in $\bb Q[t]$ of degree at most 2.
For more on $L$-series, including a precise definition of the polynomials $L_p$,
see \cite{husemoller87}, \cite{milne06}, or \cite{silverman09}.

If $E/\bb Q$ has ``good reduction'' at a prime $p$
(see again \cite{husemoller87}, \cite{milne06}, or \cite{silverman09}),
then $L_p(1) = \#E(\bb F_p)$, which is the value of interest in the Sato-Tate Conjecture.

There are many other important number-theoretic conjectures related to the $L$-series of an elliptic curve.
One of the Clay Mathematics Institute's seven famous Millenium Prize Problems,
with a \$1,000,000 bounty on its head,
is the Birch and Swinnerton-Dyer Conjecture.

\begin{conjecture}[Birch and Swinnerton-Dyer, \cite{koblitz93}]
  Let $E$ be an elliptic curve over $\bb Q$.
  Then $L(E/\bb Q, s)$ has a zero at $s = 1$ of order equal to the rank of $E(\bb Q)$.
\end{conjecture}

In order to test this conjecture against a given curve $E/\bb Q$,
one may wish to compute $L(E, 1)$, which requires computing $L_p(\frac 1 p)$ at many primes $p$,
in turn requiring one to count points on many elliptic curves.

Hasse's Theorem has a generalization to higher genus curves
(elliptic curves have genus 1) that is easy to state.
\begin{theorem}[Hasse-Weil]
  Let $C$ be an algebraic curve of genus $g$ over a finite field $\bb F_q$.
  The number of points on $C$ differs from $q + 1$ by at most $2g \sqrt q$.
  That is,
  \[ | \#E(\bb F_q) - (q + 1) | \leq 2g \sqrt q. \]
\end{theorem}

The Sato-Tate Conjecture and the Birch and Swinnerton-Dyer conjectures
(and other $L$-series-related conjectures not mentioned above,
including the Koblitz-Zywina conjecture and Lang-Trotter conjecture)
also have generalizations to higher genus curves \cite{sutherland16-2} \cite{sutherland-18}.
However, on higher-genus curves, $L_p(1)$ is no longer the number of points on the curve,
but rather the order of the divisor class group of the curve \cite{sutherland08},
which is known to be a finite Abelian group.
(In the case of elliptic curves, the order of the divisor class group is
equal to the number of points on the curve.)

To compute the order of a finite Abelian group, one may first compute the order of an element of that group
using an algorithm such as Baby-step/Giant-step or Pollard-Rho.
Leveraging Lagrange's Theorem, the order of the element must divide the order of the group.
If one has a good estimate in advance for the order of the group, this may be enough
to compute this order.
The Baby-step/Giant-step and Pollard-Rho algorithms are general algorithms that
require that one have another algorithm to perform the group operations in the group
whose order one is compuring.
Thus, in order to compute the order of the divisor class group of a curve,
and ultimately to compute terms in the $L$-series of the curve,
we need an algorithm to efficiently carry out addition in the divisor class group.

The above-mentioned conjectures have been studied on genus 1 and genus 2 curves.
Attention has turned recently towards studying them on genus 3 curves.
Genus 3 algebraic plane curves fall into two categories: hyperelliptic and non-hyperelliptic.
Non-hyperelliptic genus 3 curves are also called $C_{3,4}$ curves, a special case of the broader family of $C_{a,b}$ curves.
Therefore, fast algorithms are needed to compute in the divisor class group of a $C_{3,4}$ curve.
That is the subject of this thesis.



%%%%%%%%%%%%%%%%%%%%%%%%%%
%%%%%                %%%%%
%%%%%   Prior Work   %%%%%
%%%%%                %%%%%
%%%%%%%%%%%%%%%%%%%%%%%%%%

\subsection{Prior Work}

Several authors have found explicit formulae describing the arithmetic of genus 3 hyperelliptic curves.
\note{Cite Sutherland, Rezai Rad.}
[SOURCE] has used this to compute Sato-Tate distributions for genus 3 HEC curves.
It is worth noting that general algorithms exist for computing in the Jacobian of an arbitrary curve
but these will be slower than explicit formulae.
\note{Cite Sutherlands results}

Turning now towards the state of $C_{3,4}$ curves...

Previous work in describing arithmetic in the divisor class group of a $C_{3,4}$ curve has been done with cryptographic applications in mind.
For instance, elliptic curve Diffie-Hellman key exchange and the elliptic curve digital signature algorithm (see \note{SOURCES})
may be modified to use $C_{3,4}$ curves rather than elliptic curves.
In the cryptographic setting, researchers have been interested primarily in $C_{3,4}$ curves defined over very large finite fields.
When working over such fields, they may assume that divisors satisfy certain generacity assumptions described in Chapter \ref{chap_representation}.
Under this assumption, they have designed algorithms to efficiently add divisors,
and when this assumption is violated, they may fall back on slower algorithms that work on any algebraic curve.

In \cite{makdisi04}, Khuri-Makdisi describes an algorithm for computing in the divisor class group of an arbitrary algebraic plane curve,
in which divisors are represented by projective embeddings of line bundles.
The runtime of this algorithm is polynomial in the genus $g$ of the curve, running in time $O(g^4)$.
In \cite{arita99}, Arita gives a $O(g^3)$ algorithm to compute in the divisor class group of an arbitrary $C_{a,b}$ curve,
where divisors are identified with polynomial ideals and represented by the Gr\"obner bases of these ideals.
In \cite{harasawa00}, Harasawa and Suzuki identify divisors with $K[x]$-modules rather than ideals and represent divisors by matrices in Hermite normal form,
achieving a $O(g^2)$ algorithm.
In \cite{hess99}, Hess gives a $O(g^2)$ algorithm for determining the structure of the class group of a function field 
\note{(say more on this)}.

This is only to highlight the choice of algorithms one might fall back on when the algorithms below fail.
The asymptotic runtime of these algorithms are of little interest when we are working with curves of small genus --
$C_{3,4}$ curves all have genus 3.

In \ref{arita05-2}, Arita specializes the algorithm from \ref{arita99} to the $C_{3,4}$ case.
He classifies divisors of $C_{3,4}$ curves into 19 types based on the forms of the Gr\"obner bases by which they are represented.
The algorithm presented allows one to add divisors of any type.
This is much more general than the algorithms to follow, however it is much slower and leaves some cases handled.
In terms of speed, it requires 5I 204M to add two divisors and 5I 284M to double a divisor.
It assumes that, when adding divisors $D + D'$, that $D$ and $D'$ are disjoint,
and that, when doubling a divisor $D + D$, that $D$ has no multiple points.
Should these assumptions be violated, the proposed solution is to select a random divisor $R$ and pre-add
\[  (D + R) + (D' + (-R)) \text { or } (D + R) + (D + (-R)), \]
depending on whether one is adding or doubling.
The idea is that, when working over a large finite field, $D + R$ and $D' + R$ are likely now disjoint,
and if not, then one may repeat the process until they are.
However, on some curves over small finite fields, this recursion is not guaranteed to terminate.

Other publications are less general but much faster.
In \cite{flon08}, Flon et al assume that the curve is defined by a short-form polynomial equation
(see Equations \ref{eq_c34} and \ref{eq_c34_short_form}).
On top of assuming that divisors are disjoint and have no multiple points, 
they assume that a divisor is in ``general position'', and mimic techniques from hyperelliptic curve arithmetic ---
they represent divisors in a manner similar to the Mumford representation used for divisors of hyperelliptic curves,
and follow an algorithm similar to Cantor's algorithm.
The result is an algorithm requiring 2I 148M 15SQ for addition and 2I 165M 20SQ for doubling.

In \cite{salem07}, Abu Salem and Khuri-Makdisi make the same assumptions as in \cite{flon08}.
They represent divisors by a pair of polynomials of minimal degree and compute sums of divisors
by computing kernels of maps between Riemann-Roch spaces and quotients of Riemann-Roch spaces.
This gives an algorithm requiring 2I 117M to add and 2I 129M to double.
In an appendix in \cite{makdisi18}, Khuri-Makdisi gives an improvement bringing the operation count down to 2I 98M and 2I 110M.

The goal of this thesis is to marry the methods of Abu Salem and Khuri-Makdisi --- who have the fastest results to date ---
with the methods of Arita --- who has the most general results ---
in order to produce fast explicit formulae describing all cases of $C_{3,4}$ curve arithmetic.
More specifically,
\begin{itemize}
  \item the curve equation may be over a finite field of any size, small or large;
  \item the curve equation may be over a finite field of any characteristic, including 2 and 3;
  \item the curve equation may be in long or short form;
  \item divisors may be typical or atypical;
  \item divisors may have multiple points;
  \item divisors may be non-disjoint;
  \item and algorithms must provably terminate.
\end{itemize}
This marriage of methods is facilitated by the fact that Salem/Khuri-Makdisi's representation of typical divisors resembles type 31 divisors from Arita's classification.

%%%%%%%%%%%%%%%%%%%%%%%%%
%%%%%               %%%%%
%%%%%   Old Stuff   %%%%%
%%%%%               %%%%%
%%%%%%%%%%%%%%%%%%%%%%%%%

\begin{comment}
\section{Old Stuff}

\note{Generalizing Sato-Tate to higher genus curves requires Sato-Tate groups.
The theory is outside the scope of this thesis.}

A projective line over $\bb F_q$ has $q + 1$ rational points.
An elliptic curve over $\bb F_q$ has approximately as many rational points as does a line.
The actual number of rational points on an elliptic curve differs from that of the line by an error term
that is small compared to the size of the field $\bb F_q$.
The following theorem due to Hasse places bounds on this error term.
\begin{theorem}[Hasse's Theorem]
  Let $E$ be an elliptic curve over a finite field $\bb F_q$.
  The number of points on $E$ differs from $q + 1$ by at most $2 \sqrt q$.
  \[ | \#E(\bb F_q) - (q + 1) | \leq 2 \sqrt q \]
\end{theorem}

This theorem generalizes in the following way.
\begin{theorem}[Hasse-Weil Bound]
  Let $C$ be a curve of genus $g$ over a finite field $\bb F_q$.
  The number of points on $C$ differs from $q + 1$ by at most $2 g \sqrt q$.
  \[ | \#C(\bb F_q) - (q + 1) | \leq 2 g \sqrt q \]
\end{theorem}

Expressed differently, the Hasse-Weil Bound says that the quantity
\begin{equation}
  \frac {\#C(\bb F_q) - (q + 1)}{2 g \sqrt q}
\end{equation}
is between -1 and 1, so that for some angle $\theta_p$,
\begin{equation}
  \label{eq_sato_tate_ratio}
 \cos \theta_p = \frac {\#C(\bb F_q) - (q + 1)}{2 g \sqrt q}.
\end{equation}

Now given a fixed curve $C$, one can ask, what the probability is as $p$ varies that the error term should lie in some interval.
Given real numbers $-1 \leq a < b \leq 1$ (or angles $0 \leq \alpha < \beta \leq \pi$),
what is the probability that the ratio $\cos \theta_p \in [a, b]$ (or $\theta_p \in [\alpha, \beta]$)?
The Sato-Tate Conjecture (proven for genus 1), says that this probability follows a $\sin^2$ distribution.

\begin{conjecture}[Sato-Tate]
  Let $E$ be an elliptic curve without complex multiplication.
  \note{Complete this thought...}
  \[ \lim_{N \to \infty} \frac {\#\{ p \leq N : \alpha \leq \theta_p \leq \beta \}} {\pi(N)} = \frac 2 \pi \int_{\alpha}^{\beta} \sin^2 \theta\,d\theta. \]
\end{conjecture}

Divisor class group computations are used to compute the L-series of a genus 3 curve.
L-series is raised in
  * Birch and Swinnerton-Dyer conjecture
  * Koblitz-Zywina conjecture
  * Lang-Trotter conjecture
  * Sato-Tate conjecture
\end{comment}

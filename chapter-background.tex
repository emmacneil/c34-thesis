
%%%%%%%%%%%%%%%%%%%%%%%%%%
%%%%%                %%%%%
%%%%%   Background   %%%%%
%%%%%                %%%%%
%%%%%%%%%%%%%%%%%%%%%%%%%%

\section{Introduction}

This introduction will be light on definitions.
Full details of the topics mentioned may be found later in this thesis or in other cited works.
For now, it should be enough to know that an \defn{algebraic plane curve} is the set of points $(x,y)$
at which $f(x, y) = 0$, for some field $K$ and polynomial $f \in K[x,y]$.
Examples of algebraic plane curves include elliptic curves, hyperelliptic curves,
and the subject of this thesis, $C_{3,4}$ curves.
An \defn{elliptic curve} is the set of zeroes of a polynomial
\begin{equation}
  \label{eq_elliptic}
  y^2 + x^3 + c_4xy + c_3x^2 + c_2y + c_1x + c_0.
\end{equation}
A (ramified genus 3) \defn{hyperelliptic curve} is the set of zeroes of a polynomial
\begin{equation}
  \label{eq_genus_3_hyperelliptic}
  y^2 + x^7 + c_{10}x^3y + c_9x^6 + c_8x^2y + c_7x^5 + c_6xy + c_5x^4 + c_4y + c_3x^3 + c_2x^2 + c_1x + c_0,
\end{equation}
though hyperelliptic curves can be of any genus 2 or greater and come in a variety of flavours (ramified, split, inert).
A \defn{$C_{3,4}$ curve} is the set of zeroes of a polynomial
\begin{equation}
  \label{eq_c34_intro}
  y^3 + x^4 + c_8xy^2 + c_7x^2y + c_6x^3 + c_5y^2 + c_4xy + c_3x^2 + c_2y + c_1x + c_0.
\end{equation}
For each of these classes of curves, one might also demand that they be non-singular
\note{(put a definition of singular in chapter 4 and refer to it here)},
as we will do in Chapter \ref{chap_curves}.

If $C$ is an algebraic plane curve, a \defn{divisor} of a curve is a formal sum of points on $C$.
If $P$ and $Q$ are points on $C$, then $2P$ and $P - Q$ are examples of divisors on $C$.
In Chapter \ref{chap_divisors}, we will place an equivalence relation on divisors,
partitioning them into \defn{divisor classes}.
Together with an addition operation, this forms the \defn{divisor class group} of $C$.



%%%%%%%%%%%%%%%%%%%%%%%%%%
%%%%%                %%%%%
%%%%%   Motivation   %%%%%
%%%%%                %%%%%
%%%%%%%%%%%%%%%%%%%%%%%%%%

\subsection{Motivation}

Let $\bb F_q$ be the finite field of order $q$ and suppose $E$ is an elliptic curve defined over $\bb F_q$,
meaning that $E$ is the set of zeroes of a polynomial in $\bb F_q[x,y]$ of the form in Equation \ref{eq_elliptic}.
The following theorem, due to Helmut Hasse (1898 -- 1979), says that
the number of rational points (i.e. points in $\bb F_q \times \bb F_q$) on $E$
lies in some interval that is small compared to the order of $\bb F_q$.
\begin{theorem}[Hasse]
  Let $E$ be an elliptic curve over a finite field $\bb F_q$.
  The number of rational points on $E$ differs from $q + 1$ by at most $2 \sqrt q$. That is,
  \[ | \#E(\bb F_q) - (q + 1) | \leq 2 \sqrt q. \]
\end{theorem}
In other words,
$q + 1$ is a good estimate for the number of rational points on $E$,
especially when $q$ is large.

\begin{example}
  Let $p = 2^{31} - 1 = 2,147,483,647$, which is prime.
  The elliptic curve $y^2 + x^3 - 1 = 0$ over $\bb F_p[x,y]$ has 2,147,391,324 rational points.
\end{example}

Let $a_q = \frac {\#E(\bb F_q) - (q + 1)} {2\sqrt q}$.
Then Hasse's Theorem says that $a_p \in [-1, 1]$.
Letting $\theta_q = \arccos a_q$, then $\theta_q \in [0, \pi]$.
Given a subinterval $[\alpha, \beta] \subseteq [0, \pi]$ and a prime $p$,
one can ask what the probabilty is that $\theta_p \in [\alpha, \beta]$.
The Sato-Tate Conjecture suggests that, for most elliptic curves, this probability follows a $\sin^2$ distribution.

\begin{conjecture}[Sato-Tate, \cite{silverman92}]
  Let $E$ be an elliptic curve over $\bb Q$ without complex multiplication.
  For any interval $[\alpha, \beta] \subseteq [0, \pi]$
  \[ \lim_{N \to \infty}
    \frac {\#\{ p \leq N : \alpha \leq \theta_p \leq \beta \}} {\pi(N)} =
    \frac 2 \pi \int_{\alpha}^{\beta} \sin^2 \theta\,d\theta, \]
  where $p$ is prime and $\pi(N)$ is the prime-counting function.
\end{conjecture}

This conjecture on the number of points on an elliptic curve $E$
can be made into a statement about the $L$-series of $E$.
\begin{comment}
The $L$-series of curves are central to many conjectures in number theory.
A complete discussion of $L$-series is outside the scope of this thesis;
we present only enough to motivate the topic of this thesis.
Let $E$ be an elliptic curve defined over the rational numbers $\bb Q$,
meaning $E$ is defined by a polynomial in $\bb Q[x,y]$ of the form of Equation \ref{eq_elliptic}.
\end{comment}
The \defn{$L$-series} of $E$ is defined as
\[ L(E/\bb Q, s)  = \prod_{p \text{ is prime}} \frac 1 {L_p \left( \frac 1 {p^s} \right)},\]
where, for each prime $p$, $L_p$ is a non-zero polynomial in $\bb Q[t]$ of degree at most 2.
For more on $L$-series, including a precise definition of the polynomials $L_p$,
see \cite{husemoller87}, \cite{milne06}, or \cite{silverman09}.

If $E/\bb Q$ has ``good reduction'' at a prime $p$
(see again \cite{husemoller87}, \cite{milne06}, or \cite{silverman09}),
then $L_p(1) = \#E(\bb F_p)$, which is the value of interest in the Sato-Tate Conjecture.

There are many other important number-theoretic conjectures related to the $L$-series of an elliptic curve.
One of the Clay Mathematics Institute's seven famous Millenium Prize Problems,
with a \$1,000,000 bounty on its head,
is the Birch and Swinnerton-Dyer Conjecture.

\begin{conjecture}[Birch and Swinnerton-Dyer, \cite{koblitz93}]
  Let $E$ be an elliptic curve over $\bb Q$.
  Then $L(E/\bb Q, s)$ has a zero at $s = 1$ of order equal to the rank of $E(\bb Q)$.
\end{conjecture}

In order to test this conjecture against a given curve $E/\bb Q$,
one may wish to compute $L(E, 1)$, which requires computing $L_p(\frac 1 p)$ at many primes $p$,
in turn requiring one to count points on many elliptic curves.

Hasse's Theorem has a generalization to higher genus curves
(elliptic curves have genus 1) that is easy to state.
\begin{theorem}[Hasse-Weil]
  Let $C$ be an algebraic curve of genus $g$ over a finite field $\bb F_q$.
  The number of points on $C$ differs from $q + 1$ by at most $2g \sqrt q$.
  That is,
  \[ | \#E(\bb F_q) - (q + 1) | \leq 2g \sqrt q. \]
\end{theorem}

The Sato-Tate Conjecture and the Birch and Swinnerton-Dyer conjectures
(and other $L$-series-related conjectures not mentioned above,
including the Koblitz-Zywina conjecture and Lang-Trotter conjecture)
also have generalizations to higher genus curves \cite{sutherland16} \cite{sutherland18}.
However, on higher-genus curves, $L_p(1)$ is no longer the number of points on the curve,
but rather the order of the divisor class group of the curve \cite{kedlaya08},
which is known to be a finite Abelian group.
(In the case of elliptic curves, the order of the divisor class group is
equal to the number of points on the curve.)

To compute the order of a finite Abelian group, one may first compute the order of an element of that group
using an algorithm such as Baby-step/Giant-step or Pollard-Rho.
Leveraging Lagrange's Theorem, the order of the element must divide the order of the group.
If one has a good estimate in advance for the order of the group, this may be enough
to compute this order.
The Baby-step/Giant-step and Pollard-Rho algorithms are general algorithms that
require that one have another algorithm to perform the group operations in the group
whose order one is compuring.
Thus, in order to compute the order of the divisor class group of a curve,
and ultimately to compute terms in the $L$-series of the curve,
we need an algorithm to efficiently carry out addition in the divisor class group.

The above-mentioned conjectures have been studied on genus 1 and genus 2 curves.
Attention has turned recently towards studying them on genus 3 curves.
Genus 3 algebraic plane curves fall into two categories: hyperelliptic and non-hyperelliptic.
Non-hyperelliptic genus 3 curves are also called $C_{3,4}$ curves, a special case of the broader family of $C_{a,b}$ curves.
Therefore, fast algorithms are needed to compute in the divisor class group of a $C_{3,4}$ curve.

The goal of this thesis is find such algorithms.
More accurately, the goal of this thesis is to find \defn{explicit formulae} describing arithmetic in a $C_{3,4}$ curve's divisor class group.
In brief, a divisor class $[D]$ will be typically be represented by a triple of polynomials $\pid{f,g,h}$.
To find explicit formulae for the sum $[D''] = [D] + [D']$ is to find formulae expressing the coefficients of the
polynomials representing $[D'']$ in terms of the coefficients representing $[D]$ and $[D']$.
To compare the efficiency of our formulae in comparison to earlier works,
we will count the number of finite field inversions, multiplications, and additions needed to evaluate these formulae.



%%%%%%%%%%%%%%%%%%%%%%%%%%
%%%%%                %%%%%
%%%%%   Prior Work   %%%%%
%%%%%                %%%%%
%%%%%%%%%%%%%%%%%%%%%%%%%%

\subsection{Prior Work}

We briefly highlight prior work on genus 3 hyperelliptic curves.
For ramified genus 3 hyperelliptic curves, the most efficient published explicit formulae
are due to Nyukai \cite{nyukai06-1} \cite{nyukai06-2},
who adds divisors in 1I+67M (1 finite field inversion and 67 multiplications) and doubles divisors in 1I+68M.
For split genus 3 hyperelliptic curves, Sutherland \cite{sutherland18} recently published explicit formulae
requiring 1I+79M/1I+82M to add/double.
Rezai Rad \cite{rezairad16} \cite{rezairad19} found comparably fast explicit formulae to add/double
in the infrastructure of a hyperelliptic curve in 1I+75M/1I+86M.
The relationship between the divisor class group and the infrastructure is explored in \cite{rezairad16} and \cite{rezairad19}.
Faster hyperelliptic curve arithmetic remains a subject of ongoing research at the Unviersity of Calgary.

In \cite{kedlaya08} and \cite{harvey16}, Kedlaya, Harvey, Massierer, and Sutherland use this arithmetic to compute $L$-series of hyperelliptic curves of genus $g \leq 3$.
Sutherland notes in \cite{sutherland18} that in earlier attempts at computing $L$-series using generic divisor arithmetic algorithms (i.e. not using explicit formulae),
the majority of the computation time was spent adding and doubling divisors.
After moving to optimized, explicit formulae specific to genus 3 hyperelliptic curves, the time spend adding and doubling divisors became ``negligible''.
The benefit of finding explicit formulae describing divisor arithmetic is clear.

Turning now towards the state of $C_{3,4}$ curves, previous work in $C_{3,4}$ curve divisor class group arithmetic
has been done with cryptographic applications in mind, as this group may be used in cryptographic primitives based on the Discrete Log Problem.
In the cryptographic setting, researchers are interested primarily in $C_{3,4}$ curves defined over very large finite fields
of characteristic greater than 3.
A $C_{3,4}$ curve over such a field is isomorphic to one given by a short-form equation (see Equation \ref{eq_c34_short}), yielding faster arithmetic.
Moreover, with very high probability, one will only encounter ``typical'' divisors (see Chapter \ref{chap_representation})
and many degenerate cases need not be considered.
When these assumptions are violated, one may fall back on slower divisor addition algorithms that work on any algebraic curve.

In \cite{kmakdisi04}, Khuri-Makdisi describes an algorithm for computing in the divisor class group of an arbitrary algebraic plane curve,
in which divisors are represented by projective embeddings of line bundles.
The runtime of this algorithm is polynomial in the genus $g$ of the curve, running in time $O(g^4)$.
In \cite{arita99}, Arita gives a $O(g^3)$ algorithm to compute in the divisor class group of an arbitrary $C_{a,b}$ curve,
where divisors are identified with polynomial ideals and represented by the Gr\"obner bases of these ideals.
In \cite{harasawa00}, Harasawa and Suzuki identify divisors with $K[x]$-modules rather than ideals and represent divisors by matrices in Hermite normal form,
achieving a $O(g^2)$ algorithm.
In \cite{hess99}, Hess gives a $O(g^2)$ algorithm for determining the structure of the class group of a function field 
\note{(say more on this)}.

This is only to highlight the choice of algorithms one might fall back on when the algorithms below fail.
The asymptotic runtime of these algorithms are of little interest when we are working with curves of small genus ---
$C_{3,4}$ curves all have genus 3.

In \cite{arita05-2}, Arita specializes the algorithm from \cite{arita99} to the $C_{3,4}$ case.
He classifies divisors of $C_{3,4}$ curves into 19 types based on the forms of the Gr\"obner bases by which they are represented.
The algorithm presented allows one to add divisors of any type.
This is much more general than the algorithms to follow, requiring 5I+204M/5I+284M.
\note{Shorten this paragraph.}
Unlike the algorithms that follow, it allows one to add non-disjoint divisors (e.g. $(P + Q) + (Q + R)$)
or double divisors with multiples of a point (e.g. doubling $P + 2Q$),
although it handles this in a recursive manner that does not terminate over very small finite fields;
Arita was interested in the cryptographic setting over a large finite field.

Other publications are less general but much faster.
In \cite{flon08}, Flon \emph{et al.} assume a $C_{3,4}$ curve defined by a short-form polynomial equation
(see Equations \ref{eq_c34} and \ref{eq_c34_short}).
In addition to assuming that divisors are disjoint and have no multiple points, 
they assume that divisors being added or doubled are typical and mimic techniques from hyperelliptic curve arithmetic ---
they represent divisors in a manner similar to the Mumford representation used for divisors of hyperelliptic curves,
and follow an algorithm similar to Cantor's algorithm (see \S 10.3 in \cite{galbraith12}).
The result is an algorithm requiring 2I+148M+15SQ/2I+165M+20SQ, where SQ denotes squaring in a finite field.

In \cite{salem07}, Abu Salem and Khuri-Makdisi make the same assumptions as in \cite{flon08}.
They represent divisors by a pair of polynomials of minimal degree and compute sums of divisors
by computing kernels of maps between vector spaces in 2I+117M/2I+129M.
In an appendix in \cite{kmakdisi18},
Khuri-Makdisi gives an improvement bringing the operation count down to 2I+98M/2I+110M.

The goal of this thesis is to marry the methods of Abu Salem and Khuri-Makdisi ---
who have the fastest explicit formulae to date ---
with the methods of Arita --- whose formulae are the most general ---
in order to produce fast and fully general explicit formulae describing all cases of $C_{3,4}$ curve arithmetic.
More specifically,
\begin{itemize}
  \item the curve equation may be over a finite field of any size, small or large;
  \item the curve equation may be over a finite field of any characteristic, including 2 and 3;
  \item the curve equation may be in long or short form;
  \item divisors may be typical or atypical;
  \item divisors may have multiple points;
  \item divisors may be non-disjoint;
  \item and algorithms must provably terminate.
\end{itemize}
This marriage of methods is facilitated by the fact that Salem/Khuri-Makdisi's representation of typical divisors resembles type 31 divisors from Arita's classification.



%%%%%%%%%%%%%%%%%%%%%%%
%%%%%             %%%%%
%%%%%   Outline   %%%%%
%%%%%             %%%%%
%%%%%%%%%%%%%%%%%%%%%%%

\subsection{Thesis Outline}

We begin by reviewing some background material necessary for understanding this thesis.
Familiarity with common algebraic structures ---
including monoids, groups, rings, fields, modules, and vector spaces --- is assumed.
In chapter 2, we review Gr\"obner bases,
which are central to Arita's representation of divisors,
which we have adopted here.
In chapter 3, we review differentials, which arise when doubling divisors.

Next, we delve into theory more specific to $C_{3,4}$ curves.
In chapter 4, we introduce algebraic curves and define $C_{3,4}$ curves,
as well as some related objects such as the coordinate ring and function field.
In chapter 5, we introduce divisors on curves and give some basic properties of the divisor class group.
In chapter 6, we introduce the ideal class group and exhibit an isomorphism between the divisor class group and the ideal class group.
We use this isomorphism in chapter 7 to represent divisors by ideals, in turn represented by Gr\"obner bases.
We then define what it means for a divisor to be reduced or typical
and prove several properties of these divisors.

Beginning with chapter 8, we present algorithms to operate in the divisor class group of a $C_{3,4}$ curve.
Chapters 8, 9, and 10 describe algorithms for adding, doubling, and reducing divisor classes, respectively.
In chapter 11, we combine addition and reduction into a single add-reduce operation
as well as doubling and reduction into single double-reduce operation.
Combining these operations saves an inversion and several multiplications and additions,
giving an improvement over the current state-of-the-art in $C_{3,4}$ curve arithmetic.

The results and contributions of this thesis are summarized in the concluding chapter 12.


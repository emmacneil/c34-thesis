
%%%%%%%%%%%%%%%%%%%%%%%%%%
%%%%%                %%%%%
%%%%%   Background   %%%%%
%%%%%                %%%%%
%%%%%%%%%%%%%%%%%%%%%%%%%%

\section{Background and Motivation}

$L$-series are central to many conjectures in number theory.
A complete discussion of $L$-series is outside the scope of this thesis;
we present only enough to motivate the topic of this thesis.

Let $E$ be an elliptic curve defined over the rational numbers $\bb Q$.
The $L$-series of $E$ is
\[ L(E/\bb Q, s)  = \prod_{p \text{ is prime}} \frac 1 {L_p \left( \frac 1 {p^s} \right)},\]
where, for each prime $p$, $L_p$ is a non-zero polynomial of degree at most 2.
\note{See Silverman, or Milne for more, including a definition of $L_p$.}

By Hasse's Theorem, the number of points on an elliptic curve defined over a finite field
lies in some interval that is small compare to the size of the finite field.
\begin{theorem}[Hasse]
  Let $E$ be an elliptic curve over a finite field $\bb F_q$.
  The number of points on $E$ differs from $q + 1$ by at most $2 \sqrt q$.
  That is,
  \[ | \#E(\bb F_q) - (q + 1) | \leq 2 \sqrt q. \]
\end{theorem}

Rewritten slightly, this theorem says that
  \[ \frac {\#E(\bb F_q) - (q + 1)} {\sqrt q} \in [-2, 2]. \]
The difference between $q + 1$ and the number of rational points on the curve
lies in the interval $[-2 \sqrt q, 2 \sqrt q]$.
The scaled difference is in the interval $[-2, 2]$.
Now given a randomly chosen prime $p$,
one can ask what the probability that this scaled difference lies in some sub-interval of $[-2, 2]$.
The Sato-Tate Conjecture says that this probability follows a $\sin^2$ distribution.
\begin{conjecture}[Sato-Tate]
  Let $E$ be an elliptic curve without complex multiplication.
  \note{Complete this conjecture}
  \[ \lim_{N \to \infty}
    \frac {\#\{ p \leq N : \alpha \leq \theta_p \leq \beta \}} {\pi(N)} =
    \frac 2 \pi \int_{\alpha}^{\beta} \sin^2 \theta\,d\theta. \]
\end{conjecture}

If $E/\bb Q$ has good reduction at a prime $p$
\note{(see some source on what this means)},
then $L_p(1) = \#E(\bb F_p)$,
so that the Sato-Tate Conjecture can be seen as a statement about the number of points on a curve
or about the value of a term in an $L$-series.

One of the Clay Mathematics Institutes seven famous Millenium Prize Problems
is the Birch and Swinnerton-Dyer Conjecture.

\begin{conjecture}[Birch and Swinnerton-Dyer]
  Let $E$ be an elliptic curve over $\bb Q$.
  Then $L(E/\bb Q, s)$ has a zero at $s = 1$ of order equal to the rank of $E(\bb Q)$.
\end{conjecture}

In order to test this conjecture against a given curve $E/\bb Q$,
one may wish to compute $L(E, 1)$, which is to compute $L_p$ at many primes $p$
\note{(Verify).}
This amounts to counting points on many elliptic curves.

Hasse's Theorem has a generalization to higher genus curves that is easy to state.
\begin{theorem}[Hasse-Weil]
  Let $C$ be an algebraic curve of genus $g$ over a finite field $\bb F_q$.
  The number of points on $E$ differs from $q + 1$ by at most $2g \sqrt q$.
  That is,
  \[ | \#E(\bb F_q) - (q + 1) | \leq 2g \sqrt q. \]
\end{theorem}

The Sato-Tate Conjecture and the Birch and Swinnerton-Dyer conjectures
also have generalizations to higher genus curves.
One can reformulate these as conjectures on the $L$-series of more complex curves,
in which case $L_p(1)$ is no longer the number of points on the curve,
but rather the order of the Jacobian, or divisor class group, of the curve,
which is known to be a finite Abelian group.
In the case of elliptic curves, the order of the divisor class group is
equal to the number of points on the curve.

To compute the order of a finite Abelian group, one may compute the order of an element,
using an algorithm such as Baby-step/Giant-step or Pollard-Rho.
Leveraging Lagrange's Theorem, the order of the element must divide the order of the group.
If one has a good estimate in advance for the order of the group, this may be enough
to compute this order.
The Baby-step/Giant-step and Pollard-Rho algorithms are general algorithms that
require that one have another algorithm to perform the group operations in the group
whose order one is compuring.
Thus, in order to compute the order of the divisor class group of a curve,
and ultimately to compute terms in the $L$-series of the curve,
we need an algorithm to efficiently carry out addition in the divisor class group.
That is the subject of this thesis.




%%%%%%%%%%%%%%%%%%%%%%%%%
%%%%%               %%%%%
%%%%%   Old Stuff   %%%%%
%%%%%               %%%%%
%%%%%%%%%%%%%%%%%%%%%%%%%

\section{Old Stuff}

\note{Generalizing Sato-Tate to higher genus curves requires Sato-Tate groups.
The theory is outside the scope of this thesis.}

A projective line over $\bb F_q$ has $q + 1$ rational points.
An elliptic curve over $\bb F_q$ has approximately as many rational points as does a line.
The actual number of rational points on an elliptic curve differs from that of the line by an error term
that is small compared to the size of the field $\bb F_q$.
The following theorem due to Hasse places bounds on this error term.
\begin{theorem}[Hasse's Theorem]
  Let $E$ be an elliptic curve over a finite field $\bb F_q$.
  The number of points on $E$ differs from $q + 1$ by at most $2 \sqrt q$.
  \[ | \#E(\bb F_q) - (q + 1) | \leq 2 \sqrt q \]
\end{theorem}

This theorem generalizes in the following way.
\begin{theorem}[Hasse-Weil Bound]
  Let $C$ be a curve of genus $g$ over a finite field $\bb F_q$.
  The number of points on $C$ differs from $q + 1$ by at most $2 g \sqrt q$.
  \[ | \#C(\bb F_q) - (q + 1) | \leq 2 g \sqrt q \]
\end{theorem}

Expressed differently, the Hasse-Weil Bound says that the quantity
\begin{equation}
  \frac {\#C(\bb F_q) - (q + 1)}{2 g \sqrt q}
\end{equation}
is between -1 and 1, so that for some angle $\theta_p$,
\begin{equation}
  \label{eq_sato_tate_ratio}
 \cos \theta_p = \frac {\#C(\bb F_q) - (q + 1)}{2 g \sqrt q}.
\end{equation}

Now given a fixed curve $C$, one can ask, what the probability is as $p$ varies that the error term should lie in some interval.
Given real numbers $-1 \leq a < b \leq 1$ (or angles $0 \leq \alpha < \beta \leq \pi$),
what is the probability that the ratio $\cos \theta_p \in [a, b]$ (or $\theta_p \in [\alpha, \beta]$)?
The Sato-Tate Conjecture (proven for genus 1), says that this probability follows a $\sin^2$ distribution.

\begin{conjecture}[Sato-Tate]
  Let $E$ be an elliptic curve without complex multiplication.
  \note{Complete this thought...}
  \[ \lim_{N \to \infty} \frac {\#\{ p \leq N : \alpha \leq \theta_p \leq \beta \}} {\pi(N)} = \frac 2 \pi \int_{\alpha}^{\beta} \sin^2 \theta\,d\theta. \]
\end{conjecture}

Divisor class group computations are used to compute the L-series of a genus 3 curve.
L-series is raised in
  * Birch and Swinnerton-Dyer conjecture
  * Koblitz-Zywina conjecture
  * Lang-Trotter conjecture
  * Sato-Tate conjecture

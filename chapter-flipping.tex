%%%%%%%%%%%%%%%%%%%%%%%%%%%%%%%%%
%%%%%                       %%%%%
%%%%%   Flipping Divisors   %%%%%
%%%%%                       %%%%%
%%%%%%%%%%%%%%%%%%%%%%%%%%%%%%%%%

\section{Flipping Divisors}
\label{chap_flipping}

Let $D$ be a divisor, and $I_D$ its ideal.
To find a representative of the ideal class $[-D]$ is to compute the quotient of $p : I_D$ for any polynomial $p \in I_D$.
A different choice of $p$ results in a different representative of $[-D]$.
Choosing a $p$ of minimal order results in a reduced representative of $[-D]$.

The ideal $I_D$ is generated by a reduced Gr\"obner basis of at most three elements.
Sometimes, $I_D$ is a principal ideal, generated by a single polynomial $f$.
In this case, the negation of $D$ is
  \[ f : I_D = f : f = \pid{1}. \]
In other cases, $I_D$ has two generators, $I_D = \pid{f,g}$, and $f$ is monic and of minimal order.
In this case, to negate $D$ is to compute
  \[ f : I_D = f : \pid{f,g} = f : g. \]
The remaining case is when $I_D$ has three generators, $I_D = \pid{f,g,h}$, and $f$ is monic and of minimal order.
We compute
  \[ f : I_D = f : \pid{f,g,h} = (f : g) \cap (f : h). \]
The case where $I_D$ is principal is trivial, while the other cases boil down to computing ideals of the form $f : g$.
In this section, we discuss how to compute this.

Recall that (\note{from an earlier section}), geometrically, $A = \div(f : g)$ consists of those points on $C$ intersected by $f$, but not by $g$, counting multiplicities.
This allows us to compute the degree of $A$:
\begin{equation}
  \label{eq_flip_degree}
  \deg(\div(f : g)) = \deg(\div f) - \deg(\div(f,g)).
\end{equation}
Moreover, $A$ is reduced (\note{result also in an earlier section}).
Therefore, if the formula in (\ref{eq_flip_degree}) evaluates to 1 or 3, then $A$ is of type 11 or 31, respectively.
If it evaluates to 2, then $A$ is of type 21 or 22.
We do not need to guess which type it will be -- it is determined by the type of $D$.
Below is an example of determining the type of $A$ when $\deg A = 2$.
\begin{example}
  Suppose $D$ is of type 62.
  Then $D = \pid{f,g}$ is generated by two polynomials, with $\LM(f) = y^2$ and $\LM(g) = x^3$.
  Consider the polynomial
    \[ yf + xg - F = (f_4 - c_8)xy^2 + \dots \in I_D. \]
  Its leading monomial is divisible by $\LT(f)$, so we can subtract away an appropriate multiple of $f$ to get
    \[ (y - (f_4 - c_8)x)f + xg - F = (f_3 - f_4(f_4 - c_8) + g_4 - c_7)x^2y + \dots \in I_D. \]
  We must have $(f_3 - f_4(f_4 - c_8) + g_4 - c_7) = 0$, since $x^2y \not\in \LT(I_D)$.
  Subtracting away appropriate multiples of $f$ and $g$ to remove the $x^3$ and $y^2$ terms in this polynomial
  leaves a polynomial with leading monomial lesser than $y^2$.
  However, there is no monomial smaller than $y^2$ in $\LT(I_D)$, so we must be left with zero.
  That is, we have found polynomials $s = y + s_1x + s_0$ and $t = x + t_0$ such that $sf + tg - F = 0$.
  In particular, $tg \in \pid{f}$, so $t \in f : g$.
  Having found a polynomial with leading monomial $x$ in $f : g$, $A$ must be of type 22.
\end{example}

Given the type of $D$, the table below gives the type of its flip $A$.
\begin{table}
  \label{table_divisor_flips}
  \caption{Divisor types and their flips}
  \begin{center}
  \begin{tabular}{|c|ccccccccccc|}
    \hline
    Type of $D$ & 0 & 11 & 21 & 22 & 31 & 32 & 33 & 41 & 42 & 43 & 44 \\
    \hline
    Type of $A$ & 0 & 22 & 21 & 11 & 31 & 11 & 0  & 31 & 22 & 21 &  0 \\
    \hline \hline
    Type of $D$ & 51 & 52 & 53 & 54 & 61 & 62 & 63 & 64 & 65 \\
    \hline
    Type of $A$ & 31 & 22 & 21 & 11 & 31 & 22 & 21 & 11 & 0 \\
    \hline
  \end{tabular}
  \end{center}
\end{table}

Once the type of $A$ is known,
we know the forms of the polynomials in its reduced Gr\"obner basis.
Let $V$ be the set of of monomials found among that polynomial basis.
For instance, if $A$ is of type 31, then $V = \pid{1, x, y, x^2, xy, y^2}$.
If $A$ is of type 22, then $V = \pid{1, x, y, y^2}$.
Our goal is to find $K$-linear combinations on $V$ that, when multiplied by $g$ then reduced modulo $f$, give 0.

Among the monomials in $V$, let $m$ be the largest of them.
Let $N = -v_{P_\infty}(mg)$.
Multiplying monomials in $V$ by $g$ will result in polynomials of order at most $N$.
Let $n = N - v_{P_\infty}(f)$.
Multiplying $f$ by monomials of order at most $n$ will produce polynomials of order at most $N$.
Now we may operate in subspaces of $W^N$.
To compute a basis for $f : g$ is to compute the kernel of the composed map $M$ in
\[ \begin{tikzcd}
  V \arrow["\cdot g"]{r} \arrow[bend right,swap,"M"]{rrr} & gV \arrow["\iota"]{r} & W^N \arrow["\pi"]{r} & \frac{W^N}{fW^n}
\end{tikzcd} \]
where the individual maps, from left to right, are multiplication by $g$, canonical injection, and canonical projection.

This matrix $M$ will not have full rank.
Typically, it will have $-v_{P_\infty}(f)$ rows (as many as 9 rows for a type 61 divisor),
but it will have rank 2 or 3.
Knowing that most of the rows of the matrix will be linearly dependent,
we save ourselves some work by pre-emptively quotienting out by a subspace of $W^N$,
thereby eliminating some rows.

Let us modify the above diagram.
Let $\ell = -v_{P_\infty}(g) - 1$ in
\[ \begin{tikzcd}
    &    & \frac{W^N}{fW^n} \arrow[bend left,"\pi"]{ddd} \\
  V \arrow{r} \arrow[bend left,"M_1"]{rru} \arrow[bend right,swap,"M_2"]{rrdd} & gV \arrow{r} & W^N \arrow{d} \arrow{u} \\
    &    & \frac{W^N}{W^\ell} \arrow{d} \\
    &    & \frac{W^N}{W^\ell + fW^n}
\end{tikzcd} \]
\note{I claim that this diagram commutes, i.e. $\pi \circ M_1 = M_2$.}
\begin{lemma}
  $\ker M_1 = \ker M_2$.
\end{lemma}
\begin{proof}
  ($\ker M_1 \subseteq \ker M_2$)
  Let $p \in \ker M_1$. Then
    \[ M_2(p) = \pi(M_1(p)) = \pi(0) = 0, \]
  so $p \in \ker M_2$.

  ($\ker M_2 \subseteq \ker M_1$)
  Let $p \in \ker M_2$.
  If $p = 0$, then $p \in \ker M_1$, so suppose that $p \neq 0$.
  Then $gp \in W^\ell + fW^n$.
  So $gp = r + fs$ for some $r \in W^\ell$ and $s \in W^n$.
  Written otherwise, we have $gp - fs = r$.
  However, $-v_{P_\infty}(r) = \ell < -v_{P_\infty}(g)$, and there is no $K[C]$-linear combination of $g$ and $f$ a polynomial smaller than $g$.
  So $r = 0$, $gp = fs \in fW^8$, and $M_1(p) = 0$.
\end{proof}



%%%%%%%%%%%%%%%%%%%%%%%%%%%%%%%%%%%%%%%%%%%%%%%%%
%%%%%                                       %%%%%
%%%%%   Flipping Typical Type 61 Divisors   %%%%%
%%%%%                                       %%%%%
%%%%%%%%%%%%%%%%%%%%%%%%%%%%%%%%%%%%%%%%%%%%%%%%%

\subsection{Flipping Typical Type 61 Divisors}

Let $D$ be a typical degree 6 divisor of type 61.
So $I_D$ is given by the Gr\"obner basis $\pid{f,g,h}$, where
\begin{align*}
  f  &= x^3  + f_5y^2 + f_4xy + f_3x^2 + f_2y + f_1x + f_0 \\
  g  &= x^2y + g_5y^2 + g_4xy + g_3x^2 + g_2y + g_1x + g_0 \\
  h  &= xy^2 + h_5y^2 + h_4xy + h_3x^2 + h_2y + h_1x + h_0.
\end{align*}
Since $D$ is typical, $I_D = \pid{f,g}$, although $f$ and $g$ alone do not form a Gr\"obner basis.
To compute the flip $A$ of $D$, we compute the quotient ideal
  \[ f : I_D = f : \pid{f,g,h} = f : \pid{f,g} = f : g. \]

By Table \ref{table_divisor_flips}, $A$ is of type 31.
So $I_A$ is given by a Gr\"obner basis $\pid{f',g',h'}$, where
\begin{align*}
  f' &= x^2 + f_2y + f_1x + f_0 \\
  g' &=  xy + g_2y + g_1x + g_0 \\
  h' &= y^2 + h_2y + h_1x + h_0.
\end{align*}
These polynomials are made solely from the monomials $\{1, x, y, x^2, xy, y^2\}$.
However, since $D$ is typical, $A$ is also typical \note{(place this result somewhere)},
therefore generated by $f'$ and $g'$.
So we will therefore first find $f'$ and $g'$ and compute $h'$ afterwards.
\note{Computing $h'$ in this way is faster than the alternative.}

The monomials making up $f'$ and $g'$ are $\{1, x, y, x^2, xy\}$.
The space generated by these monomials is $V = W^7$.
We are looking for polynomials $p \in W^7$ such that $gp$ vanishes modulo $f$.
As per the previous section, this is the kernel of the map $M$ in
\[ \begin{tikzcd}
  W_A^7 \arrow[equal]{r} \arrow[leftrightarrow,swap,"\rotatebox{90}{$\simeq$}"]{d}
    & \ker M \arrow[hook]{r} \arrow[leftrightarrow,swap,"\rotatebox{90}{$\simeq$}"]{d}
    & W^7 \arrow["\simeq"]{r} \arrow[leftrightarrow,swap,"\rotatebox{90}{$\simeq$}"]{d}
    & gW^7 \arrow[hook]{r} \arrow[leftrightarrow,swap,"\rotatebox{90}{$\simeq$}"]{d}
    & W^{17} \arrow[two heads]{r} \arrow[leftrightarrow,swap,"\rotatebox{90}{$\simeq$}"]{d}
    & \frac{W^{17}}{W^9} \arrow[two heads]{r} \arrow[leftrightarrow,swap,"\rotatebox{90}{$\simeq$}"]{d}
    & \frac{W^{17}}{W^9 + fW^8}  \arrow[leftrightarrow,swap,"\rotatebox{90}{$\simeq$}"]{d} \\
   K_2^5 \arrow[equal]{r}
    & K_2^5 \arrow[hook]{r}
    & K^5 \arrow["\simeq"]{r} \arrow[bend right,swap,"M"]{rrrr}
    & K_5^{15} \arrow[hook]{r}
    & K^{15} \arrow[two heads]{r}
    & K^8 \arrow[two heads]{r}
    & K_3^8 
\end{tikzcd} \]
where $K_d^n$ denoted a $d$-dimensional subspace of $K^n$.

This diagram demands some explanation.
The top row describes at a conceptual, algebraic level what we are doing,
while bottom row describes what is happening at a computational level.

Beginning with $W^7$, we have a 5-dimensional space of polynomials upper bounded by the monomial $xy$.
This is the vector space generated by the monomials $\{1, x, y, x^2, xy\}$.

The polynomials of $W^7$ we multiply by $g$ to get the space $gW^7$.
This is a 5-dimensional space, generated by the polynomials $\{g, xg, yg, x^2g, xyg\}$.
The polynomials of this space are upper bounded by the polynomial $x^3y^2$.
So we view $gW^7$ as a 5-dimensional subspace of the 15-dimensional space $W^{17}$.

The polynomials in $gW^7$, we inject into $W^{17}$ via the canonical inclusion map.
We then take their image (via the canonical projection map) in the quotient space $W^{17}/W^9$.
This is an 8-dimensional space generated by the monomials $\{x^2y, xy^2, y^3, x^3y, x^2y^2, xy^3, y^4, x^3y^2\}$.

We then take their quotient again by the space $fW^8$ (another subspace of $W^{17}$).
Computing this quotient will require us to pick a basis for $fW^8$.
The obvious choice of basis, $\{f, xf, yf, x^2f, xyf, y^2f\}$, will not yield the best results --
rather, we modify the basis to save some multiplications.

To compute the matrix $M$, we observe the action of the above maps on the standard basis for $W^7$.

\[ \left( \begin{array}{r|rrrrr}
      & 1 & x & y & x^2 & xy \\
  \hline
    1 & 1 & 0 & 0 & 0 & 0 \\
    x & 0 & 1 & 0 & 0 & 0 \\
    y & 0 & 0 & 1 & 0 & 0 \\
  x^2 & 0 & 0 & 0 & 1 & 0 \\
   xy & 0 & 0 & 0 & 0 & 1
\end{array} \right) \]

We multiply each basis element of $W^7$ by $g$.
We then view each product as a vector in $W^{17}$.
This gives us the five column vectors, 
\[ \left( \begin{array}{r|rrrrr}
         &   g & xg & yg & x^2g & xyg \\
  \hline
  1      & g_0 &   0 &   0 &           - c_0g_3 &   0 \\
  x      & g_1 & g_0 &   0 &           - c_1g_3 &   0 \\
     y   & g_2 &   0 & g_0 &     - c_0 - c_2g_3 &   0 \\
  x^2    & g_3 & g_1 &   0 & g_0       - c_3g_3 &   0 \\
  x  y   & g_4 & g_2 & g_1 &     - c_1 - c_4g_3 & g_0 \\
     y^2 & g_5 &   0 & g_2 &     - c_2 - c_5g_3 &   0 \\
  x^3    &   0 & g_3 &   0 & g_1       - c_6g_3 &   0 \\
  x^2y   &   1 & g_4 & g_3 & g_2 - c_3 - c_7g_3 & g_1 \\
  x  y^2 &   0 & g_5 & g_4 &     - c_4 - c_8g_3 & g_2 \\
     y^3 &   0 &   0 & g_5 &     - c_5    - g_3 &   0 \\
  x^3y   &   0 &   1 &   0 & g_4 - c_6          & g_3 \\ 
  x^2y^2 &   0 &   0 &   1 & g_5 - c_7          & g_4 \\ 
  x  y^3 &   0 &   0 &   0 &     - c_8          & g_5 \\
     y^4 &   0 &   0 &   0 &       - 1          &   0 \\
  x^3y^2 &   0 &   0 &   0 &         0          &   1
\end{array} \right) \]

Next, we take their images under the quotient map $W^{17} \to W^{17}/W^9$.
This amounts to deleting the first seven rows of the matrix above,
resulting in the five column vectors below.
\[ \left( \begin{array}{r|rrrrr}
         &   g & xg & yg & x^2g & xyg \\
  \hline
  x^2y   &   1 & g_4 & g_3 & g_2 - c_3 - c_7g_3 & g_1 \\
  x  y^2 &   0 & g_5 & g_4 &     - c_4 - c_8g_3 & g_2 \\
     y^3 &   0 &   0 & g_5 &     - c_5    - g_3 &   0 \\
  x^3y   &   0 &   1 &   0 & g_4 - c_6          & g_3 \\ 
  x^2y^2 &   0 &   0 &   1 & g_5 - c_7          & g_4 \\ 
  x  y^3 &   0 &   0 &   0 &     - c_8          & g_5 \\
     y^4 &   0 &   0 &   0 &       - 1          &   0 \\
  x^3y^2 &   0 &   0 &   0 &         0          &   1
\end{array} \right) \]

We will soon be taking another quotient by the space $fW^8$,
but we must first select a basis for $fW^8$.
For the moment, take the na\"ive choice of basis, $\{f, xf, yf, x^2f, xyf, y^2\}$.
We will make some modifications later.
We view this as a subspace of $W^{17}/W^9$, giving the column basis
\[ \left( \begin{array}{r|rrrrrr}
         & f &        xf &  yf &                       x^2f &       xyf & y^2f \\
  \hline
  x^2y   & 0 & f_4 - c_7 & f_3 & f_2 - c_4 - c_7(f_3 - c_6) & f_1 - c_3 &   0 \\
  x  y^2 & 0 & f_5 - c_8 & f_4 &     - c_5 - c_8(f_3 - c_6) & f_2 - c_4 & f_1 \\
     y^3 & 0 &       - 1 & f_5 &              - (f_3 - c_6) &     - c_5 & f_2 \\
  x^3y   & 0 &         0 &   1 & f_4 - c_7                  & f_3 - c_6 &   0 \\
  x^2y^2 & 0 &         0 &   0 & f_5 - c_8                  & f_4 - c_7 & f_3 \\
  x  y^3 & 0 &         0 &   0 &       - 1                  & f_5 - c_8 & f_4 \\
     y^4 & 0 &         0 &   0 &         0                  &       - 1 & f_5 \\
  x^3y^2 & 0 &         0 &   0 &         0                  &         0 &   1
\end{array} \right) \]

Now we consider the augmented matrix
\[ \left( \begin{array}{rrrrr|rrrrr}
  f_4 - c_7 & f_3 & f_2 - c_4 - c_7(f_3 - c_6) & f_1 - c_3 &   0 & 1 & g_4 & g_3 & g_2 - c_3 - c_7g_3 & g_1 \\
  f_5 - c_8 & f_4 &     - c_5 - c_8(f_3 - c_6) & f_2 - c_4 & f_1 & 0 & g_5 & g_4 &     - c_4 - c_8g_3 & g_2 \\
        - 1 & f_5 &              - (f_3 - c_6) &     - c_5 & f_2 & 0 &   0 & g_5 &     - c_5    - g_3 &   0 \\
          0 &   1 & f_4 - c_7                  & f_3 - c_6 &   0 & 0 &   1 &   0 & g_4 - c_6          & g_3 \\ 
          0 &   0 & f_5 - c_8                  & f_4 - c_7 & f_3 & 0 &   0 &   1 & g_5 - c_7          & g_4 \\ 
          0 &   0 &       - 1                  & f_5 - c_8 & f_4 & 0 &   0 &   0 &     - c_8          & g_5 \\
          0 &   0 &         0                  &       - 1 & f_5 & 0 &   0 &   0 &       - 1          &   0 \\
          0 &   0 &         0                  &         0 &   1 & 0 &   0 &   0 &         0          &   1
\end{array} \right) \]
To compute the image $W^7$ under the map $M$ is to compute
the reductions of the columns on the right-hand side modulo the columns on the left-hand side.
The na\"ive way to do this would be costly.
To reduce column 10 (the right-most column), one might subtract column 5, then subtract $f_5$ times column 4,
then subtract appropriate multiples of each of columns 3, 2, and 1.
To reduce column 10 alone would require 16 multiplications, and to reduce all five columns would require 30
(not accounting for duplicate multiplications, however).
If the columns of the left-hand side were in reduced column echelon form,
reducing the right-hand side would only require at most 16 multiplications,
however, producing the reduced column echelon form is costly.

Luckily, computing a reduced column echelon basis is more work than is necessary.
Instead we will partially reduce the columns of the left-hand side.
We will negate the columns ending in $-1$.
We will add multiples of columns 1 and 4 to columns 2 and 5 to obtain zeros in place of the $f_5$'s.
This will cost some multiplications, but some are duplicate, and we would need to perform them regardless.
We will also add $(f_3 - c_6)$ times column 1 to column 3 to obtain a zero in place of $-(f_3 - c_6)$.
This will cost no multiplications or additions, and the newly obtained zero will save us some multiplications later. 
This gives us a partially reduced basis for $fW^8$ given by the columns of
\[ \left( \begin{array}{lllll}
  c_7 - f_4 & f_3 - f_5(c_7 - f_4) & c_4 - f_4(c_6 - f_3) - f_2 & c_3 - f_1 &     - f_5(c_3 - f_1) \\
  c_8 - f_5 & f_4 - f_5(c_8 - f_5) & c_5 - f_5(c_6 - f_3)       & c_4 - f_2 & f_1 - f_5(c_4 - f_2)\\
          1 & 0                    & 0                          & c_5       & f_2 - c_5f_5 \\
          0 & 1                    & c_7 - f_4                  & c_6 - f_3 &     - f_5(c_6 - f_3) \\
          0 & 0                    & c_8 - f_5                  & c_7 - f_4 & f_3 - f_5(c_7 - f_4) \\
          0 & 0                    & 1                          & c_8 - f_5 & f_4 - f_5(c_8 - f_5) \\
          0 & 0                    & 0                          &         1 &   0 \\
          0 & 0                    & 0                          &         0 &   1
\end{array} \right) \]

Now let
\begin{align*}
  d_1 &= c_3 - f_1 & e_1 &=     - f_5d_1 & e_5 &= f_3 - f_5d_4 \\
  d_2 &= c_4 - f_2 & e_2 &= f_1 - f_5d_2 & e_6 &= f_4 - f_5d_5 \\
  d_3 &= c_6 - f_3 & e_3 &= f_2 - f_5c_5 & e_7 &= d_2 - f_4d_3 \\
  d_4 &= c_7 - f_4 & e_4 &=     - f_5d_3 & e_8 &= c_5 + e_4 \\
  d_5 &= c_8 - f_5.
\end{align*}
Relabel the matrix above and re-augment it with the basis for $gW^7$.
\begin{comment}
\[ \left( \begin{array}{lllll}
  d_4 & e_5 & e_7 & d_1 & e_1 \\
  d_5 & e_6 & e_8 & d_2 & e_2 \\
    1 &   0 &   0 & c_5 & e_3 \\
    0 &   1 & d_4 & d_3 & e_4 \\
    0 &   0 & d_5 & d_4 & e_5 \\
    0 &   0 &   1 & d_5 & e_6 \\
    0 &   0 &   0 &   1 &   0 \\
    0 &   0 &   0 &   0 &   1
\end{array} \right) \]
\end{comment}
\[ \left( \begin{array}{rrrrr|rrrrr}
  d_4 & e_5 & e_7 & d_1 & e_1 &   1 & g_4 & g_3 & g_2 - c_3 - c_7g_3 & g_1 \\
  d_5 & e_6 & e_8 & d_2 & e_2 &   0 & g_5 & g_4 &     - c_4 - c_8g_3 & g_2 \\
    1 &   0 &   0 & c_5 & e_3 &   0 &   0 & g_5 &     - c_5    - g_3 &   0 \\
    0 &   1 & d_4 & d_3 & e_4 &   0 &   1 &   0 & g_4 - c_6          & g_3 \\ 
    0 &   0 & d_5 & d_4 & e_5 &   0 &   0 &   1 & g_5 - c_7          & g_4 \\ 
    0 &   0 &   1 & d_5 & e_6 &   0 &   0 &   0 &     - c_8          & g_5 \\
    0 &   0 &   0 &   1 &   0 &   0 &   0 &   0 &       - 1          &   0 \\
    0 &   0 &   0 &   0 &   1 &   0 &   0 &   0 &         0          &   1
\end{array} \right) \]
Now, reducing the right-hand columns modulo the left gives the matrix
\[ \begin{pmatrix}
  1 & a_1 & a_2 & a_3 & a_4 \\
  0 & a_5 & a_6 & a_7 & a_8 - a_5e_8 \\
  0 &   0 &   0 &   0 &   0 \\
  0 &   0 &   0 &   0 &   0 \\ 
  0 &   0 &   1 & a_5 & a_9 \\ 
  0 &   0 &   0 &   0 &   0 \\
  0 &   0 &   0 &   0 &   0 \\
  0 &   0 &   0 &   0 &   0
\end{pmatrix}, \]
where
\begin{align*}
  a_1 &= g_4 - e_5 \\
  a_5 &= g_5 - e_6 \\
  a_2 &= g_3 - g_5d_4 \\
  a_6 &= g_4 - g_5d_5 \\
  a_3 &= g_2 - f_1 - c_7g_3 + f_5e_7 - a_1e_5 + g_3d_4 \\
      &= g_2 - f_1 + f_5e_7 - a_1e_5 - f_4g_3 \\
  a_7 &=     - f_2 - c_8g_3 + f_5e_8 - a_1e_6 + g_3d_5 \\
      &=     - f_2 - a_1e_6 - f_5(g_3 - e_8) \\
  a_4 &= g_1 - e_1 - a_5e_7 - (g_3 - e_4 - a_5d_4)e_5 + e_3d_4 \\
  a_8 &= g_2 - e_2 - (g_3 - e_4 - a_5d_4)e_6 + e_3d_5 \\
  a_9 &= a_1 - a_5d_5
\end{align*}
The linear transformation $M$ is then given by the matrix
\[ M = \begin{pmatrix}
  1 & a_1 & a_2 & a_3 & a_4 \\
  0 & a_5 & a_6 & a_7 & a_8 - a_5e_8 \\
  0 &   0 &   1 & a_5 & a_9
\end{pmatrix}. \]
The peculiar value $a_5e_8$ is separated from rather than being integrated into the term $a_8$
because we will not in fact need to perform the multiplication $a_5e_8$.
To compute the kernel of $M$, we must compute the reduced row echelong form of $M$,
which means we will be dividing $a_5e_8$ be its row's pivot element $a_5$.

The reduced row echelon form of $M$ is
\[ M_{\text{RREF}} = \begin{pmatrix}
  1 & 0 & 0 & -u_0 & -v_0 \\
  0 & 1 & 0 & -u_1 & -v_1 \\
  0 & 0 & 1 & -u_2 & -v_2
\end{pmatrix}, \]
where
\begin{align*}
  \alpha &= \frac 1 {a_5} \\
  u_2 &= -a_5 \\
  v_2 &= -a_9 \\
  u_1 &= -(\alpha a_7 - a_6) \\
  v_1 &= -(\alpha(a_8 + a_6v_2) - e_8) \\
  u_0 &= -(a_3 + a_2u_2 + a_1u_1) \\
  v_0 &= -(a_4 + a_2v_2 + a_1v_1).
\end{align*}
The kernel of $M$ is then
\[ \ker M = \begin{pmatrix}
  u_0 & v_0 \\
  u_1 & v_1 \\
  u_2 & v_2 \\
    1 & 0 \\
    0 & 1 \\
\end{pmatrix}. \]
This gives the polynomials $f'$ and $g'$ generating $W_A^7$ and $I_A$ as
\begin{align*}
  f' &= x^2 + u_2y + u_1x + u_0 \\
  g' &= xy  + v_2y + v_1x + v_0.
\end{align*}
These polynomials were computed using 1I 28M 38A.
The third polynomial, $h'$, in the reduced Gr\"obner basis for $I_A$ may be computed in terms of $f'$ and $g'$ in an additional 7M 5A via
\begin{align*}
  w_0 &= -\alpha(u_0v_1 + v_0(v_2 - u_1)) \\
  w_1 &= -\alpha(v_1v_2 - v_0) \\
  w_2 &= v_1 - \alpha(v_2(v_2 - u_1) + u_0) \\
  h' &= y^2 + w_2y + w_1x + w_0.
\end{align*}



%%%%%%%%%%%%%%%%%%%%%%%%%%%%%%%%%%%%%%%%%%%%%%%%%
%%%%%                                       %%%%%
%%%%%   Flipping Typical Type 31 Divisors   %%%%%
%%%%%                                       %%%%%
%%%%%%%%%%%%%%%%%%%%%%%%%%%%%%%%%%%%%%%%%%%%%%%%%

\subsection{Flipping Typical Type 31 Divisors}

Let $D$ be a typical degree 3 divisor of type 31.
So $I_D$ is given by the Gr\"obner basis $\pid{f,g,h}$, where
\begin{align*}
  f  &= x^2 + f_2y + f_1x + f_0 \\
  g  &= xy  + g_2y + g_1x + g_0 \\
  h  &= y^2 + h_2y + h_1x + h_0.
\end{align*}
Since $D$ is typical, $I_D = \pid{f,g}$, although $f$ and $g$ alone do not form a Gr\"obner basis.
To compute the flip $A$ of $D$, we compute the quotient ideal
  \[ f : I_D = f : \pid{f,g,h} = f : \pid{f,g} = f : g. \]

By Table \ref{table_divisor_flips}, $A$ is also of type 31, and since $D$ is typical, so too is $A$.
As was the case when flipping a typical degree 6 divisor,
the reduced Gr\"obner basis of $I_A$ consists of three polynomials
\begin{align*}
  f' &= x^2 + f_2y + f_1x + f_0 \\
  g' &=  xy + g_2y + g_1x + g_0 \\
  h' &= y^2 + h_2y + h_1x + h_0.
\end{align*}
However, $f' = f$ and $h'$ can be computed from $f'$ and $g'$.
We therefore only need to compute $g'$.
The monomials of $g'$ are $\{1, x, y, xy\}$.
Let $V \subseteq W^7$ be the vector space generated by the monomials of $g'$.

\[ \begin{tikzcd}
  W_A^7 \arrow[equal]{r} \arrow[leftrightarrow,swap,"\rotatebox{90}{$\simeq$}"]{d}
    & \ker M \arrow[hook]{r} \arrow[leftrightarrow,swap,"\rotatebox{90}{$\simeq$}"]{d}
    & V \arrow["\simeq"]{r} \arrow[leftrightarrow,swap,"\rotatebox{90}{$\simeq$}"]{d}
    & gV \arrow[hook]{r} \arrow[leftrightarrow,swap,"\rotatebox{90}{$\simeq$}"]{d}
    & W^{14} \arrow[two heads]{r} \arrow[leftrightarrow,swap,"\rotatebox{90}{$\simeq$}"]{d}
    & \frac{W^{14}}{W^6} \arrow[two heads]{r} \arrow[leftrightarrow,swap,"\rotatebox{90}{$\simeq$}"]{d}
    & \frac{W^{14}}{W^6 + fW^8}  \arrow[leftrightarrow,swap,"\rotatebox{90}{$\simeq$}"]{d} \\
   K_2^5 \arrow[equal]{r}
    & K_2^5 \arrow[hook]{r}
    & K^5 \arrow["\simeq"]{r} \arrow[bend right,swap,"M"]{rrrr}
    & K_5^{12} \arrow[hook]{r}
    & K^{12} \arrow[two heads]{r}
    & K^8 \arrow[two heads]{r}
    & K_3^8 
\end{tikzcd} \]

We begin with the monomial basis for $V$,
multiply each monomial by $g$,
view these as vectors in $W^{14}$,
then remove the top 4 rows, placing us in $W^{14}/W^6$.


\[ \left( \begin{array}{r|rrrrr}
         & g &  xg &  yg & xyg \\
  \hline
  x  y   & 1 & g_2 & g_1 & g_0 \\
     y^2 & 0 &   0 & g_2 &   0 \\
  x^3    & 0 &   0 &   0 &   0 \\
  x^2y   & 0 &   1 &   0 & g_1 \\
  x  y^2 & 0 &   0 &   1 & g_2 \\
     y^3 & 0 &   0 &   0 &   0 \\
  x^3y   & 0 &   0 &   0 &   0 \\
  x^2y^2 & 0 &   0 &   0 &   1
\end{array} \right) \]

\[ \left( \begin{array}{r|rrrrr}
         &  xf  & yf &      x^2f & xyf & y^2f \\
  \hline
  x  y   & f_2 & f_1 &     - c_4 & f_0 &   0 \\
     y^2 &   0 & f_2 &     - c_5 &   0 & f_0 \\
  x^3    &   1 &   0 & f_1 - c_6 &   0 &   0 \\
  x^2y   &   0 &   1 & f_2 - c_7 & f_1 &   0 \\
  x  y^2 &   0 &   0 &     - c_8 & f_2 & f_1 \\
     y^3 &   0 &   0 &       - 1 &   0 & f_2 \\
  x^3y   &   0 &   0 &         0 &   1 &   0 \\
  x^2y^2 &   0 &   0 &         0 &   0 &   1
\end{array} \right) \]

\[ \left( \begin{array}{rrrrr|rrrr}
  f_2 & f_1 &     - c_4 & f_0 &   0 & 1 & g_2 & g_1 & g_0 \\
    0 & f_2 &     - c_5 &   0 & f_0 & 0 &   0 & g_2 &   0 \\
    1 &   0 & f_1 - c_6 &   0 &   0 & 0 &   0 &   0 &   0 \\
    0 &   1 & f_2 - c_7 & f_1 &   0 & 0 &   1 &   0 & g_1 \\
    0 &   0 &     - c_8 & f_2 & f_1 & 0 &   0 &   1 & g_2 \\
    0 &   0 &       - 1 &   0 & f_2 & 0 &   0 &   0 &   0 \\
    0 &   0 &         0 &   1 &   0 & 0 &   0 &   0 &   0 \\
    0 &   0 &         0 &   0 &   1 & 0 &   0 &   0 &   1
\end{array} \right) \]

\[ d_1 = c_6 - f_1 ~~~ d_2 = c_7 - f_2 \]

\[ \left( \begin{array}{rrrrr|rrrrr}
  f_2 & f_1 & c_4 & f_0 &   0 & 1 & g_2 & g_1 & g_0 \\
    0 & f_2 & c_5 &   0 & f_0 & 0 &   0 & g_2 &   0 \\
    1 &   0 & d_1 &   0 &   0 & 0 &   0 &   0 &   0 \\
    0 &   1 & d_2 & f_1 &   0 & 0 &   1 &   0 & g_1 \\
    0 &   0 & c_8 & f_2 & f_1 & 0 &   0 &   1 & g_2 \\
    0 &   0 &   1 &   0 & f_2 & 0 &   0 &   0 &   0 \\
    0 &   0 &   0 &   1 &   0 & 0 &   0 &   0 &   0 \\
    0 &   0 &   0 &   0 &   1 & 0 &   0 &   0 &   1
\end{array} \right) \]

\[ M = \begin{pmatrix}
  1 &   a_1 & g_1 & a_2 \\
  0 & - f_2 & g_2 & - f_0 + f_2(c_5 - g_1 - f_2d_2) \\
  0 &     0 &   1 & a_1 + f_2c_8
\end{pmatrix} \]

\begin{align*}
  a_1 &= g_2 - f_1 \\
  a_2 &= g_0 + f_2(c_4 - f_2d_1) - f_1(g_1 + f_2d_2)
\end{align*}

\[ M_{rref} = \begin{pmatrix}
  1 & 0 & 0 & -v_0 \\
  0 & 1 & 0 & -v_1 \\
  0 & 0 & 1 & -v_2
\end{pmatrix} \]

\begin{align*}
  \alpha &= \frac 1 {f_2} \\
  v_2 &= -(a_1 + f_2c_8) \\
  v_1 &= \alpha(f_0 - g_2v_2) + c_5 - g_1 - f_2d_2 \\
  v_0 &= a_2 + g_1v_2 + a_1v_1
\end{align*}

\[ \ker M = \begin{pmatrix} v_0 & v_1 & v_2 & 1 \end{pmatrix}^T \]

This gives us $f'$ and $g'$ in 1I 9M 13A.
We may compute $h'$ in terms of $f'$ and $g'$ in an additional 7M 5A
for a total of 1I 16M 18A.
If $D$ was computed as the flip of some other divisor,
then we might already have computed $f_2\inv$.
If we held onto that inverse, then we save 1I in computing $A$.
\note{In characteristic greater than 3, when $c_5 = c_6 = c_8 = 0$, we save an additional 2M 3A.}

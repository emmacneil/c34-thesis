\section{The Colon Ideal}
\label{appendix_colon_ideal}

Let $\frak a$ and $\frak b$ be \emph{integral} ideals of a commutative ring with identity $R$.
The \defn{ideal quotient} of $\frak a$ by $\frak b$, also called the \defn{colon ideal}, is
\[ \frak a : \frak b = \{ r \in R ~|~ r \frak b \subseteq \frak a \}. \]
When an ideal quotient involves principal ideals $\pid a$ or $\pid b$,
we may write $a : \frak b$, $\frak a : b$, and $a : b$ for brevity,
rather than $\pid a : \frak b$, $\frak a : \pid b$, and $\pid a : \pid b$.

The following proposition sums up several useful, well-known properties of the colon ideal.
\begin{proposition}
  \label{prop_colon_ideal}
  Let $R$ be a commutative ring with identity.
  Let $\frak a$, $\frak b$, and $\frak c$ be $R$-ideals. Then
  \begin{enumerate}[label=(\roman*)]
    \item $\frak a : \frak b$ is an $R$-ideal;
    \item $\frak a \subseteq \frak a : \frak b$;
    \item $\frak a : R = \frak a$;
    \item $R : \frak a = R$;
    \item $\frak a \frak b \subseteq \frak c \iff \frak a \subseteq \frak c : \frak b$;
    \item $\frak a : \frak b = R \iff \frak b \subseteq \frak a$;
    \item $\frak a : (\frak b + \frak c) = (\frak a : \frak b) \cap (\frak a : \frak c)$;
    \item $(\frak a \cap \frak b) : \frak c = (\frak a : \frak c) \cap (\frak b : \frak c)$;
    \item $(\frak a : \frak b) : \frak c = \frak a : \frak b \frak c$.
  \end{enumerate}
\end{proposition}

These properties are given as propositions in \cite{cox07},
though the statements are given for a multivariate polynomial ring over a field, $K[x_1, \ldots, x_n]$
rather than for an arbitrary commutative ring with identity,
and most of the proofs are left as exercises.
Proofs, in slightly greater generality, are given here.

\begin{proof}
  \begin{enumerate}[label=(\roman*)]
    \item Let $a, b \in \frak a : \frak b$.
          Then $a \frak b, b \frak b \subseteq \frak a$.
          Then $(a + b)\frak b = a\frak b + b\frak b \subseteq \frak a$
          (since $a\frak b + b\frak b$ is the join of $a\frak b$ and $b\frak b$ in the lattice of $R$-ideals).
          So $a + b \in \frak a : \frak b$.
          
          Let $a \in \frak a : \frak b$, $r \in R$.
          Then $a \frak b \subseteq \frak a$ and $ra \frak b \subseteq a \frak b$,
          so $ra \frak b \subseteq \frak a$ and $ra \in \frak a : \frak b$.

    \item
      We have
      \begin{align*}
             & \frak a \subseteq \frak a : b \\
        \iff & \forall a \in \frak a : a \frak b \subseteq \frak a \\
        \iff & \forall a \in \frak a : \forall b \in \frak b : ab \in \frak a.
      \end{align*}
      The last statement is true since ideals are closed under multiplication by $R$.
      
    \item
      By part (ii), we have $\frak a \subseteq \frak a : R$.
      Suppose $a \in \frak a : R$.
      Then $aR \subseteq \frak a$.
      In particular, $a = a \cdot 1_R \in \frak a$,
      so $\frak a : R \subseteq \frak a$.
      
    \item
      By definition, $R : \frak a \subseteq R$.
      By part (ii), $R \subseteq R : \frak a$.
      
    \item
      ($\implies$) Let $a \in \frak a$. Then $a \frak b \subseteq \frak a \frak b$,
      and by hypothesis, $\frak a \frak b \subseteq \frak c$,
      so $a \frak b \subseteq \frak c$,
      and $a \in \frak c : \frak b$.
      
      ($\impliedby$) Let $a \in \frak a$.
      By hypothesis, $a \in \frak c : \frak b$, so $a \frak b \subseteq \frak c$.
      Since the choice of $a$ was arbitrary, this means $\frak a \frak b \subseteq \frak c$.
      
    \item
      ($\implies$) Suppose $\frak a : \frak b = R$.
      Then $\frak b = 1_R \frak b \subseteq \frak a$.
      
      ($\impliedby$) Suppose $b \subseteq \frak a$.
      For all $r \in R$, $r \frak b \subseteq \frak b$.
      So $r \frak b \subseteq \frak a$ and $\frak a : \frak b = R$.
      
    \item
      Let $r \in R$. We have
      \begin{align*}
           & r(\frak b + \frak c) \subseteq \frak a \\
      \iff & r \frak b + r \frak c \subseteq \frak a \\
      \iff & r \frak b \subseteq \frak a \text{ and } r \frak c \subseteq \frak a.
      \end{align*}
      So
      \begin{align*}
        \frak a : (\frak b + \frak c)
          &= \{ r \in R ~|~ r(\frak b + \frak c) \subseteq \frak a \} \\
          &= \{ r \in R ~|~ r \frak b \subseteq \frak a \text{ and } r \frak c \subseteq \frak a \} \\
          &= \{ r \in R ~|~ r \frak b \subseteq \frak a \} \cap \{ r \in R ~|~ r \frak c \subseteq \frak a \} \\
          &= (\frak a : \frak b) \cap (\frak a : \frak c).
      \end{align*}
    
    \item
      Similarly to part (vii),
      \begin{align*}
        (\frak a \cap \frak b) : \frak c
          &= \{ r \in R ~|~ r \frak c \subseteq \frak a \cap \frak b \} \\
          &= \{ r \in R ~|~ r \frak c \subseteq \frak a \text{ and } r \frak c \subseteq \frak b \} \\
          &= \{ r \in R ~|~ r \frak c \subseteq \frak a \} \cap \{ r \in R ~|~ r \frak c \subseteq \frak b \} \\
          &= (\frak a : \frak c) \cap (\frak b : \frak c).
      \end{align*}
    
    \item
      Let $r \in R$. We have
      \begin{align*}
           & r \frak c \subseteq \frak a : \frak b \\
      \iff & r \frak b \frak c \subseteq \frak a & \text{by (v)} \\
      \iff & r \in \frak a : \frak b \frak c,
      \end{align*}
      so
      \begin{align*}
        (\frak a : \frak b) : \frak c
          &= \{ r \in R ~|~ r \frak c \subseteq \frak a : \frak b \} \\
          &= \{ r \in R ~|~ r \in \frak a : \frak b \frak c \} \\
          &= \frak a : \frak b \frak c.
      \end{align*}
  \end{enumerate}
\end{proof}

The following proposition and its corollary illustrate why this ideal is called the ideal quotient.
In a Dedekind domain, we have $\frak a \frak b : \frak b = \frac {\frak a \frak b}{\frak b} = \frak a$.

\begin{proposition}
  \label{prop_colon_by_prime}
  Let $R$ be a Dedekind domain.
  Let $\frak a$ be a non-zero ideal and $\frak p$ a non-zero prime ideal of $R$. Then
  \[ \frak a \frak p : \frak p = \frak a. \]
\end{proposition}
\begin{proof}
  Clearly, $\frak a \frak p \subseteq \frak a \frak p$.
  Using Proposition \ref{prop_colon_ideal}.(v),
  this gives $\frak a \subseteq \frak a \frak p : \frak p$.
  
  Suppose $\alpha \in \frak a \frak p : \frak p$.
  Then $(\alpha)\frak p \subseteq \frak a \frak p$
  and there exists a non-zero ideal $\frak b$ such that $(\alpha)\frak p = \frak a \frak b \frak p$.
  By Corollary 3.3 in \cite{conrad}, $(\alpha) = \frak a \frak b$,
  so $(\alpha) \subseteq \frak a$ and $\alpha \in \frak a$.
\end{proof}

\begin{corollary}
  \label{cor_ab_colon_b_is_a}
  Let $R$ be a Dedekind domain.
  Let $\frak a$ and $\frak b$ be non-zero ideals of $R$. Then
  \[ \frak a \frak b : \frak b = \frak a. \]
\end{corollary}
\begin{proof}
  Let $\frak b$ factor into $\frak p_1 \cdots \frak p_n$, where the $\frak p_i$'s are not necessarily distinct.
  Then
  \begin{align*}
    \frak a \frak b : \frak b
      &= (\frak a \frak p_1 \frak p_2 \dots \frak p_n) : (\frak p_1 \frak p_2 \dots \frak p_n) \\
      &= (((\frak a \frak p_1 \frak p_2 \dots \frak p_n : \frak p_1) : \frak p_2) : \dots) : \frak p_n
        & \text{by Prop \ref{prop_colon_ideal}.ix} \\
      &= ((\frak a \frak p_2 \dots \frak p_n : \frak p_2) : \dots) : \frak p_n
        & \text{by Prop \ref{prop_colon_by_prime}} \\
      &= \dots & \text{induction} \\
      &= \frak a.
  \end{align*}
\end{proof}

\begin{corollary}
  \label{cor_ideal_times_quotient_is_principal}
  Let $R$ be a Dedekind domain and let $\frak a$ be a non-zero $R$-ideal.
  Let $a \in \frak a$. Then
  \[ \frak a (a : \frak a) = \pid a. \]
\end{corollary}
\begin{proof}
  We have $\frak a \subseteq \pid a$,
  so there is an $R$-ideal $\frak b$ such that $\frak a \frak b = \pid a$.
  Then by Corollary \ref{cor_ab_colon_b_is_a},
  \[ a : \frak a = (\frak a \frak b) : \frak a = \frak b, \]
  and
  \[ \frak a (a : \frak a) = \frak a \frak b = \pid a. \]
\end{proof}


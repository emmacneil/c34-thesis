%%%%%%%%%%%%%%%%%%%%%%%%
%%%%%              %%%%%
%%%%%   Addition   %%%%%
%%%%%              %%%%%
%%%%%%%%%%%%%%%%%%%%%%%%

\section{Addition}
\label{chap_addition}

We present in this chapter a general algorithm for adding reduced divisors
in the divisor class group of a $C_{3,4}$ curve $C$.
Given two reduced divisors $D$ and $D'$, we wish to find a divisor $D'' \equiv D + D'$.
More accurately, we wish to find a reduced Gr\"obner basis for an ideal $I_{D''} \equiv I_{D}I_{D'}$.
The divisor $D''$ produced by this algorithm will usually not be reduced.
We will see in Chapter \ref{chap_reduction} how to reduce $D''$,
and in Chapter \ref{chap_typical_case} how to combine addition and reduction into a single add-reduce operation.

We will make the following assumptions on $D$ and $D'$:
\begin{enumerate}[label=(\roman*)]
  \item $D$ and $D'$ are reduced;
  \item $D \neq D'$;
  \item $\deg D \geq \deg D'$.
\end{enumerate}
When one of $D$ or $D'$ is unreduced, it can be reduced using algorithms presented in Chapter \ref{chap_reduction}.
When $D = D'$, their sum is $2D$, which may be computed using algorithms in Chapter \ref{chap_doubling}.
As for the third assumption,
the addition algorithm described in this chapter boils down to constructing and row-reducing a matrix.
The dimensions of this matrix depends on the degree of $D'$.
By assuming $D'$ is of lesser degree, we may work with a smaller matrix, resulting in faster computations.
Should $D'$ be of greater degree than $D$, we may swap them before adding, as divisor addition is commutative.

The addition algorithm presented in this chapter
is based on the algorithm presented by Abu Salem and Khuri-Makdisi in \cite{salem07},
but extended to operate on divisors not considered in \cite{salem07}.
The authors in \cite{salem07} only considered adding disjoint typical degree 3 divisors,
and their algorithm would only return a meaningful value
when the sum of the divisors was also typical ---
otherwise it would return a value indicating an error has occurred.
They also assumed that $C$ is defined over a large finite field
by an equation of the short form \ref{eq_c34_short}.

Their algorithm may be briefly described as follows.
To add two disjoint typical degree 3 divisors $D$ and $D'$,
they first identify these divisors with the vector spaces $W_D^{x^2y}$ and $W_{D'}^{x^2y}$.
These vector spaces will be defined properly below.
The sum $D + D'$ is also identified with the vector space $W_{D + D'}^{x^2y}$
and is related to $W_D^{x^2y}$ and $W_{D'}^{x^2y}$ via
\[ W_{D + D'}^{x^2y} = W_D^{x^2y} \cap W_{D'}^{x^2y}. \]
The intersection of vector spaces is computed by computing the kernel of the quotient
\begin{center}
\begin{tikzcd}
  W_{D + D'}^{x^2y} \arrow[hook]{r}{\ker M_{\text{add}}} &
  W_D^{x^2y} \arrow[hook]{r}{\iota} \arrow[bend left]{rr}{M_{\text{add}}} &
  W^{x^2y} \arrow[two heads]{r}{\pi} &
  \frac {W^{x^2y}} {W_{D'}^{x^2y}}
\end{tikzcd},
\end{center}
where $\iota$ and $\pi$ indicate the canonical inclusion and quotient maps,
and $M_{\text{add}} = \pi \circ \iota$ is their composition.

In this chapter, we will generalize this algorithm to cover all cases of divisor addition
by making three key observations.
\begin{enumerate}[label=(\roman*)]
  \item When $D$ and $D'$ are non-disjoint, the kernel does not give $D + D'$, but rather $L = \lcm(D, D')$.
  \item $G = \gcd(D, D')$ may be computed by computing the image of the quotient.
  \item The monomial $x^2y$ indicates an upper bound on a search space ---
        by choosing a larger monomial $m > x^2y$, one can account for the case where
        $D + D'$ is atypical.
\end{enumerate}
The generalization of Abu Salem and Khuri-Makdisi's addition therefore requires
selecting an appropriate bounding monomial $m$ and computing the kernel and image of $M_{\text{add}}$ in
\begin{center}
\begin{tikzcd}
  W_L^m \arrow[hook]{r}{\ker M_{\text{add}}} &
  W_D^m \arrow[hook]{r}{\iota} \arrow[bend left]{rr}{M_{\text{add}}} &
  W^m \arrow[two heads]{r}{\pi} &
  \frac {W^m} {W_{D'}^m} \arrow[two heads]{r}{\im M_{\text{add}}} &
  \frac {W_G^m} {W_{D'}^m}
\end{tikzcd}.
\end{center}
In the sections to follow, we explain the steps in detail.
We will define the space $W_D^m$ and how it relates to $D$ and $I_D$.
We will see how to compute $W_L^m$ and $W_G^m$.
By using the relation
\begin{equation}
  \label{eq_lcm_plus_gcd}
  D + D' = L + G
\end{equation}
and the fact that, typically, $G = 0$,
we give an algorithm for general divisor addition.
The atypical case where $G \neq 0$ is handled recursively,
so we must also demonstrate that this algorithm terminates.



%%%%%%%%%%%%%%%%%%%%%%%%%%%%%%%%%%%%%%%%%%%%%
%%%%%                                   %%%%%
%%%%%   Translating to Linear Algebra   %%%%%
%%%%%                                   %%%%%
%%%%%%%%%%%%%%%%%%%%%%%%%%%%%%%%%%%%%%%%%%%%%

\subsection{The Vector Space $W_D^m$}

For any divisor $D$, the ideal $I_D$ has structure as an infinite-dimensional $K$-vector space,
which we will denote by $W_D$.
This is merely the vector space where addition of polynomials and scalar multiplication are as in $I_D$,
but we forget multiplication between polynomials.
Any $K$-basis for $W_D$ also generates $I_D$ as an ideal.
However, the infinitely large basis for $W_D$ is difficult to compute with.
We may restrict to a finite subspace without losing any information.
Let $\cal M$ be the set of monomials of $K[x,y]$ and let $m \in \cal M$.
Define the $K$-vector spaces
\begin{align*}
  W^m &:= \{ f \in K[C] ~|~ \LM(f) \leq m \}
%      &= \Span\{ \mu \in \cal M ~:~ \mu \leq m \}
\end{align*}
and
\begin{align*}
  W_D^m &:= W_D \cap W^m \\
        &= \{ f \in I_D ~|~ \LM(f) \leq m \}.
\end{align*}
\begin{comment}
\begin{remark}
  Since $-\nu_{P_\infty}(x^4) = -\nu_{P_\infty}(y^3)$, $W^{x^4} = W^{y^3}$.
  In practice, it is easier to assume $m$ is not divisible by $y^3$ and think of $W^m$ as the space
  \[ W^m = \{ f \in K[x,y] ~|~ \LM(f) \leq m, ~\text{$f$ is reduced modulo $y^3$} \}. \]
\end{remark}
\end{comment}
\begin{proposition}
  \label{prop_vector_space_basis_generates_I}
  Let $G = \{g_1, \ldots, g_k\}$ be a reduced Gr\"obner basis for $I_D$.
  Let $m = \LM(g_k) > \dots > \LM(g_1)$.
  Let $B$ be a $K$-basis for $W_D^m$.
  Then $I_D = \pid B$.
\end{proposition}
\begin{proof}
  Certainly $\pid B \subseteq I_D$, since $B \subseteq W_D^m \subseteq I_D$.
  To show $I_D \subseteq \pid B$, it suffices to show $G \subseteq \pid B$, whence $I_D = \pid G \subseteq \pid B$.
  
  Let $g_i \in G$.
  Then $\LM(g_i) \leq m$, so $g_i \in W_D^m$.
  Therefore $g_i$ is a $K$-linear combination of elements of $B$,
  implying that $g_i \in \pid B$.
\end{proof}

An echelon basis for $W_D^m$ is a basis $B$ where no two distinct $b, b' \in B$ have the same leading monomial.

\begin{proposition}
  Let $I_D$, $G$, and $m$ be as in Proposition \ref{prop_vector_space_basis_generates_I}.
  If $B$ is an echelon basis for $W_D^m$, then $B$ is a Gr\"obner basis for $I_D$.
\end{proposition}
\begin{proof}
  It suffices to show that for all $f \in I_D$, there is a $b \in B$ such that $\LM(b) | \LM(f)$.
  This is easy to demonstrate, since there is a $g_i \in G$ such that $\LM(g_i) | \LM(f)$,
  and there is a $b \in B$ such that $\LM(b) = \LM(g_i)$.
\end{proof}

\begin{corollary}
  \label{cor_reduced_echelon_basis_gives_reduced_groebner_basis}
  Let $I_D$, $G$, and $m$ be as in Proposition \ref{prop_vector_space_basis_generates_I}.
  If $B$ is a reduced echelon basis for $W_D^m$,
  then there is a subset of $B$ that is a reduced Gr\"obner basis for $I_D$.
\end{corollary}
\begin{proof}
  Take the set
  \begin{equation}
    \label{eq_reduced_basis}
    S = \{ b \in B ~|~ \nexists b' \in B-\{b\} : \LM(b') | \LM(b) \}.
  \end{equation}
  For any $s \in S$, by construction, $\LM(s)$ is not divisible by the leading term of any other $s' \in S$.
  Moreover, no term in $s$ is divible by the leading term of any other $s'$
  by virtue of $B$ being a reduced echelon basis.
\end{proof}
Given an ideal $I_D$, we may produce the vector space $W_D^m$.
By Corollary \ref{cor_reduced_echelon_basis_gives_reduced_groebner_basis},
given a vector space $W_D^m$, we may reproduce the ideal $I_D$,
as long as $m$ was chosen to be sufficiently large.
In particular, $m$ must be at least as large as
the largest leading monomial appearing in the reduced Gr\"obner basis for $I_D$.

The space $W_D^m$ is simply a Riemann-Roch space under a different notation.\footnote{Namely, $W_D^m = \cal L(-D - \nu_{P_\infty}(m)P_\infty)$.}
It is a well-known fact in algebraic geometry that Riemann-Roch spaces are finite-dimensional
(Proposition 1.4.9 in \cite{stichtenoth09}).
We give the dimensions explicitly:

\begin{theorem}
  \label{thm_dim_W}
  For a divisor $D$ and monomial $m$, $W^m$ and $W_D^m$ are finite-dimensional.
  In particular,
  \begin{align*}
    \dim W^m &= \begin{cases}
                  1 & m = 1 \\
                  2 & m = x \\
                  3 & m = y \\
                  3i + 4j - 2 & m = x^iy^j > y
                \end{cases}. \\
    \dim W_D^m &= \# \{ \mu \in \cal M ~:~ \mu \leq m, ~\mu \in \LT(I_D) \}.
  \end{align*}
  For sufficiently large $m$,
  \[ \dim W_D^m = \dim W^m - \deg D.\]
\end{theorem}
\begin{comment}
\begin{remark}
In theory, ``sufficiently large'' means $m \geq m'$,
where $m'$ is the maximum of the set $\{ \mu \in \cal M ~:~ \mu \not\in \LT(I_D) \}$,
as in the proof to follow.
In practice, when computing $D + D$', if $\deg(D + D') = d$,
we will choose $m$ to be the largest monomial found among all reduced Gr\"obner bases of divisors of degree $d$.
\end{remark}
\end{comment}
\begin{proof}
  The dimension of $W^m$ is the number of monomials in $K[C]$ less than or equal to $m$.
  For readability, define $f(i,j) = \dim W^{x^iy^j}$.
  Recalling the $C_{3,4}$ order on monomials in $K[C]$,
    \[ 1 < x < y < x^2 < xy < y^2 < x^3 < x^2y < xy^2 < x^4 < x^3y < x^2y^2 < \dots, \]
  it is clear that
  \begin{align*}
    f(0,0) &= 1 & f(2,0) &= 4 \\
    f(1,0) &= 2 & f(1,1) &= 5 \\
    f(0,1) &= 3 & f(0,2) &= 6,
  \end{align*}
  that for $i \geq 0$, $f(2 + i,0) = 3i + f(2, 0)$, and that for $0 \leq j \leq 2$, $f(i,j) = j + f(i+j,0)$.
  By the curve equation, for $k \geq 0$, $f(i, 3k + j) = f(4k + i, j)$.
  Putting these all together.
  \begin{align*}
    f(i,j)
      &= f(i, 3k + \ell) \\
      &= f(4k + i, \ell) \\
      &= \ell + f(4k + i + \ell, 0) \\
      &= \ell + f(2 + 4k + i + \ell - 2, 0) \\
      &= \ell + 3(4k + i + \ell - 2) + f(2, 0) \\
      &= 12k + 3i + 4\ell - 2 \\
      &= 3i + 4j - 2.
  \end{align*}
  
  For the second claim, let $B$ be a reduced echelon basis for $W_D^m$. Then
  \[ \mu \leq m \text{ and } \mu \in \LT(I_D) \iff \exists b \in B ~:~ \LT(b) = \mu. \]
  To see this, suppose $\mu \leq m$ and $\mu \in \LT(I_D)$.
  Then there is a monic $f \in I_D$ such that $\LT(f) = \mu$.
  Then $f \in W_D^m$, so $f$ is a $K$-linear combination on $B$,
  so there is a $b \in B$ with $\LT(b) = \LT(f) = \mu$.
  The converse is trivial, since $\LT(b) \leq m$ and $\LT(b) \in \LT(I_D)$.
  
  For the final claim, the set $\{ \mu \in \cal M ~:~ \mu \not\in \LT(I_D) \}$ has a maximum $m'$.
  Hence for $m \geq m'$,
  \begin{align*}
    \dim W_D^m
      &= \# \{ \mu \in \cal M ~:~ \mu \leq m, ~\mu \in \LT(I_D) \} \\
      &= \# \{ \mu \in \cal M ~:~ \mu \leq m \} - \# \{ \mu \in \cal M ~:~ \mu \leq m, ~\mu \not\in \LT(I_D) \} \\
      &= \dim W^m - \deg D.
  \end{align*}
  The last equality holds by Corollary \ref{cor_dim_R_mod_I}.
\end{proof}



%%%%%%%%%%%%%%%%%%%%%%%%%%%%%%%%
%%%%%                      %%%%%
%%%%%   Kernel and Image   %%%%%
%%%%%                      %%%%%
%%%%%%%%%%%%%%%%%%%%%%%%%%%%%%%%

\subsection{Computing $W_L^m$ and $W_G^m$}

Let $L = \lcm(D, D')$ and $G = \gcd(D, D')$.
Just as $I_L = I_D \cap I_{D'}$ and $I_G = I_D + I_{D'}$,
so too are $W_L^m = W_D^m \cap W_{D'}^m$ and $W_G^m = W_D^m + W_{D'}^m$
(without restriction on the size of $m$).
We now address the question of how to compute $W_L^m$ and $W_G^m$.

Ultimately, the purpose of computing $W_L^m$ and $W_G^m$ is to recover $I_L$ and $I_G$.
For that, we must first choose a sufficiently large monomial $m$ to bound our vector spaces.
The monomial $m$ must be at least as large as the largest leading monomial in the Gr\"obner bases of $I_L$ and $I_G$.
We do not know \emph{a priori} what those Gr\"obner bases are,
but we may place an upper bound on the degrees of $L$ and $G$,
then refer to Table \ref{tab_divisor_types} to place a bound on $m$.
For example, if $D$ and $D'$ are both of degree 3, then $D + D'$ is of degree 6,
and $L$ and $G$ are no larger than degree 6.
Referring to Table \ref{tab_divisor_types},
no divisor of degree 6 (or smaller) has a monomial larger than $x^4$,
so $m = x^4$ is sufficiently large for our purposes.

After choosing $m$, the following theorem tells us how to compute $W_L^m$ and $W_G^m$.
Strictly speaking, it gives us a quotient of $W_G^m$,
but we will see later that we have enough information to get $W_G^m$ itself.

\begin{theorem}
  Let $D$ and $D'$ be divisors, let $m$ be a monomial, and let $M_{\text{add}}$ be the composed map
  \begin{center}
    \begin{tikzcd}
      W_D^m \arrow[hook]{r}{\iota} \arrow[bend left]{rr}{M_{\text{add}}} & W^m \arrow[two heads]{r}{\pi} & \frac {W^m} {W_{D'}^m}
    \end{tikzcd}.
  \end{center}
  where $\iota$ is the canonical inclusion map and $\pi$ is the canonical quotient map.
  Then $\ker M_{\text{add}} = W_L^m$ and $\im M_{\text{add}} = W_G^m / W_{D'}^m$,
  \begin{center}
    \begin{tikzcd}
      W_L^m \arrow[hook]{r}{\ker M_{\text{add}}} & 
      W_D^m \arrow[hook]{r}{\iota} \arrow[bend left]{rr}{M_{\text{add}}} & 
      W^m \arrow[two heads]{r}{\pi} & 
      \frac {W^m} {W_{D'}^m} \arrow[two heads]{r}{\im M_{\text{add}}} & 
      \frac {W_G^m} {W_{D'}^m}
    \end{tikzcd}.
  \end{center}
\end{theorem}
\begin{proof}
  The kernel of $M_{\text{add}}$ consists precisely of those elements in $W_D^m$ that vanish modulo $W_{D'}^m$.
  An element vanishes modulo $W_{D'}^m$ if and only if it is a member of $W_{D'}^m$, so
  \[ \ker M_{\text{add}} = W_D^m \cap W_{D'}^m = W_L^m. \]

  As for the image, it is enough to show that $\im M_{\text{add}}$ is isomorphic to and contained in $W_G^m / W_{D'}^m$.
  By the first and second isomorphism theorems for vector spaces
  (see Theorems IV.1.7 and IV.1.9 in \cite{hungerford}, or many other algebra and linear algebra texts),
  \[ \im M_{\text{add}} \overset{\text{(1st)}}{\cong} \frac {W_D^m} {W_L^m}
           = \frac {W_D^m} {W_D^m \cap W_{D'}^m}
           \overset{\text{(2nd)}}{\cong} \frac {W_D^m + W_{D'}^m} {W_{D'}^m}
           = \frac {W_G^m} {W_{D'}^m}. \]
  Let $f \in W_D^m$.
  Then $M_{\text{add}}(f) = [f] \in \frac {W^m} {W_{D'}^m}$.
  However $f$ is also in $W_D^m + W_{D'}^m = W_G^m$,
  therefore $[f]$ is also in $\frac {W_G^m} {W_{D'}^m}$.
\end{proof}

\begin{proposition}
  \label{prop_deg_L_G}
  The dimensions of $W_L^m$ and $W_G^{m'}$ are
  \begin{enumerate}[label=(\roman*)]
    \item $\dim W_L^m    = \nullity M_{\text{add}}$;
    \item $\dim \frac {W_G^m} {W_{D'}^m} = \rank M_{\text{add}}$.
  \end{enumerate}
  For a sufficiently large monomial $m$,
  \begin{enumerate}[label=(\roman*)]
    \setcounter{enumi}{2}
    \item $\deg L = \deg D  + \rank M_{\text{add}}$;
    \item $\deg G = \deg D' - \rank M_{\text{add}}$;
    \item $D + D' = L \iff G = 0 \iff \rank M_{\text{add}} = \deg D'$.
  \end{enumerate}
\end{proposition}
\begin{proof}
  Parts (i) and (ii) are immediate from the Rank-nullity Theorem.
  For part (iii),
  \begin{align*}
    \deg L
      &= \dim W^m - \dim W_L^m & \text{Theorem \ref{thm_dim_W}} \\
      &= \dim W^m - \nullity M_{\text{add}} & \text{part (i)} \\
      &= \dim W^m - \dim W_D^m + \rank M_{\text{add}} & \text{rank-nullity} \\
      &= \deg D + \rank M_{\text{add}} & \text{Theorem \ref{thm_dim_W}}.
  \end{align*}
  For part (iv),
  \begin{align*}
    \deg G
      &= \deg G + \deg L - \deg L \\
      &= \deg D' + \deg D - \deg L & L + G = D + D' \\
      &= \deg D' - \rank M_{\text{add}} & \text{part (iii)}.
  \end{align*}
  Finally, for part (v), it is clear that $D + D' = L \iff G = 0$ by equation \ref{eq_lcm_plus_gcd},
  and by part (iv), $G = 0 \iff \rank M_{\text{add}} = \deg D'$ .
\end{proof}

In the following subsections, we work out some extended examples of computing $L$ and $G$.



%%%%%%%%%%%%%%%%%%%%%%%%%%%%%%%%%%
%%%%%                        %%%%%
%%%%%   Infinite Recursion   %%%%%
%%%%%                        %%%%%
%%%%%%%%%%%%%%%%%%%%%%%%%%%%%%%%%%

\subsection{The Addition Algorithm}

We are now able to compute $L = \lcm(D, D')$ and $G = \gcd(D, D')$.
The goal of this chapter is to compute a divisor $D''$ equivalent to $D + D'$.
By Proposition \ref{prop_deg_L_G}, if $M_{\text{add}}$ has full rank, then $D'' = L$.
This occurs if and only if $D$ and $D'$ are disjoint ($G = 0$).

When $D$ and $D'$ are non-disjoint,
we compute $D + D'$ by computing $L$ and $G$, reducing $L$ ($\deg G \leq 2$, so $G$ is already reduced),
then adding $\bar{\bar L} + G$, introducing an element of recursion to our addition.
If not handled properly, this can lead to infinite recursion.
For example, suppose $D = P + 2Q$ and $D' = P + Q$.
Then $L = P + 2Q = D$ (and is already reduced) and $G = P + Q = D'$.
Attempting to compute $D + D'$ by $\bar{\bar L} + G$ brings us full circle.
In this section, we will identify three cases arising in divisor addition.
We will handle each case in such a way that any time we recursively add,
the smaller of the two divisors in the next iteration
is strictly smaller than in the previous iteration,
forcing our algorithm to eventually terminate.

We begin with a proposition characterizing the case $\rank M_{\text{add}} = 0$.
\begin{proposition}
  \label{prop_rank_M}
  The following are equivalent:
  \begin{enumerate}[label=(\roman*)]
    \item $L = D$;
    \item $G = D'$;
    \item $D' \leq D$;
    \item $\rank M_{\text{add}} = 0$.
  \end{enumerate}
\end{proposition}
\begin{proof}
  \begin{description}
    \item[(i) $\iff$ (ii):]
      Immediate from the relation $D + D' = L + G$.
    \item[(i) $\implies$ (iii):]
      $D' \leq L$ and $L = D$, hence $D' \leq D$.
    \item[(iii) $\implies$ (iv):]
      If $D' \leq D$, then $I_D \subseteq I_{D'}$ and $W_D^m \subseteq W_{D'}^m$.
      Every element in $W_D^m$ vanishes under $M_{\text{add}}$,
      $\nullity M_{\text{add}} = \dim W_D^m$ and $\rank M_{\text{add}} = 0$.
    \item[(iv) $\implies$ (i):]
      Suppose $\rank M_{\text{add}} = 0$.
      By Proposition \ref{prop_deg_L_G}, $\deg L = \deg D$.
      Since $D \leq L$ and $D$ and $L$ have the same degree, $L = D$.
  \end{description}
\end{proof}
\begin{corollary}
  \[ G < D' \iff \rank M_{\text{add}} > 0. \]
\end{corollary}

We may now identify three cases:
\begin{enumerate}[label=(\roman*)]
  \item $\rank M_{\text{add}} = \deg D'$;
  \item $0 < \rank M_{\text{add}} < \deg D'$; or
  \item $\rank M_{\text{add}} = 0$.
\end{enumerate}
In the first case, we return $L$ and our algorithm terminates.
In the second case, we recursively compute $\bar{\bar L} + G$;
the degree of $G$ is strictly smaller than the degree of $D'$.
In the third case, $\bar{\bar L} = D$ and $G = D'$, so recursively adding $\bar{\bar L} + G$ leads to an infinite loop.
We must treat this case separately.

By Proposition \ref{prop_rank_M}, if $\rank M_{\text{add}} = 0$, then $D' \leq D$.
Since we are assuming $D' \neq D$, we have $D' < D$.
Since $D$ is reduced and of degree at most 3, $\deg D'$ is either 1 or 2.
We consider each of these cases separately, beginning with the latter.

If $D'$ is of degree 2, then $D = D' + A$ for some degree 1 divisor $A$, which we must find.
Once we have, we compute
  \[ D + D' = \bar{\bar{2D'}} + A. \]
Now $\deg A < \deg D'$, so our algorithm is one step closer towards termination.
Finding $A$ amounts to finding $x_0, y_0 \in K$ such that
  \[ I_D = I_{D'}\pid{x + x_0, y + y_0}. \]
To do this, we note that $I_{D'} = \pid{f',g'}$ has a Gr\"obner basis of two elements.
To find $x_0, y_0$, we note that $f'(x + x_0)$ and $f'(y + y_0)$ must be in $I_D$.
So reducing $f'(x + x_0)$ and $f'(y + y_0)$ modulo $f, g, h$ must give 0.
This gives us a simple system of linear equations to solve for $x_0$ and $y_0$.
Formulae solving this system are available at \cite{github}.

If $D'$ is of degree 1, then $D = nD' + A$ for some $A$ disjoint from $D'$ (possibly $A = 0$)
and some $1 \leq n \leq 3$.
We must find $A$ and $n$ and compute
  \[ D + D' = \bar{\bar{(n + 1)D'}} + A. \]
To find $A$ and $n$, one repeats the process from the previous case.
Find a divisor $A$ of degree $\deg D - \deg D'$ such that $D = D' + A$
by reducing $f'(x + x_0)$ and $f'(y + y_0)$ modulo $I_D$ and solving a system of linear equations.
It is then easy to check\footnote{
$D$ consists of one point. Check if the generators of $I_A$ are zero at that point.}
if $D' \leq A$.
If not, $n = 1$ and we have the divisor $A$ we need.
Otherwise, increment $n$ and repeat the process by finding $A'$ such that $A = D' + A'$, and so on.

Now $A$ is disjoint\footnote
{Except for an exceptional case where $(n + 1)D'$ is type 32 or 33,
in which case $\bar{\bar{(n + 1)D'}}$ is degree 0 or 1.
In this case, the algorithm will still terminate.}
from $\bar{\bar{(n + 1)D'}}$, so that the algorithm will terminate in the next recursive step.
Computing $(n + 1)D'$ requires us to be able to double, triple, or quadruple a point.
Doubling is the topic of Chapter \ref{chap_doubling} and quadrupling is accomplished by doubling twice.
In Chapter \ref{chap_doubling}, two doubling algorithms are presented.
When the addition algorithm is required to double, it uses Algorithm \ref{alg_doubling_fast},
which provably terminates.
Formulae for tripling a point are available at \cite{github}.

\begin{comment}
Having now shown that we can compute $D + D'$ in a way that will terminate,
we must now show how to find the relevant values $D''$, $P$, and $n$ where necessary.

Suppose $\deg D' = 2$, and we wish to find $P$ such that $D = P + D'$.
Since $\deg D > \deg D'$, we must have $\type D = 31$.
It may be that $\type D' = 21$ or $\type D' = 22$, so we consider each case separately.

Suppose first that $D'$ is of type 21. 
Let $P$ be represented by the type 11 divisor $\pid{s, t} = \pid{x + s_0, y + t_0}$.
Let $D$ and $D'$ be
\begin{align*}
  D &= \pid{f, g, h} & D' &= \pid{f', g'} \\
  f &= x^2 + f_2y + f_1x + f_0 & f' &= y + f'_1x + f'_0 \\
  g &= xy + g_2y + g_1x + g_0 & g' &= x^2 + g'_1x + g'_0 \\
  h &= y^2 + h_2y + h_1x + h_0.
\end{align*}

We need to solve for $s_0$ and $t_0$ such that
\[ \pid{f, g, h} = \pid{s, t}\pid{f', g', h'}. \]
A solution exists and must satisfy $sf' - f'_1f - g \in \pid{f,g,h}$.
We have that
  \[ sf' - f'_1f - g = (s_0 - f_2f'_1 - g_2)y + (f'_0 + f'_1s_0 - f_1f'_1 - g_1)x + (f'_0s_0 - f_0f'_1 - g_0). \]
It must be that each of these coefficients are zero, otherwise $f, g, h$ would not form a reduced Gr\"obner basis.
Therefore
  \[ s_0 = f_2f'_1 + g_2. \]
Likewise, we compute
  \[ tf' - f'_1g - h = (f'_0 + t_0 - f'_1g_2 - h_0)y + (f'_1t_0 - f'_1g_1 - h_1)x + (f'_0t_0 - f'_1g_0 - h_0) \]
and conclude
  \[ t_0 = - f'_0 + f'_1g_2 + h_0. \]

When $D'$ is of type 22, the same strategy yields $s_0$ and $t_0$.
We reduce $sf' = (x + s_0)(x + f'_0)$ and $tf' = (y + t_0)(x + f'_0)$ modulo $I_D$,
argue these reductions are zero, and conclude
\begin{align*}
  s_0 &= - f'_0 - f_1 \\
  t_0 &= g_1.
\end{align*}

\begin{proposition}
  Let $D$ be a typical type 31 divisor and $D'$ a type 11 divisor.
  Suppose $D' < D$.
  Then there is a type 21 divisor $D''$ such that $D = D' + D''$,
  given by $D'' = \pid{y + s_1x + s_0, x^2 + t_1x + t_0}$ where
    \begin{align*}
      s_1 &= \frac{-(g_2 - f'_0)} {f_2} \\
      s_0 &= g_1 + s_1(f_1 - f'_0) \\
      t_1 &= g_2 + f_1 - f'_0 \\
      t_0 &= f_1g_2 - f_2g_1 - f'_0t_1 + f_0.
    \end{align*}
\end{proposition}
\begin{proof}
  There certainly exists a degree 2 divisor $D''$ such that $D = D' + D''$.
  Either $D''$ is of type 21 or 22.
  The sum of a type 11 divisor with a type 22 divisor is never a typical divisor \note{(justify this?)}
  so $D''$ must be of type 21, given by polynomials $s = y + s_1x + s_0$ and $t = x^2 + t_1x + t_0$.
  Solving $\pid{f, g, h} = \pid{s,t}\pid{f',g'}$ gives $s_0, s_1, t_0, t_1$ as given in the theorem statement.
\end{proof}

\begin{proposition}
  Let $D$ be a semi-typical type 31 divisor.
  The ideal of $I_D$ of $D$ is generated by $\pid{f,g,h}$ with $f = x^2 + f_1x + f_0$.
  Let $D'$ be a type 11 divisor.
  Suppose $D' < D$. Then
  \begin{enumerate}[label=(\roman*)]
    \item $f$ has two (not necessarily distinct) rational roots.
  \end{enumerate}
  There is a degree 2 divisor $D''$ such that $D = D' + D''$ and
  \begin{enumerate}[label=(\roman*)]
    \setcounter{enumi}{1}
    \item if $f$ has distinct roots then
      \[ \type(D'') = \begin{cases} 21 & f'_0 = g_2 \\ 22 & f'_0 \neq g_2 \end{cases}\; \]
    \item if $f$ has a double root and $g'_0 = g_1$ then $\type(D'') = 21$
      \[ \type(D'') = \begin{cases} 22 & g'_0 = g_1 \\ 21 & g'_0 \neq g_1 \end{cases}\; \]
    \item if $\type(D'') = 21$, then $D'' = \pid{y + s_1x + s_0, x^2 + t_1x + t_0}$ where
    \begin{align*}
      s_1 &= \frac{g_0 - f'_0g_1} {f'_0(f_1 - f'_0) - f_0} \\
      s_0 &= g_1 + s_1(f_1 - f'_0) \\
      t_1 &= f_1 \\
      t_0 &= f_0 \;
    \end{align*}
    \item if $\type(D'') = 22$, then $D'' = \pid{x + s_0, y^2 + t_2y + t_0}$ where
    \begin{align*}
      s_0 &= f_1 - f'_0 \\
      t_2 &= h_2 \\
      t_0 &= h_0 - h_1s_0.
    \end{align*}
  \end{enumerate}
\end{proposition}
\note{These formulas still need testing.}
\note{I'm not convinced $s_1$ is correct if $f$ has a double root.}
\begin{proof}
  \begin{enumerate}[label=(\roman*)]
    \item
      Since $D' < D$, we have $I_D \subset I_{D'}$.
      In particular $f \in I_{D'}$, so $f = af'+ bg'$ for some polynomials $a$ and $b$.
      However, there is no $y$-term in $f$, so $b = 0$.
      One may then conclude that the rational root of $f'$ is also a root of $f$.
      Having one rational root, the second root of $f$ must also be rational.
    
    \item
      The divisor $D$ is the sum of a type 11 divisor $P$ and a type 22 divisor $Q + R$.
      $D''$ is of type 22 if and only if $D' = P$.
      \note{And this is determined by $D' = P \iff f'_0 \neq g_2$.}
    
    \item
      The divisor $D$ is the sum three points sharing the same $x$-coordinate.
      They cannot all be distinct, or else $D$ would be type 33 (principal).
      So at least two points in $D$ are equal, say $D = 2P + Q$.
      There are three cases:
      $P \neq Q$ and neither point is a special point;
      $P \neq Q$ and $Q$ is a special point;
      and $P = Q$ and $P$ is a special point.
      
      The cases we have ruled out:
      If $P \neq Q$ and $P$ is a special point, then $2P + Q$ is type 33.
      If $P = Q$ and $P$ is a regular point or hyperflex, then $\ord_P(f) < 3$.
      If $P = Q$ and $P$ is an inflection point, then $D = 2P + Q = 3P$ is type 33.
      
      In all three cases, $g$ defines a ``cross'' centered on $P$
      One can determine whether $D' = P$ by $D' = P \iff g'_0 = g_1$.
      Also, $P + Q$ is type $22$ in all three cases.
      
      So we have
      \begin{align*}
        g'_0 = g_1
          &\implies D' = P \\
          &\implies D = 2P + Q = P + D'' = D' + D'' \\
          &\implies D'' = P + Q \\
          &\implies \type(D'') = 22.
      \end{align*}
      Otherwise, $\type(D'') = 21$.
    
    \item
      Solving $sf' - s_1f - g = 0$ gives the stated formulas for $s_1$ and $s_0$.
      We note that $D'' < D$ implies $\pid{f,g,h} \subset \pid{s,t}$,
      $f \in \pid{s,t}$, and finally $t = f$.
    
    \item
      Solving $sf' - f = 0$ gives $s_0$.
      We note that $D'' < D$ implies $\pid{f,g,h} \subset \pid{s,t}$,
      $h \in \pid{s,t}$, and $h = t + h_1s$.
      Solving this simple equation gives $t_0$ and $t_2$.
  \end{enumerate}
\end{proof}
\end{comment}



%%%%%%%%%%%%%%%%%%%%%%%%%%%%%%%%%%%%%%
%%%%%                            %%%%%
%%%%%   The Addition Algorithm   %%%%%
%%%%%                            %%%%%
%%%%%%%%%%%%%%%%%%%%%%%%%%%%%%%%%%%%%%

We summarize the addition algorithm in Algorithm \ref{alg_addition}.
By ``return $L$'', we mean return a reduced Gr\"obner basis for $I_L$,
which is the subset of the reduced echelon basis for $\ker M_{\text{add}}$ satisfying Equation \ref{eq_reduced_basis}.

\begin{algorithm}
  \label{alg_addition}
  \caption{Divisor Addition}
  {\bf Input:} Two reduced divisors $D$ and $D'$ satisfying $D \neq D'$ and $\deg D \geq \deg D'$,
  represented by the reduced Gr\"obner bases of their ideals $I_D$ and $I_{D'}$ \\
  {\bf Output:} A divisor $D''$ equivalent to $D + D'$, represented by its ideal $I_{D''}$
  \begin{algorithmic}[1]
    \If {$D' = 0$}
      \State \Return $D$ \label{alg_addition:return_0}
    \EndIf
    \State Compute $M_{\text{add}} : W_D^m \to W_{D'}^m$
    \State Compute $\rref(M_{\text{add}})$ and $\rank M_{\text{add}}$ and $\ker M_{\text{add}}$
    \If {$\rank M_{\text{add}} = \deg D'$}
      \State Compute $L = \lcm(D, D')$
      \State \Return $L$ \label{alg_addition:return_1}
    \EndIf
    \If {$\rank M_{\text{add}} > 0$}
      \State Compute $L = \lcm(D, D')$
      \State Compute $G = \gcd(D, D')$
      \State \Return $\bar{\bar L} + G$ \label{alg_addition:return_2}
    \EndIf
    \If {$\rank M_{\text{add}} = 0$}
      \If {$\deg D' = 2$}
        \State Compute $A$ such that $D = A + D'$
        \State \Return $\bar{\bar{2D'}} + A$ \label{alg_addition:return_3}
      \EndIf
      \If {$\deg D' = 1$}
        \State Compute $A$ and largest $n$ such that $D = A + nD'$
        \State \Return $\bar{\bar{(n+1)D'}} + A$ \label{alg_addition:return_4}
      \EndIf
    \EndIf
  \end{algorithmic}
\end{algorithm}

\begin{comment}
This algorithm requires that we be able to reduce divisors
in lines \ref{alg_addition:return_2}, \ref{alg_addition:return_3}, and \ref{alg_addition:return_4},
to double divisors in lines \ref{alg_addition:return_3} and \ref{alg_addition:return_4},
or possibly to triple degree 1 divisors in line \ref{alg_addition:return_4}.
Assume for now that we have terminating algorithms for these operations.
We wish to show that the addition Algorithm \ref{alg_addition}.

This algorithm has five return statements.
If the algorithm reaches \ref{alg_addition:return_0} or \ref{alg_addition:return_1},
then the algorithm terminates.
If the algorithm returns at line \ref{alg_addition:return_4},
then $(n + 1)D'$ and $A$ will be disjoint,
and the algorithm will reach at line \ref{alg_addition:return_1} in its next iteration
(or line \ref{alg_addition:return_0} if $A = 0$).
At lines \ref{alg_addition:return_2} and \ref{alg_addition:return_3},
$G$ and $A$ are of lesser degree than $D'$.
Eventually, a recursive call will involve $D'$ of degree 0 --
in which case the next interation reaches line \ref{alg_addition:return_0} --
or 1 --
in which case $\rank M$ is either 1 or 0
and the next iteration reaches line \ref{alg_addition:return_1} or \ref{alg_addition:return_4} and terminates.

Whether or not this algorithm terminates therefore depends on whether we have terminating algorithms
for reducing divisors, doubling divisors, and tripling points.
These are the topics of the next two chapters.
\end{comment}


%%%%%%%%%%%%%%%%%%%%%%%%%%%%%%%%
%%%%%                      %%%%%
%%%%%   Disjoint Example   %%%%%
%%%%%                      %%%%%
%%%%%%%%%%%%%%%%%%%%%%%%%%%%%%%%

\subsection{Example -- Computing $\ker M_{\text{add}}$}

Let $C$ be the $C_{3,4}$ curve over $\bb F_{11}$ defined by the polynomial $y^3 + x^4 + 1$.
Let $D$ and $D'$ be type 31 divisors with
\begin{align*}
  D  &= \pid{f, g, h}     & D' &= \pid{f', g', h'} \\
  f  &= x^2 + 3y + 7x + 5 & f' &= x^2 + 6y + 3x - 2 \\
  g  &= xy + 2y + 2x + 9  & g' &= xy + 5y + 5x + 9 \\
  h  &= y^2 + 4y + 2x + 3 & h' &= y^2 - y - x + 5.
\end{align*}
These divisors are disjoint, with
\begin{align*}
  D &= 2 \cdot (7 : 6 : 1) + (10 : 4 : 1) \\
  D' &= (5 : 1 : 1) + (2\alpha + 6 : 7\alpha : 1) + (9\alpha + 3 : 4\alpha + 6 : 1)
\end{align*}
where $\alpha \in \bb F_{11^2}$ is a root of $x^2 + 7x + 2$.

The sum $D + D'$ is a degree 6 divisor. Referring to Table \ref{tab_divisor_types},
we see that no generator of any reduced Gr\"obner basis of a degree 6 divisor has a monomial larger than $x^4$.
Therefore $W^{x^4}$ will be a sufficiently large space in which to perform our computations.
We proceed by computing the matrix $M_{\text{add}}$ in
\begin{center}
  \begin{tikzcd}
    W_L^{x^4} \arrow[hook]{r}{\ker M_{\text{add}}} & 
    W_D^{x^4} \arrow[hook]{r}{\iota} \arrow[bend left]{rr}{M_{\text{add}}} & 
    W^{x^4} \arrow[two heads]{r}{\pi} & 
    \frac {W^{x^4}} {W_{D'}^{x^4}} \arrow[two heads]{r}{\im M_{\text{add}}} &
    \frac {W_G^{x^4}} {W_{D'}^{x^4}} \\ &
    K^7 \arrow[hook]{r} \arrow[no head]{u}{\rotatebox{90}{$\simeq$}} &
    K^{10} \arrow[two heads]{r} \arrow[no head]{u}{\rotatebox{90}{$\simeq$}} &
    K^3 \arrow[no head]{u}{\rotatebox{90}{$\simeq$}}
  \end{tikzcd},
\end{center}
where the bottom row serves to show the dimensions of these spaces.
The dimensions of $\ker M_{\text{add}}$ and $\im M_{\text{add}}$ are not yet known until we do some more calculations.

The spaces $W_D^{x^4}$ and $W_{D'}^{x^4}$ are 7-dimensional.
Any seven polynomials from one of these spaces with different valuations at $P_{\infty}$
will form an echelon basis for that space.
Choosing polynomials with different valuations at infinity amounts to choosing polynomials with different leading monomials while also taking into account the linear dependence on the monomials
$1, x, \ldots, x^4, y^3$.
The most obvious choice for bases of $W_D^{x^4}$ and $W_{D'}^{x^4}$, then, are
\begin{align*}
  W_D^{x^4} &= \Span_K\{ f, g, h, xf, xg, xh, x^2f \} \\
  W_{D'}^{x^4} &= \Span_K\{ f', g', h', xf', xg', xh', x^2f' \}.
\end{align*}
We reduce the basis of $W_D^{x^4}$ modulo $W_{D'}^{x^4}$ to get the matrix
\[ M_{\text{add}} = \begin{pmatrix}
  7 & 0 & 9 & 2 & 10 & 5 & 2 \\
  4 & 8 & 3 & 10 & 2 & 8 & 6 \\
  8 & 8 & 5 & 2 & 0 & 1 & 7
\end{pmatrix}, \]
where, e.g., the reduction of $f$ is $\bar f = 8y + 4x + 7$,
the reduction of $g$ is $\bar g = 8y + 8x$, etc.
This matrix $M_{\text{add}}$ has the reduced row echelon form (RREF)
and kernel
\[ \rref(M_{\text{add}}) = \begin{pmatrix}
  1 & 0 & 6 & 0 & 6 & 9 & 2 \\
  0 & 1 & 7 & 0 & 9 & 8 & 10 \\
  0 & 0 & 0 & 1 & 6 & 4 & 5
\end{pmatrix} ~~~ \ker M_{\text{add}} =
\begin{pmatrix}
  -6 & -6 & -9 & -2 \\
  -7 & -9 & -8 & -10 \\
   1 &  0 &  0 &  0 \\
   0 & -6 & -4 & -5 \\
   0 &  1 &  0 &  0 \\
   0 &  0 &  1 &  0 \\
   0 &  0 &  0 &  1
\end{pmatrix}. \]
The kernel is 4-dimensional, spanned by the vectors
\begin{align*}
  \ker M_{\text{add}}
    &= \Span_K \left\{ \begin{array}{l}
               h -  7g - 6f, \\
        xg - 6xf -  9g - 6f, \\
        xh - 4xf -  8g - 9f, \\
      x^2f - 5xf - 10g - 2f \end{array} \right\} \\
    &= \Span_K \left\{ \begin{array}{l}
      y^2 + 4xy + 5x^2 + 5y + x - 2, \\
      x^2y + 5x^3 - 3xy - 2x^2 - 3y - 4x - 1, \\
      xy^2 - 4x^3 - 5xy - 2x^2 + y + 3x + 4, \\
      x^4 + 3x^2y + 2x^3 - 3xy + x^2 - 4y - 4x - 1
    \end{array} \right\}.
\end{align*}
These four polynomials form a basis for $W_L^{x^4}$ and a Gr\"obner basis for $I_L$,
but not a \emph{reduced} Gr\"obner basis.
By Proposition \ref{prop_deg_L_G}, $\deg L = 6$ and $\deg G = 0$.
We have just determined that $I_L$ contains polynomials with leading monomials $y^2$ and $x^2y$,
so by Arita's classification of divisors, $L$ must be of type 63 and the first two polynomials
alone form a reduced Gr\"obner basis,
  \[ I_L = \pid{ y^2 + 4xy + 5x^2 + 5y + x - 2, ~x^2y + 5x^3 - 3xy - 2x^2 - 3y - 4x - 1}.\]

\begin{remark}
In this example, we computed the kernel of $M_{\text{add}}$ to get four polynomials, though two were not needed.
Consequently, two columns of the matrix $M_{\text{add}}$ were not needed.
An efficient implementation of divisor arithmetic will avoid computing unnecessary columns of $M_{\text{add}}$
by computing them only as they become necessary.
\end{remark}



%%%%%%%%%%%%%%%%%%%%%%%%%%%%%%%%%%%%
%%%%%                          %%%%%
%%%%%   Non-Disjoint Example   %%%%%
%%%%%                          %%%%%
%%%%%%%%%%%%%%%%%%%%%%%%%%%%%%%%%%%%

\begin{comment}
\subsection{An example -- non-disjoint case}

Consider again the curve $C : y^3 + x^4 + 1$ over $\bb F_{11}$.
Let $D$ and $D'$ be type 31 divisors with
\begin{align*}
  D  &= \pid{f, g, h}      & D' &= \pid{f', g', h'} \\
  f  &= x^2 + 10           & f' &= x^2 + 3y + 2x + 2 \\
  g  &= xy + y + 9x + 9    & g' &= xy + 4x + 5 \\
  h  &= y^2 + 10y + 4x + 5 & h' &= y^2 + y + 2x + 3 .\\
\end{align*}
These divisors are non-disjoint. We have
\begin{align*}
  D &= (1 : 2 : 1) + (10 : 2\alpha + 4 : 1) + (10 : 10\alpha + 8 : 1) \\
  D' &= (1 : 2 : 1) + (3 : 9 : 1) + (5 : 6 : 1)
\end{align*}
where $\alpha \in \bb F_{11^2}$ is a root of $x^2 + 7x + 2$.

\begin{center}
  \begin{tikzcd}
    W_L^{x^4} \arrow[hook]{r}{\ker M} & 
    W_D^{x^4} \arrow[hook]{r}{\iota} \arrow[bend left]{rr}{M} & 
    W^{x^4} \arrow[two heads]{r}{\pi} & 
    \frac {W^{x^4}} {W_{D'}^{x^4}} \arrow[two heads]{r}{\im M} &
    W_G^y \\ &
    K^7 \arrow[hook]{r} \arrow[no head]{u}{\rotatebox{90}{$\simeq$}} &
    K^{10} \arrow[two heads]{r} \arrow[no head]{u}{\rotatebox{90}{$\simeq$}} &
    K^3 \arrow[no head]{u}{\rotatebox{90}{$\simeq$}}.
  \end{tikzcd}
\end{center}

As in the previous example, we reduce the basis of $W_{D}^{x^4}$ modulo the basis for $W_{D'}^{x^4}$ to get the matrix
\[ M = \begin{pmatrix}
  8 & 4 & 2 & 8 & 7 & 6 & 10 \\
  9 & 5 & 2 & 2 & 1 & 6 & 2 \\
  8 & 1 & 9 & 6 & 7 & 5 & 5
\end{pmatrix}. \]
This matrix has the reduced row echelon form
\[ M = \begin{pmatrix}
  1 & 0 & 6 & 8 & 5 & 7 & 5 \\
  0 & 1 & 5 & 8 & 0 & 4 & 9 \\
  0 & 0 & 0 & 0 & 0 & 0 & 0
\end{pmatrix}. \]
We can see the $M$ does not have full rank.
Hence $L$ will not be of degree 6 and $G$ will be non-zero.
The kernel of $M$ is
\begin{align*}
  \ker M
    &= \Span \left\{ \begin{array}{l}
      h - 5g - 6f, \\
      xf - 8g - 8f, \\
      xg - 5f, \\
      xh - 4g - 7f, \\
      x^2f - 9g - 5f \end{array} \right\} \\
    &= \Span \left\{ \begin{array}{l}
      y^2 + 6xy + 5x^2 + 5y + 3x + 10, \\
      x^3 + 3xy + 3x^2 + 3y + 4x + 2, \\
      x^2y + xy + 4x^2 + 9x + 5, \\
      xy^2 + 6xy + 8x^2 + 7y + 2x + 4, \\
      x^4 + 2xy + 5x^2 + 2y + 7x + 1
    \end{array} \right\}
\end{align*}
and $L$ is a type 51 divisor whose ideal $I_L$ has the reduced Gr\"obner basis
\[ I_L = \left\langle\begin{array}{l}
  y^2 + 6xy + 5x^2 + 5y + 3x + 10, \\ 
  x^3 + 3xy + 3x^2 + 3y + 4x + 2, \\ 
  x^2y + xy + 4x^2 + 9x + 5\end{array}\right\rangle. \]
The pivot columns of $M$ determine its image.
The first and second columns are pivots, thus the image of $M$ is given by the column basis
\[ \im M = \begin{pmatrix}
  8 & 4 \\
  9 & 5 \\
  8 & 1 
\end{pmatrix}. \]
That is, $\im M$ is spanned by the vectors $8y + 9x + 8$ and $y + 5x + 8$.
This column basis is reducible to 
\[ \im M = \begin{pmatrix}
  10 & 9 \\
   1 & 0 \\
   0 & 1 
\end{pmatrix}, \]
so that $I_G = \pid{x + 10, y + 9} = \pid{x - 1, y - 2}$,
which agrees with $D$ and $D'$ sharing the point $(1 : 2 : 1)$ in common.

Now we have $L$ and $G$ of degrees 5 and 1, respectively, such that $D + D' = L + G$.
To finish, we must compute the reduction $\bar{\bar L}$ of $L$ and compute $D + D' = \bar{\bar L} + G$.
The reduction $\bar{\bar L}$ will be of type 31, thus the problem of adding $D$ and $D'$
is reduced to adding a degree 3 divisor $\bar{\bar L}$ with a degree 1 divisor $G$.
\end{comment}


%%%%%%%%%%%%%%%%%%%%%%%%%%%%%%%%%%%%%%%%%%%%
%%%%%                                  %%%%%
%%%%%   Another Non-Disjoint Example   %%%%%
%%%%%                                  %%%%%
%%%%%%%%%%%%%%%%%%%%%%%%%%%%%%%%%%%%%%%%%%%%

\subsection{Example -- Computing $\im M_{\text{add}}$}

Consider again the $C_{3,4}$ curve $C$ defined by the polynomial $y^3 + x^4 + 1$ over $\bb F_{11}$.
Let $D$ and $D'$ be type 31 divisors with
\begin{align*}
  D  &= \pid{f, g, h}      & D' &= \pid{f', g', h'}  \\
  f  &= x^2 +  y + 5x + 1  & f' &= x^2 - 3y - 5x - 3 \\
  g  &=  xy + 2y - 3x + 2  & g' &=  xy - 4y + 4x - 4 \\
  h  &= y^2 + 3y + 3x + 2  & h' &= y^2 -  y + 4x - 2.
\end{align*}
These divisors are non-disjoint, with
\begin{align*}
  D  &= (0 : 10 : 1) + (3 : 8 : 1) + (1 : 4 : 1) \\
  D' &= (0 : 10 : 1) + (3 : 8 : 1) + (6 : 1 : 1).
\end{align*}

As in the previous example, we reduce the basis of $W_{D}^{x^4}$ modulo the basis for $W_{D'}^{x^4}$ to get the matrix
\[ M_{\text{add}} = \begin{pmatrix}
   4 & 6 & 4 & 2 & 3 & 2 & 1 \\
  10 & 4 & 2 & 5 & 2 & 5 & 8 \\
   4 & 6 & 4 & 2 & 3 & 2 & 1
\end{pmatrix}. \]
This matrix has the reduced row echelon form
\[ \rref(M_{\text{add}}) = \begin{pmatrix}
  1 & 7 & 1 & 6 & 9 & 6 & 3 \\
  0 & 0 & 0 & 0 & 0 & 0 & 0 \\
  0 & 0 & 0 & 0 & 0 & 0 & 0
\end{pmatrix}. \]
We can see the $M_{\text{add}}$ does not have full rank.
In this case, $\rank M_{\text{add}} = 1$.
By Proposition \ref{prop_deg_L_G},
$\deg L = 4$, and $G$ is non-zero with $\deg G = 2$.
We have a non-trivial image to compute.
\begin{comment}
\begin{align*}
  \ker M
    &= \Span \left\{ \begin{array}{l}
      g - 7f, \\
      h -  f, \\
      xf - 6f, \\
      xg - 9f, \\
      xh - 6f, \\
      x^2f - 3f \end{array} \right\} 
    = \Span \left\{ \begin{array}{l}
      xy + 4x^2 - 5y - 5x - 5, \\
      y^2 - x^2 + 2y - 2x + 1, \\
      x^3 + xy - x^2 + 5y + 4x + 5, \\
      x^2y + 2xy - x^2 + 2y + x + 2, \\
      xy^2 + 3xy - 3x^2 + 5y + 5x + 5, \\
      x^4 + x^2y + 5x^3 - 2x^2 - 3y - 4x - 3
    \end{array} \right\}.
\end{align*}
Comparing the result to Table \ref{tab_divisor_types},
$L$ is degree 4 and $I_L$ has polynomials with leading monomials $xy$, $y^2$, and $x^3$,
so $L$ must be type 41 and $I_L$ is generated by the first three polynomials.
After reducing the third polynomial modulo the first to eliminate the $xy$ term,
we have a reduced Gr\"obner basis for $I_L$,
\[ I_L = \left\langle\begin{array}{l}
      xy  + 4x^2 - 5y - 5x - 5, \\
      y^2 -  x^2 + 2y - 2x + 1, \\
      x^3 - 5x^2 -  y - 2x - 1
  \end{array}\right\rangle. \]
\end{comment}
The image of $M_{\text{add}}$ is determined by its pivot columns.
In this case, there is only one pivot column --- the first.
The image of $M_{\text{add}}$ is therefore given by the column basis
\[ \im M_{\text{add}} = \Span_K \begin{pmatrix} 4 \\ 10 \\ 4 \end{pmatrix}
         = \Span_K \begin{pmatrix} 1 \\ -3 \\ 1 \end{pmatrix}. \]
That is, $\im M_{\text{add}} = \frac {W_G^m}{W_{D'}^m}$ is spanned by the polynomial $u = y - 3x + 1$.
Since $G$ has degree 2 and contains a polynomial with leading monomial $y$,
$G$ must be of type 21 and we must find its other generator $v$ with leading monomial $x^2$.
Since $I_D \subseteq I_G$, $f$ must be an element of $I_G$, and we may reduce $f$ modulo $u$ to obtain $v$.
This gives
  \[ I_G = \pid{u,v} = \pid{y - 3x + 1, ~x^2 - 3x}. \]
Note that $u$ and $v$ intersect at the points $(0 : 10 : 1)$ and $(3 : 8 : 1)$, as expected.

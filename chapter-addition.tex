%%%%%%%%%%%%%%%%%%%%%%%%
%%%%%              %%%%%
%%%%%   Addition   %%%%%
%%%%%              %%%%%
%%%%%%%%%%%%%%%%%%%%%%%%

\section{Addition}
\label{chap_addition}

The addition algorithm presented in this chapter is based on the algorithm presented in \cite{salem07},
but extended to operate on divisors not considered \cite{salem07}.
In particular, the authors in \cite{salem07} only considered adding disjoint typical degree 3 divisors,
and their algorithm would only return a meaningful value when the sum of the divisors was also typical ---
otherwise it would return an error code.
They also assumed the $C$ is defined over a large finite field by an equation of the short form \ref{eq_c34_short}.
The modification to the algorithm in this chapter will operate on typical or atypical divisors of degree 3 or less
and non-disjoint divisors,
and it will return a correct result (up to equivalence in the divisor class group)
even when the sum of the divisors is atypical.
It also works for the more general curve equation \ref{eq_c34}.

In \cite{arita05-2}, Arita's algorithm also works in this more general setting.
As we are adopting Arita's classification of divisors into types,
this chapter is in a sense a marriage of Khuri-Makdisi/Abu Salem's and Arita's algorithms.
Khuri-Makdisi and Abu Salem's algorithm was a significant speed-up over Arita's in the typical setting that they considered.
The generalization in this chapter maintains that speed improvement.
Chapter \ref{chap_typical_case} will show a speed-up over Khuri-Makdisi/Abu Salem as well.

We wish to add two divisors $D$ and $D'$ and receive a divisor $D'' \equiv D + D'$.
We make the following assumptions on $D$ and $D'$:
\begin{enumerate}[label=(\roman*)]
  \item $D$ and $D'$ are reduced;
  \item $D \neq D'$;
  \item $\deg D \geq \deg D'$.
\end{enumerate}
When $D$ or $D'$ is unreduced, they can be reduced using algorithms presented in Chapter \ref{chap_reduction}.
When $D = D'$, their sum is $2D$, which may be computed using algorithms in Chapter \ref{chap_doubling}.
As for the third assumption,
the addition algorithm described in this chapter boils down to constructing and row-reducing a matrix.
The dimensions of this matrix depends on the degree of $D'$.
By assuming $D'$ is of lesser degree, we may work with a smaller matrix, resulting in faster computations.
Should $D'$ be of greater degree than $D$, we may swap them before adding.

The algorithm for addition is based on a few key observations.
The first is that, after recognizing that adding divisors amounts to multiplying ring ideals,
this ring-theoretic problem can be reduced to a linear algebra problem by representing our ideals by vector spaces.
The authors in \cite{salem07} and \cite{arita05-2} give two different ways of translating the problem into linear algebra.
The method of \cite{salem07} is described below, with some generalizations.

The next observation is that for any divisors $D$ and $D'$,
\begin{equation}
  \label{eq_lcm_plus_gcd}
  D + D' = \lcm(D, D') + \gcd(D, D'),
\end{equation}
just as for ideals $I$ and $J$ in a Dedekind domain, $IJ = (I \cap J)(I + J)$.
Typically, $D$ and $D'$ are disjoint, in which case $\gcd(D, D') = 0$ and $D + D' = \lcm(D, D')$.
This is analogous to the case for ideals in a commutative ring with identity,
where when $I$ and $J$ are coprime, $IJ = I \cap J$.

In \cite{salem07}, the authors compute the $\lcm$ of two divisors by computing the kernel of a matrix.
The third observation is that the $\gcd$ may be computed from the image of the same matrix.
By computing the $\lcm$, we have already done most of the work required to compute the $\gcd$.
This observation is what allows us to extend the algorithm in \cite{salem07} to the non-disjoint case.



%%%%%%%%%%%%%%%%%%%%%%%%%%%%%%%%%%%%%%%%%%%%%
%%%%%                                   %%%%%
%%%%%   Translating to Linear Algebra   %%%%%
%%%%%                                   %%%%%
%%%%%%%%%%%%%%%%%%%%%%%%%%%%%%%%%%%%%%%%%%%%%

\subsection{Translating to Linear Algebra}

For any divisor $A$, the ideal $I_A$ has structure as an infinite-dimensional $K$-vector space.
Any $K$-basis for $I_A$ as a $K$-vector space also generates $I_A$ as an ideal.
However, the infinitely many basis vectors for $I_A$ is more information than we need.
We may restrict to a finite subspace without losing any information.

Define the spaces
\begin{align*}
  W^m &= \{ f \in K[C] ~:~ \LM(f) \leq m \} \\
      &= \Span\{ \mu \in \cal M ~:~ \mu \leq m \}
\end{align*}
and
\begin{align*}
  W_D^m &= \{ f \in I_D ~:~ \LM(f) \leq m \}.
\end{align*}
These spaces are simply Riemann-Roch spaces under a different notation.\footnote
{Namely, $W_D^{x^iy^j} = \cal L(-D + (3i+4j)P_\infty)$.}
It is a well-known fact in algebraic geometry that Riemann-Roch spaces are finite-dimensional.
We give the dimensions explicitly:

\begin{theorem}
  \label{thm_dim_W}
  For a divisor $D$ and monomial $m$, $W^m$ and $W_D^m$ are finite-dimensional.
  In particular,
  \begin{align*}
    \dim W^m &= \begin{cases}
                  1 & m = 1 \\
                  2 & m = x \\
                  3 & m = y \\
                  3i + 4j - 2 & m = x^iy^j > y
                \end{cases}. \\
    \dim W_D^m &= \# \{ \mu \in \cal M ~:~ \mu \leq m, ~\mu \in \LT(I_D) \}.
  \end{align*}
  For sufficiently large $m$,
  \[ \dim W_D^m = \dim W^m - \deg D.\]
\end{theorem}
\begin{remark}
In theory, ``sufficiently large'' means $m \geq m'$,
where $m'$ is the maximum of the set $\{ \mu \in \cal M ~:~ \mu \not\in \LT(I_D) \}$,
as in the proof to follow.
In practice, when computing $D + D$', if $\deg(D + D') = d$,
we will choose $m$ to be the largest monomial found among all reduced Gr\"obner bases of divisors of degree $d$.
\end{remark}
\begin{proof}
  The dimension of $W^m$ is the number of monomials less than or equal to $m$,
  though modulo the curve equation $C$, $y^3$ can be replaced by $x^4$ plus smaller monomials.
  For readability, define by $f(i,j) = \dim W^{x^iy^j}$.
  Recalling the monomial order (eliminating powers of $y$ greater than 2),
    \[ 1 < x < y < x^2 < xy < y^2 < x^3 < x^2y < xy^2 < x^4 < x^3y < x^2y^2 < \dots, \]
  it is clear that
  \begin{align*}
    f(0,0) &= 1 && f(2,0) &= 4 \\
    f(1,0) &= 2 && f(1,1) &= 5 \\
    f(0,1) &= 3 && f(0,2) &= 6,
  \end{align*}
  that for $i \geq 0$, $f(2 + i,0) = 3i + f(2, 0)$, and that for $0 \leq j \leq 2$, $f(i,j) = j + f(i+j,0)$.
  By the curve equation, for $k \geq 0$, $f(i, 3k + j) = f(4k + i, j)$.
  Putting these all together.
  \begin{align*}
    f(i,j)
      &= f(i, 3k + \ell) & j = 3k + \ell, 0 \leq \ell \leq 2 \\
      &= f(4k + i, \ell) \\
      &= \ell + f(4k + i + \ell, 0) \\
      &= \ell + f(2 + 4k + i + \ell - 2, 0) \\
      &= \ell + 3(4k + i + \ell - 2) + f(2, 0) \\
      &= 12k + 3i + 4\ell - 2 \\
      &= 3i + 4j - 2.
  \end{align*}
  
  For the second claim, we show that there is a basis for $W_D^m$ with one vector per monomial in $\LT(I_D)$.
  Certainly if $\mu$ is a monomial in $\LT(I_D)$ if and only if there is a polynomial $f \in I_D$ with $\LM(f) = \mu$.
  If $f, g \in W_D^m$ are distinct but have the same leading monomial,
  we can subtract a multiple of one from the other to get a polynomial $h$ with a smaller leading monomial.
  An inductive argument then shows that $f$ is a linear combination of $g$ and some smaller polynomials.
  
  For the final claim, the set $\{ \mu \in \cal M ~:~ \mu \not\in \LT(I_D) \}$ has a maximum $m'$.
  Hence for $m \geq m'$,
  \begin{align*}
    \dim W_D^m
      &= \# \{ \mu \in \cal M ~:~ \mu \leq m, ~\mu \in \LT(I_D) \} \\
      &= \# \{ \mu \in \cal M ~:~ \mu \leq m \} - \# \{ \mu \in \cal M ~:~ \mu \leq m, ~\mu \not\in \LT(I_D) \} \\
      &= \dim W^m - \deg D.
  \end{align*}
\end{proof}

\begin{proposition}
  Let $D$ and $D'$ be divisors and let $L = \lcm(D, D')$ and $G = \gcd(D, D')$.
  Let $m$ be a monomial. Then
    \[ W_L^m = W_D^m \cap W_{D'}^m \]
  and
    \[ W_G^m = W_D^m + W_{D'}^m. \]
\end{proposition}
\begin{proof}
  Considering $I_{(-)}$ as an infinite-dimensional vector space,
  \begin{align*}
    W_L^m &= W^m \cap  I_L                  & W_G^m &= W^m \cap I_G \\
          &= W^m \cap (I_D \cap I_{D'})            &&= W^m \cap (I_D + I_{D'}) \\
          &= W^m \cap (W_D^m \cap W_{D'}^m)        &&= W^m \cap (W_D^m + W_{D'}^m) \\
          &= W_D^m \cap W_{D'}^m                   &&= W_D^m + W_{D'}^m.
  \end{align*}
\end{proof}

To compute $L$, we first choose a sufficiently large monomial $m$ to bound our vector spaces.
We choose $m$ to be large enough to represent any generator of $I_L$, whatever the form of $I_L$ may turn out to be.
If $d = \deg(D + D')$, $L$ will have degree at most $d$.
We look at all forms of ideals for divisors of degree $d$ in Table \ref{tab_divisor_types},
and pick $m$ to be the largest monomial found among these.
For example, if $d = 2$, then $m = y^2$. If $d = 6$, then $m = x^4$.

After choosing $m$, we next compute $L$ by taking the intersection of the spaces $W_D^m$ and $W_{D'}^m$.
That is, we wish to find all polynomials in both $W_D^m$ and $W_{D'}^m$.
Roughly speaking, this is the same as computing the kernel of the quotient map $W_D^m \to W_D^m/W_{D'}^m$.
An element in $W_D^m$ vanishes under this map if and only if it is also in $W_{D'}^m$.
More accurately, we must first inject $W_D^m$ and $W_{D'}^m$ into a common space for this quotient to be properly defined.
Let $M$ be the matrix of the composite linear transformation
\begin{center}
\begin{tikzcd}
  W_D^m \arrow[hook]{r}{\iota} \arrow[bend left]{rr}{M} & W^m \arrow[two heads]{r}{\pi} & \frac {W^m} {W_{D'}^m}
\end{tikzcd}
\end{center}
where $\iota$ is the canonical inclusion map and $\pi$ is the canonical quotient map.

\begin{proposition}
  \label{prop_im_M}
  The image of $M$ is $\frac{W_G^m}{W_{D'}^m}$.
\end{proposition}
That is, while the kernel of $M$ is exactly $W_L^m$,
the image of $M$ is not exactly $W_G^m$.
We will see that we do have enough information to recover $W_G^m$.
\begin{proof}
  It is enough to show that $\im M$ is isomorphic to and contained in $W_G^m / W_{D'}^m$.
  By the first and second isomorphism theorems for vector spaces
  (see Theorems IV.1.7 and IV.1.9 in \cite{hungerford}, or many other algebra and linear algebra texts),
  \[ \im M \overset{\text{(1st)}}{\cong} \frac {W_D^m} {W_L^m}
           = \frac {W_D^m} {W_D^m \cap W_{D'}^m}
           \overset{\text{(2nd)}}{\cong} \frac {W_D^m + W_{D'}^m} {W_{D'}^m}
           = \frac {W_G^m} {W_{D'}^m}. \]
  Let $f \in W_D^m$.
  Then $M(f) = [f] \in \frac {W^m} {W_{D'}^m}$.
  However $f$ is also in $W_D^m + W_{D'}^m = W_G^m$,
  therefore $[f]$ is also in $\frac {W_G^m} {W_{D'}^m}$.
\end{proof}
The full picture is now
\begin{center}
  \begin{tikzcd}
    W_L^m \arrow[hook]{r}{\ker M} & 
    W_D^m \arrow[hook]{r}{\iota} \arrow[bend left]{rr}{M} & 
    W^m \arrow[two heads]{r}{\pi} & 
    \frac {W^m} {W_{D'}^m} \arrow[two heads]{r}{\im M} & 
    \frac {W_G^m} {W_{D'}^m}
  \end{tikzcd}
\end{center}

Since $D'$ is assumed to be reduced and therefore one of only a few possible types,
we can write $\im M$ even more explicitly.
\[
  \im M = \begin{cases}
    W_G^1 & \type(D') = 11 \\
    W_G^x & \type(D') = 21 \\
    W_G^y \cap \Span\{ 1, y \} & \type(D') = 22 \\
    W_G^y & \type(D') = 31
  \end{cases}.
\]

\begin{proposition}
  \label{prop_deg_L_G}
  The dimensions of $W_L^m$ and $W_G^{m'}$ are
  \begin{enumerate}[label=(\roman*)]
    \item $\dim W_L^m    = \nullity M$;
    \item $\dim \frac {W_G^m} {W_{D'}^m} = \rank M$.
  \end{enumerate}
  For a sufficiently large monomial $m$,
  \begin{enumerate}[label=(\roman*)]
    \setcounter{enumi}{2}
    \item $\deg L = \deg D  + \rank M$;
    \item $\deg G = \deg D' - \rank M$;
    \item $D + D' = L \iff G = 0 \iff \rank M = \deg D'$.
  \end{enumerate}
\end{proposition}
\begin{proof}
  Parts (i) and (ii) are immediate from the Rank-nullity Theorem.
  For part (iii),
  \begin{align*}
    \deg L
      &= \dim W^m - \dim W_L^m & \text{theorem \ref{thm_dim_W}} \\
      &= \dim W^m - \nullity M & \text{part (i)} \\
      &= \dim W^m - \dim W_D^m + \rank M & \text{rank-nullity} \\
      &= \deg D + \rank M & \text{theorem \ref{thm_dim_W}}.
  \end{align*}
  For part (iv),
  \begin{align*}
    \deg G
      &= \deg G + \deg L - \deg L \\
      &= \deg D' + \deg D - \deg L & L + G = D + D' \\
      &= \deg D' - \rank M & \text{part (iii)}.
  \end{align*}
  Finally, for part (v), it is clear that $D + D' = L \iff G = 0$ by equation \ref{eq_lcm_plus_gcd}.
  Suppose $D + D' = L$. Then
  \begin{align*}
    D + D' &= L \\
    \deg D + \deg D' &= \deg L \\
    \deg D + \deg D' &= \deg D + \rank M & \text{part (iii)} \\
    \deg D' &= \rank M.
  \end{align*}
  Conversely, suppose $\deg D' = \rank M$.
  Then by part (iv), $\deg G = 0$.
  Since $G$ is an effective divisor, $G = 0$.
\end{proof}

In the following subsections, we work out some extended examples of computing $L$ and $G$.



%%%%%%%%%%%%%%%%%%%%%%%%%%%%%%%%
%%%%%                      %%%%%
%%%%%   Disjoint Example   %%%%%
%%%%%                      %%%%%
%%%%%%%%%%%%%%%%%%%%%%%%%%%%%%%%

\subsection{Example -- Computing $\ker M$}

Let $C : y^3 + x^4 + 1$ be a $C_{3,4}$ curve over $\bb F_{11}$.
Let $D$ and $D'$ be type 31 divisors with
\begin{align*}
  D  &= \pid{f, g, h}     & D' &= \pid{f', g', h'} \\
  f  &= x^2 + 3y + 7x + 5 & f' &= x^2 + 6y + 3x - 2 \\
  g  &= xy + 2y + 2x + 9  & g' &= xy + 5y + 5x + 9 \\
  h  &= y^2 + 4y + 2x + 3 & h' &= y^2 - y - x + 5.
\end{align*}
These divisors are disjoint, with
\begin{align*}
  D &= 2 \cdot (7 : 6 : 1) + (10 : 4 : 1) \\
  D' &= (5 : 1 : 1) + (2\alpha + 6 : 7\alpha : 1) + (9\alpha + 3 : 4\alpha + 6 : 1)
\end{align*}
where $\alpha \in \bb F_{11^2}$ is a root of $x^2 + 7x + 2$.

The sum $D + D'$ is a degree 6 divisor. Referring to Table \ref{tab_divisor_types},
we see that no generator of any reduced Gr\"obner basis of a degree 6 divisor has a monomial larger than $x^4$.
Therefore $W^{x^4}$ will be a sufficiently large space in which to perform our computations.
We proceed by computing the matrix $M$ in
\begin{center}
  \begin{tikzcd}
    W_L^{x^4} \arrow[hook]{r}{\ker M} & 
    W_D^{x^4} \arrow[hook]{r}{\iota} \arrow[bend left]{rr}{M} & 
    W^{x^4} \arrow[two heads]{r}{\pi} & 
    \frac {W^{x^4}} {W_{D'}^{x^4}} \arrow[two heads]{r}{\im M} &
    \frac {W_G^{x^4}} {W_{D'}^{x^4}} \\ &
    K^7 \arrow[hook]{r} \arrow[no head]{u}{\rotatebox{90}{$\simeq$}} &
    K^{10} \arrow[two heads]{r} \arrow[no head]{u}{\rotatebox{90}{$\simeq$}} &
    K^3 \arrow[no head]{u}{\rotatebox{90}{$\simeq$}}
  \end{tikzcd},
\end{center}
where the bottom row serves to show the dimensions of these spaces.
The dimensions of $\ker M$ and $\im M$ are not yet known until we do some more calculations.

The spaces $W_D^{x^4}$ and $W_{D'}^{x^4}$ are 7-dimensional, spanned by the polynomials
\begin{align*}
  W_D^{x^4} &= \Span\{ f, g, h, xf, xg, xh, x^2f \} \\
  W_{D'}^{x^4} &= \Span\{ f', g', h', xf', xg', xh', x^2f' \}.
\end{align*}
We reduce the basis of $W_D^{x^4}$ modulo $W_{D'}^{x^4}$ to get the matrix
\[ M = \begin{pmatrix}
  7 & 0 & 9 & 2 & 10 & 5 & 2 \\
  4 & 8 & 3 & 10 & 2 & 8 & 6 \\
  8 & 8 & 5 & 2 & 0 & 1 & 7
\end{pmatrix}, \]
where, e.g., the reduction of $f$ is $\bar f = 8y + 4x + 7$,
the reduction of $g$ is $\bar g = 8y + 8x$, etc.
This matrix $M$ has the reduced row echelon form
and kernel
\[ M_{\text{rref}} = \begin{pmatrix}
  1 & 0 & 6 & 0 & 6 & 9 & 2 \\
  0 & 1 & 7 & 0 & 9 & 8 & 10 \\
  0 & 0 & 0 & 1 & 6 & 4 & 5
\end{pmatrix} ~~~ \ker M =
\begin{pmatrix}
  -6 & -6 & -9 & -2 \\
  -7 & -9 & -8 & -10 \\
   1 &  0 &  0 &  0 \\
   0 & -6 & -4 & -5 \\
   0 &  1 &  0 &  0 \\
   0 &  0 &  1 &  0 \\
   0 &  0 &  0 &  1
\end{pmatrix}. \]
The kernel is 4-dimensional, spanned by the vectors
\begin{align*}
  \ker M
    &= \Span \left\{ \begin{array}{l}
               h -  7g - 6f, \\
        xg - 6xf -  9g - 6f, \\
        xh - 4xf -  8g - 9f, \\
      x^2f - 5xf - 10g - 2f \end{array} \right\} \\
    &= \Span \left\{ \begin{array}{l}
      y^2 + 4xy + 5x^2 + 5y + x - 2, \\
      x^2y + 5x^3 - 3xy - 2x^2 - 3y - 4x - 1, \\
      xy^2 - 4x^3 - 5xy - 2x^2 + y + 3x + 4, \\
      x^4 + 3x^2y + 2x^3 - 3xy + x^2 - 4y - 4x - 1
    \end{array} \right\}.
\end{align*}
These four polynomials form a basis for $W_L^{x^4}$ and a Gr\"obner basis for $I_L$,
but not a \emph{reduced} Gr\"obner basis.
By Proposition \ref{prop_deg_L_G}, $\deg L = 6$ and $\deg G = 0$.
We have just determined that $I_L$ contains polynomials with leading monomials $y^2$ and $x^2y$,
so by Arita's classification of divisors, $L$ must be of type 63 and the first two polynomials
alone form a reduced Gr\"obner basis,
  \[ I_L = \pid{ y^2 + 4xy + 5x^2 + 5y + x - 2, ~x^2y + 5x^3 - 3xy - 2x^2 - 3y - 4x - 1}.\]

\begin{remark}
In this example, we computed the kernel of $M$ to get four polynomials, though two were unneeded.
Consequently, two columns of the matrix $M$ were unneeded.
An efficient implementation of divisor arithmetic will avoid computing unnecessary columns of $M$
by computing them only as they become necessary.
\end{remark}



%%%%%%%%%%%%%%%%%%%%%%%%%%%%%%%%%%%%
%%%%%                          %%%%%
%%%%%   Non-Disjoint Example   %%%%%
%%%%%                          %%%%%
%%%%%%%%%%%%%%%%%%%%%%%%%%%%%%%%%%%%

\begin{comment}
\subsection{An example -- non-disjoint case}

Consider again the curve $C : y^3 + x^4 + 1$ over $\bb F_{11}$.
Let $D$ and $D'$ be type 31 divisors with
\begin{align*}
  D  &= \pid{f, g, h}      & D' &= \pid{f', g', h'} \\
  f  &= x^2 + 10           & f' &= x^2 + 3y + 2x + 2 \\
  g  &= xy + y + 9x + 9    & g' &= xy + 4x + 5 \\
  h  &= y^2 + 10y + 4x + 5 & h' &= y^2 + y + 2x + 3 .\\
\end{align*}
These divisors are non-disjoint. We have
\begin{align*}
  D &= (1 : 2 : 1) + (10 : 2\alpha + 4 : 1) + (10 : 10\alpha + 8 : 1) \\
  D' &= (1 : 2 : 1) + (3 : 9 : 1) + (5 : 6 : 1)
\end{align*}
where $\alpha \in \bb F_{11^2}$ is a root of $x^2 + 7x + 2$.

\begin{center}
  \begin{tikzcd}
    W_L^{x^4} \arrow[hook]{r}{\ker M} & 
    W_D^{x^4} \arrow[hook]{r}{\iota} \arrow[bend left]{rr}{M} & 
    W^{x^4} \arrow[two heads]{r}{\pi} & 
    \frac {W^{x^4}} {W_{D'}^{x^4}} \arrow[two heads]{r}{\im M} &
    W_G^y \\ &
    K^7 \arrow[hook]{r} \arrow[no head]{u}{\rotatebox{90}{$\simeq$}} &
    K^{10} \arrow[two heads]{r} \arrow[no head]{u}{\rotatebox{90}{$\simeq$}} &
    K^3 \arrow[no head]{u}{\rotatebox{90}{$\simeq$}}.
  \end{tikzcd}
\end{center}

As in the previous example, we reduce the basis of $W_{D}^{x^4}$ modulo the basis for $W_{D'}^{x^4}$ to get the matrix
\[ M = \begin{pmatrix}
  8 & 4 & 2 & 8 & 7 & 6 & 10 \\
  9 & 5 & 2 & 2 & 1 & 6 & 2 \\
  8 & 1 & 9 & 6 & 7 & 5 & 5
\end{pmatrix}. \]
This matrix has the reduced row echelon form
\[ M = \begin{pmatrix}
  1 & 0 & 6 & 8 & 5 & 7 & 5 \\
  0 & 1 & 5 & 8 & 0 & 4 & 9 \\
  0 & 0 & 0 & 0 & 0 & 0 & 0
\end{pmatrix}. \]
We can see the $M$ does not have full rank.
Hence $L$ will not be of degree 6 and $G$ will be non-zero.
The kernel of $M$ is
\begin{align*}
  \ker M
    &= \Span \left\{ \begin{array}{l}
      h - 5g - 6f, \\
      xf - 8g - 8f, \\
      xg - 5f, \\
      xh - 4g - 7f, \\
      x^2f - 9g - 5f \end{array} \right\} \\
    &= \Span \left\{ \begin{array}{l}
      y^2 + 6xy + 5x^2 + 5y + 3x + 10, \\
      x^3 + 3xy + 3x^2 + 3y + 4x + 2, \\
      x^2y + xy + 4x^2 + 9x + 5, \\
      xy^2 + 6xy + 8x^2 + 7y + 2x + 4, \\
      x^4 + 2xy + 5x^2 + 2y + 7x + 1
    \end{array} \right\}
\end{align*}
and $L$ is a type 51 divisor whose ideal $I_L$ has the reduced Gr\"obner basis
\[ I_L = \left\langle\begin{array}{l}
  y^2 + 6xy + 5x^2 + 5y + 3x + 10, \\ 
  x^3 + 3xy + 3x^2 + 3y + 4x + 2, \\ 
  x^2y + xy + 4x^2 + 9x + 5\end{array}\right\rangle. \]
The pivot columns of $M$ determine its image.
The first and second columns are pivots, thus the image of $M$ is given by the column basis
\[ \im M = \begin{pmatrix}
  8 & 4 \\
  9 & 5 \\
  8 & 1 
\end{pmatrix}. \]
That is, $\im M$ is spanned by the vectors $8y + 9x + 8$ and $y + 5x + 8$.
This column basis is reducible to 
\[ \im M = \begin{pmatrix}
  10 & 9 \\
   1 & 0 \\
   0 & 1 
\end{pmatrix}, \]
so that $I_G = \pid{x + 10, y + 9} = \pid{x - 1, y - 2}$,
which agrees with $D$ and $D'$ sharing the point $(1 : 2 : 1)$ in common.

Now we have $L$ and $G$ of degrees 5 and 1, respectively, such that $D + D' = L + G$.
To finish, we must compute the reduction $\bar{\bar L}$ of $L$ and compute $D + D' = \bar{\bar L} + G$.
The reduction $\bar{\bar L}$ will be of type 31, thus the problem of adding $D$ and $D'$
is reduced to adding a degree 3 divisor $\bar{\bar L}$ with a degree 1 divisor $G$.
\end{comment}


%%%%%%%%%%%%%%%%%%%%%%%%%%%%%%%%%%%%%%%%%%%%
%%%%%                                  %%%%%
%%%%%   Another Non-Disjoint Example   %%%%%
%%%%%                                  %%%%%
%%%%%%%%%%%%%%%%%%%%%%%%%%%%%%%%%%%%%%%%%%%%

\subsection{Example -- Computing $\im M$}

Consider again the curve $C : y^3 + x^4 + 1$ over $\bb F_{11}$.
Let $D$ and $D'$ be type 31 divisors with
\begin{align*}
  D  &= \pid{f, g, h}      & D' &= \pid{f', g', h'}  \\
  f  &= x^2 +  y + 5x + 1  & f' &= x^2 - 3y - 5x - 3 \\
  g  &=  xy + 2y - 3x + 2  & g' &=  xy - 4y + 4x - 4 \\
  h  &= y^2 + 3y + 3x + 2  & h' &= y^2 -  y + 4x - 2.
\end{align*}
These divisors are non-disjoint, with
\begin{align*}
  D  &= (0 : 10 : 1) + (3 : 8 : 1) + (1 : 4 : 1) \\
  D' &= (0 : 10 : 1) + (3 : 8 : 1) + (6 : 1 : 1).
\end{align*}

As in the previous example, we reduce the basis of $W_{D}^{x^4}$ modulo the basis for $W_{D'}^{x^4}$ to get the matrix
\[ M = \begin{pmatrix}
   4 & 6 & 4 & 2 & 3 & 2 & 1 \\
  10 & 4 & 2 & 5 & 2 & 5 & 8 \\
   4 & 6 & 4 & 2 & 3 & 2 & 1
\end{pmatrix}. \]
This matrix has the reduced row echelon form
\[ M = \begin{pmatrix}
  1 & 7 & 1 & 6 & 9 & 6 & 3 \\
  0 & 0 & 0 & 0 & 0 & 0 & 0 \\
  0 & 0 & 0 & 0 & 0 & 0 & 0
\end{pmatrix}. \]
We can see the $M$ does not have full rank.
In this case, $\rank M = 1$.
By Proposition \ref{prop_deg_L_G},
$\deg L = 4$, and $G$ is non-zero with $\deg G = 2$.
We have a non-trivial image to compute.
\begin{comment}
\begin{align*}
  \ker M
    &= \Span \left\{ \begin{array}{l}
      g - 7f, \\
      h -  f, \\
      xf - 6f, \\
      xg - 9f, \\
      xh - 6f, \\
      x^2f - 3f \end{array} \right\} 
    = \Span \left\{ \begin{array}{l}
      xy + 4x^2 - 5y - 5x - 5, \\
      y^2 - x^2 + 2y - 2x + 1, \\
      x^3 + xy - x^2 + 5y + 4x + 5, \\
      x^2y + 2xy - x^2 + 2y + x + 2, \\
      xy^2 + 3xy - 3x^2 + 5y + 5x + 5, \\
      x^4 + x^2y + 5x^3 - 2x^2 - 3y - 4x - 3
    \end{array} \right\}.
\end{align*}
Comparing the result to Table \ref{tab_divisor_types},
$L$ is degree 4 and $I_L$ has polynomials with leading monomials $xy$, $y^2$, and $x^3$,
so $L$ must be type 41 and $I_L$ is generated by the first three polynomials.
After reducing the third polynomial modulo the first to eliminate the $xy$ term,
we have a reduced Gr\"obner basis for $I_L$,
\[ I_L = \left\langle\begin{array}{l}
      xy  + 4x^2 - 5y - 5x - 5, \\
      y^2 -  x^2 + 2y - 2x + 1, \\
      x^3 - 5x^2 -  y - 2x - 1
  \end{array}\right\rangle. \]
\end{comment}
The image of $M$ is determined by its pivot columns.
In this case, there is only one pivot column --- the first.
The image of $M$ is therefore given by the column basis
\[ \im M = \Span \begin{pmatrix} 4 \\ 10 \\ 4 \end{pmatrix}
         = \Span \begin{pmatrix} 1 \\ -3 \\ 1 \end{pmatrix}. \]
That is, $\im M = \frac {W_G^m}{W_{D'}^m}$ is spanned by the polynomial $u = y - 3x + 1$.
Since $G$ has degree 2 and contains a polynomial with leading monomial $y$,
$G$ must be of type 21 and we must find its other generator $v$ with leading monomial $x^2$.
Since $I_D \subseteq I_G$, $f$ must be an element of $I_G$, and we may reduce $f$ modulo $u$ to obtain $v$.
This gives
  \[ I_G = \pid{u,v} = \pid{y - 3x + 1, ~x^2 - 3x}. \]
Note that $u$ and $v$ intersect at the points $(0 : 10 : 1)$ and $(3 : 8 : 1)$, as expected.



%%%%%%%%%%%%%%%%%%%%%%%%%%%%%%%%%%
%%%%%                        %%%%%
%%%%%   Infinite Recursion   %%%%%
%%%%%                        %%%%%
%%%%%%%%%%%%%%%%%%%%%%%%%%%%%%%%%%

\subsection{Avoiding Infinite Recursion}

When $D$ and $D'$ are non-disjoint, we compute $D + D' = \bar{\bar L} + G$.
We recursively apply the addition algorithm on $\bar{\bar L}$ and $G$.
If not handled properly, this can lead to infinite recursion.
For example, suppose $D = P + 2Q$ and $D' = P + Q$.
Then $L = P + 2Q = D$ (and is already reduced) and $G = P + Q = D'$.
Attempting to compute $D + D'$ by $\bar{\bar L} + G$ brings us full circle.
We should only compute $\bar{\bar L} + G$ when we are certain that $L$ and $G$ are distinct from $D$ and $D'$.

\begin{proposition}
  \label{prop_rank_M}
  The following are equivalent:
  \begin{enumerate}[label=(\roman*)]
    \item $L = D$;
    \item $G = D'$;
    \item $D' \leq D$;
    \item $\rank M = 0$.
  \end{enumerate}
\end{proposition}
\begin{proof}
  \begin{description}
    \item[(i) $\iff$ (ii):]
      Immediate from Equation \ref{eq_lcm_plus_gcd}.
    \item[(i) $\implies$ (iii):]
      $D' \leq L$ and $L = D$, hence $D' \leq D$.
    \item[(iii) $\implies$ (iv):]
      If $D' \leq D$, then $I_D \subseteq I_{D'}$ and $W_D^m \subseteq W_{D'}^m$.
      Every element in $W_D^m$ vanishes under $M$,
      $\nullity M = \dim W_D^m$ and $\rank M = 0$.
    \item[(iv) $\implies$ (i):]
      Suppose $\rank M = 0$.
      By proposition \ref{prop_deg_L_G}, $\deg L = \deg D$.
      Since $D \leq L$ and $D$ and $L$ have the same degree, $L = D$.
  \end{description}
\end{proof}
\begin{corollary}
  \[ G < D' \iff \rank M > 0. \]
\end{corollary}

When $\rank M = \deg D'$, we can compute $D + D' = L$.
When $0 < \rank M < \deg D'$, we can compute $D + D' = \bar{\bar L} + G$, reducing to a ``smaller'' case with $G < D'$.
When $\rank M = 0$, this strategy leads to infinite recursion, and we must do something else to ensure this terminates.



%%%%%%%%%%%%%%%%%%%%%%%%%%
%%%%%                %%%%%
%%%%%   Rank M = 0   %%%%%
%%%%%                %%%%%
%%%%%%%%%%%%%%%%%%%%%%%%%%

%\subsection{When $\rank M = 0$}

By Proposition \ref{prop_rank_M}, if $\rank M = 0$, then $D' \leq D$.
Since we are assuming $D' \neq D$, we have $D' < D$.
Since $D$ is reduced and of degree at most 3, $\deg D'$ is either 1 or 2.
We consider each of these cases separately, beginning with the latter.

If $D'$ is of degree 2, then $D = D' + P$ for some point $P$, which we must find.
Once we have, we compute
  \[ D + D' = \bar{\bar{2D'}} + P, \]
thereby reducing to the case of adding a reduced divisor to a degree 1 divisor.
Finding $P$ amounts to finding $x_0, y_0 \in K$ such that
  \[ I_D = I_{D'}\pid{x + x_0, y + y_0}. \]
\note{Give explicit formulas for this in the appendix.}

If $D'$ is of degree 1, then $D = nD' + A$ where $D'$ and $A$ are disjoint and $1 \leq n \leq 3$.
We must find $D''$ and $n$ and compute
  \[ D + D' = \bar{\bar{(n + 1)D'}} + A. \]
\note{Give explicit formulas for this in the appendix.}
Computing $(n + 1)D'$ requires us to be able to double, triple, or quadruple a point.
Doubling is the topic of the next chapter and quadrupling is accomplished by doubling twice.
\note{Explicit formulas for tripling also to be given in appendix? Or next chapter?}

\begin{comment}
Having now shown that we can compute $D + D'$ in a way that will terminate,
we must now show how to find the relevant values $D''$, $P$, and $n$ where necessary.

Suppose $\deg D' = 2$, and we wish to find $P$ such that $D = P + D'$.
Since $\deg D > \deg D'$, we must have $\type D = 31$.
It may be that $\type D' = 21$ or $\type D' = 22$, so we consider each case separately.

Suppose first that $D'$ is of type 21. 
Let $P$ be represented by the type 11 divisor $\pid{s, t} = \pid{x + s_0, y + t_0}$.
Let $D$ and $D'$ be
\begin{align*}
  D &= \pid{f, g, h} & D' &= \pid{f', g'} \\
  f &= x^2 + f_2y + f_1x + f_0 & f' &= y + f'_1x + f'_0 \\
  g &= xy + g_2y + g_1x + g_0 & g' &= x^2 + g'_1x + g'_0 \\
  h &= y^2 + h_2y + h_1x + h_0.
\end{align*}

We need to solve for $s_0$ and $t_0$ such that
\[ \pid{f, g, h} = \pid{s, t}\pid{f', g', h'}. \]
A solution exists and must satisfy $sf' - f'_1f - g \in \pid{f,g,h}$.
We have that
  \[ sf' - f'_1f - g = (s_0 - f_2f'_1 - g_2)y + (f'_0 + f'_1s_0 - f_1f'_1 - g_1)x + (f'_0s_0 - f_0f'_1 - g_0). \]
It must be that each of these coefficients are zero, otherwise $f, g, h$ would not form a reduced Gr\"obner basis.
Therefore
  \[ s_0 = f_2f'_1 + g_2. \]
Likewise, we compute
  \[ tf' - f'_1g - h = (f'_0 + t_0 - f'_1g_2 - h_0)y + (f'_1t_0 - f'_1g_1 - h_1)x + (f'_0t_0 - f'_1g_0 - h_0) \]
and conclude
  \[ t_0 = - f'_0 + f'_1g_2 + h_0. \]

When $D'$ is of type 22, the same strategy yields $s_0$ and $t_0$.
We reduce $sf' = (x + s_0)(x + f'_0)$ and $tf' = (y + t_0)(x + f'_0)$ modulo $I_D$,
argue these reductions are zero, and conclude
\begin{align*}
  s_0 &= - f'_0 - f_1 \\
  t_0 &= g_1.
\end{align*}

\begin{proposition}
  Let $D$ be a typical type 31 divisor and $D'$ a type 11 divisor.
  Suppose $D' < D$.
  Then there is a type 21 divisor $D''$ such that $D = D' + D''$,
  given by $D'' = \pid{y + s_1x + s_0, x^2 + t_1x + t_0}$ where
    \begin{align*}
      s_1 &= \frac{-(g_2 - f'_0)} {f_2} \\
      s_0 &= g_1 + s_1(f_1 - f'_0) \\
      t_1 &= g_2 + f_1 - f'_0 \\
      t_0 &= f_1g_2 - f_2g_1 - f'_0t_1 + f_0.
    \end{align*}
\end{proposition}
\begin{proof}
  There certainly exists a degree 2 divisor $D''$ such that $D = D' + D''$.
  Either $D''$ is of type 21 or 22.
  The sum of a type 11 divisor with a type 22 divisor is never a typical divisor \note{(justify this?)}
  so $D''$ must be of type 21, given by polynomials $s = y + s_1x + s_0$ and $t = x^2 + t_1x + t_0$.
  Solving $\pid{f, g, h} = \pid{s,t}\pid{f',g'}$ gives $s_0, s_1, t_0, t_1$ as given in the theorem statement.
\end{proof}

\begin{proposition}
  Let $D$ be a semi-typical type 31 divisor.
  The ideal of $I_D$ of $D$ is generated by $\pid{f,g,h}$ with $f = x^2 + f_1x + f_0$.
  Let $D'$ be a type 11 divisor.
  Suppose $D' < D$. Then
  \begin{enumerate}[label=(\roman*)]
    \item $f$ has two (not necessarily distinct) rational roots.
  \end{enumerate}
  There is a degree 2 divisor $D''$ such that $D = D' + D''$ and
  \begin{enumerate}[label=(\roman*)]
    \setcounter{enumi}{1}
    \item if $f$ has distinct roots then
      \[ \type(D'') = \begin{cases} 21 & f'_0 = g_2 \\ 22 & f'_0 \neq g_2 \end{cases}\; \]
    \item if $f$ has a double root and $g'_0 = g_1$ then $\type(D'') = 21$
      \[ \type(D'') = \begin{cases} 22 & g'_0 = g_1 \\ 21 & g'_0 \neq g_1 \end{cases}\; \]
    \item if $\type(D'') = 21$, then $D'' = \pid{y + s_1x + s_0, x^2 + t_1x + t_0}$ where
    \begin{align*}
      s_1 &= \frac{g_0 - f'_0g_1} {f'_0(f_1 - f'_0) - f_0} \\
      s_0 &= g_1 + s_1(f_1 - f'_0) \\
      t_1 &= f_1 \\
      t_0 &= f_0 \;
    \end{align*}
    \item if $\type(D'') = 22$, then $D'' = \pid{x + s_0, y^2 + t_2y + t_0}$ where
    \begin{align*}
      s_0 &= f_1 - f'_0 \\
      t_2 &= h_2 \\
      t_0 &= h_0 - h_1s_0.
    \end{align*}
  \end{enumerate}
\end{proposition}
\note{These formulas still need testing.}
\note{I'm not convinced $s_1$ is correct if $f$ has a double root.}
\begin{proof}
  \begin{enumerate}[label=(\roman*)]
    \item
      Since $D' < D$, we have $I_D \subset I_{D'}$.
      In particular $f \in I_{D'}$, so $f = af'+ bg'$ for some polynomials $a$ and $b$.
      However, there is no $y$-term in $f$, so $b = 0$.
      One may then conclude that the rational root of $f'$ is also a root of $f$.
      Having one rational root, the second root of $f$ must also be rational.
    
    \item
      The divisor $D$ is the sum of a type 11 divisor $P$ and a type 22 divisor $Q + R$.
      $D''$ is of type 22 if and only if $D' = P$.
      \note{And this is determined by $D' = P \iff f'_0 \neq g_2$.}
    
    \item
      The divisor $D$ is the sum three points sharing the same $x$-coordinate.
      They cannot all be distinct, or else $D$ would be type 33 (principal).
      So at least two points in $D$ are equal, say $D = 2P + Q$.
      There are three cases:
      $P \neq Q$ and neither point is a special point;
      $P \neq Q$ and $Q$ is a special point;
      and $P = Q$ and $P$ is a special point.
      
      The cases we have ruled out:
      If $P \neq Q$ and $P$ is a special point, then $2P + Q$ is type 33.
      If $P = Q$ and $P$ is a regular point or hyperflex, then $\ord_P(f) < 3$.
      If $P = Q$ and $P$ is an inflection point, then $D = 2P + Q = 3P$ is type 33.
      
      In all three cases, $g$ defines a ``cross'' centered on $P$
      One can determine whether $D' = P$ by $D' = P \iff g'_0 = g_1$.
      Also, $P + Q$ is type $22$ in all three cases.
      
      So we have
      \begin{align*}
        g'_0 = g_1
          &\implies D' = P \\
          &\implies D = 2P + Q = P + D'' = D' + D'' \\
          &\implies D'' = P + Q \\
          &\implies \type(D'') = 22.
      \end{align*}
      Otherwise, $\type(D'') = 21$.
    
    \item
      Solving $sf' - s_1f - g = 0$ gives the stated formulas for $s_1$ and $s_0$.
      We note that $D'' < D$ implies $\pid{f,g,h} \subset \pid{s,t}$,
      $f \in \pid{s,t}$, and finally $t = f$.
    
    \item
      Solving $sf' - f = 0$ gives $s_0$.
      We note that $D'' < D$ implies $\pid{f,g,h} \subset \pid{s,t}$,
      $h \in \pid{s,t}$, and $h = t + h_1s$.
      Solving this simple equation gives $t_0$ and $t_2$.
  \end{enumerate}
\end{proof}
\end{comment}



%%%%%%%%%%%%%%%%%%%%%%%%%%%%%%%%%%%%%%
%%%%%                            %%%%%
%%%%%   The Addition Algorithm   %%%%%
%%%%%                            %%%%%
%%%%%%%%%%%%%%%%%%%%%%%%%%%%%%%%%%%%%%

\subsection{The Addition Algorithm}

We conclude this chapter by summing up its contents in Algorithm \ref{alg_addition}, below.

\begin{algorithm}[hb!]
  \label{alg_addition}
  \caption{Divisor Addition}
  {\bf Input:} Two reduced divisors $D$ and $D'$ represented by their ideals $I_D$ and $I_{D'}$
  satisfying $D \neq D'$ and $\deg D \geq \deg D'$ \\
  {\bf Output:} A divisor $D''$ equivalent to $D + D'$, represented by its ideal $I_{D''}$
  \begin{algorithmic}[1]
    \If {$D' = 0$}
      \State \Return $D$ \label{alg_addition:return_0}
    \EndIf
    \State Compute $M : W_D^m \to W_{D'}^m$
    \State Compute $M_{\text{rref}}$ and $\rank M$
    \If {$\rank M = \deg D'$}
      \State Compute $L = \lcm(D, D')$
      \State \Return $L$ \label{alg_addition:return_1}
    \EndIf
    \If {$\rank M > 0$}
      \State Compute $L = \lcm(D, D')$
      \State Compute $G = \gcd(D, D')$
      \State \Return $\bar{\bar L} + G$ \label{alg_addition:return_2}
    \EndIf
    \If {$\rank M = 0$}
      \If {$\deg D' = 2$}
        \State Compute $A$ such that $D = A + D'$
        \State \Return $\bar{\bar{2D'}} + A$ \label{alg_addition:return_3}
      \EndIf
      \If {$\deg D' = 1$}
        \State Compute $A$ and largest $n$ such that $D = A + nD'$
        \State \Return $\bar{\bar{(n+1)D'}} + A$ \label{alg_addition:return_4}
      \EndIf
    \EndIf
  \end{algorithmic}
\end{algorithm}

This algorithm requires that we be able to reduce divisors
in lines \ref{alg_addition:return_2}, \ref{alg_addition:return_3}, and \ref{alg_addition:return_4},
to double divisors in lines \ref{alg_addition:return_3} and \ref{alg_addition:return_4},
or possibly to triple degree 1 divisors in line \ref{alg_addition:return_4}.
Assume for now that we have terminating algorithms for these operations.
We wish to show that the addition Algorithm \ref{alg_addition}.

This algorithm has five return statements.
If the algorithm reaches \ref{alg_addition:return_0} or \ref{alg_addition:return_1},
then the algorithm terminates.
If the algorithm returns at line \ref{alg_addition:return_4},
then $(n + 1)D'$ and $A$ will be disjoint,
and the algorithm will reach at line \ref{alg_addition:return_1} in its next iteration
(or line \ref{alg_addition:return_0} if $A = 0$).
At lines \ref{alg_addition:return_2} and \ref{alg_addition:return_3},
$G$ and $A$ are of lesser degree than $D'$.
Eventually, a recursive call will involve $D'$ of degree 0 --
in which case the next interation reaches line \ref{alg_addition:return_0} --
or 1 --
in which case $\rank M$ is either 1 or 0
and the next iteration reaches line \ref{alg_addition:return_1} or \ref{alg_addition:return_4} and terminates.

Whether or not this algorithm terminates therefore depends on whether we have terminating algorithms
for reducing divisors, doubling divisors, and tripling points.
These are the topics of the next two chapters.

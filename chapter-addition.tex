%%%%%%%%%%%%%%%%%%%%%%%%
%%%%%              %%%%%
%%%%%   Addition   %%%%%
%%%%%              %%%%%
%%%%%%%%%%%%%%%%%%%%%%%%

\section{Addition}
\label{chap_addition}

The addition algorithm presented in this chapter is based on the algorithm presented in \cite{salem07},
but extended to operate on divisors not considered \cite{salem07}.
In particular, the authors in \cite{salem07} only considered adding disjoint typical degree 3 divisors,
and their algorithm would only return a meaningful value when the sum of the divisors was also typical ---
otherwise it would return an error code.
They also assumed the $C$ is defined over a large finite field by an equation of the short form \ref{eq_c34_short}.
The modification to the algorithm in this chapter will operate on typical or atypical divisors of degree 3 or less
and non-disjoint divisors,
and it will return a correct result (up to equivalence in the divisor class group)
even when the sum of the divisors is atypical.
It also works for the more general curve equation \ref{eq_c34}.

In \cite{arita05-2}, Arita's algorithm also works in this more general setting.
As we are adopting Arita's classification of divisors into types,
this chapter is in a sense a marriage of Khuri-Makdisi/Abu Salem's and Arita's algorithms.
Khuri-Makdisi and Abu Salem's algorithm was a significant speed-up over Arita's in the typical setting that they considered.
The generalization in this chapter maintains that speed improvement.
A later chapter \note{(which?)} will show a speed-up over Khuri-Makdisi/Abu Salem as well.

We wish to add two divisors $D$ and $D'$ and receive a divisor $D'' \equiv D + D'$.
We make the following assumptions on $D$ and $D'$:
\begin{enumerate}[label=(\roman*)]
  \item $D$ and $D'$ are reduced;
  \item $D \neq D'$;
  \item $\deg D \geq \deg D'$.
\end{enumerate}
When $D$ or $D'$ is unreduced, they can be reduced using algorithms presented in Chapter \ref{chap_reduction}.
When $D = D'$, their sum is $2D$, which may be computed using algorithms in Chapter \ref{chap_doubling}.
As for the third assumption,
the addition algorithm described in this chapter boils down to constructing and row-reducing a matrix.
The dimensions of this matrix depends on the degree of $D'$.
By assuming $D'$ is of lesser degree, we may work with a smaller matrix, resulting in faster computations.
Should $D'$ be of greater degree than $D$, we may swap them before adding.

The algorithm for addition is based on a few key observations.
The first is that, after recognizing that adding divisors amounts to multiplying ring ideals,
this ring-theoretic problem can be reduced to a linear algebra problem by representing our ideals by vector spaces.
The authors in \cite{salem07} and \cite{arita05-2} give two different ways of translating the problem into linear algebra.
The method of \cite{salem07} is described below, with some generalizations.

The next observation is that for any divisors $D$ and $D'$,
  \[D + D' = \lcm(D, D') + \gcd(D, D'),\]
just as for ideals $I$ and $J$ in a Dedekind domain, $IJ = (I \cap J)(I + J)$.
Typically, $D$ and $D'$ are disjoint, in which case $\gcd(D, D') = 0$ and $D + D' = \lcm(D, D')$.
This is analogous to the case for ideals in a commutative ring with identity,
where when $I$ and $J$ are coprime, $IJ = I \cap J$.

In \cite{salem07}, the authors compute the $\lcm$ of two divisors by computing the kernel of a matrix.
The third observation is that the $\gcd$ may be computed from the image of the same matrix.
By computing the $\lcm$, we have already done most of the work required to compute the $\gcd$.
This observation is what allows us to extend the algorithm in \cite{salem07} to the non-disjoint case.

\subsection{Translating to Linear Algebra}

For any divisor $A$, the ideal $I_A$ has structure as an infinite-dimensional $K$-vector space.
Any $K$-basis for $I_A$ as a $K$-vector space also generates $I_A$ as an ideal.
However, the infinitely many basis vectors for $I_A$ is more information than we need.
We may restrict to a finite subspace without losing any information.

Define the spaces
\begin{align*}
  W^m &= \{ f \in K[C] ~:~ \LM(f) \leq m \} \\
      &= \Span\{ \mu \in \cal M ~:~ \mu \leq m \}
\end{align*}
and
\begin{align*}
  W_D^m &= \{ f \in I_D ~:~ \LM(f) \leq m \}.
\end{align*}
These spaces are simply Riemann-Roch spaces under a different notation.\footnote
{Namely, $W_D^{x^iy^j} = \cal L(-D + (3i+4j)P_\infty)$.}
It is a well-known fact in algebraic geometry that Riemann-Roch spaces are finite-dimensional.
We give the dimensions explicitly:

\begin{theorem}
  \label{thm_dim_W}
  For a divisor $D$ and monomial $m$, $W^m$ and $W_D^m$ are finite-dimensional.
  In particular,
  \begin{align*}
    \dim W^m &= \begin{cases}
                  1 & m = 1 \\
                  2 & m = x \\
                  3 & m = y \\
                  3i + 4j - 2 & m = x^iy^j > y
                \end{cases}. \\
    \dim W_D^m &= \# \{ \mu \in \cal M ~:~ \mu \leq m, ~\mu \in \LT(I_D) \}.
  \end{align*}
  For sufficiently large $m$,
  \[ \dim W_D^m = \dim W^m - \deg D.\]
\end{theorem}
\begin{proof}
  The dimension of $W^m$ is the number of monomials less than or equal to $m$,
  though modulo the curve equation $C$, $y^3$ can be replaced by $x^4$ plus smaller monomials.
  For readability, define by $f(i,j) = \dim W^{x^iy^j}$.
  Recalling the monomial order (eliminating powers of $y$ greater than 2),
    \[ 1 < x < y < x^2 < xy < y^2 < x^3 < x^2y < xy^2 < x^4 < x^3y < x^2y^2 < \dots, \]
  it is clear that
  \begin{align*}
    f(0,0) &= 1 && f(2,0) &= 4 \\
    f(1,0) &= 2 && f(1,1) &= 5 \\
    f(0,1) &= 3 && f(0,2) &= 6,
  \end{align*}
  that for $i \geq 0$, $f(2 + i,0) = 3i + f(2, 0)$, and that for $0 \leq j \leq 2$, $f(i,j) = j + f(i+j,0)$.
  By the curve equation, for $k \geq 0$, $f(i, 3k + j) = f(4k + i, j)$.
  Putting these all together.
  \begin{align*}
    f(i,j)
      &= f(i, 3k + \ell) & j = 3k + \ell, 0 \leq \ell \leq 2 \\
      &= f(4k + i, \ell) \\
      &= \ell + f(4k + i + \ell, 0) \\
      &= \ell + f(2 + 4k + i + \ell - 2, 0) \\
      &= \ell + 3(4k + i + \ell - 2) + f(2, 0) \\
      &= 12k + 3i + 4\ell - 2 \\
      &= 3i + 4j - 2.
  \end{align*}
  
  For the second claim, we show that there is a basis for $W_D^m$ with one vector per monomial in $\LT(I_D)$.
  Certainly if $\mu$ is a monomial in $\LT(I_D)$ if and only if there is a polynomial $f \in I_D$ with $\LM(f) = \mu$.
  If $f, g \in W_D^m$ are distinct but have the same leading monomial,
  we can subtract a multiple of one from the other to get a polynomial $h$ with a smaller leading monomial.
  An inductive argument then shows that $f$ is a linear combination of $g$ and some smaller polynomials.
  
  For the final claim, the set $\{ \mu \in \cal M ~:~ \mu \not\in \LT(I_D) \}$ has a maximum $m'$.
  Hence for $m \geq m'$,
  \begin{align*}
    \dim W_D^m
      &= \# \{ \mu \in \cal M ~:~ \mu \leq m, ~\mu \in \LT(I_D) \} \\
      &= \# \{ \mu \in \cal M ~:~ \mu \leq m \} - \# \{ \mu \in \cal M ~:~ \mu \leq m, ~\mu \not\in \LT(I_D) \} \\
      &= \dim W^m - \deg D.
  \end{align*}
\end{proof}

\begin{proposition}
  Let $D$ and $D'$ be divisors and let $L = \lcm(D, D')$ and $G = \gcd(D, D')$.
  Let $m$ be a monomial. Then
    \[ W_L^m = W_D^m \cap W_{D'}^m \]
  and
    \[ W_G^m = W_D^m + W_{D'}^m. \]
\end{proposition}
\begin{proof}
  \begin{align*}
    W_L^m &= W^m \cap (I_L)                 & W_G^m &= W^m \cap I_G \\
          &= W^m \cap (I_D \cap I_{D'})            &&= W^m \cap (I_D + I_{D'}) \\
          &= W^m \cap (W_D^m \cap W_{D'}^m)        &&= W^m \cap (W_D^m + W_{D'}^m) \\
          &= W_D^m \cap W_{D'}^m                   &&= W_D^m + W_{D'}^m.
  \end{align*}
\end{proof}

To compute $L$ \note{(Still need to choose a sufficiently large $m$)}, we take the intersection of the spaces $W_D^m$ and $W_{D'}^m$.
That is, we wish to find all polynomials in both $W_D^m$ and $W_{D'}^m$.
Roughly speaking, this is the same as computing the kernel of the quotient map $W_D^m \to W_D^m/W_{D'}^m$.
An element in $W_D^m$ vanishes under this map if and only if it is also in $W_{D'}^m$.
More accurately speaking, we must first inject $W_D^m$ and $W_{D'}^m$ into a common space for this quotient to be properly defined.
Let $M$ be the matrix of the composite linear transformation
\begin{center}
\begin{tikzcd}
  W_D^m \arrow[hook]{r}{\iota} \arrow[bend left]{rr}{M} & W^m \arrow[two heads]{r}{\pi} & \frac {W^m} {W_{D'}^m}
\end{tikzcd}
\end{center}
where $\iota$ is the canonical inclusion map and $\pi$ is the canonical quotient map.

\begin{proposition}
  \label{prop_ker_im_M}
  The kernel of $M$ is $W_L^m$.
  The image of $M$ is $W_G^m \cap \Span \{ \mu \in \cal M ~:~ \mu \not \in \LT(I_D) \}$.
\end{proposition}
That is, while the kernel of $M$ is exactly $W_L^m$,
the image of $M$ is only those elements of $W_G^m$ that are generated by monomials not in $\LT(I_D)$.
We will see that this is enough to recover all of $W_G^m$.
\note{Write an example of this later.}
\begin{proof}
  \note{TODO}
\end{proof}

Most of the time, $\im M$ is simply $W_G^{m'}$ for some monomial $m' < m$.
When $D'$ is a reduced divisor, i.e. of type 11, 21, 22, or 31, we have
\[
  \im M = \begin{cases}
    W_G^1 & \type(D') = 11 \\
    W_G^x & \type(D') = 21 \\
    W_G^y \cap \Span\{ 1, y \} & \type(D') = 22 \\
    W_G^y & \type(D') = 31
  \end{cases}.
\]

We will frequently make use of the diagram
\begin{center}
  \begin{tikzcd}
    W_L^m \arrow[hook]{r}{\ker M} & 
    W_D^m \arrow[hook]{r}{\iota} \arrow[bend left]{rr}{M} & 
    W^m \arrow[two heads]{r}{\pi} & 
    \frac {W^m} {W_{D'}^m} \arrow[two heads]{r}{\im M} & 
    W_G^{m'}
  \end{tikzcd}
\end{center}
with the understanding that $W_G^{m'}$ is an abuse of notation in the case when $D'$ is of type 22.

\begin{proposition}
  \label{prop_deg_L_G}
  The dimensions of $W_L^m$ and $W_G^{m'}$ are
  \begin{enumerate}[label=(\roman*)]
    \item $\dim W_L^m    = \nullity M$;
    \item $\dim W_G^{m'} = \rank M$.
  \end{enumerate}
  For a sufficiently large monomial $m$,
  \begin{enumerate}[label=(\roman*)]
    \setcounter{enumi}{2}
    \item $\deg L = \deg D  + \rank M$;
    \item $\deg G = \deg D' - \rank M$;
    \item $D + D' = L \iff \rank M = \deg D'$.
  \end{enumerate}
\end{proposition}
\begin{proof}
  By proposition \ref{prop_ker_im_M}, $W_L^m = \ker M$ and $W_G^{m'} = \im M$,
  and (i) and (ii) then follow by definition of rank and nullity.
  As for (iii), we have
  \begin{align*}
    \deg L
      &= \dim W^m - \dim W_L^m & \text{proposition \ref{prop_dim_W}} \\
      &= \dim W^m - \nullity M & \text{part (i)} \\
      &= \dim W^m - \dim W_D^m + \rank M & \text{rank-nullity theorem} \\
      &= \deg D + \rank M & \text{proposition \ref{prop_dim_W}}.
  \end{align*}
  For part (iv),
  \begin{align*}
    \deg G
      &= \deg G + \deg L - \deg L \\
      &= \deg D' + \deg D - \deg L & L + G = D + D' \\
      &= \deg D' - \rank M & \text{part (iii)}.
  \end{align*}
  Finally, for part (v), suppose $D + D' = L$. Then
  \begin{align*}
    D + D' &= L \\
    \deg D + \deg D' &= \deg L \\
    \deg D + \deg D' &= \deg D + \rank M & \text{part (iii)} \\
    \deg D' &= \rank M.
  \end{align*}
  Conversely, suppose $\deg D' = \rank M$.
  Then by part (iv), $\deg G = 0$.
  Since $G$ is an effective divisor, $G = 0$.
  Therefore $D + D' = L + G = L$.
\end{proof}

In the following subsections, we work out some extended examples of computing $L$ and $G$.

%%%%%%%%%%%%%%%%%%%%%%%%%%%%%%%%%%%%%%%%%%%%%%%%%%%%%%%%%%%%%%%%%%%%

\subsection{An example -- disjoint case}

Let $C : y^3 + x^4 + 1$ be a $C_{3,4}$ curve over $\bb F_{11}$.
Let $D$ and $D'$ be type 31 divisors with
\begin{align*}
  D  &= \pid{f, g, h}     & D' &= \pid{f', g', h'} \\
  f  &= x^2 + 3y + 7x + 5 & f' &= x^2 + 6y + 3x - 2 \\
  g  &= xy + 2y + 2x + 9  & g' &= xy + 5y + 5x + 9 \\
  h  &= y^2 + 4y + 2x + 3 & h' &= y^2 - y - x + 5.
\end{align*}
These divisors are disjoint, with
\begin{align*}
  D &= 2 \cdot (7 : 6 : 1) + (10 : 4 : 1) \\
  D' &= (5 : 1 : 1) + (2\alpha + 6 : 7\alpha : 1) + (9\alpha + 3 : 4\alpha + 6 : 1)
\end{align*}
where $\alpha \in \bb F_{11^2}$ is a root of $x^2 + 7x + 2$.

Now $L = \lcm(D, D') = D + D'$ is a degree 6 divisor, and by \note{(insert table number)},
we see that no generator of any reduced Gr\"obner basis of a degree 6 polynomial
has a monomial larger than $x^4$.
We proceed by computing the matrix $M$ in
\begin{center}
  \begin{tikzcd}
    W_L^{x^4} \arrow[hook]{r}{\ker M} & 
    W_D^{x^4} \arrow[hook]{r}{\iota} \arrow[bend left]{rr}{M} & 
    W^{x^4} \arrow[two heads]{r}{\pi} & 
    \frac {W^{x^4}} {W_{D'}^{x^4}} \\ &
    K^7 \arrow[hook]{r} \arrow[no head]{u}{\rotatebox{90}{$\simeq$}} &
    K^{10} \arrow[two heads]{r} \arrow[no head]{u}{\rotatebox{90}{$\simeq$}} &
    K^3 \arrow[no head]{u}{\rotatebox{90}{$\simeq$}}.
  \end{tikzcd}
\end{center}
The spaces $W_D^{x^4}$ and $W_{D'}^{x^4}$ are 7-dimensional, spanned by the vectors
\begin{align*}
  W_D^{x^4} &= \Span\{ f, g, h, xf, xg, xh, x^2f \} \\
  W_{D'}^{x^4} &= \Span\{ f', g', h', xf', xg', xh', x^2f' \}.
\end{align*}
We reduce the basis of $W_D^{x^4}$ modulo $W_{D'}^{x^4}$ to get the matrix
\[ M = \begin{pmatrix}
  7 & 0 & 9 & 2 & 10 & 5 & 2 \\
  4 & 8 & 3 & 10 & 2 & 8 & 6 \\
  8 & 8 & 5 & 2 & 0 & 1 & 7
\end{pmatrix}, \]
where, e.g., the reduction of $f$ is $\bar f = 8y + 4x + 7$,
the reduction of $g$ is $\bar g = 8y + 8x$, etc.
This matrix $M$ has the reduced row echelon form
\[ M_{\text{rref}} = \begin{pmatrix}
  1 & 0 & 6 & 0 & 6 & 9 & 2 \\
  0 & 1 & 7 & 0 & 9 & 8 & 10 \\
  0 & 0 & 0 & 1 & 6 & 4 & 5
\end{pmatrix} \]
and kernel
\[ \ker M =
\begin{pmatrix}
  -6 & -6 & -9 & -2 \\
  -7 & -9 & -8 & -10 \\
   1 &  0 &  0 &  0 \\
   0 & -6 & -4 & -5 \\
   0 &  1 &  0 &  0 \\
   0 &  0 &  1 &  0 \\
   0 &  0 &  0 &  1
\end{pmatrix} =
\begin{pmatrix}
  5 & 5 & 2 & 9 \\
  4 & 2 & 3 & 1 \\
  1 & 0 & 0 & 0 \\
  0 & 5 & 7 & 6 \\
  0 & 1 & 0 & 0 \\
  0 & 0 & 1 & 0 \\
  0 & 0 & 0 & 1
\end{pmatrix} \]
The kernel is 4-dimensional, spanned by the vectors
\begin{align*}
  \ker M
    &= \Span \left\{ \begin{array}{l}
      h + 4g + 5f, \\
      xg + 5xf + 2g + 5f, \\
      xh + 7xf + 3g + 2f, \\
      x^2f + 6xf + g + 9f \end{array} \right\} \\
    &= \Span \left\{ \begin{array}{l}
      y^2 + 4xy + 5x^2 + 5y + x - 2, \\
      x^2y + 5x^3 - 3xy - 2x^2 - 3y - 4x - 1, \\
      xy^2 - 4x^3 - 5xy - 2x^2 + y + 3x + 4, \\
      x^4 + 3x^2y + 2x^3 - 3xy + x^2 - 4y - 4x - 1
    \end{array} \right\}
\end{align*}
Label these four polynomials, from top to bottom, by $u$, $v$, $w$, and $z$
Then $u,v,w,z$ form a basis for $W_L^{x^4}$, but not a reduced Gr\"obner basis for $I_L$.
We can see that $\LM(u)$ divides $\LM(w)$, hence $w$ is unneeded.
Likewise 
\begin{align*}
  z &\equiv z - C \\
    &= z - (y^3 + x^4 + 1) \\
    &= -y^3 + 3x^2y + 2x^3 - 3xy + x^2 - 4y - 4x - 2
\end{align*}
and $\LM(u)$ divides $\LM(z - C)$, hence $z$ is also unneeded.
\note{Double check this reasoning.}

Therefore $L$ is the type 63 divisor whose ideal $I_L$ is generated by the reduced Gr\"obner basis
  \[ I_L = \pid{ y^2 + 4xy + 5x^2 + 5y + x - 2, x^2y + 5x^3 - 3xy - 2x^2 - 3y - 4x - 1}.\]

In the example above, we computed the reduced row echelon form and kernel of a $3 \times 7$ matrix,
however two columns -- those producing the polynomials $w$ and $z$ -- were unneeded.
In finding explicit formulas describing the divisor addition operation, we should be careful not to compute unneeded values.

The image of $M$ also gives us important information.
\begin{center}
\begin{tikzcd}
  W_L^m \arrow[hook]{r}{\ker M} & W_D^m \arrow[hook]{r}{\iota} \arrow[bend left]{rr}{M} & W^m \arrow[two heads]{r}{\pi} & \frac {W^m} {W_{D'}^m} \arrow[two heads]{r}{\im M} & W_G^{m'}
\end{tikzcd}
\end{center}
The image gives us the space of polynomials in $I_G$ upper bounded by a monomial $m'$.
This $m'$ is determined by $D'$.
  \[ m' = \max \{ \mu \in \cal M ~:~ \mu \not\in \LT(I_{D'}) \}. \]



%%%%%%%%%%%%%%%%%%%%%%%%%%%%%%%%%%%%%%%%%%%%%%%%%%%%%%%

\subsection{An example -- non-disjoint case}

Consider again the curve $C : y^3 + x^4 + 1$ over $\bb F_{11}$.
Let $D$ and $D'$ be type 31 divisors with
\begin{align*}
  D  &= \pid{f, g, h}      & D' &= \pid{f', g', h'} \\
  f  &= x^2 + 10           & f' &= x^2 + 3y + 2x + 2 \\
  g  &= xy + y + 9x + 9    & g' &= xy + 4x + 5 \\
  h  &= y^2 + 10y + 4x + 5 & h' &= y^2 + y + 2x + 3 .\\
\end{align*}
These divisors are non-disjoint. We have
\begin{align*}
  D &= (1 : 2 : 1) + (10 : 2\alpha + 4 : 1) + (10 : 10\alpha + 8 : 1) \\
  D' &= (1 : 2 : 1) + (3 : 9 : 1) + (5 : 6 : 1)
\end{align*}
where $\alpha \in \bb F_{11^2}$ is a root of $x^2 + 7x + 2$.

\begin{center}
  \begin{tikzcd}
    W_L^{x^4} \arrow[hook]{r}{\ker M} & 
    W_D^{x^4} \arrow[hook]{r}{\iota} \arrow[bend left]{rr}{M} & 
    W^{x^4} \arrow[two heads]{r}{\pi} & 
    \frac {W^{x^4}} {W_{D'}^{x^4}} \arrow[two heads]{r}{\im M} &
    W_G^y \\ &
    K^7 \arrow[hook]{r} \arrow[no head]{u}{\rotatebox{90}{$\simeq$}} &
    K^{10} \arrow[two heads]{r} \arrow[no head]{u}{\rotatebox{90}{$\simeq$}} &
    K^3 \arrow[no head]{u}{\rotatebox{90}{$\simeq$}}.
  \end{tikzcd}
\end{center}

As in the previous example, we reduce the basis of $W_{D}^{x^4}$ modulo the basis for $W_{D'}^{x^4}$ to get the matrix
\[ M = \begin{pmatrix}
  8 & 4 & 2 & 8 & 7 & 6 & 10 \\
  9 & 5 & 2 & 2 & 1 & 6 & 2 \\
  8 & 1 & 9 & 6 & 7 & 5 & 5
\end{pmatrix}. \]
This matrix has the reduced row echelon form
\[ M = \begin{pmatrix}
  1 & 0 & 6 & 8 & 5 & 7 & 5 \\
  0 & 1 & 5 & 8 & 0 & 4 & 9 \\
  0 & 0 & 0 & 0 & 0 & 0 & 0
\end{pmatrix}. \]
We can see the $M$ does not have full rank.
Hence $L$ will not be of degree 6 and $G$ will be non-zero.
The kernel of $M$ is
\begin{align*}
  \ker M
    &= \Span \left\{ \begin{array}{l}
      h - 5g - 6f, \\
      xf - 8g - 8f, \\
      xg - 5f, \\
      xh - 4g - 7f, \\
      x^2f - 9g - 5f \end{array} \right\} \\
    &= \Span \left\{ \begin{array}{l}
      y^2 + 6xy + 5x^2 + 5y + 3x + 10, \\
      x^3 + 3xy + 3x^2 + 3y + 4x + 2, \\
      x^2y + xy + 4x^2 + 9x + 5, \\
      xy^2 + 6xy + 8x^2 + 7y + 2x + 4, \\
      x^4 + 2xy + 5x^2 + 2y + 7x + 1
    \end{array} \right\}
\end{align*}
and $L$ is a type 51 divisor whose ideal $I_L$ has the reduced Gr\"obner basis
\[ I_L = \left\langle\begin{array}{l}
  y^2 + 6xy + 5x^2 + 5y + 3x + 10, \\ 
  x^3 + 3xy + 3x^2 + 3y + 4x + 2, \\ 
  x^2y + xy + 4x^2 + 9x + 5\end{array}\right\rangle. \]
The pivot columns of $M$ determine its image.
The first and second columns are pivots, thus the image of $M$ is given by the column basis
\[ \im M = \begin{pmatrix}
  8 & 4 \\
  9 & 5 \\
  8 & 1 
\end{pmatrix}. \]
That is, $\im M$ is spanned by the vectors $8y + 9x + 8$ and $y + 5x + 8$.
This column basis is reducible to 
\[ \im M = \begin{pmatrix}
  10 & 9 \\
   1 & 0 \\
   0 & 1 
\end{pmatrix}, \]
so that $I_G = \pid{x + 10, y + 9} = \pid{x - 1, y - 2}$,
which agrees with $D$ and $D'$ sharing the point $(1 : 2 : 1)$ in common.

Now we have $L$ and $G$ of degrees 5 and 1, respectively, such that $D + D' = L + G$.
To finish, we must compute the reduction $\bar{\bar L}$ of $L$ and compute $D + D' = \bar{\bar L} + G$.
The reduction $\bar{\bar L}$ will be of type 31, thus the problem of adding $D$ and $D'$
is reduced to adding a degree 3 divisor $\bar{\bar L}$ with a degree 1 divisor $G$.



%%%%%%%%%%%%%%%%%%%%%%%%%%%%%%%%%%%%%%%%%%%%%%%%%%%%%%%%%%%%%%%

\subsection{Avoiding Infinite Recursion}

When $D$ and $D'$ are disjoint, we compute $D + D'$ by computing $L = \lcm(D, D')$ via the kernel of a quotient map.
This we can guarantee will terminate.
When $D$ and $D'$ are non-disjoint, we compute $L = \lcm(D, D')$ and $G = \gcd(D, D')$, then compute
$D + D' \equiv \bar{\bar L} + G$.
This alone, however, is not guaranteed to terminate.
For example, suppose $D = P + 2Q$ is of degree 3 and $D' = P + Q$ is of degree 2.
Then $L = P + 2Q = D$ is already reduced and $G = P + Q = D'$ and we compute
  \[ D + D' = L + G \equiv \bar{\bar L} + G = L + G = D + D', \]
getting us nowhere.

We must identify which cases may lead to infinite recursion.
We should only compute $L + G$ when we are certain that $G$ is of lesser degree than $D$ and $D'$.
The degree of $G$ is
  \[ \deg G = \deg D' - \rank M. \]
If we assume that $\deg D \geq \deg D'$, then $G$ is of lesser degree than $D$ and $D'$ if and only if $\rank M > 0$.
We need only be concerned about infinite recursion in cases where $\rank M = 0$,
which happens if and only if $D > D'$.

\begin{proposition}
  The following are equivalent:
  \begin{enumerate}[label=(\roman*)]
    \item $L = D$;
    \item $G = D'$;
    \item $\rank M = 0$.
  \end{enumerate}
\end{proposition}
\begin{proof}
  \begin{description}
    \item[(i) $\implies$ (ii):]
      Suppose $L = D$. Then
        \[ G = G + L - L = D' + D - L = D' + L - L = D'. \]
    \item[(ii) $\implies$ (iii):]
      Suppose $G = D'$.
      Then by proposition \ref{prop_deg_L_G},
        \[ \rank M = \deg D' - \deg G = \deg(D' - G) = \deg 0 = 0. \]
    \item[(iii) $\implies$ (i):]
      Suppose $\rank M = 0$.
      By proposition \ref{prop_deg_L_G}, $\deg L = \deg D + \rank M = \deg D$.
      Since $D$ and $L$ are effective divisors of equal degree and $L \leq D$, we must have $L = D$.
  \end{description}
\end{proof}
\begin{corollary}
  \[ G < D' \iff \rank M > 0. \]
\end{corollary}

When $\rank M = \deg D'$, we can compute $D + D' = L$.
When $0 < \rank M < \deg D'$, we can compute $D + D' = \bar{\bar L} + G$, reducing to a ``smaller'' case with $G < D'$.
When $\rank M = 0$, this strategy leads to infinite recursion, and we must do something else to ensure this terminates.



%%%%%%%%%%%%%%%%%%%%%%%%%%%%%%%%%%%%%%%%%%%%%%%%%%%%%%

\subsection{When $\rank M = 0$}

First we identify under which circumstances the rank of $M$ may be zero.
Then we determine what to do under these circumstances.

If the rank of $M$ is zero, then $D' \leq D$ \note{(should this be a proposition?)}.
If $D$ and $D'$ are of the same degree, this cannot happen since $D$ and $D'$ are assumed to be distinct.
So $\rank M$ can only be zero if $\deg D' < \deg D$.
We handle the situation differently depending on whether $D'$ is of degree 1 or 2.

If $D'$ is of degree 2, then $D = P + D'$ for some point $P$, which we must find.
Once we have, we compute
  \[ D + D' = \bar{\bar{2D'}} + P, \]
thereby reducing to the case of adding a reduced divisor to a degree 1 divisor.
We must now prove that \emph{that} case terminates.

If $D'$ is of degree 1, then $D' = P$ for some point $P$ and $D = D'' + nP$ for some integer $1 \leq n \leq 3$ and some divisor $D''$ disjoint from $D'$.
We must find $D''$ and $n$ and compute
  \[ D + D' = D'' + ((n + 1)P). \]
For each possible value of $n$, we argue that this will terminate.
\begin{description}
  \item[$n = 1$:]
    The degree of $D$ is at most 3, so $\deg D'' = \deg D - n \leq 2$ and $D''$ is reduced.
    Clearly, $\deg((n + 1)P) = 2$, so it, too, is reduced.
    Since $D''$ and $2P$ are disjoint and reduced, computing $D'' + 2P$ will terminate.
  \item[$n = 2$:]
    If $\deg D = 2$, then $D'' = 0$ and $D + D' = 3P$.
    \note{We see in a later section how to triple a point.}
    
    If $\deg D = 3$, then $\deg D'' = 1$ and $D + D' = D'' + 3P$.
    If $3P$ is of type 31, then $D''$ and $3P$ are disjoint and reduced, so their addition will terminate.
    If $3P$ is of type 33, then $D + D' = D'' + 3P \equiv D''$.
    Otherwise, if $3P$ is of type 32, then we compute $D + D' = D'' + \bar{\bar{3P}}$,
    which is the sum of a type 11 with a type 22 divisor.
    Thus we have yet again reduced to a ``smaller'' case.
    In the worst case, $D''$ and $\bar{\bar{3P}}$ are not disjoint,
    but this lands us in one of the above cases, which we have shown terminate.
  \item[$n = 3$:]
    The divisor $D''$ is zero and $(n + 1)P = 4P$.
    The divisor $2P$ is reduced and we compute $D + D' = 4P$ by doubling $2P$.
    \note{In the chapter on doubling, we argue that this will terminate.}
\end{description}

Having now shown that we can compute $D + D'$ in a way that will terminate,
we must now show how to find the relevant values $D''$, $P$, and $n$ where necessary.

Suppose $\deg D' = 2$, and we wish to find $P$ such that $D = P + D'$.
Since $\deg D > \deg D'$, we must have $\type D = 31$.
It may be that $\type D' = 21$ or $\type D' = 22$, so we consider each case separately.

Suppose first that $D'$ is of type 21. 
Let $P$ be represented by the type 11 divisor $\pid{s, t} = \pid{x + s_0, y + t_0}$.
Let $D$ and $D'$ be
\begin{align*}
  D &= \pid{f, g, h} & D' &= \pid{f', g'} \\
  f &= x^2 + f_2y + f_1x + f_0 & f' &= y + f'_1x + f'_0 \\
  g &= xy + g_2y + g_1x + g_0 & g' &= x^2 + g'_1x + g'_0 \\
  h &= y^2 + h_2y + h_1x + h_0.
\end{align*}

We need to solve for $s_0$ and $t_0$ such that
\[ \pid{f, g, h} = \pid{s, t}\pid{f', g', h'}. \]
A solution exists and must satisfy $sf' - f'_1f - g \in \pid{f,g,h}$.
We have that
  \[ sf' - f'_1f - g = (s_0 - f_2f'_1 - g_2)y + (f'_0 + f'_1s_0 - f_1f'_1 - g_1)x + (f'_0s_0 - f_0f'_1 - g_0). \]
It must be that each of these coefficients are zero, otherwise $f, g, h$ would not form a reduced Gr\"obner basis.
Therefore
  \[ s_0 = f_2f'_1 + g_2. \]
Likewise, we compute
  \[ tf' - f'_1g - h = (f'_0 + t_0 - f'_1g_2 - h_0)y + (f'_1t_0 - f'_1g_1 - h_1)x + (f'_0t_0 - f'_1g_0 - h_0) \]
and conclude
  \[ t_0 = - f'_0 + f'_1g_2 + h_0. \]

When $D'$ is of type 22, the same strategy yields $s_0$ and $t_0$.
We reduce $sf' = (x + s_0)(x + f'_0)$ and $tf' = (y + t_0)(x + f'_0)$ modulo $I_D$,
argue these reductions are zero, and conclude
\begin{align*}
  s_0 &= - f'_0 - f_1 \\
  t_0 &= g_1.
\end{align*}

\begin{proposition}
  Let $D$ be a typical type 31 divisor and $D'$ a type 11 divisor.
  Suppose $D' < D$.
  Then there is a type 21 divisor $D''$ such that $D = D' + D''$,
  given by $D'' = \pid{y + s_1x + s_0, x^2 + t_1x + t_0}$ where
    \begin{align*}
      s_1 &= \frac{-(g_2 - f'_0)} {f_2} \\
      s_0 &= g_1 + s_1(f_1 - f'_0) \\
      t_1 &= g_2 + f_1 - f'_0 \\
      t_0 &= f_1g_2 - f_2g_1 - f'_0t_1 + f_0.
    \end{align*}
\end{proposition}
\begin{proof}
  There certainly exists a degree 2 divisor $D''$ such that $D = D' + D''$.
  Either $D''$ is of type 21 or 22.
  The sum of a type 11 divisor with a type 22 divisor is never a typical divisor \note{(justify this?)}
  so $D''$ must be of type 21, given by polynomials $s = y + s_1x + s_0$ and $t = x^2 + t_1x + t_0$.
  Solving $\pid{f, g, h} = \pid{s,t}\pid{f',g'}$ gives $s_0, s_1, t_0, t_1$ as given in the theorem statement.
\end{proof}

\begin{proposition}
  Let $D$ be a semi-typical type 31 divisor.
  The ideal of $I_D$ of $D$ is generated by $\pid{f,g,h}$ with $f = x^2 + f_1x + f_0$.
  Let $D'$ be a type 11 divisor.
  Suppose $D' < D$. Then
  \begin{enumerate}[label=(\roman*)]
    \item $f$ has two (not necessarily distinct) rational roots.
  \end{enumerate}
  There is a degree 2 divisor $D''$ such that $D = D' + D''$ and
  \begin{enumerate}[label=(\roman*)]
    \setcounter{enumi}{1}
    \item if $f$ has distinct roots then
      \[ \type(D'') = \begin{cases} 21 & f'_0 = g_2 \\ 22 & f'_0 \neq g_2 \end{cases}\; \]
    \item if $f$ has a double root and $g'_0 = g_1$ then $\type(D'') = 21$
      \[ \type(D'') = \begin{cases} 22 & g'_0 = g_1 \\ 21 & g'_0 \neq g_1 \end{cases}\; \]
    \item if $\type(D'') = 21$, then $D'' = \pid{y + s_1x + s_0, x^2 + t_1x + t_0}$ where
    \begin{align*}
      s_1 &= \frac{g_0 - f'_0g_1} {f'_0(f_1 - f'_0) - f_0} \\
      s_0 &= g_1 + s_1(f_1 - f'_0) \\
      t_1 &= f_1 \\
      t_0 &= f_0 \;
    \end{align*}
    \item if $\type(D'') = 22$, then $D'' = \pid{x + s_0, y^2 + t_2y + t_0}$ where
    \begin{align*}
      s_0 &= f_1 - f'_0 \\
      t_2 &= h_2 \\
      t_0 &= h_0 - h_1s_0.
    \end{align*}
  \end{enumerate}
\end{proposition}
\note{These formulas still need testing.}
\note{I'm not convinced $s_1$ is correct if $f$ has a double root.}
\begin{proof}
  \begin{enumerate}[label=(\roman*)]
    \item
      Since $D' < D$, we have $I_D \subset I_{D'}$.
      In particular $f \in I_{D'}$, so $f = af'+ bg'$ for some polynomials $a$ and $b$.
      However, there is no $y$-term in $f$, so $b = 0$.
      One may then conclude that the rational root of $f'$ is also a root of $f$.
      Having one rational root, the second root of $f$ must also be rational.
    
    \item
      The divisor $D$ is the sum of a type 11 divisor $P$ and a type 22 divisor $Q + R$.
      $D''$ is of type 22 if and only if $D' = P$.
      \note{And this is determined by $D' = P \iff f'_0 \neq g_2$.}
    
    \item
      The divisor $D$ is the sum three points sharing the same $x$-coordinate.
      They cannot all be distinct, or else $D$ would be type 33 (principal).
      So at least two points in $D$ are equal, say $D = 2P + Q$.
      There are three cases:
      $P \neq Q$ and neither point is a special point;
      $P \neq Q$ and $Q$ is a special point;
      and $P = Q$ and $P$ is a special point.
      
      The cases we have ruled out:
      If $P \neq Q$ and $P$ is a special point, then $2P + Q$ is type 33.
      If $P = Q$ and $P$ is a regular point or hyperflex, then $\ord_P(f) < 3$.
      If $P = Q$ and $P$ is an inflection point, then $D = 2P + Q = 3P$ is type 33.
      
      In all three cases, $g$ defines a ``cross'' centered on $P$
      One can determine whether $D' = P$ by $D' = P \iff g'_0 = g_1$.
      Also, $P + Q$ is type $22$ in all three cases.
      
      So we have
      \begin{align*}
        g'_0 = g_1
          &\implies D' = P \\
          &\implies D = 2P + Q = P + D'' = D' + D'' \\
          &\implies D'' = P + Q \\
          &\implies \type(D'') = 22.
      \end{align*}
      Otherwise, $\type(D'') = 21$.
    
    \item
      Solving $sf' - s_1f - g = 0$ gives the stated formulas for $s_1$ and $s_0$.
      We note that $D'' < D$ implies $\pid{f,g,h} \subset \pid{s,t}$,
      $f \in \pid{s,t}$, and finally $t = f$.
    
    \item
      Solving $sf' - f = 0$ gives $s_0$.
      We note that $D'' < D$ implies $\pid{f,g,h} \subset \pid{s,t}$,
      $h \in \pid{s,t}$, and $h = t + h_1s$.
      Solving this simple equation gives $t_0$ and $t_2$.
  \end{enumerate}
\end{proof}



%%%%%%%%%%%%%%%%%%%%%%%%%%%%%%%%%%%%%%

\subsection{The Addition Algorithm}

\begin{center}
  \begin{algorithm}
    \caption{Disjoint Divisor Addition}
    {\bf Input:} Two divisors $D$ and $D'$, represented by their ideals $I_D$ and $I_{D'}$
    {\bf Output:} A divisor $D''$ equivalent to $D + D'$.
    \begin{algorithmic}[1]
      \State Compute $M : W_D^m \to W_{D'}^m$
      \State Compute $M_{\text{rref}}$ and $\rank M$
      \State Compute $L = \lcm(D, D')$
      \If {$\rank M = \deg D'$}
        \State \Return $L$
      \EndIf
      \If {$\rank M > 0$}
        \State Compute $G = \gcd(D, D')$
        \State \Return $\bar{\bar L} + G$
      \EndIf
      \If {$\rank M = 0$}
        \If {$\deg D' = 2$}
          \State Compute $A$ such that $D = A + D'$
        \EndIf
      \EndIf
    \end{algorithmic}
  \end{algorithm}
\end{center}
\note{Finish this}

%%%%%%%%%%%%%%%%%%%%%%%%%%%%%%%%%%%%%%%%%%
%%%%%                                %%%%%
%%%%%   Implementation and Testing   %%%%%
%%%%%                                %%%%%
%%%%%%%%%%%%%%%%%%%%%%%%%%%%%%%%%%%%%%%%%%

\section{Implementation and Testing}
\label{chap_implementation}

An implementation in Sage of $C_{3,4}$ curve divisor class group arithmetic
is available at \cite{github}.
This implementation uses the explicit formulae derived at the end of Chapter \ref{chap_typical_case},
as well as explicit formulae for the many, many atypical cases,
derived following the methods described in
Chapters \ref{chap_addition}, \ref{chap_doubling}, and \ref{chap_reduction}.

The implementation defines {\tt C34Curve} and {\tt C34CurveDivisor} classes from which to instantiate objects.
A {\tt C34Curve} may be constructed by specifying a base field and the curve coefficients.
\begin{verbatim}
  sage: C = C34Curve(GF(41), [1,2,3,4,5,6,7,8,9]); C
  C34 curve defined by y^3 + x^4 + 9*x*y^2 + 8*x^2*y + 7*x^3 + 6*y^2
  + 5*x*y + 4*x^2 + 3*y + 2*x + 1 over Finite Field of size 41
\end{verbatim}
An exception is thrown if the resulting curve would be singular.
Alternatively, a {\tt C34Curve} over a specified base field may be generated randomly.
\begin{verbatim}
  sage: C = C34Curve.random_curve(GF(97)); C
  C34 curve defined by y^3 + x^4 - 41*x^2*y - 48*x*y + 43*x^2 + 48*y
  + 7*x over Finite Field of size 97
\end{verbatim}

A {\tt C34CurveDivisor} may be instantiated in several ways.
A {\tt C34Curve} can create them randomly:
\begin{verbatim}
  sage: D1 = C.random_divisor(); D1
  <x^2 + 46*y - 24*x + 40, x*y + 46*y - 46*x + 27, y^2 + 8*y + 23*x - 20>
  sage: D2 = C.random_divisor_of_type(31, typical = False); D2
  <x^2 + 34*x - 4, x*y + 31*y + 30*x - 40, y^2 + 8*y - 35*x + 11>
  sage: D3 = C.random_divisor_of_type(32); D3
  <y + 4*x + 46, x^3 - 21*x^2 + 29*x - 9>
\end{verbatim}
Details on random divisor generation are given in Section \ref{sec_random_divisors}.
A divisor may also be constructed by specifying its points,
\begin{verbatim}
  sage: P1 = C.point(96, 76)
  sage: P2 = C.random_rational_point()
  sage: D = C.divisor([P1, P2]); D
  <y - 47*x - 26, x^2 + 31*x + 30>
  sage: D.formal_sum()
  [((67 : 71 : 1), 1), ((96 : 76 : 1), 1)]
\end{verbatim}
or by specifying the polynomials forming a Gr\"obner basis.
\begin{verbatim}
  sage: R = C.polynomial_ring()
  sage: x, y = R.gens()
  sage: f = y - 41*x + 46
  sage: g = x^2 - 17*x - 37
  sage: D = C.divisor([f,g]); D
  <y - 41*x + 46, x^2 - 17*x - 37>
  sage: D.formal_sum()
  [((5 : 62 : 1), 1), ((12 : 58 : 1), 1)]
  sage: D.type
  21
\end{verbatim}

Operators {\tt +} and {\tt -} are overloaded to act on divisors,
while {\tt *} is overloaded to allow scalar multiplication of divisors by integers.
\begin{verbatim}
  sage: D1 = C.random_divisor(); D2 = C.random_divisor(); D1; D2
  <x^2 + 24*y - 22*x + 37, x*y - 25*y - 25*x + 37, y^2 + 12*y - 24*x - 27>
  <x^2 - 36*y - 12*x + 31, x*y - 22*y + 16*x + 1, y^2 - 26*y - 36*x + 35>
  sage: D1 + D2
  <x^2 + 4*y + 38*x - 41, x*y - 36*y + 21*x + 23, y^2 + 22*y - 25*x + 14>
  sage: D1 - D1
  <1>
  sage: D3 = 999*D2; D3; D3.formal_sum()
  <x^2 - 41*y + x - 24, x*y - 18*y + 22*x + 33, y^2 + 45*y + x + 40>
  [((30 : 15 : 1), 1),
   ((22*z2 + 31 : 74*z2 + 19 : 1), 1),
   ((75*z2 + 53 : 23*z2 + 93 : 1), 1)]
\end{verbatim}
Here, {\tt z2} is a primitive element of the extension $\bb F_{97^2}/\bb F_{97}$.

The arithmetic is fully implemented even for atypical divisors.
\begin{verbatim}
  sage: D1 = C.random_divisor_of_type(11); D1
  <x + 47, y + 36>
  sage: D2 = C.random_divisor_of_type(22); D2
  <x - 5, y^2 - 35*y + 8>
  sage: D1 + D2
  <x^2 + 42*x - 41, x*y - 5*y + 36*x + 14, y^2 - 35*y + 12*x + 45>
\end{verbatim}
More complete documentation is available at \cite{github}.



%%%%%%%%%%%%%%%%%%%%%%%
%%%%%             %%%%%
%%%%%   Testing   %%%%%
%%%%%             %%%%%
%%%%%%%%%%%%%%%%%%%%%%%

\subsection{Testing}

The formulae describing the typical cases in Chapter \ref{chap_typical_case}
and the remaining formulae for the atypical cases found at \cite{github}
were tested via unit testing and random testing.
Producing test cases requires generating random divisors on curves.
We describe below the testing methodology,
and in Section \ref{sec_random_divisors} we describe how random divisors were generated.



\subsubsection{Unit Testing}

The code for adding distinct divisors is divided up into several subroutines.
For every pair of non-zero reduced divisor types (11, 21, 22, and 31),
there is one subroutine,
e.g. {\tt add\_11\_11}, {\tt add\_21\_11}, {\tt add\_21\_21}, etc.
For every non-zero reduced divisor type, there is a doubling subroutine,
e.g. {\tt double\_11}, {\tt double\_21}, etc.
There are also flip and reduction subroutines for every divisor type, reduced or not.

Addition and doubling subroutines consist of several conditional branches.
In the case of addition subroutines,
for each possible type of the divisors' least common mulitple,
there is a conditional branch.
These branches also correspond to the different possiblities for
the form (i.e. positions of the pivot elements)
of the reduced row echelon form of $M_{\text{add}}$,
the matrix whose construction was described in Chapter \ref{chap_addition}.
For example, when adding a type 21 divisor $D_1$ to a type 11 divisor $D_2$,
their least common multiple $L = \lcm(D_1, D_2)$ may be of type 31, 32, or 21.
Consequently, there are 3 branches in {\tt add\_21\_11} corresponding to each of these possibilities.
The case where $L$ is of type 21 (this happens when $D_2 < D_1$)
further breaks down into the case where $D_1 = 2D_2$,
in which case a tripling subroutine must be called,
and the case where $D_1 \neq 2D_2$, in which case doubling is necessary.

Doubling subroutines also consist of several different conditional branches.
As was the case with addition,
these branches correspond to the different possible positions
of the pivot elements in $\rref(M_{\text{doub}})$.
Unit tests were written to test each branch.

Unit tests were written to test every branch of each addition and doubling subroutine.
For each subroutine, the most commonly executed branch is tested several times
using curves over each of the finite fields $\bb F_{2}$, $\bb F_{2^4}$,
$\bb F_{3}$, $\bb F_{3^3}$, $\bb F_{31}$, $\bb F_{31^2}$, and $\bb F_{1009}$.
The other branches are tested once, usually over $\bb F_{997}$ or $\bb F_{1009}$.
The reason for only testing once is due to the difficulty in constructing instances of the atypical cases.
These rare cases get tested further during random testing, discussed in the next subsection.

These particular finite fields were chosen to test the code's correctness
over fields of small order, where degenerate cases are more likely to be encountered;
in characteristics 2 and 3, which were ignored by previous authors;
and over finite fields of non-prime order.
Over a very small field such as $\bb F_2$, it is possible that unit tests might accidentally pass
even if the formulae being tested are incorrect.
The fields $\bb F_{31^2}$, $\bb F_{997}$, and $\bb F_{1009}$ were chosen
as they are large enough that this is unlikely to happen.

Unit tests were also written for each reduction and flipping subroutine.
These subroutines are tested on curves over each of the fields
$\bb F_{2}$, $\bb F_{2^4}$, $\bb F_{3}$, $\bb F_{3^3}$, $\bb F_{31}$, $\bb F_{31^2}$, and $\bb F_{1009}$.
Flipping subroutines for semitypical divisors (types 31, 41, 51, and 61) have three branches,
one for each of the cases $\pid{f,g} = \pid{f,g,h}$, $\pid{f,g} \neq \pid{f,g,h} = \pid{f,h}$,
and $\pid{f,g} \neq \pid{f,g,h} \neq \pid{f,h}$.
Each of these cases are tested over each of the above fields.



\subsubsection{Random Testing}

Unit testing was useful in detecting errors while putting the explicit formulae into code
or when making modifications to that code.
However, unit testing alone is not sufficient to test the correctness of the formulae.
It is not enough to test subroutines over a few select fields and curves.
Random testing was also done to ensure correctness.
That is, many hundreds of thousands of random divisors were added,
on random curves over many base fields of small order.
The results of the additions were compared against another library.

More specifically, for every prime power $2 \leq q \leq 31$,
100 random $C_{3,4}$ curves were generated.
For each such curve $C$, 100 pairs of random reduced divisors $(D_1, D_2)$ were generated.
These divisors may be of any type representing a reduced divisor (types 0, 11, 21, 22, and 31).
See Section \ref{sec_random_divisors} for details on how these were generated.
For each pair, we add $D_3 := D_1 + D_2$.
To test whether the addition was correctly computed,
we construct the ideals $I_{D_1}$ and $I_{D_2}$ using Sage's built-in ring ideal classes.
We multiply $J := I_{D_1} I_{D_2}$.
The test passes if the reduced Gr\"obner basis of $J$
matches the reduced Gr\"obner basis we computed for $D_3$.

Doubling is tested analogously.
It proceeds as above,
except for each trial we need only generate a single random reduced divisor, $D_1$,
rather than a pair.
Then we compute $D_2 := 2D_1$, $J := I_{D_1}^2$,
and compare the reduced Gr\"obner bases of $J$ and $D_2$.



%%%%%%%%%%%%%%%%%%%%%%%%%%%%%%%%%%%%%%%%%
%%%%%                               %%%%%
%%%%%   Random divisor generation   %%%%%
%%%%%                               %%%%%
%%%%%%%%%%%%%%%%%%%%%%%%%%%%%%%%%%%%%%%%%

\subsection{Random divisor generation}
\label{sec_random_divisors}

In order to generate test cases, it is necessary to be able to generate random divisors on curves.
Once one is able to generate reduced divisors,
it is straightforward to extend the method to generate unreduced divisors of any desired type as well.
Recall that if $D$ is a reduced divisor,
then the minimal polynomial in $I_D$ has leading monomial $x^2$ or smaller
(see Theorem \ref{thm_reduced_divisor_types} and Table \ref{tab_divisor_types}).
The general idea is to randomly choose a polynomial $f$ with $\LM(f) \leq x^2$,
then construct a divisor $D$ for which the minimal polynomial of $I_D$ is $f$.

The details are as follows.
Choose random coefficients $f_0, f_1, f_2, f_3 \in K$ and set $f := f_3x^2 + f_2y + f_1x + f_0$.
If $f$ is constant, then return the zero divisor.
Otherwise, the rest proceeds differently
depending on whether the leading monomial of $f$ is $x^2$, $y$, or $x$.
We will describe the most difficult case, when $\LM(f) = x^2$.
In this case, we are about to construct a type 31 divisor, $D$.
the other cases are handled similarly, but are greatly simplified by not needing to consider whether the divisor will be typical or atypical nor needing to generate a third polynomial.

Suppose $f_2 \neq 0$.
Then $D$ will be a typical type 31 divisor.
We will find polynomials $g, h \in K[C]$ such that $I_D = \pid{f,g,h}$ is the ideal of a type 31 divisor.
Since $\pid{f,g,h} = \pid{kf,g,h}$ for any scalar $k \in K$, we simply set $f_2 := 1$,
so that $f := f_3x^2 + y + f_1x + f_0$.
Set $\bar F(x) = F(x, -f_3x^2 - f_1x - f_0) \in K[x]$.
Geometrically, $f$ is a parabola that intersects $C$ at 6 finite points.
The $x$-coordinates of these points are the roots of $\bar F$.

We now construct a degree 3 polynomial $g \in K[x]$ that divides $\bar F$.
This is done by factoring $\bar F$ and choosing a random subset of its divisors whose product is of degree 3.
If no such $g$ exists (perhaps $\bar F$ is irreducible or has only quadratic factors), then we must restart and generate a new polynomial $f$.

Finally, we make $f$ monic by setting $f := - \frac 1 {f_3} f$,
reduce $g$ modulo $f$ by setting $g := g \pmod f$,
then compute $h$ via
\[ h = \frac{(y + g_1)f - (x + f_1 - g_2)g}{f_2} \]
and return $I_D = \pid{f,g,h}$.

Suppose instead $f_2 = 0$.
Set $f_3 := 1$, so that $f = x^2 + f_1x + f_0$.
Then $D$ will be an atypical type 31 divisor.
Every atypical type 31 divisor arises as the sum of a type 11 with a type 22 divisor.
Thus, we must check if $f$ factors into linear terms $f = (x + x_1)(x + x_2)$.
If not, restart and generate a new polynomial $f$.
Otherwise, generate a type 11 divisor with minimal polynomial $f = x + x_1$ and a type 22 divisor with minimal polynomial $f = x + x_2$.

If $\LM(f) = y$, then $D$ will be of type 21.
If $\LM(f) = x$, then $D$ will be of type 11 or 22.
Since type 11 and type 22 divisors are in bijection with one another,
simply generate a type 11 divisor, then randomly decide whether to flip it afterwards.
Generating type 11 and type 21 divisors is similar to generating type 31 divisors:
Reduce the curve equation $F$ modulo $f$ to get a univariate polynomial $\bar F$,
randomly generate a degree $\deg D$ factor $g$ of $\bar F$,
and return $I_D = \pid{f, g \pmod f}$.

Following this method allows one to generate random reduced divisors
in such a way that no reduced divisor is left out.
Every reduced divisor will occur with probability greater than 0.
However, reduced divisors will not occur with uniform probability.
If there are many divisors with the same minimal polynomial,
each one will occur with lesser frequency than a divisor that shares a minimal polynomial only with its flip.


Forcing random divisors to be of a particular type:

The method above allows one to randomly generate a reduced divisor.
One can generate a random reduced divisor of a chosen type (0, 11, 21, 22, or 31)
by forcing certain coefficients of the random polynomial $f$ to be 0 or 1.

It is then possible to generate a random unreduced divisor of any desired type
by first generating a \emph{reduced} divisor $A$, then finding a divisor $D$ of the desired type
whose flip is $\bar D = A$.

To generate a random divisor of type $T$:
\begin{itemize}
  \item
    Let $T_A$ be the type of the flip of a divisor of type $T$ (see Table \ref{tab_large_flip_type}).
  \item
    Generate a reduced divisor $A$ of type $T_A$.
  \item
    Let $m$ be the leading monomial of the minimum polynomial in the ideal of a divisor of type $T_D$
    (see Table \ref{tab_divisor_types}).
  \item
    Choose a random polynomial $u$ in $I_A$ with $\LM(u) = m$.
  \item
    Compute $I_D = u : I_A$.
    In a Sage implementation, this can be done using Sage's built-in ideal arithmetic, for example.
  \item
    Return $D$.
\end{itemize}



%%%%%%%%%%%%%%%%%%%%%%%%%%%%%%%%%%%%%%%%%%%%%%
%%%%%                                    %%%%%
%%%%%   Comparison to State-of-the-art   %%%%%
%%%%%                                    %%%%%
%%%%%%%%%%%%%%%%%%%%%%%%%%%%%%%%%%%%%%%%%%%%%%

\subsection{Comparison to State-of-the-art}

Putting together the operation counts from the previous sections,
we sum up the costs of adding and doubling typical divisors.
\begin{theorem}
  \label{thm_op_count}
  Let $C$ be a $C_{3,4}$ curve over a field $K$.
  Suppose that $\Char K = 0$ or $\Char K > 3$ and that $C$ is given by a polynomial in short form
  (Equation \ref{eq_c34_short}).
  Let $D$ and $D'$ be typical type 31 divisors on $C$,
  given by reduced Gr\"obner bases of their ideals $I_D$ and $I_{D'}$.
  \begin{enumerate}[label=(\roman*)]
    \item
    If $D + D'$ is a typical type 61 divisor,
    then a reduced Gr\"obner basis for the ideal of $\bar{\bar{D + D'}}$ can be computed in 1I+111M+3S+99A.
    \item
    If $2D$ is a typical type 61 divisor, 
    then a reduced Gr\"obner basis for the ideal of $\bar{\bar {2D}}$ can be computed in 1I+135M+3S+116A.
  \end{enumerate}
\end{theorem}
\begin{proof}
  Lemmas \ref{lem_Madd_op_count}, \ref{lem_Mdoub_op_count}, \ref{lem_UV_op_count}, and \ref{lem_fgh_op_count}.
\end{proof}

Now compare this to the current state-of-the-art given by
Abu Salem and Khuri-Makdisi.\footnote{
As \cite{kmakdisi18} and \cite{salem07} provided only counts for inversions and multiplications,
their results being 2I+98M/2I+110M for addition/doubling. The counts in this table are my own.}
\begin{center}
\begin{tabular}{|l|rrrrr|rrrrr|}
  \hline
  & \multicolumn{5}{c|}{Addition} & \multicolumn{5}{c|}{Doubling} \\
  Author                  & I &  M & S & C &   A & I &   M & S & C & A \\
  \hline
  Abu Salem/Khuri-Makdisi \cite{kmakdisi18,salem07}
                          & 2 & 97 &  1 & 2 & 132 & 2 & 107 &  3 & 2 & 151 \\
  This thesis             & 1 & 111 & 3 & 0 &  99 & 1 & 135 &  3 & 0 & 116 \\
  \hline
\end{tabular}
\end{center}
Abu Salem and Khuri-Makdisi also consider division by 2 in a finite field as being of negligeable cost.
Here, division by 2 is labeled under C, multiplication by a constant.

The explicit formulae produced by this thesis use considerably fewer additions, but considerably more multiplications.
More importantly, they use one fewer inversion.
Whether or not the formulae in this thesis are an improvement over the state-of-the-art depends
on the cost of computing an inverse versus a product.
While it is claimed in \cite{salem07} that an inversion costs approximately as much as 3 to 10 multiplications,
\cite{dahmen07} claims that the cost may range anywhere from 4 to 80 multiplications, or higher on embedded devices,
while \cite{hankerson04} assumes a cost of 80 multiplications.
I have elected to create Sage implementations of both sets of formulae,
those from Chapter \ref{chap_typical_case} and those from \cite{salem07} and \cite{kmakdisi18},
to compare the number of additions and doublings each can compute in a unit of time.
These implementations are available at \cite{github}.

In order to do the fairest comparison between the two, the following steps were taken.
Let $p$ be the prime number $p = 2^{28} - 57 = 268,435,399$
and let $C$ be the $C_{3,4}$ curve over $\bb F_p$ defined by
 \[y^3 + x^4 - 114167898x^2y - 126472665xy - 20302892x^2 + 56254855y + 63973501x + 65135542.\]
The coefficients of this curve were chosen at random.
The prime $p$ was chosen as it is the largest prime number less than $2^{28}$,
a choice of prime corresponding to a real application where this $C_{3,4}$ curve arithmetic might be used;
in \cite{sutherland16}, Sutherland computes Sato-Tate distributions
on genus 3 ramified hyperelliptic curves over primes $p < 2^{28}$,
Let $D_1$ and $D_2$ be the divisors
\begin{align*}
  D_1 &= \left\langle \begin{array}{r}
    x^2 -  58199944y + 26531881x - 32437186 \\
     xy - 121857018y + 87572390x - 13153072 \\
    y^2 +   9642394y - 58287154x + 13450942
  \end{array} \right\rangle \\
  D_2 &= \left\langle \begin{array}{r}
    x^2 - 91512999y -  67449632x + 64403020 \\
     xy - 92024952y - 121602716x + 97210118 \\
    y^2 + 43402044y - 11043566x  - 21890587
  \end{array} \right\rangle
\end{align*}
chosen at random, and compute the Fibonacci-like sequence 
  \[ D_{i+2} = D_{i+1} + D_i, ~~ i > 0. \]
The implementation of the formulae from Chapter \ref{chap_typical_case} was able to compute
the first 6,532,301 elements of this sequence in 10 minutes.
The implementation of the formulae of Abu Salem and Khuri-Makdisi was able to compute
the first 5,514,964 in the same period of time.
This represents a speed-up of approximately 18.4\%.
None of the divisors encountered in this sequence were atypical,
nor was there ever a case where $D_{i+1} = D_{i}$
(in which case we would be doubling, not adding).

To compare doubling, let $D_1$ be as above,
and repeatedly double $D_1$, computing the sequence
  \[ D_{i+1} = 2D_i, ~~ i > 0. \]
Note that the $D_2$ produced by this sequence is not the same as the $D_2$ from above.
The implementation of the formulae from Chapter \ref{chap_typical_case} was able to compute
the first 6,156,174 elements of this sequence in 10 minutes.
The implementation of the formulae of Abu Salem and Khuri-Makdisi was able to compute
the first 5,249,322 in the same period of time.
This represents a speed-up of approximately 17.3\%.
None of the divisors encountered in this sequence were atypical.

These tests were run on a server with an 80 core 2.8GHz Intel Xeon E7-8891 and 256GB of RAM,
running the operating system Red Hat Enterprise Linux 7.
The initial results suggest an approximate speed-up of 17\%--18\%.
However, there has been some variability in the results.
Re-running the same tests (same sequences, curves, initial divisors, and computer)
at different times has produced results, recording speed-ups anywhere between 15\% and 27\%.
More testing is underway to ensure less variability in the results,
and to perform comparisons over other curves and fields.

The tests were repeated on an Aspire E 15 laptop computer,
with a 4 core 2.5GHz Intel i5-7200U processor and 8GB of RAM,
running the operating system Ubuntu 18.04.3 LTS.
This time, the tests were performed over several curves
over finite fields of order approximately $2^{28}$.
The results are recorded in Table \ref{tab_timing_comparison}.
In each trial, a different prime number $p$ is chosen;
a curve $C$ over the field $\bb F_p$ and divisors $D$ and $D'$ on $C$ are chosen;
and the sequences described above are computed.
Table \ref{tab_timing_comparison} records how many terms in the sequence each algorithm was able to compute in 10 minutes.
Since random curve and divisor generation is slow,
timing began only after the initial curve and divisors were generated.

\begin{table}[h]
\caption{Efficiency gains over several curves and finite fields}
\label{tab_timing_comparison}
\begin{center}
\begin{tabular}{|r|l|rrr|rrr|}
  \hline
        &     & \multicolumn{3}{c|}{Additions} & \multicolumn{3}{c|}{Doublings} \\
  Trial & $p$ & MacNeil & AS/K-M & Adv. & MacNeil & AS/K-M & Adv. \\
  \hline
   1 & $2^{28} + 3$ & 8725 & 7324 & 19.1\% & 8623 & 6657 & 29.5\% \\
   2 & $2^{28} + 7$ & 6981 & 6621 & 5.4\% & 7298 & 7195 & 1.4\% \\
   3 & $2^{28} + 37$ & 8664 & 7528 & 15.1\% & 8204 & 7645 & 7.3\% \\
   4 & $2^{28} + 67$ & 8975 & 7333 & 22.4\% & 8656 & 7150 & 21.1\% \\
   5 & $2^{28} + 81$ & 8613 & 7649 & 12.6\% & 8356 & 7507 & 11.3\% \\
   6 & $2^{28} + 105$ & 8983 & 7865 & 14.2\% & 8639 & 7594 & 13.8\% \\
   7 & $2^{28} + 121$ & 8549 & 7782 & 9.9\% & 8549 & 7550 & 13.2\% \\
   8 & $2^{28} + 123$ & 8814 & 7618 & 15.7\% & 8413 & 7565 & 11.2\% \\
   9 & $2^{28} + 141$ & 8845 & 7717 & 14.6\% & 8510 & 7582 & 12.2\% \\
  10 & $2^{28} + 175$ & 8793 & 7767 & 13.2\% & 8450 & 7576 & 11.5\% \\
  11 & $2^{28} + 183$ & 8735 & 7689 & 13.6\% & 8521 & 7645 & 11.5\% \\
  12 & $2^{28} + 193$ & 8854 & 7710 & 14.8\% & 8517 & 7556 & 12.7\% \\
  13 & $2^{28} + 213$ & 8750 & 7826 & 11.8\% & 8592 & 7641 & 12.4\% \\
  14 & $2^{28} + 241$ & 8745 & 6967 & 25.5\% & 8286 & 7121 & 16.4\% \\
  15 & $2^{28} + 255$ & 8748 & 7828 & 11.8\% & 8635 & 7687 & 12.3\% \\
  16 & $2^{28} + 267$ & 8760 & 7795 & 12.4\% & 8497 & 7056 & 20.4\% \\
  17 & $2^{28} + 291$ & 8392 & 7665 & 9.5\% & 8110 & 7653 & 6.0\% \\
  18 & $2^{28} + 295$ & 8918 & 6570 & 35.7\% & 8171 & 6596 & 23.9\% \\
  19 & $2^{28} + 301$ & 8589 & 7865 & 9.2\% & 8621 & 7624 & 13.1\% \\
  20 & $2^{28} + 357$ & 8928 & 7925 & 12.7\% & 8556 & 7641 & 12.0\% \\
  \hline
  \multicolumn{2}{|r|}{Total:} & 173361 & 151044 & 14.8\% & 168204 & 148241 & 13.5\% \\
  \hline
\end{tabular}
\end{center}
\end{table}



\subsection{Summary of Operation Costs}

Here we list all subroutines implemented in Sage at \cite{github}, beginning with addition subroutines. There is one addition subroutine for every pair of non-zero reduced divisor types
(11, 21, 21, and 31).
The subroutines branch conditionally on the type of the divisor $L = \lcm(D, D')$,
where $D$ and $D'$ are the input divisors.
Operation counts are given for every branch.
When divisors are non-disjoint (i.e. when $\deg L < \deg D + \deg D'$),
the recursive addition algorithm requires that a subroutine call another addition or doubling subroutine.
This leads to an explosion of more cases to consider.
Rather than consider each of these individually, operation counts for the worst case are given in
Table \ref{tab_addition_costs}.
Operation counts in this table also include the cost of reducing $D + D'$.

The computational cost of adding two divisors
\begin{longtable}{|l|rrrr|c|}
  \caption{Operation counts for addition subroutines\label{tab_addition_costs}}\\
  \hline
  & \multicolumn{4}{c|}{Operation count} & \\
  Subroutine & I & M & S & A & Type$(L)$ \\
  \hline  
  {\tt add\_11\_11}  & -- & -- & -- & -- & 21 \\
  {\tt add\_11\_11}  & -- & -- & -- & -- & 22 \\
  \hline  
  {\tt add\_21\_11}  & -- & -- & -- & -- & 31 \\
  {\tt add\_21\_11}  & -- & -- & -- & -- & 32 \\
  {\tt add\_21\_11}  & -- & -- & -- & -- & 21 \\
  \hline
  {\tt add\_21\_21}  & -- & -- & -- & -- & 41 \\
  {\tt add\_21\_21}  & -- & -- & -- & -- & 43 \\
  {\tt add\_21\_21}  & -- & -- & -- & -- & 44 \\
  {\tt add\_21\_21}  & -- & -- & -- & -- & 31 \\
  {\tt add\_21\_21}  & -- & -- & -- & -- & 32 \\
  \hline
  {\tt add\_21\_22}  & -- & -- & -- & -- & 41 \\
  {\tt add\_21\_22}  & -- & -- & -- & -- & 42 \\
  {\tt add\_21\_22}  & -- & -- & -- & -- & 31 \\
  {\tt add\_21\_22}  & -- & -- & -- & -- & 32 \\
  \hline
  {\tt add\_22\_11}  & -- & -- & -- & -- & 31 \\
  {\tt add\_22\_11}  & -- & -- & -- & -- & 33 \\
  {\tt add\_22\_11}  & -- & -- & -- & -- & 22 \\
  \hline
  {\tt add\_22\_22}  & -- & -- & -- & -- & 43 \\
  {\tt add\_22\_22}  & -- & -- & -- & -- & 33 \\
  \hline
  {\tt add\_31\_11}  & -- & -- & -- & -- & 41 \\
  {\tt add\_31\_11}  & -- & -- & -- & -- & 42 \\
  {\tt add\_31\_11}  & -- & -- & -- & -- & 43 \\
  {\tt add\_31\_11}  & -- & -- & -- & -- & 31 \\
  \hline
  {\tt add\_31\_21}  & -- & -- & -- & -- & 51 \\
  {\tt add\_31\_21}  & -- & -- & -- & -- & 52 \\
  {\tt add\_31\_21}  & -- & -- & -- & -- & 53 \\
  {\tt add\_31\_21}  & -- & -- & -- & -- & 54 \\
  {\tt add\_31\_21}  & -- & -- & -- & -- & 41 \\
  {\tt add\_31\_21}  & -- & -- & -- & -- & 42 \\
  {\tt add\_31\_21}  & -- & -- & -- & -- & 43 \\
  {\tt add\_31\_21}  & -- & -- & -- & -- & 31 \\
  \hline
  {\tt add\_31\_22}  & -- & -- & -- & -- & 51 \\
  {\tt add\_31\_22}  & -- & -- & -- & -- & 53 \\
  {\tt add\_31\_22}  & -- & -- & -- & -- & 54 \\
  {\tt add\_31\_22}  & -- & -- & -- & -- & 41 \\
  {\tt add\_31\_22}  & -- & -- & -- & -- & 42 \\
  {\tt add\_31\_22}  & -- & -- & -- & -- & 43 \\
  {\tt add\_31\_22}  & -- & -- & -- & -- & 31 \\
  \hline
  {\tt add\_31\_31\_high\_char}  & -- & -- & -- & -- & 61 \\
  {\tt add\_31\_31}  & -- & -- & -- & -- & 61 \\
  {\tt add\_31\_31}  & -- & -- & -- & -- & 62 \\
  {\tt add\_31\_31}  & -- & -- & -- & -- & 63 \\
  {\tt add\_31\_31}  & -- & -- & -- & -- & 64 \\
  {\tt add\_31\_31}  & -- & -- & -- & -- & 65 \\
  {\tt add\_31\_31}  & -- & -- & -- & -- & 51 \\
  {\tt add\_31\_31}  & -- & -- & -- & -- & 52 \\
  {\tt add\_31\_31}  & -- & -- & -- & -- & 53 \\
  {\tt add\_31\_31}  & -- & -- & -- & -- & 54 \\
  {\tt add\_31\_31}  & -- & -- & -- & -- & 41 \\
  {\tt add\_31\_31}  & -- & -- & -- & -- & 42 \\
  {\tt add\_31\_31}  & -- & -- & -- & -- & 43 \\
  \hline
\end{longtable}

There are fewer cases to consider when doubling divisors.

\begin{longtable}{|l|rrrr|c|}
  \caption{Operation counts for addition subroutines\label{tab_doubling_costs}}\\
  \hline
  & \multicolumn{4}{c|}{Operation count} & \\
  Subroutine & I & M & S & A & Type$(D_1)$ \\
  \hline
  {\tt double\_11}  & -- & -- & -- & -- & 21 \\
  {\tt double\_11}  & -- & -- & -- & -- & 22 \\
  \hline  
  {\tt double\_21}  & -- & -- & -- & -- & 41 \\
  {\tt double\_21}  & -- & -- & -- & -- & 42 \\
  {\tt double\_21}  & -- & -- & -- & -- & 43 \\
  {\tt double\_21}  & -- & -- & -- & -- & 44 \\
  \hline
  {\tt double\_22}  & -- & -- & -- & -- & 42 \\
  {\tt double\_22}  & -- & -- & -- & -- & 43 \\
  {\tt double\_22}  & -- & -- & -- & -- & 44 \\
  \hline
  {\tt double\_31\_high\_char}  & -- & -- & -- & -- & 61 \\
  {\tt double\_31t}  & -- & -- & -- & -- & 61 \\
  {\tt double\_31}  & -- & -- & -- & -- & 61 \\
  {\tt double\_31}  & -- & -- & -- & -- & 62 \\
  {\tt double\_31}  & -- & -- & -- & -- & 63 \\
  {\tt double\_31}  & -- & -- & -- & -- & 64 \\
  {\tt double\_31}  & -- & -- & -- & -- & 65 \\
  {\tt double\_31}  & -- & -- & -- & -- & 51 \\
  {\tt double\_31}  & -- & -- & -- & -- & 52 \\
  {\tt double\_31}  & -- & -- & -- & -- & 53 \\
  {\tt double\_31}  & -- & -- & -- & -- & 54 \\
  {\tt double\_31}  & -- & -- & -- & -- & 41 \\
  {\tt double\_31}  & -- & -- & -- & -- & 42 \\
  {\tt double\_31}  & -- & -- & -- & -- & 43 \\
  {\tt double\_31}  & -- & -- & -- & -- & 44 \\
  \hline
\end{longtable}

\begin{comment}
Divisors of types 0, 11, 21, 22, 31 already reduced,
therefore no reduction subroutines are needed for these types.
Divisors of types 33, 44, 65 are principal divisors;
they may be reduced at no cost by returning the zero divisor.
Reduction subroutines for types 41, 51, and 61
are implemented only for typical divisors.
Atypical divisors of types 41, 51, and 61 are reduced by flipping twice.
\end{comment}
\begin{table}[h]
  \caption{Operation counts for reduction subroutines}
  \label{tab_reduction_costs}
\begin{center}
\begin{tabular}{|l|rrrr|}
  \hline
  & \multicolumn{4}{c|}{Operation count} \\
  Subroutine & I & M & S & A \\
  \hline  
  {\tt reduce\_32}  & 0 & 8 & 0 & 11 \\
  {\tt reduce\_33}  & 0 & 0 & 0 & 0 \\
  {\tt reduce\_41t} & 1 & 23 & 1 & 28 \\
  {\tt reduce\_42}  & 0 & 0 & 0 & 1 \\
  {\tt reduce\_43}  & 0 & 6 & 0 & 11 \\
  {\tt reduce\_44}  & 0 & 0 & 0 & 0 \\
  {\tt reduce\_51t} & 1 & 24 & 0 & 32 \\
  {\tt reduce\_52}  & 0 & 1 & 0 & 3 \\
  {\tt reduce\_53}  & 0 & 12 & 0 & 14 \\
  {\tt reduce\_54}  & 0 & 7 & 0 & 10 \\
  {\tt reduce\_61t} & 1 & 35 & 0 & 46 \\
  {\tt reduce\_62}  & 0 & 2 & 0 & 5 \\
  {\tt reduce\_63}  & 0 & 8 & 0 & 13 \\
  {\tt reduce\_64}  & 0 & 12 & 0 & 21 \\
  {\tt reduce\_65}  & 0 & 0 & 0 & 0 \\
  \hline
\end{tabular}
\end{center}
\end{table}

\begin{table}[h]
  \caption{Operation counts for flipping subroutines}
  \label{tab_flipping_costs}
\begin{center}
\begin{tabular}{|l|rrrr|l|}
  \hline
  & \multicolumn{4}{c|}{Operation count} & \\
  Subroutine & I & M & S & A & Case \\
  \hline  
  {\tt flip\_11}  & -- & -- & -- & -- & \\
  {\tt flip\_21}  & -- & -- & -- & -- & \\
  {\tt flip\_22}  & -- & -- & -- & -- & \\
  {\tt flip\_31}  & -- & -- & -- & -- & $\pid{f,g} = \pid{f,g,h}$ \\
  {\tt flip\_31}  & -- & -- & -- & -- & $\pid{f,g} \neq \pid{f,g,h}$ \\
  {\tt flip\_32}  & -- & -- & -- & -- & \\
  {\tt flip\_33}  & -- & -- & -- & -- & \\
  {\tt flip\_41}  & -- & -- & -- & -- & $\pid{f,g} = \pid{f,g,h}$ \\
  {\tt flip\_41}  & -- & -- & -- & -- & $\pid{f,g} \neq \pid{f,g,h} = \pid{f,h}$ \\
  {\tt flip\_41}  & -- & -- & -- & -- & $\pid{f,g} \neq \pid{f,g,h} \neq \pid{f,h}$ \\
  {\tt flip\_42}  & -- & -- & -- & -- & \\
  {\tt flip\_43}  & -- & -- & -- & -- & \\
  {\tt flip\_44}  & -- & -- & -- & -- & \\
  {\tt flip\_51}  & -- & -- & -- & -- & $\pid{f,g} = \pid{f,g,h}$ \\
  {\tt flip\_51}  & -- & -- & -- & -- & $\pid{f,g} \neq \pid{f,g,h} = \pid{f,h}$ \\
  {\tt flip\_51}  & -- & -- & -- & -- & $\pid{f,g} \neq \pid{f,g,h} \neq \pid{f,h}$ \\
  {\tt flip\_52}  & -- & -- & -- & -- & \\
  {\tt flip\_53}  & -- & -- & -- & -- & \\
  {\tt flip\_54}  & -- & -- & -- & -- & \\
  {\tt flip\_61}  & -- & -- & -- & -- & $\pid{f,g} = \pid{f,g,h}$ \\
  {\tt flip\_61}  & -- & -- & -- & -- & $\pid{f,g} \neq \pid{f,g,h} = \pid{f,h}$ \\
  {\tt flip\_61}  & -- & -- & -- & -- & $\pid{f,g} \neq \pid{f,g,h} \neq \pid{f,h}$ \\
  {\tt flip\_62}  & -- & -- & -- & -- & \\
  {\tt flip\_63}  & -- & -- & -- & -- & \\
  {\tt flip\_64}  & -- & -- & -- & -- & \\
  {\tt flip\_65}  & -- & -- & -- & -- & \\
  \hline
\end{tabular}
\end{center}
\end{table}





%%%%%%%%%%%%%%%%%%%%%%%%%%%%%%%%
%%%%%                      %%%%%
%%%%%   The Typical Case   %%%%%
%%%%%                      %%%%%
%%%%%%%%%%%%%%%%%%%%%%%%%%%%%%%%

\section{Explicit Formulae}
\label{chap_typical_case}

The three previous chapters discuss adding, doubling, and reducing arbitrary divisors.
With the methods described in those chapters, we may add any two reduced divisors,
double any reduced divisor, and reduce any unreduced divisor.

Suppose $C$ is a $C_{3,4}$ curve over the finite field $\bb F_q$.
As $q \to \infty$, the chance that a divisor of $C$ chosen uniformly at random is atypical
tends toward $\frac 1 q$ (Theorem 2.10 in \cite{kmakdisi18}).
Consequently, if we are adding or doubling divisors,
we will spend most of our time ---
almost all of it if $q$ is very large ---
adding and doubling typical type 31 divisors and reducing typical type 61 divisors.
Any algorithm which requires adding and doubling divisors will benefit greatly from
having the arithmetic in those cases highly optimized.
In this chapter, we will apply the methods described in
Chapters \ref{chap_addition}, \ref{chap_doubling}, and \ref{chap_reduction}
and derive explicit formulae to efficiently handle
the typical cases arising in divisor class group arithmetic:
adding and reducing two disjoint typical type 31 divisors,
and doubling and reducing a typical type 31 divisor.

We saw in Chapter \ref{chap_addition} that in order to add two divisors,
we must first compute the kernel of a matrix $M_{\text{add}}$.
Likewise, to double a divisor, we must first compute the kernel of a matrix $M_{\text{doub}}$.
In the following two sections, we will see explicit formulae describing the entries of these matrices.
After computing the elements of the matrices,
the rest of the algorithm ---
row-reducing the matrices, computing their kernel, then reducing the resulting divisor ---
proceeds identically.

The formulae we derive in this chapter are more efficient than the current state-of-the-art published in \cite{kmakdisi18}.
The savings come primarily from two places.
Firstly, after computing polynomials $u,v$ for the basis of $\ker M_{\text{add}}$ (or $\ker M_{\text{doub}}$),
the constant coefficients of $u$ and $v$ are not needed in the rest of the calculations.
Therefore, we do not compute their constant coefficients, saving several multiplications.
Secondly, computing the reduced row echelon form of $M_{\text{add}}$ or $M_{\text{doub}}$ requires one inversion
operation, while reducing a divisor requires another inversion.
It is possible to compute the inverses of two finite field elements while performing only one inversion operation,
with a technique called Montgomery's Trick,
a mathematical case of killing two birds with one stone.
\begin{example}
  \label{ex_two_birds_one_stone}
  Let $a, b$ be non-zero elements of some finite field.
  Rather than inverting $a$ and $b$ separately, compute
  \[ c := ab, ~~ d := \frac 1 c, ~~ \alpha := bd, ~~ \beta := ad. \]
  Then $\alpha = a\inv$ and $\beta = b\inv$, while only one inversion operation was performed.
  One inversion is saved at the cost of 3 multipiclations.
\end{example}
Inversions in a finite field are expensive operations;
it is worth eliminating one inversion even at the cost of many multiplications,
although the exact number of multiplications one should be willing to trade is dependent on
the software/hardware implementation of the finite field arithmetic and the size of the finite field.
The formulae we find for adding and doubling cost, respectively, 1I+111M+3S+99A and 1I+135M+3S+116A,
where I, M, S, A refer to the number of inversion, multiplications, squares, and additions in a finite field.
We require more multiplications, but fewer inversions than \cite{kmakdisi18},
so we will compare Sage implementations of the formulae below and the formulae in \cite{kmakdisi18}
and see that the trade-off is worthwhile, at least in Sage.
We refer later to literature that suggests that our formulae should be more efficient in other implementations,
hardware or software.

For the remainder of this chapter, we will assume that $C$ is a $C_{3,4}$ curve
over a field $K$ with $\Char K = 0$ or $\Char K > 3$, defined by a polynomial $F$.
The curve $C$ will be assumed to be given by an equation in short form (Equation \ref{eq_c34_short}),
i.e. with coefficients $c_5$, $c_6$, and $c_8$ all equal to 0.
Recall that over a field of characteristic other than 2 or 3, a $C_{3,4}$ curve may always be transformed to one in short form.
The resulting formulae are collected and reproduced in one place at the end of Section \ref{sec_reducing}.



%%%%%%%%%%%%%%%%%%%%%%%%%%%%%%%
%%%%%                     %%%%%
%%%%%   Addition Matrix   %%%%%
%%%%%                     %%%%%
%%%%%%%%%%%%%%%%%%%%%%%%%%%%%%%

\subsection{Constructing the Matrix $M_{\text{add}}$.}

We wish to add two type 31 divisors $D$ and $D'$,
given by ideals $I_D = \pid{f,g,h}$ and $I_{D'} = \pid{f',g',h'}$, where
\begin{align*}
  f &= x^2 + f_2y + f_1x + f_0 & f' &= x^2 + f'_2y + f'_1x + f'_0 \\
  g &=  xy + g_2y + g_1x + g_0 & g' &=  xy + g'_2y + g'_1x + g'_0 \\
  h &= y^2 + h_2y + h_1x + h_0 & h' &= y^2 + h'_2y + h'_1x + h'_0.
\end{align*}
We will assume that $D$ and $D'$ are disjoint and that $D + D'$ is a typical type 61 divisor.
We will be able to detect whether these assumptions are violated,\footnote{
Non-disjointedness is detected when $M_{\text{add}}$ does not have full rank,
and $D+D'$ can be confirmed to be typical via Lemma \ref{lem_fgh_is_fg_61}.}
and if so, we may fall back on the addition algorithm from Chapter \ref{chap_addition}.
We need not assume that $D$ or $D'$ are typical;
it is possible for $D + D'$ to be typical even if one or both of $D$ and $D'$ is not,
and the formulae will still be correct.

We consider the map $M_{\text{add}}$ in 
\[  \begin{tikzcd}
      W_{D+D'}^{x^2y} \arrow[hook]{r}{\ker M_{\text{add}}} & 
      W_D^{x^2y} \arrow[hook]{r}{\iota} \arrow[bend left]{rr}{M_{\text{add}}} & 
      W^{x^2y} \arrow[two heads]{r}{\pi} & 
      \frac {W^{x^2y}} {W_{D'}^{x^2y}}
    \end{tikzcd}. \]
Since we are assuming $D$ and $D'$ are disjoint, the kernel gives their sum $D + D'$ rather than merely $\lcm(D,D')$.
We are assuming that $D + D'$ is typical, hence the third polynomial in the Gr\"obner basis of $I_{D+D'}$ is not needed,
and the other two polynomials in the basis contain no monomial larger than $x^2y$.

We construct the matrix 
\[ M_{\text{add}} =
\begin{pmatrix}
  a_1 & a_2 & a_3 & a_4 & a_5 \\
  a_6 & a_7 & a_8 & a_9 & a_{10} \\
  a_{11} & a_{12} & a_{13} & a_{14} & a_{15}
\end{pmatrix}. \]
The space $W_D^{x^2y}$ is spanned by the basis $\{f,g,h,xf,xg\}$.
The columns of $M_{\text{add}}$, from left to right, are the reductions of $f,g,h,xf,xg$ modulo $f',g',h'$.
Therefore, the first three columns are given by
\begin{align*}
  a_1    &= f_0 - f'_0 & a_2    &= g_0 - g'_0 & a_3    &= h_0 - h'_0 \\
  a_6    &= f_1 - f'_1 & a_7    &= g_1 - g'_1 & a_8    &= h_1 - h'_1 \\
  a_{11} &= f_2 - f'_2 & a_{12} &= g_2 - g'_2 & a_{13} &= h_2 - h'_2.
\end{align*}
We may view multiplication by $x$ as an endomorphism on $\frac {W^{x^2y}}{W_{D'}^{x^2y}}$, given by the matrix
\[ T_x = \begin{pmatrix}
    0 & -f'_0 & -g'_0 \\
    1 & -f'_1 & -g'_1 \\
    0 & -f'_2 & -g'_2
  \end{pmatrix}. \]
We use this matrix to compute the reductions of $xf$ and $xg$ modulo $f',g',h'$.
This gives the last two columns of $M_{\text{add}}$, via
\[ 
  \begin{pmatrix}
    a_4    & a_5    \\
    a_9    & a_{10} \\
    a_{14} & a_{15}
  \end{pmatrix} = 
  \begin{pmatrix}
    0 & -f'_0 & -g'_0 \\
    1 & -f'_1 & -g'_1 \\
    0 & -f'_2 & -g'_2
  \end{pmatrix}
  \begin{pmatrix}
    a_1    & a_2    \\
    a_6    & a_7    \\
    a_{11} & a_{12}
  \end{pmatrix}.
\]
This results in
\begin{align*}
  a_4    &=     - f'_0a_6 - g'_0a_{11} & a_5    &=     - f'_0a_7 - g'_0a_{12} \\
  a_9    &= a_1 - f'_1a_6 - g'_1a_{11} & a_{10} &= a_2 - f'_1a_7 - g'_1a_{12} \\
  a_{14} &=     - f'_2a_6 - g'_2a_{11} & a_{15} &=     - f'_2a_7 - g'_2a_{12}.
\end{align*}

\begin{lemma}
  \label{lem_Madd_op_count}
  The matrix $M_{\text{add}}$ may be computed in 12M+17A.
\end{lemma}


%%%%%%%%%%%%%%%%%%%%%%%%%%%%%%%
%%%%%                     %%%%%
%%%%%   Doubling Matrix   %%%%%
%%%%%                     %%%%%
%%%%%%%%%%%%%%%%%%%%%%%%%%%%%%%

\subsection{Constructing the Matrix $M_{\text{doub}}$.}

We wish to double a type 31 divisor $D$, given by the ideal $I_D = \pid{f,g,h}$, where
\begin{align*}
  f &= x^2 + f_2y + f_1x + f_0 \\
  g &=  xy + g_2y + g_1x + g_0 \\
  h &= y^2 + h_2y + h_1x + h_0.
\end{align*}
We will assume that $D$ is typical, so that $f_2 \neq 0$.
Let the inverse of $f_2$ be $\phi = \frac 1 {f_2}$.
We will assume that $\phi$ is given as an input to the doubling algorithm.
This same assumption is made by \cite{salem07} and \cite{kmakdisi18},
justified by the fact that $\phi$ may be included in the outputs of the reduction algorithm,
and that typically one is adding or doubling divisors that were previously reduced by the reduction algorithm.

We must find $g', h' \in K[C]$ such that $fh' \equiv gg' \pmod F$.
The polynomials $g'$ and $h'$ are of the forms
\begin{align*}
  g' &= xy            + g'_2y + g'_1x + g'_0 \\
  h' &= y^2 + h'_3x^2 + h'_2y + h'_1x + h'_0.
\end{align*}
Equating coefficients of $g'$ and $h'$ in the equation $fh' - gg' - f_2F = 0$
gives the solution
\begin{align*}
  h'_3 &= f_2 \\
  g'_2 &= f_1 - g_2 \\
  h'_2 &= \phi(g'_2g_2 - f_0) \\ 
  g'_1 &= f_2(f_2 - c_7) + h'_2 - g_1 \\
  h'_1 &= -f_1f_2 \\
  g'_0 &= h'_2f_1 + f_2(h'_1 - c_4) - g'_2g_1 - g'_1g_2 - g_0 \\
  h'_0 &= f_2(c_3 - f_0) - h'_1f_1 + g'_1g_1.
\end{align*}
\begin{comment}
% For arbitrary field characteristic
\begin{align*}
  h'_3 &= f_2 \\
  g'_2 &= - c_8f_2 + f_1 - g_2 \\
  h'_2 &= c_5 + \phi(g'_2g_2 - f_0) \\ 
  g'_1 &= f_2(f_2 - c_7) + h'_2 - g_1 \\
  h'_1 &= f_2(c_6 - f_1) \\
  g'_0 &= h'_2f_1 + f_2(h'_1 - c_4) - g'_2g_1 - g'_1g_2 - g_0 \\
  h'_0 &= f_2(c_3 - f_0) - h'_1f_1 + g'_1g_1.
\end{align*}
\end{comment}
We consider the map $M_{\text{doub}}$ in 
\[  \begin{tikzcd}
      W_{2D}^{x^2y} \arrow[hook]{r}{\ker M_{\text{doub}}} & 
      W_D^{x^2y} \arrow[hook]{r}{d} \arrow[bend left]{rr}{M_{\text{doub}}} & 
      W^{x^2y} \arrow[two heads]{r}{\pi} & 
      \frac {W^{x^2y}} {W_{D'}^{x^2y}}
    \end{tikzcd}, \]
where $d(f) = g'$ and $d(g) = h'$.
The space $W_D^{x^2y}$ is spanned by the basis
  \[ W_D^{x^2y} = \left\{ f,g,h,xf,xg \right\} = \left\{ f,g,\frac{(y + g_1)f - (x + f_1 - g_2)g}{f_2},xf,xg \right\}. \]
We construct the matrix 
\[ M_{\text{doub}} =
\begin{pmatrix}
  a_1 & a_2 & a_3 & a_4 & a_5 \\
  a_6 & a_7 & a_8 & a_9 & a_{10} \\
  a_{11} & a_{12} & a_{13} & a_{14} & a_{15}
\end{pmatrix}. \]
The columns of $M_{\text{add}}$, from left to right,
are the reductions of $g',h',\phi((y + g_1)g' - (x + f_1 - g_2)h'),xg',xh'$ modulo $f',g',h'$.
The first two columns are therefore given by
\begin{align*}
  a_1    &= g'_0 - g_0 & a_2    &= h'_0 - h_0 - f_0f_2 \\
  a_6    &= g'_1 - g_1 & a_7    &= h'_1 - h_1 - f_1f_2 \\
  a_{11} &= g'_2 - g_2 & a_{12} &= h'_2 - h_2 - f_2f_2.
\end{align*}
However, observe that $h'_1 = -f_1f_2$, so $a_7 = 2h'_1 - h_1$.
The last two columns, like in the previous section, may be computed by
\[ 
  \begin{pmatrix}
    a_4    & a_5    \\
    a_9    & a_{10} \\
    a_{14} & a_{15}
  \end{pmatrix} = 
  \begin{pmatrix}
    0 & -f'_0 & -g'_0 \\
    1 & -f'_1 & -g'_1 \\
    0 & -f'_2 & -g'_2
  \end{pmatrix}
  \begin{pmatrix}
    a_1    & a_2    \\
    a_6    & a_7    \\
    a_{11} & a_{12}
  \end{pmatrix},
\]
giving
\begin{align*}
  a_4    &=     - f_0a_6 - g_0a_{11} & a_5    &=     - f_0a_7 - g_0a_{12} \\
  a_9    &= a_1 - f_1a_6 - g_1a_{11} & a_{10} &= a_2 - f_1a_7 - g_1a_{12} \\
  a_{14} &=     - f_2a_6 - g_2a_{11} & a_{15} &=     - f_2a_7 - g_2a_{12}.
\end{align*}

Just as multiplication by $x$ may be seen as an endomorphism on $\frac {W^{x^2y}}{W_D^{x^2y}}$,
so too may multiplication by $y$. The matrix of this endomorphism is
\[ T_y = \begin{pmatrix}
    0 & -g'_0 & -h'_0 \\
    0 & -g'_1 & -h'_1 \\
    1 & -g'_2 & -h'_2
  \end{pmatrix}. \]
Column 3 is then computed by
\begin{align*}
  C_3 &= \phi \left( T_y(C_1) + g_1C_1 - T_x(C_2) - (f_1 - g_2)C_2 \right) \\
      &= \phi \left( T_y(C_1) + g_1C_1 - C_5 - (f_1 - g_2)C_2 \right)
\end{align*}
Noting that $f_1 - g_2 = g'_2$,
\begin{align*}
  a_3    &= \phi \left(     - g_0a_6 - h_0a_{11} + g_1a_1    - a_5    - g'_2a_2    \right) \\
  a_8    &= \phi \left(              - h_1a_{11}             - a_{10} - g'_2a_7    \right) \\
  a_{13} &= \phi \left( a_1 - g_2a_6 - (h_2 - g_1)a_{11}     - a_{15} - g'_2a_{12} \right) .
\end{align*}
\begin{comment}
\begin{align*}
  a_3    &= \phi \left(     - g_0a_6 - h_0a_{11} + g_1a_1    - a_5    - (f_1 - g_2)a_2    \right) \\
  a_8    &= \phi \left(              - h_1a_{11}             - a_{10} - (f_1 - g_2)a_7    \right) \\
  a_{13} &= \phi \left( a_1 - g_2a_6 - (h_2 - g_1)a_{11}     - a_{15} - (f_1 - g_2)a_{12} \right) .
\end{align*}
\end{comment}

\begin{lemma}
  \label{lem_Mdoub_op_count}
  The matrix $M_{\text{doub}}$ may be computed in 36M+44A.
\end{lemma}
\begin{proof}
  The coefficients of $g'$ and $h'$ require 11M+13A to compute as written,
  however, the terms $g'_1g_1$, $g'_1g_2 + g'_2g_1$, and $g'_2g_2$ all appear.
  Given $g'_1g_1$ and $g'_2g_2$, rather than compute $g'_1g_2 + g'_2g_1$ in 2M+1A,
  we may compute it in 1M+4A via Karatsuba multiplication:
  \[ g'_1g_2 + g'_2g_1 = (g'_1 + g'_2)(g_1 + g_2) - g'_1g_1 - g'_1g_2. \]
  Thus the coefficients of $g'$ and $h'$ are obtained in 10M+16A.
  
  The first column of $M_{\text{doub}}$ costs 3A to compute.
  The second column costs 2M+6A to compute, counting doubling $h'_1$ as 1 addition.
  Columns 4 and 5 each cost 6M+4A, while column 3 costs 12M+11A.
\end{proof}



%%%%%%%%%%%%%%%%%%%%%%%%%%%%%%%%
%%%%%                      %%%%%
%%%%%   Computing Kernel   %%%%%
%%%%%                      %%%%%
%%%%%%%%%%%%%%%%%%%%%%%%%%%%%%%%

\subsection{Computing Kernel}

After computing the matrix $M_{\text{add}}$ or $M_{\text{doub}}$,
the addition and doubling algorithms proceed identically,
so let $M$ be one of the above matrices.
If we are adding, let $M = M_{\text{add}}$ and $A = D + D'$.
If we are doubling, let $M = M_{\text{doub}}$ and $A = 2D$ instead.

In Chapters \ref{chap_addition} and \ref{chap_doubling},
we found a reduced Gr\"obner basis for $A$ by finding a reduced echelon basis for $\ker M$.
Reducing $M$ to its reduced row echelon form typically requires an inversion in $\bb F_q$
and this ensures that the resulting Gr\"obner basis for $A$ is monic.
In the following, we delay inversion until the next section.
Instead, we compute a multiple of $\rref(M)$.
Consequently, we obtain a pair $\{U,V\}$ of non-monic polynomials that generate $I_A$.

Recall that the elements of $M$ are
\[ M =
\begin{pmatrix}
  a_1 & a_2 & a_3 & a_4 & a_5 \\
  a_6 & a_7 & a_8 & a_9 & a_{10} \\
  a_{11} & a_{12} & a_{13} & a_{14} & a_{15}
\end{pmatrix}. \]
If the first column of $M$ is zero, then $A$ is not a type 61 divisor,
so we must abort and fall back on other methods described in this thesis.
If the first column is non-zero, but $a_1 = 0$, then swap rows and relabel the elements so that $a_1 \neq 0$.
We now put $M$ into echelon form,
\[ M' =
\begin{pmatrix}
  a_1 & a_2 & a_3 & a_4 & a_5 \\
    0 & b_1 & b_2 & b_3 & b_4 \\
    0 &   0 & b_5 & b_6 & b_7
\end{pmatrix}, \]
via
\begin{align*}
  d_1 &= a_1a_{12} - a_2a_{11} & b_1 &= a_1a_7    - a_2a_6 & b_5 &= b_1a_{13} - d_1a_8    + d_2a_3 \\
  d_2 &= a_6a_{12} - a_7a_{11} & b_2 &= a_1a_8    - a_3a_6 & b_6 &= b_1a_{14} - d_1a_9    + d_2a_4 \\
      &                        & b_3 &= a_1a_9    - a_4a_6 & b_7 &= b_1a_{15} - d_1a_{10} + d_2a_5 \\
      &                        & b_4 &= a_1a_{10} - a_5a_6.
\end{align*}
If $b_1 = 0$ or $b_5 = 0$, then $A$ is not of type 61, so we must abort.
Otherwise, we reduce $M'$ further to
\[ M'' =
\begin{pmatrix}
  Z & 0 & 0 & A_1 & A_2 \\
  0 & Z & 0 & B_1 & B_2 \\
  0 & 0 & Z & C_1 & C_2
\end{pmatrix}, \]
where
\begin{align*}
  Y &= a_1b_1 & e_1 &= b_3b_5 - b_2b_6 \\
  Z &= Yb_5   & e_2 &= b_4b_5 - b_2b_7
\end{align*}
\begin{align*}
  A_1 &= b_1(a_4b_5 - b_6a_3) - a_2e_1 & B_1 &= a_1e_1 & C_1 &= Yb_6 \\
  A_2 &= b_1(a_5b_5 - b_7a_3) - a_2e_2 & B_2 &= a_1e_2 & C_2 &= Yb_7
\end{align*}
This matrix $M''$ is $Z \cdot \rref(M)$.
If we have made it this far without aborting, then $a_1$, $b_1$, and $b_5$ are non-zero, so $Z$ is non-zero too.

The kernel of $M$ is $\Span_K\{U,V\}$, where
\begin{align*}
  U &= Zxf - C_1h - B_1g - A_1f \\
  V &= Zxg - C_2h - B_2g - A_2f.
\end{align*}
Let $U_1, \ldots, U_5$ be the coefficients of $x, y, x^2, xy, y^2$ in $U$
and let $V_1, \ldots, V_5$ be the coefficients of $x, \ldots, y^2$ in $V$.
These are given explicitly by
\begin{align*}
  U_1 &= Zf_0 - C_1h_1 - B_1g_1 - A_1f_1 & V_1 &= Zg_0 - C_2h_1 - B_2g_1 - A_2f_1 \\
  U_2 &=      - C_1h_2 - B_1g_2 - A_1f_2 & V_2 &=      - C_2h_2 - B_2g_2 - A_2f_2 \\
  U_3 &= Zf_1 - A_1 & V_3 &= Zg_1 - A_2 \\
  U_4 &= Zf_2 - B_1 & V_4 &= Zg_2 - B_2 \\
  U_5 &=      - C_1 & V_5 &=      - C_2.
\end{align*}
Note that in the computations to follow in the next section,
we will not need the constant coefficients of $U$ and $V$.

\begin{lemma}
  \label{lem_UV_op_count}
  Given $M$, the non-constant coefficients of the polynomials $U$ and $V$ may be computed in 57M+32A.
\end{lemma}
\begin{proof}
  The elements $b_i$ of the matrix $M'$ are computed in a total of 21M+12A.
  The matrix $M''$ is then computed in an additional 18M+6A.
  Given $M''$, the coefficients $U_i, V_i$ for $1 \leq i \leq 5$ cost a total of 18M+14A.
\end{proof}



%%%%%%%%%%%%%%%%%%%%%%%%
%%%%%              %%%%%
%%%%%   Reducing   %%%%%
%%%%%              %%%%%
%%%%%%%%%%%%%%%%%%%%%%%%

\subsection{Reducing}
\label{sec_reducing}

We now have two (non-monic) polynomials $U$ and $V$.
Let $u = \frac U Z$ and $v = \frac V Z$.
The type 61 divisor $A$ has the ideal $I_A = \pid{u, v}$,
where $u$ and $v$ are the first two of three polynomials needed for the reduced Gr\"obner basis of $I_A$.
We will not need the third polynomial $w$ of $I_A$'s Gr\"obner basis,
but we note that by Lemma \ref{lem_fgh_is_fg_61},
there are polynomials $r,s \in K[C]$ and a constant $t \in K$
such that $ru + sv + tw \equiv 0 \pmod F$.
The divisor $A$ is typical if and only if $t \neq 0$.
Moreover, $-t$ will become the coefficient of $y$ in $f''$ in the ideal of the reduced divisor
$I_{\bar{\bar{A}}} = \pid{f'', g'', h''}$.

The inverse of $-t$ will be needed, as will the inverse of $Z$.
Let their inverses be $\zeta = \frac 1 Z$ and $\tau = -\frac 1 t$.
As alluded to in the introduction to this chapter,
we will compute $\zeta$ and $\tau$ using only a single inversion.
However, it is more complicated than in Example \ref{ex_two_birds_one_stone}.
We know $Z$ but we do not yet know $-t$. We compute
\begin{align*}
  z_0 &= U_5^2 + Z(u_4 - v_5) &  z_1 &= Zz_0 \\
  z_2 &= \frac 1 {z_1} & z_3 &= Zz_2 \\
  \zeta &= z_0z_2 &  \tau &= Z^2z_3.
\end{align*}
\begin{comment}
\begin{align*}
  z_0 &= U_5^2 - Z(U_5c_8 - u_4 + v_5) \\
  z_1 &= Zz_0 \\
  z_2 &= \frac 1 {z_1} \\
  z_3 &= Zz_2 \\
  \zeta &= z_0z_2 \\
  \tau  &= Z^2z_3.
\end{align*}
\end{comment}
Compare the value of $t$ in Lemma \ref{lem_fgh_is_fg_61} to the value of $z_0$ above.
We confirm that $-t$ and $\tau$ are inverses:
\[ \frac 1 \tau = \frac 1 {Z^2z_3} 
                = \frac 1 {Z^3z_2} 
                = \frac {z_1} {Z^3} 
                = \frac {z_0} {Z^2}
                = \frac {U_5^2 + Z(U_4 - V_5)} {Z^2}
                = u_5^2 + u_4 - v_5
                = -t. \]
\begin{comment}
\[ \frac 1 \tau = \frac 1 {Z^2z_3} 
                = \frac 1 {Z^3z_2} 
                = \frac {z_1} {Z^3} 
                = \frac {z_0} {Z^2}
                = \frac {U_5^2 - Z(U_5c_8 - U_4 + V_5)} {Z^2}
                = u_5^2 - u_5c_8 + u_4 - v_5
                = -t. \]
\end{comment}
Now let $u_1, \ldots, u_5$ and $v_1, \ldots, v_5$ be the coefficients of $x, y, \ldots, y^2$ in $u$ and $v$.
Compute them by
\[ u_i = \zeta U_i, ~~v_i = \zeta V_i, ~~1 \leq i \leq 5. \]

Now $I_A = \pid{u,v}$.
Following Theorem \ref{thm_fast_reduction},
in order to reduce $A$, we find polynomials
$f'' = x^2 + f''_2y + f''_1x + f''_0$ and
$G = xy + G_3x^2 + G_2y + G_1x + G_0$ such that $f''v \equiv Gu \pmod F$.
We would find that $G_3 = u_5$ and we would then compute the reduction of $G$ modulo $f''$ by $g'' = G - G_3 f''$.
We do both at once by instead computing $f''$ and $g'' = xy + g''_2y + g''_1x + g''_0$
such that $f''v \equiv (g'' + u_5f'')u \pmod F$.
Equating coefficients in the equation\footnote{
$f_2v_5 - u_5(x + f'_2u_5 + g_2)$ is chosen so that the left-hand side contains no $y^3$ term.}
\[f''v - (g'' + u_5f'')u - (f_2v_5 - u_5(x + f'_2u_5 + g_2))F = 0 \]
gives
\begin{align*}
  f''_2 &= u_5^2 + u_4 - v_5 \\
  r_0   &= u_5(f''_2 + u_4 - c_7) + u_3 - v_4 \\
  r_1   &= f''_2(f''_2 - u_4) \\
  g''_1 &= r_1 - u_5(u_3 + r_0) + v_3 \\
  g''_2 &= -u_4u_5 + v_4 - r_0 + \tau(u_4r_0 - u_5g''_1 - u_2) \\
  f''_1 &= r_0 + g''_2 \\
  f''_0 &= -c_7(r_1 + g''_2u_5) + u_5(f''_2u_3 + f''_1u_4 - c_4 + u_2)
         + g''_2u_3 + g''_1u_4 - f''_2v_3 - f''_1v_4 + u_1 - v_2 \\
  g''_0 &= u_5(c_3 - f''_0 - u_1 - f''_1u_3) - g''_1u_3 + f''_1v_3 + v_1.
\end{align*}
In deriving these formulae, it can be shown that $r_0 = f''_1 - g''_2$.
\begin{comment}
\begin{align*}
  G_3 &= u_5 \\
  f''_2 &= u_5^2 + u_4 - v_5 \\
  G_2 &= v_4 + u_5v_5 + \tau(u_5(u_3u_5 + v_5(u_4 - c_7) - v_3) + v_5(u_3 - v_4) - u_2) \\
  e_3 &= f''_2v_5 - G_2u_5 \\
  f''_1 &= u_5(u_4 - c_7) + G_2 + u_3 - v_4 \\
  G_1 &= -u_3u_5 - e_3 + v_3 \\
  f''_0 &= c_7e_3 + u_5(u_2 - c_4) + G_2u_3 + G_1u_4 - f''_2v_3 - f''_1v_4 + u_1 - v_2 \\
  G_0 &= u_5(c_3 - u_1) - G_1u_3 + f''_1v_3 + v_1.
\end{align*}
\end{comment}
\begin{comment}
\begin{align*}
  G_3 &= u_5 \\
  f''_2 &= u_5(u_5 - c_8) + u_4 - v_5 \\
  G_2 &= v_4 + v_5(u_5 - c_8) + \tau(u_5(u_5(u_3 - c_6) + v_5(u_4 - c_7) + c_5 - v_3) + v_5(u_3 - v_4) - u_2) \\
  e_3 &= f''_2v_5 - G_2u_5 \\
  f''_1 &= u_5(u_4 - c_7) + G_2 + u_3 - v_4 \\
  G_1 &= u_5(c_6 - u_3) - e_3 + v_3 \\
  f''_0 &= c_7e_3 + u_5(u_2 - c_4) + G_2u_3 + G_1u_4 - f''_2v_3 - f''_1v_4 + u_1 - v_2 \\
  G_0 &= -c_6e_3 + u_5(c_3 - u_1) - G_1u_3 + f''_1v_3 + v_1.
\end{align*}
\end{comment}

Finally, we compute $h'' = y^2 + h''_2y + h''_1x + h''_0$,
the third polynomial in the reduced Gr\"obner basis of $I_{\bar{\bar A}} = \pid{f'', g'', h''}$.
Since $\bar{\bar A}$ is typical, $h''$ is given by
\[ h'' = \frac {(y + g''_1)f'' - (x + f''_1 - g''_2)g''} {f''_2}, \]
though we note that $f''_1 - g''_2 = r_0$.
The coefficients of $h''$ are given by
\begin{align*}
  h''_0 &= \tau(f''_0g''_1 - g''_0r_0) \\
  h''_1 &= \tau(g''_1g''_2 - g''_0) \\
  h''_2 &= g''_1 + \tau(f''_0 - g''_2r_0).
\end{align*}

\begin{lemma}
  \label{lem_fgh_op_count}
  Given $Z, U_1, \ldots, U_5, V_1, \ldots, V_5$,
  the reduced Gr\"obner basis $\{f'',g'',h''\}$ for $I_{\bar{\bar A}}$ can be computed in 1I+42M+3S+40A.
\end{lemma}
\begin{proof}
  It costs 1I+5M+2S+3A to compute $\tau$ and $\zeta$;
  10M to compute $u_1, \ldots, u_5, v_1, \ldots, v_5$;
  20M+1S+33A to compute $f''$ and $g''$; and
  7M+4A to compute $h''$.
\end{proof}



\subsection{Explicit Formulae}

We collect the lemmas from the previous sections
to get a statement about the total cost of adding divisors,
in the typical case.

\begin{theorem}
  Let $D$ and $D'$ be type 31 divisors,
  represented by reduced Gr\"obner bases of their ideals $I_D$ and $I_{D'}$.
  \begin{enumerate}[label=(\roman*)]
    \item
    Let $D'' = \bar{\bar{D + D'}}$.
    If $D''$ is typical, then a reduced Gr\"obner basis for $I_{D''}$
    can be computed in 1I+111M+3S+99A.

    \item
    Let $D'' = \bar{\bar{2D}}$.
    If $D$ and $D''$ are typical, then a reduced Gr\"obner basis for $I_{D''}$
    can be computed in 1I+135M+3S+116A.
  \end{enumerate}
\end{theorem}
\begin{proof}
  For part (i), add the operation counts in Lemmas
  \ref{lem_Madd_op_count}, \ref{lem_UV_op_count}, and \ref{lem_fgh_op_count}.
  For part (ii), add the operation counts in Lemmas
  \ref{lem_Mdoub_op_count}, \ref{lem_UV_op_count}, and \ref{lem_fgh_op_count}.
\end{proof}

The formulae presented in the previous sections are also collected
in Tables \ref{tab_add_31_31} and \ref{tab_double_31}.

%%%%%%%%%%%%%%%%%%%%%%%%%%%%%%%%%%%%%%%%%%%%
%%%%%   Explicit Formulae : ADDITION   %%%%%
%%%%%%%%%%%%%%%%%%%%%%%%%%%%%%%%%%%%%%%%%%%%

\begin{longtable}{|l|r|}
  \caption{Explicit Formulae for adding two type 31 divisors (typical case)\label{tab_add_31_31}}\\
  \hline
  Addition & 1I+111M+3S+99A \\
  Input: $I_D = \pid{f, g, h}$, $I_{D'} = \pid{f',g',h'}$ & \\
  $\begin{aligned}[t]
  f &= x^2 + f_2y + f_1x + f_0 & f' = x^2 + f'_2y + f'_1x + f'_0 \\
  g &=  xy + g_2y + g_1x + g_0 & g' =  xy + g'_2y + g'_1x + g'_0 \\
  h &= y^2 + h_2y + h_1x + h_0 & h' = y^2 + h'_2y + h'_1x + h'_0
  \end{aligned}$ & \\
  Output: $I_{\bar{\bar{D + D'}}} = \pid{f'', g'', h''}$, $(f''_2)\inv$ & \\
  $\begin{aligned}[t]
  f'' &= x^2 + f''_2y + f''_1x + f''_0 \\
  g'' &=  xy + g''_2y + g''_1x + g''_0 \\
  h'' &= y^2 + h''_2y + h''_1x + h''_0 
  \end{aligned}$ & \\
  \hline \hline
  Compute elements $a_i$ of $M_{\text{add}}$ & 12M+17A \\
  \hline
  $\begin{aligned}[t]
    a_1    &= f_0 - f'_0 & a_2    &= g_0 - g'_0 & a_3    &= h_0 - h'_0 \\
    a_6    &= f_1 - f'_1 & a_7    &= g_1 - g'_1 & a_8    &= h_1 - h'_1 \\
    a_{11} &= f_2 - f'_2 & a_{12} &= g_2 - g'_2 & a_{13} &= h_2 - h'_2
  \end{aligned}$ & \\
  $\begin{aligned}[t]
    a_4    &=     - f'_0a_6 - g'_0a_{11} & a_5    &=     - f'_0a_7 - g'_0a_{12} \\
    a_9    &= a_1 - f'_1a_6 - g'_1a_{11} & a_{10} &= a_2 - f'_1a_7 - g'_1a_{12} \\
    a_{14} &=     - f'_2a_6 - g'_2a_{11} & a_{15} &=     - f'_2a_7 - g'_2a_{12}
  \end{aligned}$ & \\
  If $a_1, a_6, a_{11}$ are all zero, then abort. & \\
  If $a_1 = 0$ but $a_6$ or $a_{11}$ is non-zero, then swap rows so that $a_1 \neq 0$. & \\
  \hline \hline
  Compute row echelon form $M'_{\text{add}}$ of $M_{\text{add}}$ & 21M+12A \\
  \hline
  $\begin{aligned}[t]
    d_1 &= a_1a_{12} - a_2a_{11} & d_2 &= a_6a_{12} - a_7a_{11} 
  \end{aligned}$ & \\
  $\begin{aligned}[t]
    b_1 &= a_1a_7    - a_2a_6 & b_5 &= b_1a_{13} - d_1a_8    + d_2a_3 \\
    b_2 &= a_1a_8    - a_3a_6 & b_6 &= b_1a_{14} - d_1a_9    + d_2a_4 \\
    b_3 &= a_1a_9    - a_4a_6 & b_7 &= b_1a_{15} - d_1a_{10} + d_2a_5 \\
    b_4 &= a_1a_{10} - a_5a_6
  \end{aligned}$ & \\
  If $b_1 =0$ or $b_5 = 0$, then abort. & \\
  \hline \hline
  Compute $Z \cdot \rref\left(M_{\text{add}}\right)$ & 18M+6A \\
  \hline
  $\begin{aligned}[t]
    Y &= a_1b_1 & e_1 &= b_3b_5 - b_2b_6 \\
    Z &= Yb_5   & e_2 &= b_4b_5 - b_2b_7
  \end{aligned}$ & \\
  $\begin{aligned}[t]
    A_1 &= b_1(a_4b_5 - b_6a_3) - a_2e_1 & B_1 &= a_1e_1 & C_1 &= Yb_6 \\
    A_2 &= b_1(a_5b_5 - b_7a_3) - a_2e_2 & B_2 &= a_1e_2 & C_2 &= Yb_7
  \end{aligned}$ & \\
  \hline \hline
    Compute $\ker M_{\text{add}}$ & 18M+14A \\
  \hline
  $\begin{aligned}[t]
    U1 &= Zf_0 - C_1h_1 - B_1g_1 - A_1f_1 & V1 &= Zg_0 - C_2h_1 - B_2g_1 - A_2f_1 \\
    U2 &=      - C_1h_2 - B_1g_2 - A_1f_2 & V2 &=      - C_2h_2 - B_2g_2 - A_2f_2 \\
    U3 &= Zf_1 - A_1 & V3 &= Zg_1 - A_2 \\
    U4 &= Zf_2 - B_1 & V4 &= Zg_2 - B_2 \\
    U5 &=      - C_1 & V5 &=      - C_2.
  \end{aligned}$ & \\
  \hline \hline
    Compute $\zeta, \tau$ & 1I+5M+2S+3A \\
  \hline
  $\begin{aligned}[t]
    z_0 &= U_5^2 + Z(u_4 - v_5) & z_1 &= Zz_0 \\
    z_2 &= \frac 1 {z_1} & z_3 &= Zz_2 \\
    \zeta &= z_0z_2 & \tau  &= Z^2z_3
  \end{aligned}$ & \\
  If $z_0 = 0$, then abort. & \\
  \hline \hline
    Compute $u_1, \ldots, u_5, v_1, \ldots, v_5$ & 10M \\
  \hline
  $\begin{aligned}[t]
    u_1 &= \zeta U_1 & u_4 &= \zeta U_4 & v_1 &= \zeta V_1 & v_4 &= \zeta V_4 \\
    u_2 &= \zeta U_2 & u_5 &= \zeta U_5 & v_2 &= \zeta V_2 & v_5 &= \zeta V_5 \\
    u_3 &= \zeta U_3                    & v_3 &= \zeta V_3
  \end{aligned}$ & \\
  \hline  \hline
    Compute $f'', g'', h''$ & 27M+1S+37A \\
  \hline
    $f''_2 = u_5^2 + u_4 - v_5$ & \\
    $r_0   = u_5(f''_2 + u_4 - c_7) + u_3 - v_4$ & \\
    $r_1   = f''_2(f''_2 - u_4)$ & \\
    $g''_1 = r_1 - u_5(u_3 + r_0) + v_3$ & \\
    $g''_2 = -u_4u_5 + v_4 - r_0 + \tau(u_4r_0 - u_5g''_1 - u_2)$ & \\
    $f''_1 = r_0 + g''_2$ & \\
    $\begin{aligned}[t]
    f''_0 &= -c_7(r_1 + g''_2u_5) + u_5(f''_2u_3 + f''_1u_4 - c_4 + u_2) \\
          &~~~ + g''_2u_3 + g''_1u_4 - f''_2v_3 - f''_1v_4 + u_1 - v_2
    \end{aligned}$ & \\
    $g''_0 = u_5(c_3 - f''_0 - u_1 - f''_1u_3) - g''_1u_3 + f''_1v_3 + v_1$ & \\
    $h''_0 = \tau(f''_0g''_1 - g''_0r_0)$ & \\
    $h''_1 = \tau(g''_1g''_2 - g''_0)$ & \\
    $h''_2 = g''_1 + \tau(f''_0 - g''_2r_0)$ & \\
  \hline  \hline
    Output $f''_0, f''_1, f''_2, g''_0, g''_1, g''_2, h''_0, h''_1, h''_2$, and $(f''_2)\inv = \tau$ & \\
  \hline
\end{longtable}

%%%%%%%%%%%%%%%%%%%%%%%%%%%%%%%%%%%%%%%%%%%%
%%%%%   Explicit Formulae : DOUBLING   %%%%%
%%%%%%%%%%%%%%%%%%%%%%%%%%%%%%%%%%%%%%%%%%%%

\begin{longtable}{|l|r|}
  \caption{Explicit Formulae for doubling type 31 divisors (typical case)\label{tab_double_31}}\\
  \hline
  Doubling & 1I+135M+3S+116A \\
  Input: $I_D = \pid{f, g, h}$, $\phi$ & \\
  $\begin{aligned}[t]
    f &= x^2 + f_2y + f_1x + f_0 & f_2 \neq 0 \\
    g &=  xy + g_2y + g_1x + g_0 & \phi = \frac 1 {f_2} \\
    h &= y^2 + h_2y + h_1x + h_0
  \end{aligned}$ & \\
  Output: $I_{\bar{\bar{2D}}} = \pid{f'', g'', h''}$, $(f''_2)\inv$ & \\
  $\begin{aligned}[t]
    f'' &= x^2 + f''_2y + f''_1x + f''_0 \\
    g'' &=  xy + g''_2y + g''_1x + g''_0 \\
    h'' &= y^2 + h''_2y + h''_1x + h''_0 
  \end{aligned}$ & \\
  \hline \hline
  Compute polynomials $g'$ and $h'$ such that $fh' \equiv gg' \pmod F$ & 10M+16A \\
  \hline
  $h'_3 = f_2$ & \\
  $g'_2 = f_1 - g_2$ & \\
  $k_0  = g_2g'_2$ & \\
  $h'_2 = \phi(k_0 - f_0)$ & \\
  $g'_1 = f_2(f_2 - c_7) + h'_2 - g_1$ & \\
  $k_1  = g_1g'_1$ & \\
  $h'_1 = -f_1f_2$ & \\
  $\begin{aligned}[t]
    g'_0 &= h'_2f_1 + f_2(h'_1 - c_4) \\
         & ~~~ - (g_1 + g_2)(g'_1 + g'_2) + k_0 + k_1 - g_0 \\
  \end{aligned}$ & \\
  $h'_0 = f_2(c_3 - f_0) - h'_1f_1 + k_1$ & \\
  \hline \hline
  Compute elements $a_i$ of $M_{\text{doub}}$ & 26M+28A \\
  \hline
  $\begin{aligned}[t]
    a_1    &= g'_0 - g_0 & a_2    &= h'_0 - h_0 - f_0f_2 \\
    a_6    &= g'_1 - g_1 & a_7    &= h'_1 - h_1 - f_1f_2 \\
    a_{11} &= g'_2 - g_2 & a_{12} &= h'_2 - h_2 - f_2f_2
  \end{aligned}$ & \\
  $\begin{aligned}[t]
    a_4    &=     - f_0a_6 - g_0a_{11} & a_5    &=     - f_0a_7 - g_0a_{12} \\
    a_9    &= a_1 - f_1a_6 - g_1a_{11} & a_{10} &= a_2 - f_1a_7 - g_1a_{12} \\
    a_{14} &=     - f_2a_6 - g_2a_{11} & a_{15} &=     - f_2a_7 - g_2a_{12}
  \end{aligned}$ & \\
  $\begin{aligned}[t]
    a_3    &= \phi \left(     - g_0a_6 - h_0a_{11} + g_1a_1    - a_5    - g'_2a_2    \right) \\
    a_8    &= \phi \left(              - h_1a_{11}             - a_{10} - g'_2a_7    \right) \\
    a_{13} &= \phi \left( a_1 - g_2a_6 - (h_2 - g_1)a_{11}     - a_{15} - g'_2a_{12} \right)
  \end{aligned}$ & \\
  If $a_1, a_6, a_{11}$ are all zero, then abort. & \\
  If $a_1 = 0$ but $a_6$ or $a_{11}$ is non-zero, then swap rows so that $a_1 \neq 0$. & \\
  \hline \hline
  Compute row echelon form $M'_{\text{doub}}$ of $M_{\text{doub}}$ & 21M+12A \\
  \hline
  $\begin{aligned}[t]
    d_1 &= a_1a_{12} - a_2a_{11} & d_2 &= a_6a_{12} - a_7a_{11} 
  \end{aligned}$ & \\
  $\begin{aligned}[t]
    b_1 &= a_1a_7    - a_2a_6 & b_5 &= b_1a_{13} - d_1a_8    + d_2a_3 \\
    b_2 &= a_1a_8    - a_3a_6 & b_6 &= b_1a_{14} - d_1a_9    + d_2a_4 \\
    b_3 &= a_1a_9    - a_4a_6 & b_7 &= b_1a_{15} - d_1a_{10} + d_2a_5 \\
    b_4 &= a_1a_{10} - a_5a_6
  \end{aligned}$ & \\
  If $b_1 =0$ or $b_5 = 0$, then abort. & \\
  \hline \hline
  Compute $Z \cdot \rref\left(M_{\text{doub}}\right)$ & 18M+6A \\
  \hline
  $\begin{aligned}[t]
    Y &= a_1b_1 & e_1 &= b_3b_5 - b_2b_6 \\
    Z &= Yb_5   & e_2 &= b_4b_5 - b_2b_7
  \end{aligned}$ & \\
  $\begin{aligned}[t]
    A_1 &= b_1(a_4b_5 - b_6a_3) - a_2e_1 & B_1 &= a_1e_1 & C_1 &= Yb_6 \\
    A_2 &= b_1(a_5b_5 - b_7a_3) - a_2e_2 & B_2 &= a_1e_2 & C_2 &= Yb_7
  \end{aligned}$ & \\
  \hline \hline
    Compute $\ker M_{\text{doub}}$ & 18M+14A \\
  \hline
  $\begin{aligned}[t]
    U1 &= Zf_0 - C_1h_1 - B_1g_1 - A_1f_1 & V1 &= Zg_0 - C_2h_1 - B_2g_1 - A_2f_1 \\
    U2 &=      - C_1h_2 - B_1g_2 - A_1f_2 & V2 &=      - C_2h_2 - B_2g_2 - A_2f_2 \\
    U3 &= Zf_1 - A_1 & V3 &= Zg_1 - A_2 \\
    U4 &= Zf_2 - B_1 & V4 &= Zg_2 - B_2 \\
    U5 &=      - C_1 & V5 &=      - C_2.
  \end{aligned}$ & \\
  \hline \hline
    Compute $\zeta, \tau$ & 1I+5M+2S+3A \\
  \hline
  $\begin{aligned}[t]
    z_0 &= U_5^2 + Z(u_4 - v_5) & z_1 &= Zz_0 \\
    z_2 &= \frac 1 {z_1} & z_3 &= Zz_2 \\
    \zeta &= z_0z_2 & \tau  &= Z^2z_3
  \end{aligned}$ & \\
  If $z_0 = 0$, then abort. & \\
  \hline \hline
    Compute $u_1, \ldots, u_5, v_1, \ldots, v_5$ & 10M \\
  \hline
  $\begin{aligned}[t]
    u_1 &= \zeta U_1 & u_4 &= \zeta U_4 & v_1 &= \zeta V_1 & v_4 &= \zeta V_4 \\
    u_2 &= \zeta U_2 & u_5 &= \zeta U_5 & v_2 &= \zeta V_2 & v_5 &= \zeta V_5 \\
    u_3 &= \zeta U_3                    & v_3 &= \zeta V_3
  \end{aligned}$ & \\
  \hline  \hline
    Compute $f'', g'', h''$ & 27M+1S+37A \\
  \hline
    $f''_2 = u_5^2 + u_4 - v_5$ & \\
    $r_0   = u_5(f''_2 + u_4 - c_7) + u_3 - v_4$ & \\
    $r_1   = f''_2(f''_2 - u_4)$ & \\
    $g''_1 = r_1 - u_5(u_3 + r_0) + v_3$ & \\
    $g''_2 = -u_4u_5 + v_4 - r_0 + \tau(u_4r_0 - u_5g''_1 - u_2)$ & \\
    $f''_1 = r_0 + g''_2$ & \\
    $\begin{aligned}[t]
    f''_0 &= -c_7(r_1 + g''_2u_5) + u_5(f''_2u_3 + f''_1u_4 - c_4 + u_2) \\
          &~~~ + g''_2u_3 + g''_1u_4 - f''_2v_3 - f''_1v_4 + u_1 - v_2
    \end{aligned}$ & \\
    $g''_0 = u_5(c_3 - f''_0 - u_1 - f''_1u_3) - g''_1u_3 + f''_1v_3 + v_1$ & \\
    $h''_0 = \tau(f''_0g''_1 - g''_0r_0)$ & \\
    $h''_1 = \tau(g''_1g''_2 - g''_0)$ & \\
    $h''_2 = g''_1 + \tau(f''_0 - g''_2r_0)$ & \\
  \hline  \hline
    Output $f''_0, f''_1, f''_2, g''_0, g''_1, g''_2, h''_0, h''_1, h''_2$, and $(f''_2)\inv = \tau$ & \\
  \hline
\end{longtable}


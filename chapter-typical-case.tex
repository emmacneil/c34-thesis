%%%%%%%%%%%%%%%%%%%%%%%%%%%%%%%%
%%%%%                      %%%%%
%%%%%   The Typical Case   %%%%%
%%%%%                      %%%%%
%%%%%%%%%%%%%%%%%%%%%%%%%%%%%%%%

\section{Typical Case}
\label{chap_typical_case}

\begin{itemize}
  \item Algorithms for typical case
  \item Construct matrix for addition
  \item Construct matrix for doubling
  \item Row-reducing matrix and computing kernel identical for both after matrix constructed
  \item Typically, u, v are enough.
  \item Constant coefficients of u and v are not needed
  \item Computing RREF requires one inversion
  \item Reducing requires another inversion
  \item Save time by performing both inversions simultaneously
\end{itemize}

%%%%%%%%%%%%%%%%%%%%%%%%%%%%%%%
%%%%%                     %%%%%
%%%%%   Addition Matrix   %%%%%
%%%%%                     %%%%%
%%%%%%%%%%%%%%%%%%%%%%%%%%%%%%%

\subsection{Constructing the Matrix $M_{\text{add}}$.}

We wish to add two type 31 divisors $D$ and $D'$,
given by ideals $I_D = \pid{f,g,h}$ and $I_{D'} = \pid{f',g',h'}$, where
\begin{align*}
  f &= x^2 + f_2y + f_1x + f_0 & f' &= x^2 + f'_2y + f'_1x + f'_0 \\
  g &=  xy + g_2y + g_1x + g_0 & g' &=  xy + g'_2y + g'_1x + g'_0 \\
  h &= y^2 + h_2y + h_1x + h_0 & h' &= y^2 + h'_2y + h'_1x + h'_0.
\end{align*}
We need not assume that $D$ or $D'$ are typical.
It is possible for the sum of a typical and an atypical divisor to be typical
and for the sum of two atypical divisors to be typical.
The formulae to follow work equally well whether or not $D$ or $D'$ are typical,
though they will fail if the sum $D + D'$ is atypical.

We construct the matrix 
\[ M_{\text{add}} =
\begin{pmatrix}
  a_1 & a_2 & a_3 & a_4 & a_5 \\
  a_6 & a_7 & a_8 & a_9 & a_{10} \\
  a_{11} & a_{12} & a_{13} & a_{14} & a_{15}
\end{pmatrix}. \]
The first three columns are simply the reductions of $f$, $g$, and $h$ modulo $f'$, $g'$, and $h'$, respectively.
\begin{align*}
  a_1    &= f_0 - f'_0 & a_2    &= g_0 - g'_0 & a_3    &= h_0 - h'_0 \\
  a_6    &= f_1 - f'_1 & a_7    &= g_1 - g'_1 & a_8    &= h_1 - h'_1 \\
  a_{11} &= f_2 - f'_2 & a_{12} &= g_2 - g'_2 & a_{13} &= h_2 - h'_2.
\end{align*}
The last two columns are the reductions of $xf$ and $xg$ modulo $f,g,h$.
We may view multiplication by $x$ as a linear transformation $T_x$ on the space $W^{x^2y}/W_{D'}^{x^2y}$.
The matrix of $T_x$ is
\[ T_x = \begin{pmatrix}
    0 & -f'_0 & -g'_0 \\
    1 & -f'_1 & -g'_1 \\
    0 & -f'_2 & -g'_2
  \end{pmatrix}. \]
Therefore we compute the last two columns via
\[ 
  \begin{pmatrix}
    a_4    & a_5    \\
    a_9    & a_{10} \\
    a_{14} & a_{15}
  \end{pmatrix} = 
  \begin{pmatrix}
    0 & -f'_0 & -g'_0 \\
    1 & -f'_1 & -g'_1 \\
    0 & -f'_2 & -g'_2
  \end{pmatrix}
  \begin{pmatrix}
    a_1    & a_2    \\
    a_6    & a_7    \\
    a_{11} & a_{12}
  \end{pmatrix}
\]
to get
\begin{align*}
  a_4    &=     - f'_0a_6 - g'_0a_{11} & a_5    &=     - f'_0a_7 - g'_0a_{12} \\
  a_9    &= a_1 - f'_1a_6 - g'_1a_{11} & a_{10} &= a_2 - f'_1a_7 - g'_1a_{12} \\
  a_{14} &=     - f'_2a_6 - g'_2a_{11} & a_{15} &=     - f'_2a_7 - g'_2a_{12}.
\end{align*}

\begin{lemma}
  The matrix $M_{\text{add}}$ may be computed in 12M+17A.
\end{lemma}


%%%%%%%%%%%%%%%%%%%%%%%%%%%%%%%
%%%%%                     %%%%%
%%%%%   Doubling Matrix   %%%%%
%%%%%                     %%%%%
%%%%%%%%%%%%%%%%%%%%%%%%%%%%%%%

\subsection{Constructing the Matrix $M_{\text{doub}}$.}

We wish to double a type 31 divisor $D$, given by the ideal $I_D = \pid{f,g,h}$, where
\begin{align*}
  f &= x^2 + f_2y + f_1x + f_0 \\
  g &=  xy + g_2y + g_1x + g_0 \\
  h &= y^2 + h_2y + h_1x + h_0.
\end{align*}
We will assume that $D$ is typical, so that $f_2 \neq 0$.
Let the inverse of $f_2$ be $\phi = \frac 1 {f_2}$.
We will assume that $\phi$ is given as an input to the doubling algorithm.

We must find $g', h' \in K[C]$ such that $fh' \equiv gg' \pmod F$.
The polynomials $g'$ and $h'$ are of the forms
\begin{align*}
  g' &= xy            + g'_2y + g'_1x + g'_0 \\
  h' &= y^2 + h'_3x^2 + h'_2y + h'_1x + h'_0.
\end{align*}
Solving the equation $fh' - gg' - f_2F = 0$, gives the solution
\begin{align*}
  h'_3 &= f_2 \\
  g'_2 &= f_1 - g_2 \\
  h'_2 &= \phi(g'_2g_2 - f_0) \\ 
  g'_1 &= f_2(f_2 - c_7) + h'_2 - g_1 \\
  h'_1 &= -f_1f_2 \\
  g'_0 &= h'_2f_1 + f_2(h'_1 - c_4) - g'_2g_1 - g'_1g_2 - g_0 \\
  h'_0 &= f_2(c_3 - f_0) - h'_1f_1 + g'_1g_1.
\end{align*}
\begin{comment}
% For arbitrary field characteristic
\begin{align*}
  h'_3 &= f_2 \\
  g'_2 &= - c_8f_2 + f_1 - g_2 \\
  h'_2 &= c_5 + \phi(g'_2g_2 - f_0) \\ 
  g'_1 &= f_2(f_2 - c_7) + h'_2 - g_1 \\
  h'_1 &= f_2(c_6 - f_1) \\
  g'_0 &= h'_2f_1 + f_2(h'_1 - c_4) - g'_2g_1 - g'_1g_2 - g_0 \\
  h'_0 &= f_2(c_3 - f_0) - h'_1f_1 + g'_1g_1.
\end{align*}
\end{comment}
\note{12M, or 11M with Karatsuba.}
\[ M_{\text{doub}} =
\begin{pmatrix}
  a_1 & a_2 & a_3 & a_4 & a_5 \\
  a_6 & a_7 & a_8 & a_9 & a_{10} \\
  a_{11} & a_{12} & a_{13} & a_{14} & a_{15}
\end{pmatrix}. \]
\begin{align*}
  a_1    &= g'_0 - g_0 & a_2    &= h'_0 - h_0 - f_0f_2 \\
  a_6    &= g'_1 - g_1 & a_7    &= h'_1 - h_1 - f_1f_2 \\
  a_{11} &= g'_2 - g_2 & a_{12} &= h'_2 - h_2 - f_2f_2 \\
\end{align*}
However, observe that $h'_1 = -f_1f_2$, so $a_7 = 2h'_1 - h_1$.
\[ 
  \begin{pmatrix}
    a_4    & a_5    \\
    a_9    & a_{10} \\
    a_{14} & a_{15}
  \end{pmatrix} = 
  \begin{pmatrix}
    0 & -f'_0 & -g'_0 \\
    1 & -f'_1 & -g'_1 \\
    0 & -f'_2 & -g'_2
  \end{pmatrix}
  \begin{pmatrix}
    a_1    & a_2    \\
    a_6    & a_7    \\
    a_{11} & a_{12}
  \end{pmatrix}.
\]

\begin{align*}
  a_4    &=     - f_0a_6 - g_0a_{11} & a_5    &=     - f_0a_7 - g_0a_{12} \\
  a_9    &= a_1 - f_1a_6 - g_1a_{11} & a_{10} &= a_2 - f_1a_7 - g_1a_{12} \\
  a_{14} &=     - f_2a_6 - g_2a_{11} & a_{15} &=     - f_2a_7 - g_2a_{12}.
\end{align*}

Column 3 is computed by
\begin{align*}
  C_3 &= \phi \left( T_y(C_1) + g_1C_1 - T_x(C_2) - (f_1 - g_2)C_2 \right) \\
      &= \phi \left( T_y(C_1) + g_1C_1 - C_5 - (f_1 - g_2)C_2 \right)
\end{align*}
Noting that $f_1 - g_2 = g'_2$,
\begin{align*}
  a_3    &= \phi \left(     - g_0a_6 - h_0a_{11} + g_1a_1    - a_5    - g'_2a_2    \right) \\
  a_8    &= \phi \left(              - h_1a_{11}             - a_{10} - g'_2a_7    \right) \\
  a_{13} &= \phi \left( a_1 - g_2a_6 - (h_2 - g_1)a_{11}     - a_{15} - g'_2a_{12} \right) .
\end{align*}
\begin{comment}
\begin{align*}
  a_3    &= \phi \left(     - g_0a_6 - h_0a_{11} + g_1a_1    - a_5    - (f_1 - g_2)a_2    \right) \\
  a_8    &= \phi \left(              - h_1a_{11}             - a_{10} - (f_1 - g_2)a_7    \right) \\
  a_{13} &= \phi \left( a_1 - g_2a_6 - (h_2 - g_1)a_{11}     - a_{15} - (f_1 - g_2)a_{12} \right) .
\end{align*}
\end{comment}

\begin{lemma}
  The matrix $M_{\text{doub}}$ may be computed in 36M+42A.
\end{lemma}
\begin{proof}
  The coefficients of $g'$ and $h'$ require 11M+13A to compute as written,
  however, the terms $g'_1g_1$, $g'_1g_2 + g'_2g_1$, and $g'_2g_2$ all appear.
  Given $g'_1g_1$ and $g'_2g_2$, rather than compute $g'_1g_2 + g'_2g_1$ in 2M+1A,
  we may compute it in 1M+4A via
  \[ g'_1g_2 + g'_2g_1 = (g'_1 + g'_2)(g_1 + g_2) - g'_1g_1 - g'_1g_2. \]
  Thus the coefficients of $g'$ and $h'$ are obtained in 10M+16A.
  
  The first column of $M_{\text{doub}}$ costs 3A to compute.
  The second column costs 2M+5A to compute, counting doubling $h'_1$ as 1 addition.
  Columns 4 and 5 each cost 6M+4A, while column 3 costs 12M+10A.
\end{proof}



%%%%%%%%%%%%%%%%%%%%%%%%%%%%%%%%
%%%%%                      %%%%%
%%%%%   Computing Kernel   %%%%%
%%%%%                      %%%%%
%%%%%%%%%%%%%%%%%%%%%%%%%%%%%%%%

\subsection{Computing Kernel}

After computing the matrix $M_{\text{add}}$ or $M_{\text{doub}}$,
the addition and doubling algorithms proceed identically,
so let $M$ be one of the above matrices.
If we are adding, let $M = M_{\text{add}}$ and $A = D + D'$.
If we are doubling, let $M = M_{\text{doub}}$ and $A = 2D$ instead.

In Chapters \ref{chap_addition} and \ref{chap_doubling},
we would find a reduced Gr\"obner basis for $A$ by finding a reduced echelon basis for $\ker M$.
Reducing $M$ to its reduced row echelon form typically requires an inversion in $\bb F_q$
and this ensures that the resulting Gr\"obner basis for $A$ is monic.
In the following, we delay inversion until the next section.
Instead, we compute a multiple of $\rref(M)$.
Consequently, we obtain a pair $\{U,V\}$ of non-monic polynomials that generate $I_A$.

Recall that the elements of $M$ were
\[ M =
\begin{pmatrix}
  a_1 & a_2 & a_3 & a_4 & a_5 \\
  a_6 & a_7 & a_8 & a_9 & a_{10} \\
  a_{11} & a_{12} & a_{13} & a_{14} & a_{15}
\end{pmatrix}. \]
If the first column of $M$ is zero, then $A$ is not a type 61 divisor,
so we must abort and fall back on other methods described in this thesis.
If the first column is non-zero, but $a_1 = 0$, then swap rows and relabel the elements so that $a_1 \neq 0$.
We now put $M$ into echelon form,
\[ M' =
\begin{pmatrix}
  a_1 & a_2 & a_3 & a_4 & a_5 \\
    0 & b_1 & b_2 & b_3 & b_4 \\
    0 &   0 & b_5 & b_6 & b_7
\end{pmatrix}. \]
\begin{align*}
  d_1 &= a_1a_{12} - a_2a_{11} & b_1 &= a_1a_7    - a_2a_6 & b_5 &= b_1a_{13} - d_1a_8    + d_2a_3 \\
  d_2 &= a_6a_{12} - a_7a_{11} & b_2 &= a_1a_8    - a_3a_6 & b_6 &= b_1a_{14} - d_1a_9    + d_2a_4 \\
      &                        & b_3 &= a_1a_9    - a_4a_6 & b_7 &= b_1a_{15} - d_1a_{10} + d_2a_5 \\
      &                        & b_4 &= a_1a_{10} - a_5a_6
\end{align*}
If $b_1 = 0$ or $b_5 = 0$, then $A$ is not of type 61, so we must abort.
Otherwise, we reduce $M'$ further to
\[ M'' =
\begin{pmatrix}
  Z & 0 & 0 & A_1 & A_2 \\
  0 & Z & 0 & B_1 & B_2 \\
  0 & 0 & Z & C_1 & C_2
\end{pmatrix}. \]
\begin{align*}
  Y &= a_1b_1 & e_1 &= b_3b_5 - b_2b_6 \\
  Z &= Yb_5   & e_2 &= b_4b_5 - b_2b_7
\end{align*}
\begin{align*}
  A_1 &= b_1(a_4b_5 - b_6a_3) - a_2e_1 & B_1 &= a_1e_1 & C_1 &= Yb_6 \\
  A_2 &= b_1(a_5b_5 - b_7a_3) - a_2e_2 & B_2 &= a_1e_2 & C_2 &= Yb_7
\end{align*}
This matrix $M''$ is $Z \cdot \rref(M)$.
If we have made it this far without aborting, then $a_1$, $b_1$, and $b_5$ are non-zero, so $Z$ is non-zero too.

The kernel of $M$ is $\Span_K\{U,V\}$, where
\begin{align*}
  U &= Zxf - C_1h - B_1g - A_1f \\
  V &= Zxg - C_2h - B_2g - A_2f.
\end{align*}
Let $U_1, \ldots, U_5$ be the coefficients of $x, \ldots, y^2$ in $U$
and let $V_1, \ldots, V_5$ be the coefficients of $x, \ldots, y^2$ in $V$.
These are given explicitly by
\begin{align*}
  U1 &= Zf_0 - C_1h_1 - B_1g_1 - A_1f_1 & V1 &= Zg_0 - C_2h_1 - B_2g_1 - A_2f_1 \\
  U2 &=      - C_1h_2 - B_1g_2 - A_1f_2 & V2 &=      - C_2h_2 - B_2g_2 - A_2f_2 \\
  U3 &= Zf_1 - A_1 & V3 &= Zg_1 - A_2 \\
  U4 &= Zf_2 - B_1 & V4 &= Zg_2 - B_2 \\
  U5 &=      - C_1 & V5 &=      - C_2.
\end{align*}
Note that in the computations to follow, we will not need the constant coefficients of $U$ and $V$.

\begin{lemma}
  Given $M$, the non-constant coeffiencts of the polynomials $U$ and $V$ may be computed in 57M+32A.
\end{lemma}
\begin{proof}
  The elements $b_i$ of the matrix $M'$ are computed in a total of 21M+12A.
  The matrix $M''$ is then computed in an additional 18M+6A.
  Given $M''$, the coefficients $U_i, V_i$ for $1 \leq i \leq 5$ cost a total of 18M+14A.
\end{proof}



%%%%%%%%%%%%%%%%%%%%%%%%
%%%%%              %%%%%
%%%%%   Reducing   %%%%%
%%%%%              %%%%%
%%%%%%%%%%%%%%%%%%%%%%%%

\subsection{Reducing}

We now have two (non-monic) polynomials $U$ and $V$.
Let $u = \frac U Z$ and $v = \frac V Z$.
The type 61 divisor $A$ has the ideal $I_A = \pid{u, v}$,
where $u$ and $v$ are the first two of three polynomials needed for the reduced Gr\"obner basis of $I_A$.
We will not need the third polynomial $w$ of $I_A$'s Gr\"obner basis,
but we note that by Lemma \ref{lem_fgh_is_fg_61},
there are polynomials $r,s \in K[C]$ and a constant $t \in K$
such that $ru + sv + tw \equiv 0 \pmod F$.
The divisor $A$ is typical if and only if $t \neq 0$.
Moreover, $-t$ will become the coefficient of $y$ in $f''$ in the ideal of the reduced divisor
$I_{\bar{\bar{A}}} = \pid{f'', g'', h''}$.

The inverse of $-t$ will be needed, as will the inverse of $Z$.
Let their inverses be $\zeta = \frac 1 Z$ and $\tau = -\frac 1 t$.
We compute them using only a single inversion operation by
\begin{align*}
  z_0 &= U_5^2 + Z(u_4 - v_5) \\
  z_1 &= Zz_0 \\
  z_2 &= \frac 1 {z_1} \\
  z_3 &= Zz_2 \\
  \zeta &= z_0z_2 \\
  \tau  &= Z^2z_3.
\end{align*}
\begin{comment}
\begin{align*}
  z_0 &= U_5^2 - Z(U_5c_8 - u_4 + v_5) \\
  z_1 &= Zz_0 \\
  z_2 &= \frac 1 {z_1} \\
  z_3 &= Zz_2 \\
  \zeta &= z_0z_2 \\
  \tau  &= Z^2z_3.
\end{align*}
\end{comment}
Note the value of $t$ according to Lemma \ref{lem_fgh_is_fg_61} and the value of $z_0$ above.
We have $z_0 = Z^2t$, so
\[ \frac 1 \tau = \frac 1 {Z^2z_3} 
                = \frac 1 {Z^3z_2} 
                = \frac {z_1} {Z^3} 
                = \frac {z_0} {Z^2}
                = \frac {U_5^2 + Z(U_4 - V_5)} {Z^2}
                = u_5^2 + u_4 - v_5
                = -t. \]
\begin{comment}
\[ \frac 1 \tau = \frac 1 {Z^2z_3} 
                = \frac 1 {Z^3z_2} 
                = \frac {z_1} {Z^3} 
                = \frac {z_0} {Z^2}
                = \frac {U_5^2 - Z(U_5c_8 - U_4 + V_5)} {Z^2}
                = u_5^2 - u_5c_8 + u_4 - v_5
                = -t. \]
\end{comment}
Now let $u_1, \ldots, u_5$ and $v_1, \ldots, v_5$ be the coefficients of $x, y, \ldots, y^2$ in $u$ and $v$.
Compute them by
\[ u_i = \zeta U_i, ~~v_i = \zeta V_i, ~~1 \leq i \leq 5. \]

Following Theorem \ref{thm_fast_reduction},
we find polynomials $f'' = x^2 + f''_2y + f''_1x + f''_0$
and $G = xy + G_3x^2 + G_2y + G_1x + G_0$ such that $I_{\bar{\bar A}} = \pid{f'', G}$.
\begin{align*}
  G_3 &= u_5 \\
  f''_2 &= u_5^2 + u_4 - v_5 \\
  G_2 &= v_4 + u_5v_5 + \tau(u_5(u_3u_5 + v_5(u_4 - c_7) - v_3) + v_5(u_3 - v_4) - u_2) \\
  e_3 &= f''_2v_5 - G_2u_5 \\
  f''_1 &= u_5(u_4 - c_7) + G_2 + u_3 - v_4 \\
  G_1 &= -u_3u_5 - e_3 + v_3 \\
  f''_0 &= c_7e_3 + u_5(u_2 - c_4) + G_2u_3 + G_1u_4 - f''_2v_3 - f''_1v_4 + u_1 - v_2 \\
  G_0 &= u_5(c_3 - u_1) - G_1u_3 + f''_1v_3 + v_1.
\end{align*}
\begin{comment}
\begin{align*}
  G_3 &= u_5 \\
  f''_2 &= u_5(u_5 - c_8) + u_4 - v_5 \\
  G_2 &= v_4 + v_5(u_5 - c_8) + \tau(u_5(u_5(u_3 - c_6) + v_5(u_4 - c_7) + c_5 - v_3) + v_5(u_3 - v_4) - u_2) \\
  e_3 &= f''_2v_5 - G_2u_5 \\
  f''_1 &= u_5(u_4 - c_7) + G_2 + u_3 - v_4 \\
  G_1 &= u_5(c_6 - u_3) - e_3 + v_3 \\
  f''_0 &= c_7e_3 + u_5(u_2 - c_4) + G_2u_3 + G_1u_4 - f''_2v_3 - f''_1v_4 + u_1 - v_2 \\
  G_0 &= -c_6e_3 + u_5(c_3 - u_1) - G_1u_3 + f''_1v_3 + v_1.
\end{align*}
\end{comment}
Next reduce $G$ modulo $f''$ to get $g'' = xy + g''_2y + g''_1x + g''_0$.
\begin{align*}
  g''_2 &= G_2 - u_5f''_2 \\
  g''_1 &= G_1 - u_5f''_1 \\
  g''_0 &= G_0 - u_5f''_0.
\end{align*}
Finally, compute $h'' = y^2 + h''_2y + h''_1x + h''_0$,
the third polynomial in the reduced Gr\"obner basis of $I_{\bar{\bar A}} = \pid{f'', g'', h''}$.
\begin{align*}
  h''_0 &= \tau(f''_0g''_1 + g''_0(g''_2 - f''_1)) \\
  h''_1 &= \tau(g''_1g''_2 - g''_0) \\
  h''_2 &= g''_1 + \tau(g''_2(g''_2 - f''_1) + f''_0).
\end{align*}

\begin{lemma}
  Given $Z, U_1, \ldots, U_5, V_1, \ldots, V_5$,
  the reduced Gr\"obner basis for $I_{\bar{\bar A}}$ can be computed in ???
\end{lemma}
\begin{proof}
  tau and zeta 1I+ 5M+2SQ+2A
  ui and vi       10M
  f'' and G    
\end{proof}

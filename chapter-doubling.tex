%%%%%%%%%%%%%%%%%%%%%%%%
%%%%%              %%%%%
%%%%%   Doubling   %%%%%
%%%%%              %%%%%
%%%%%%%%%%%%%%%%%%%%%%%%

\section{Doubling}
\label{chap_doubling}

Framework for doubling:

Similar to framework for adding.
Rather than find polynomials in $I_D$ that vanish modulo $I_{D'}$,
we want to find polynomials in $I_D$ whose differentials vanish modulo $I_D$.
By Theorem \note{at end of differentials chapter}, such polynomials have a zeros
of higher multiplicity.
If $D = P + Q + R$, and $P,Q,R$ are distinct, this gives $2D = 2P + 2Q + 2R$.
However, if they are not distinct,
say $P \neq Q = R$, then $D = P + 2Q$ and we get instead $D' = 2P + 3Q$.
Proceeding in this fashion ony guarantees that $D$'s points increase in multiplicity,
but not that they double in multiplicity.
In the event that the multiplicity does not double,
we can recover the missing portion from the image of a matrix.

The picture is this.
Let $D$ be the divisor we wish to double.
\begin{center}
  \begin{tikzcd}
    W_{D'}^m \arrow[hook]{r}{\ker M} &
    W_D^m \arrow{r}{d} \arrow[bend left]{rr}{M} &
    \Omega_{K[C]/K} \arrow[two heads]{r}{\pi} &
    \frac {\Omega_{K[C]/K}} {W_D} \arrow[two heads]{r}{\im M} &
    \frac {W_{D''}^m} {W_D^m}
  \end{tikzcd}.
\end{center}
The map $d$ sends a polynomial $f$ to its differential $df$.
Therefore $W_{D'}$, the kernel of $M$, is exactly the polynomials in $W_D^m$ whose
differentials vanish modulo $W_D^m$.
This gives a divisor $D' > D$.
The image of $M$ gives us a divisor $D'' < D$ with the property that $D = D' + D''$.

The map $d$ influences the efficiency of this method.
There are different choices of generator for $\Omega$,
yielding different choices of map $d$.

Let $D$ be a type 11, 21, 22, or typical 31 divisor.
Then $I_D = \pid{f, g}$ and $I_{\bar D} = \pid{f, g'}$ are generated by two polynomials.
There is a polynomial $h'$ such that $fh' \equiv gg' \pmod C$,
Let $A = \div(g', h')$.
Then $\gcd(D, A) = 0$.

There is a generator $\omega$ for $\Omega$ such that
\[ df = g'\omega ~\text{ and }~ dg = -h'\omega. \]

\begin{conjecture}
  If $af + bg \in I$ and $ag' - bh' \in I$,
  then $af + bg \in I^2$.
\end{conjecture}
\begin{proof}
  Suppose $ag' - bh' \in I$.
  \begin{align*}
    af + bg &\equiv af + bg \pmod I^2 \\
    afg' + bgg' &\equiv afg' + bgg' \pmod I^2 \\
    afg' + bgg' &\equiv afg' + bfh' \pmod I^2 \\
    afg' + bgg' - bfh' &\equiv afg' \pmod I^2 \\
    f(ag' - bh') &\equiv (af - bg)g' \pmod I^2 \\
    \frac 1 {g'}f(ag' - bh') &\equiv (af - bg) \pmod I^2 \\
    af - bg &\equiv 0 \pmod I^2.
  \end{align*}
\end{proof}
\begin{conjecture}
  If $af + bg \in I^2$,
  then $af + bg \in I$ and $ag' - bh' \in I$.
\end{conjecture}
\begin{proof}
  Let $z \in I^2$.
  Then $z = af^2 + bfg + cg^2$.
  And $dz = f^2da + 2afdf + fgdb + bgdf + bfdg + g^2dc + 2cgdg \in I$.
\end{proof}


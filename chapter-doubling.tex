%%%%%%%%%%%%%%%%%%%%%%%%
%%%%%              %%%%%
%%%%%   Doubling   %%%%%
%%%%%              %%%%%
%%%%%%%%%%%%%%%%%%%%%%%%

\section{Doubling}
\label{chap_doubling}

Abu Salem and Khuri-Makdisi, in \cite{salem07}, present an algorithm for doubling a divisor $D$.
Their algorithm assumes that $D$ is a reduced typical divisor,
and that no point in the support of $D$ has order greater than 1.
In this chapter, we generalize their algorithm to apply to reduced divisors of any type,
typical or atypical, and without regard to the multiplicities of points appearing in the divisor.

As usual, let $C$ be a $C_{3,4}$ curve over a field $K$, defined by a polynomial $F$.
Given a divisor $D$ on $C$, analogously to Chapter \ref{chap_addition},
we wish compute a divisor equivalent to $2D$ in the divisor class group.
More precisely, we want to compute a reduced Gr\"obner basis for an ideal equivalent to $I_D^2$ in the ideal class group.

The doubling algorithm for divisors on $C$ is very similar to the addition algorithm from the previous chapter.
The only significant change is that the map $M_{\text{add}} = \pi \circ \iota$ is instead $M_{\text{doub}} = \pi \circ d$;
the canonical inclusion map $\iota$ is replaced with the map sending a polynomial $f$ to its differential $df$.
The picture is now this:
\begin{center}
  \begin{tikzcd}
    W_{D_1}^m \arrow[hook]{r}{\ker M_{\text{doub}}} &
    W_D^m \arrow{r}{d} \arrow[bend left]{rr}{M_{\text{doub}}} &
    W^{m'} \arrow[two heads]{r}{\pi} &
    \frac {W^{m'}} {W_D^{m'}} \arrow[two heads]{r}{\im M_{\text{doub}}} &
    \frac {W_{D_0}^{m'}} {W_D^{m'}}
  \end{tikzcd}.
\end{center}
Rather than returning the least common multiple and greatest common divisor,
computing the kernel and image now yields divisors $D_1$ and $D_0$ satisfying $D_0 < D < D_1$.
Typically, $D_1 = 2D$ and $D_0 = 0$.

Notice the presence of a monomial $m'$ in the above diagram.
Given a polynomial $f \in K[C]$, the monomials appearing in $df$ will be larger than the monomials appearing in $f$.
Therefore we will need a larger monomial $m'$ to bound some spaces.

The module of K\"ahler differentials, $\Omega_{K[C]/K}$, is generated as a $K[C]$-module by a single element,
but there is more than one choice of generator.
The map $d$, and therefore the computation of the composed map $M_{\text{doub}}$, depends on this choice of generator.
In \cite{arita05-2}, Arita uses the generator seen in Proposition \ref{prop_differential_generator}.
In \cite{salem07}, Abu Salem and Khuri-Makdisi make a different choice of generator
that allows for more efficient computations.
The latter choice of generator was used in \cite{salem07} to double typical reduced divisors.
We will generalize the method to apply to atypical divisors as well,
although there is one case for which we will use the former choice of generator
(but see Section \ref{sec_future_work}, Future Work).
We discuss these two choices and the resulting algorithms in the two sections to follow.



%%%%%%%%%%%%%%%%%%%%%%%%%%%
%%%%%                 %%%%%
%%%%%   Slow Method   %%%%%
%%%%%                 %%%%%
%%%%%%%%%%%%%%%%%%%%%%%%%%%

\subsection{A General Method}
\label{sec_slow_doubling}

Recall from Proposition \ref{prop_differential_generator}
that the module of K\"ahler differentials $\Omega_{K[C]/K}$ has a generator $dz$
and a derivation $d$ defined by
\begin{align*}
  d : K[C] &\to \Omega_{K[C]/K} \\
  f &\mapsto (f_xF_y - f_yF_x)dz.
\end{align*}
The derivation $d$ is $K$-linear.
We may view $K[C]$ and $\Omega_{K[C]/K}$ as $K$-vector spaces and $d$ a linear transformation between them.
However, just like in the previous chapter, we will restrict ourselves to finite-dimensional vector subspaces of these.
Let $m$ be the largest leading monomial appearing in a reduced Gr\"obner basis of a divisor of degree $\deg(2D)$.
Let $B$ be an echelon basis for $W_D^m$, and let $m'$ be the monomial
\[ m' := \max\{\LM(f_xF_y - f_yF_x) ~|~ f \in B \}. \]
We may now restrict the domain of $d$ and view it as a map
\begin{align*}
  d : W_D^m &\to W^{m'} \\
  f &\mapsto f_xF_y - f_yF_x.
\end{align*}

Now consider the diagram from earlier,
\begin{center}
  \begin{tikzcd}
    W_{D_1}^m \arrow[hook]{r}{\ker M_{\text{doub}}} &
    W_D^m \arrow{r}{d} \arrow[bend left]{rr}{M_{\text{doub}}} &
    W^{m'} \arrow[two heads]{r}{\pi} &
    \frac {W^{m'}} {W_D^{m'}} \arrow[two heads]{r}{\im M_{\text{doub}}} &
    \frac {W_{D_0}^{m'}} {W_D^{m'}}
  \end{tikzcd}.
\end{center}
The kernel of $M_{\text{doub}}$ is exactly the set of polynomials in $W_D^m$ whose differentials vanish modulo $I_D$.
Recall from Corollary \ref{cor_reduced_echelon_basis_gives_reduced_groebner_basis},
a subset of the reduced echelon basis for $\ker M_{\text{doub}}$ gives the reduced Gr\"obner basis for $D_1$,
and the subset is given explicitly by Equation \ref{eq_reduced_basis}.
By Theorem \ref{thm_differential_increases_order},
\[ D_1 = \sum_{P \in \supp(D)} (\ord_P(D) + 1)P, \]
where the sum is taken only over finite $P$.
\begin{example}
  Let $P,Q,R$ be distinct points.
  If $D = P + Q + R$, then $D_1 = 2P + 2Q + 2R$.
  If $D = P + 2Q$, then $D_1 = 2P + 3Q$.
  If $D = 3P$, then $D_1 = 4P$.
\end{example}

If all points in $D$ appear with order 1, then $D_1 = 2D$.
If, however, $D$ has a point appearing with order greater than 1, then $D_1 < 2D$.
The difference between $D_1$ and $2D$ is recovered by $D_0$.
We have
\[ D_0 = \sum_{P \in \supp(D)} (\ord_P(D) - 1)P, \]
and
\[ D_0 + D_1 = 2D. \]

This gives us a doubling algorithm similar to the addition algorithm.
We construct the matrix $M_{\text{doub}}$, then compute $\ker M_{\text{doub}}$, and if necessary (when $\rank M_{\text{doub}} < \deg D$)
compute $\im M_{\text{doub}}$, thereby giving us $D_0$ and $D_1$.
If $M_{\text{doub}}$ has full rank, then $D_1 = 2D$, so we return $D_1$.
Otherwise, use the addition algorithm to compute $\bar{\bar{D_1}} + D_0$.
This is summarized in Algorithm \ref{alg_doubling_slow}.
As was noted before Algorithm \ref{alg_addition},
By ``return $D_1$'', we mean return a reduced Gr\"obner basis for $I_{D_1}$,
which is the subset of the reduced echelon basis for $\ker M_{\text{doub}}$ satisfying Equation \ref{eq_reduced_basis}.
The addition in line \ref{alg_doubling:return_2} is performed by Algorithm \ref{alg_addition}.

\begin{algorithm}
  \caption{Divisor Doubling}
  \label{alg_doubling_slow}
  {\bf Input:} A reduced divisor $D$, represented by the reduced Gr\"obner basis of its ideal $I_D$ \\
  {\bf Output:} A divisor $D'$ equivalent to $2D$, represented by the reduced Gr\"obner basis of its ideal $I_{D'}$
  \begin{algorithmic}[1]
    \If {$D = 0$}
      \State \Return $D$ \label{alg_doubling:return_0}
    \EndIf
    \State Compute $M_{\text{doub}}$
    \State Compute $\rref(M_{\text{doub}})$, $\rank M_{\text{doub}}$, and $\ker M_{\text{doub}}$
    \State Compute $D_1$
    \If {$\rank M_{\text{doub}} = \deg D'$}
      \State \Return $D_1$ \label{alg_doubling:return_1}
    \EndIf
    \State Compute $D_0$
    \State \Return $\bar{\bar{D_1}} + D_0$ \label{alg_doubling:return_2}
  \end{algorithmic}
\end{algorithm}

This algorithm either returns a divisor, or else calls upon the addition algorithm,
Algorithm \ref{alg_addition}, which terminates.

This algorithm has two significant drawbacks.
Firstly, the difference between the leading monomials of $f$ and $df$ can be relatively large.
The larger the gap, the longer it takes to reduce $df$ modulo $I_D$.
In the next section, we present a faster alternative, Algorithm \ref{alg_doubling_fast},
that minimizes that gap between $\LM(f)$ and $\LM(df)$ and never requires computing a second divisor $D_0$.
Secondly, were the addition algorithm to call Algorithm \ref{alg_doubling_slow} to do its doubling
rather than Algorithm \ref{alg_doubling_fast},
it would lead to an infinite loop on some inputs.
Thus we require Algorithm \ref{alg_doubling_fast} in the next section,
which provably terminates, but does not handle one rare case.

If $D$ is a type 31 divisor with ideal and reduced Gr\"obner basis $I_D = \pid{f,g,h}$,
then there is a small possibility that $\pid{f,g} \neq I_D \neq \pid{f,h}$.
This case is handled by Algorithm \ref{alg_doubling_slow}, but not by Algorithm \ref{alg_doubling_fast}.
Thus, we will use Algorithm \ref{alg_doubling_slow} to double type 31 divisors satisfying this rare condition.
All other cases will be doubled by Algorithm \ref{alg_doubling_fast}.



%%%%%%%%%%%%%%%%%%%%%%%%%%%
%%%%%                 %%%%%
%%%%%   Fast Method   %%%%%
%%%%%                 %%%%%
%%%%%%%%%%%%%%%%%%%%%%%%%%%

\subsection{A Faster Method}

Let $D$ be a typical type 31 divisor, with ideal $I_D = \pid{f,g,h}$,
where $\{f,g,h\}$ is a reduced Gr\"obner basis.
Since $D$ is typical, $I_D = \pid{f,g}$.
Suppose also that $D$ is the sum of three distinct points.
The flip $\bar D$ of $D$ has the ideal $I_{\bar D} = \pid{f, g'}$ for some $g'$.
Define
\[ \cal A_D := \frac {\Omega_{K[C]/K}} {\pid{f,g} \Omega_{K[C]/K}}. \]

In \cite{salem07}, rather than consider a generator $dz$ for $\Omega_{K[C]/K}$,
Abu Salem and Khuri Makdisi instead consider a generator $\omega$ for $\cal A_D$.
They show that there exists such a generator with the properties that
\[ df = g'\omega ~\text{ and }~ dg = -h'\omega, \]
where $h'$ satisfies $fh' \equiv -gg'$.
This choice of generator and the resulting map $d$ yielded the most efficient arithmetic at the time.

Their proof of existance of $\omega$ relied on the fact that $D$ consists of distinct points,
a reasonable assumption for the cryptographic applications they were exploring.
We will show that their idea applies in greater generality than was considered in \cite{salem07},
to include most atypical divisors as well.\footnote
{Again, see Section \ref{sec_future_work}, Future Work. This can likely be extended to all divisors.}
The next two lemmas and the theorem following them are a generalization of the three-part Lemma 5.3 in \cite{salem07}.
We change notation from $f,g,g',h'$ to $f,r,r',s'$ to avoid confusion;
$\pid{f,g,h}$ indicates a Gr\"obner basis for a $I_D$, while $\pid{f,r}$ does not.

\begin{lemma}
  \label{lem_frrs}
  Let $D$ be a non-zero reduced divisor
  and suppose $I_D = \pid{f, r}$, where $f$ is the minimum polynomial in $I_D$.
  Then $I_{\bar D} = \pid{f, r'}$ for some $r' \in K[C]$
  and there is a polynomial $s' \in K[C]$ such that $fs' \equiv -rr' \pmod F$.
  Let $A = \div(r', s')$. Then $\gcd(D, A) = 0$ and $A + \bar D = \div r'$.
\end{lemma}
\begin{proof}
  The existance of $s'$ comes from the definition of the colon ideal\footnote{
  See Appendix \ref{appendix_colon_ideal} for properties of the colon ideal.
  The notation $f : r$ is short for $\pid f : \pid r$, the colon ideal of $\pid f$ by $\pid r$.}
  and the fact that $\pid{f,r'} = f : r$.
  Since $D + \bar D = \div f$,
  \begin{align*}
    \pid f
      &= I_D I_{\bar D} \\
      &= \pid{f, r} \pid{f, r'} \\
      &= \pid{f^2, fr, fr', rr'} \\
      &= \pid{f^2, fr, fr', fs'} \\
      &= \pid f \pid{f, r, r', s'},
  \end{align*}
  which implies
  \[ \pid 1 = \pid{f, r, r', s'} = \pid{f, r} + \pid{r', s'} = I_D + I_A. \]
  In the divisor class group, this translates to
  \[ 0 = \gcd(D, A). \]
  As for the claim regarding $\div r'$,
  \begin{align*}
    I_AI_{\bar D}
      &= \pid{r', s'}\pid{f, r'} \\
      &= \pid{fr', (r')^2, fs', r's'} \\
      &= \pid{fr', (r')^2, rr', r's'} \\
      &= \pid{r'}\pid{f, r', r, s'} = \pid{r'},
  \end{align*}
  hence $A + \bar D = \div r'$.
\end{proof}

\begin{lemma}
  \label{lem_frrs2}
  Let $I_D$, $r'$, and $s'$ be as in Lemma \ref{lem_frrs}.
  Let $\frak p$ be a prime factor of $I_D$.
  At least one of $r'$ and $s'$ is a unit in $K[C]/\frak p$.
\end{lemma}
\begin{proof}
  Since $\frak p$ is a maximal ideal,
  to say that $r'$ is a unit in $K[C]/\frak p$ is equivalent to saying $r' \not\in \frak p$.
  If $r'$ and $s'$ are both in $\frak p$, then $I_A = \pid{r', s'} \subseteq \frak p$.
  Then $I_D + I_A \subseteq \frak p$, a contradiction since $I_D + I_A = \pid 1$.
\end{proof}

\begin{theorem}
  \label{thm_frrs}
  Let $I_D$, $f$, $r$, $r'$, and $s'$ be as in Lemma \ref{lem_frrs}.
  Let $a, b \in K[C]$. Then
  \[ af + br \in I_D^2 \iff ar' - bs' \in I_D. \]
\end{theorem}
\begin{proof}
  ($\implies$)
  Suppose $af + br \in I_D^2$.
  Using the fact that $\div f = D + \bar D$
  and $\div r' = A + \bar D$ (Lemma \ref{lem_frrs}),
  \begin{align*}
    \pid{af + br} &\subseteq I_D^2 \\
    \div I_D^2 &\leq \div(af + br) \\
    2D &\leq \div(af + br) \\
    2D + \div r' &\leq \div(afr' + brr') \\
      &= \div(afr' - bfs') \\
      &= \div f + \div(ar' - bs') \\
    2D + \bar D + A &\leq D + \bar D + \div(ar' - bs') \\
    D + A &\leq \div(ar' - bs') \\
    D &\leq \div(ar' - bs') \\
    \pid{ar' - bs'} &\subseteq I_D.
  \end{align*}
  
  ($\impliedby$)
  Suppose $ar' - bs' \in I_D$.
  Let $\frak p^k$ be a prime power factor of $I_D$.
  We will show instead that if $ar' - bs' \in \frak p^k$, then $af + br \in \frak p^{2k}$.
  
  By Lemma \ref{lem_frrs2}, at least one of $r'$ and $s'$ is a unit modulo $\frak p$.
  Without loss of generality, suppose $r'$ is a unit.
  Then $r'$ is a unit modulo $\frak p^{2k}$.
  Because $fs' \equiv -rr'$,
  \begin{align*}
    f(ar' - bs') &\equiv (af + br)r' \pmod{\frak p^{2k}} \\
    0 &\equiv (af + br)r' \pmod{\frak p^{2k}}
      & f, ar' - bs' \in \frak p^k \\
    0 &\equiv (af + br) \pmod{\frak p^{2k}}
      & \text{$r'$ a unit}.
  \end{align*}
\end{proof}

Now suppose $D$ is a reduced divisor,
with ideal $I_D$ generated by two polynomials $\pid{f,r}$,
one of which, $f$, is the minimum polynomial in $I_D$.
Let $d' : W_D^m \to W^{m'}$ be the map defined by
\begin{align*}
  d' : W_D^m &\to W^{m'} \\
  af + bg &\mapsto ar' - bs'.
\end{align*}
We have a new diagram
\begin{center}
  \begin{tikzcd}
    W_{2D}^m \arrow[hook]{r}{\ker M_{\text{doub}}'} &
    W_D^m \arrow{r}{d'} \arrow[bend left]{rr}{M_{\text{doub}}'} &
    W^{m'} \arrow[two heads]{r}{\pi} &
    \frac {W^{m'}} {W_D^{m'}}
  \end{tikzcd}.
\end{center}
The kernel of $M_{\text{doub}}'$ consists exactly of the polynomials $af + br \in W_D^m$ for which $ar' - bs' \in I_D$.
By Theorem \ref{thm_frrs}, this is exactly $W_{2D}^m$.
There is never a need to compute a basis for the image of $M_{\text{doub}}'$
or to compute a second divisor such as $D_0$ from Section \ref{sec_slow_doubling}.

An advantage to this method is that the difference between $\LM(f)$ and $\LM(r')$
is smaller than the difference between $\LM(f)$ and $\LM(f_xF_y - f_yF_x)$.
Consequently, reducing $r'$ modulo $\pid{f,g,h}$ is faster than reducing $f_xF_y - f_yF_x$.

This results in Algorithm \ref{alg_doubling_fast}.
We will use this algorithm to double $D$
when we have a pair of polynomials that generate $I_D$.
When $D$ is type 11, 21, or 22, the reduced Gr\"obner basis for $I_D$ is such a pair.
When $D$ is a typical type 31 divisor, then $\pid{f,g,h}$ is a Gr\"obner basis
and the pair $(f,g)$ has the desired properties.
Recall that this case is characterized by $f_2 \neq 0$.
If $D$ is an atypical type 31 divisor, then $I_D \neq \pid{f,g}$,
but it may be the case that $I_D = \pid{f,h}$.
This case is characterized by $h_2c_8 + f_1c_7 - c_4 + h_1 = 0$.
(Recall that $c_i$ are the coefficient of $F$. See Equation \ref{eq_c34}.)
In the remaining case where $\pid{f,g} \neq I_D \neq \pid{f,h}$, we may use Algorithm \ref{alg_doubling_slow}.
\begin{algorithm}
  \label{alg_doubling_fast}
  \caption{Fast Divisor Doubling}
  {\bf Input:} A reduced divisor $D$, represented by the reduced Gr\"obner basis of its ideal $I_D$.
  This Gr\"obner basis is $\{f,g,h\}$ if $D$ is of type 31, otherwise $\{f,g\}$ \\
  {\bf Output:} A divisor $D'$ equivalent to $2D$, represented by the reduced Gr\"obner basis of its ideal $I_{D'}$
  \begin{algorithmic}[1]
    \If {$D = 0$}
      \State \Return $D$
    \ElsIf {$0 < \deg D < 3$}
      \State $r \gets g$
    \ElsIf {$\deg D = 3$}
      \If {$I_D = \pid{f,g}$}
        \State $r \gets g$
      \ElsIf {$I_D = \pid{f, h}$}
        \State $r \gets h$
      \Else
        \State Compute $D'$ via Algorithm \ref{alg_doubling_slow} on input $D$
        \State \Return $D'$
      \EndIf
    \EndIf
    \State Find $r', s'$ such that $fs' \equiv -rr'$
    \State Compute $M_{\text{doub}}'$
    \State Compute $\rref(M_{\text{doub}}')$ and $\ker M_{\text{doub}}'$.
    \State \Return $2D$
  \end{algorithmic}
\end{algorithm}



%%%%%%%%%%%%%%%%%%%%%%%%%%%%%%%%%%%%%%%%
%%%%%                              %%%%%
%%%%%   Example of Slow Doubling   %%%%%
%%%%%                              %%%%%
%%%%%%%%%%%%%%%%%%%%%%%%%%%%%%%%%%%%%%%%

\subsection{Example of General Doubling}

Consider the $C_{3,4}$ curve $C : F(x,y) = 0$ over $K = \bb F_{31}$,
given by $F(x,y) = y^3 + x^4 + 1$.
Let $I_D = \pid{f,g,h}$, where
\begin{align*}
  f &= x^2 - 10 \\
  g &= xy + 14y + 10x - 15 \\
  h &= y^2 + 12y - 11
\end{align*}
is a reduced Gr\"obner basis.
Then $\pid{f,g} \neq I_D \neq \pid{f,h}$.

The divisor $2D$ has degree 6.
Referring to Table \ref{tab_divisor_types}, no degree 6 divisor has a Gr\"obner basis containing a monomial larger than $x^4$.
Thus, we take $m = x^4$ as a bound and construct the space $W_D^m$.
This space $W_D^m$ is 7-dimensional, with an echelon basis
\[ W_D^m = \Span_K\{ f, g, h, xf, xg, xh, x^2f \}. \]
We compute the differentials of these basis elements,
\begin{align*}
  df      &= 6xy^2 \\
  dg      &= -7x^4 + 6x^3 - y^2 - 3 \\
  dh      &= -8x^3y + 14x^3 \\
  d(xf)   &= 9x^2y^2 + y^2 \\
  d(xg)   &= -10x^5 - 5x^4 - 2xy^2 - 14y^2 - 6x - 11 \\
  d(xh)   &= -11x^4y + 9x^4 - 2y^2 - 3y - 5 \\
  d(x^2f) &= 12x^3y^2 + 2xy^2
\end{align*}
The largest monomial appearing in the differentials is $x^3y^2$, so $d$ is a map $W_D^{x^4} \to W^{x^3y^2}$.
We reduce the differentials modulo $W_D^{x^3y^2}$ (equivalently, we reduce modulo $I_D$), to get
\begin{align*}
  \bar{df}      &= -15y + 11x + 5 \\
  \bar{dg}      &= 12y - 2x - 1 \\
  \bar{dh}      &= 4y + 10x + 9 \\
  \bar{d(xf)}   &= -7y + 9 \\
  \bar{d(xg)}   &= -13y + 3x + 5 \\
  \bar{d(xh)}   &= 6y + 5 \\
  \bar{d(x^2f)} &= 5y - 14x - 12.
\end{align*}
Therefore $M_{\text{doub}}$ is the matrix
\[ M_{\text{doub}} = \begin{pmatrix}
  5 & -1 & 9 & 9 & 5 & 5 & -12 \\
  11 & -2 & 10 & 0 & 3 & 0 & -14 \\
  -15 & 12 & 4 & -7 & -13 & 6 & 5
\end{pmatrix}. \]
It has reduced row echelon form and kernel
\[ \rref(M_{\text{doub}}) = \begin{pmatrix}
  1 & 0 & 0 &   9 & -14 &  4 & 10 \\
  0 & 1 & 0 & -15 &  14 & -8 &  0 \\
  0 & 0 & 1 &  15 &   3 & -6 &  0
\end{pmatrix}, ~~
\ker M_{\text{doub}} = \begin{pmatrix}
   -9 &  14 & -4 & -10 \\
   15 & -14 &  8 &   0 \\
  -15 &  -3 &  6 &   0 \\
    1 &   0 &  0 &   0 \\
    0 &   1 &  0 &   0 \\
    0 &   0 &  1 &   0 \\
    0 &   0 &  0 &   1
\end{pmatrix}. \]
Therefore $W_{D_1}^m$ is
\begin{align*}
  W_{D_1}^m &= \Span_K
\left\{ \begin{array}{l}
    xf - 15h + 15g - 9f, \\
    xg - 3h - 14g + 14f, \\
    xh + 6h + 8g - 4f, \\
    x^2f - 10f
\end{array} \right\} \\
&= \Span_K
\left\{ \begin{array}{l}
    x^3 - 15y^2 + 15xy - 9x^2 - y - 15x - 1, \\
    x^2y - 3y^2 - 7x^2 - 15y + 10, \\ 
    xy^2 + 6y^2 - 11xy - 4x^2 - 2y + 7x + 9, \\
    x^4 + 11x^2 + 7
\end{array} \right\}.
\end{align*}
Since $M_{\text{doub}}$ had full rank, $I_{D_1} = I_{2D}$.
So $2D$ is of type 61 and the first three polynomials
in the computed basis for $W_{D_1}^m$ form a reduced Gr\"obner basis for $I_{2D}$.



%%%%%%%%%%%%%%%%%%%%%%%%%%%%%%%%%%%%%%%%
%%%%%                              %%%%%
%%%%%   Example of Fast Doubling   %%%%%
%%%%%                              %%%%%
%%%%%%%%%%%%%%%%%%%%%%%%%%%%%%%%%%%%%%%%

\subsection{Example of Faster Doubling}
\label{sec_fast_doubling_example}

Consider again the $C_{3,4}$ curve $C : F(x,y) = 0$ over $K = \bb F_{31}$,
given by $F(x,y) = y^3 + x^4 + 1$.
Let $I_D = \pid{f,g,h}$ be the typical type 31 divisor given by
\begin{align*}
  f &= x^2 + 4y + 4x + 6 \\
  g &= xy + y + 5x - 10 \\
  h &= y^2 - 2y - 4x + 15.
\end{align*}
Then $I_D = \pid{f,r}$, where $r = g$.

There exist polynomials
\begin{align*}
  r' &= xy + r'_2y + r'_1x + r'_0 \\
  s' &= y^2 + s'_3x^2 + s'_2y + s'_1x + s'_0
\end{align*}
such that $fs' \equiv rr' \pmod F$.
The coefficients of $r'$ and $s'$ may be found by equating coefficients in $fs' - rr' - f_2F = 0$
and solving the resulting system of linear equations:
\begin{align*}
  s'_3 - 4 &= 0 \\
  -r'_2 + 4 - r_2 &= 0 \\
  4s'_3 - s'_1 + s'_2 - r_1 &= 0 \\
  4s'_3 + s'_1 &= 0 \\
  4s'_2 - s'_2r_2 + 6 &= 0 \\
  4s'_2 + 4s'_1 - s'_2r_1 - r'_1r_2 - r'_0 - r_0 &= 0 \\
  6s'_3 + 4s'_1 - r'_1r_1 + s'_0 &= 0.
\end{align*}
This gives $r' = xy + 3y - 13x + 3$ and $s' = y^2 + 4x^2 + 7y + 15x + 6$.

An echelon basis for $W_D^m$ is given by
\[ W_D^m = \Span_K\{ f, g, h, xf, xg, xh, x^2f\}. \]
We must compute the image under $M_{\text{doub}}'$ of each of these basis elements.
$M_{\text{doub}}'(f)$ and $M_{\text{doub}}'(g)$ are, respectively, the reductions of $r'$ and $s'$ modulo $I_D$,
\begin{align*}
  M_{\text{doub}}'(f) &= 2y + 13x + 13 \\
  M_{\text{doub}}'(g) &= -7y + 3x - 2.
\end{align*}
We have $h = ((y + 5)f - (x + 3)r)/4$,
so compute $M_{\text{doub}}'(h)$ by reducing $((y + 5)r' - (x + 3)s')/4$ modulo $I_D$.
\[ M_{\text{doub}}'(h) = 10y + 10x - 5. \]
The first three columns of $M_{\text{doub}}'$ are
\[ M_{\text{doub}}' = \begin{pmatrix}
   13 & -2 & -5 & * & * & * & * \\
   13 &  3 & 10 & * & * & * & * \\
    2 & -7 & 10 & * & * & * & *
\end{pmatrix}. \]
Following suit for $xf, xg, xh$ and $x^2f$,
\[ M_{\text{doub}}' = \begin{pmatrix}
   13 & -2 & -5 &  4 &   5 &  9 & 2 \\
   13 &  3 & 10 & 13 & -10 & -2 & 5 \\
    2 & -7 & 10 &  8 &  -5 & 12 & 2
\end{pmatrix} \]
with reduced row echelon form and kernel
\[ \rref(M_{\text{doub}}') = \begin{pmatrix}
  1 & 0 & 0 &   1 & -13 & -11 &  0 \\
  0 & 1 & 0 & -13 & -11 & -11 & -9 \\
  0 & 0 & 1 &   7 &  13 &   5 & -3
\end{pmatrix}, ~~
\ker M_{\text{doub}}' = \begin{pmatrix}
   -1 &  13 & 11 & 0 \\
   13 &  11 & 11 & 9 \\
   -7 & -13 & -5 & 3 \\
    1 &   0 &  0 & 0 \\
    0 &   1 &  0 & 0 \\
    0 &   0 &  1 & 0 \\
    0 &   0 &  0 & 1
\end{pmatrix}. \]
Therefore $W_{2D}^m$ is
\begin{align*}
  W_{2D}^m &= \Span_K
\left\{ \begin{array}{l}
    xf   -  7h + 13g -   f, \\
    xg   - 13h + 11g + 13f, \\
    xh   -  5h + 11g + 11f, \\
    x^2f -  3h -  9g
\end{array} \right\} \\
&= \Span_K
\left\{ \begin{array}{l}
     x^3 -  7y^2 - 14xy +  3x^2 - 8y +  2x +  7, \\
    x^2y - 13y^2 + 12xy - 13x^2 - 4y -  6x - 10, \\
    xy^2 -  5y^2 +  9xy +  7x^2 + 3y + 10x +  5, \\
     x^4 + 4x^2y + 4x^3 + 3y^2 + 9xy + 6x^2 + 3y + 2x - 14
\end{array} \right\}.
\end{align*}
So $2D$ is of type 61 and the first three polynomials form a reduced Gr\"obner basis for $I_{2D}$.
